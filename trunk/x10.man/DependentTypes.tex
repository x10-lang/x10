\documentclass[fullpage]{article}
\def\csharp{{\sf C\#}}
\usepackage[pdftex]{graphicx}
\def\Xten{{\sf X10}}
\def\java{{\sf Java}}
\newtheorem{example}{Example}[section]

\begin{document}
\title{Adding dependent types to X10}
\author{
Vijay Saraswat \\
IBM TJ Watson Research Center \\
PO Box 704, Yorktown Heights\\
NY 10598\\
({\sc Draft Version 0.2})\\
(Please do not cite)\\
(Send comments to {\tt vsaraswa@us.ibnm.com}.)
}

\date{24 December 2004}
\maketitle

\begin{abstract}
\Xten{} is a new modern statically typed object-oriented (OO) language
designed for high productivity in the high performance computing (HPC)
domain. \Xten{}, like most OO languages is designed around the notion
of objects, as instances of {\em classes}. However, \Xten{} places
equal emphasis on {\em arrays}, a central data-structure in High
Performance Computing. In particular, \Xten{} supports dense,
distributed multi-dimensional arrays of value and reference types,
built over index sets known as {\em regions}, and mappings from index
sets to places, known as {\em distributions}.  \Xten{} supports a rich
algebra of operations over regions, distributions and arrays.

In designing a static type system for \Xten{} a central problem
arises. It becomes necessary to permit types, such as {\tt region(2)},
the type of all {\tt 2}-dimensional regions, {\tt int[5]}, the type of
all arrays of {\tt int} of length {\tt 5}, and {\tt int[region(2)]},
the type of all arrays over two dimensional regions. The underlying
general idea is that of {\em dependent types} -- types that are
parametrized by {\em values}, just as generic types are types
paramterized by other types.

In this paper we develop a general syntactic and semantic framework
for user-defined dependent types in the context of modern class-based
OO languages such as \java{}, \csharp{} and \Xten{}. The design
supports the definition of dependent and generic interfaces, classes
and methods.  As in staged languages, it distinguishes between
compile-time evaluation and run-time evaluation. Dependent types are
checked at compile-time; expressions occurring in them are perforce
evaluated at compile-time.  We outline the design for a compiler which
can use an extensible constraint-solver for type-checking. No
extension of an underlying Virtual Machine is necessary.

\end{abstract}

\section{Introduction}
\subsection{The basic idea}

Our basic idea about the introduction of dependent types in
class-based statically typed OO languages is as follows. Broadly, we
follow the spirit of generic types, but use values instead of
types. 

We permit the definition of a class {\tt C} to specify a {\em
parameter list}, a list of typed parameters {\tt (T1 x1, \ldots, Tk
xk)} similar in syntactic structure to a method argument list. Each
parameter in this list is treated as a {\tt public final} instance
field of the class. We also permit the specification of a {\em where
clause} in the class definition. A where clause is a boolean
expression on the parameters separated from the parameter list with a
``{\tt :}''.  All instances of the class created at runtime must
satisfy the where clause associated with the class.

It is now straightforward to see what a parametric type should be.  It
should simply by the name of a class (the {\em base class} of the
type) followed by a where clause (a boolean expression on the
parameters of that class), called the {\em condition} of the
parametric type.\footnote{By insisting that the condition reference
only parameters, we ensure that whether an object belongs to a
parametric type or not does not depend on the state of the object. By
explicitly marking final fields as parameters we ensure that only
those fields can be used to construct parametric types which were
intended by the programmer to be used in types. An ``accidentally''
final field cannot be used to construct a parametric type.}  The
denotation (semantic interpretation) of such a type is the set of all
instances of subtypes of the base class whose parameters satisfy the
where clause. Clearly, for the denotation of a parametric type $t$ to
be non-empty it must be the case that the condition of the parametric
type is consistent with the where clause (if any) of the base class of
$t$. We shall require that the type of any variable declaration must
be non-empty.\footnote{Types that are not parametric are always
non-empty, assuming that there is at least one invocation of a
constructor for the base class which does not throw an exception.} For
simplicity we shall permit the syntax {\tt C(t1,\ldots, tk)} for the
type {\tt C (:x1 = t1 \&\& \ldots \&\& xk = tk)} (assuming that the
parameter list for {\tt C} specifies the {\tt k} parameters {\tt
x1,\ldots, xk}).

Parametric types naturally come equipped with a {\em subtyping structure}:
type $t_1$ is a subtype of $t_2$ if the denotation of $t_1$ is a
subset of $t_2$. Clearly this definition satisfies Liskov's
Substitution Principle.

We will permit constructors and methods to specify preconditions as
where clauses on parameters. For an invocation of a method (or
constructor) to be type-correct it must be the case that the
associated where clause is statically known to be satisfied.

We will permit {\em dependent methods} (similar to generic methods): a
method may specify a list of parameters in the same way that a class
may. The parameters may be used to construct dependent types for the
arguments and return value of the method.

We will permit a type to be simultaneously a dependent and a generic type.
For instance, we will permit the type:
{\footnotesize
\begin{verbatim}
  List(3)<region(1)>  
\end{verbatim}}
\noindent This is the type of all lists with {\tt 3} elements, each of
which is a 1-dimensional region.

We will permit {\em static parameters}. These are static final fields
of classes marked as parameters. Some static parameters are {\em
configuration} parameters: they may be directly assigned a value from
the invocation environment for a program (e.g.{} via  a command line or
a config file).

In order to facilitate compile-time type-checking we shall require
that all expressions occurring in parametric types must reference only
constants or (instance or static) parameters visible at that type, and
involve methods that have been specially marked as parametric. This is
necessary because it must be possible for the compiler to reason about
the execution of these methods (e.g.{} by executing them
symbolically). Further details are discussed below.

\paragraph{Rest of this paper}
In the next section I flesh out the syntactic details of the current
proposal and consider a small example, {\tt List(n)}. After that I
present the code for {\tt region}, {\tt point}, {\tt distribution} and
{\tt arrays} in \Xten. Most of the methods are {\tt extern} methods
(defined natively); however their signature shows how considerable
information about the structure of the datatypes and the operations on
them can be expressed through the type system.

\section{Dependent Types}
A dependent type is a type that takes values as arguments; such a
value is called a {\em parameter}. Thus a dependent type is just like
a generic type except that the arguments to the type are values, not
types. Syntactically, these values are enclosed in parentheses, rather
than in angle brackets. For instance:

{\footnotesize
\begin{verbatim}
  List(10)<Cell> // The type of all lists of Cells of length 10.
\end{verbatim}}

The primary mechanism for the introduction of dependent types in
\Xten{} is the {\em dependent class declaration}.  

%% Figure out dependent interface definitions as well.

\subsection{Dependent class definitions}

Dependent classes are specified by providing a parameter list (similar
in structure to a method definition) right after the name of the
class. This list is said to define the {\em explicit parameters} of
the class.  Each parameter introduces a {\tt final} instance field of
the same name and type in the class; hence parameter names must be
distinct from field names. The list of parameters may be followed by a
``{\tt :}'' (read as ``where'') and a boolean expression involving the
explicit parameters (called the {\em where clause}). An instance of a
class can only be created for actual values for the explicit parameter
which can statically be known to satisfy the where clause. 

\begin{example}[List]\label{List}
Consider the class {\tt List}:
{\footnotesize
\begin{verbatim}
  public value class List( int n : n >= 0) <Node> {
    nullable Node node = null;
    nullable List(n-1)<Node> rest = null;

    /** Returns the empty list. Defined only when the parameter n 
        has the value 0. Invocation: new List(0)<Node>().
     */
    public List (: n=0) () {}

    /** Returns a list of length 1 containing the given node.
        Invocation: new List(1)<Node>( node ).
     */
    public List (: n=1) (Node node) {
      this.node = node;
      this.rest = new List(0)<Node>();
    }
    public List (: n >= 2) {Node node, List(n-1)<Node> rest) {
        this.node = node;
        this.rest = rest;
    }
    public (int m : m >= 0) List(n+m)<Node> append(List<m>Node l) {
      return 
          (n == 0) ? l : new List(n+m)<Node>(node, rest.append(m)(l));
    }
  }  
\end{verbatim}}
Intuitively, this definition states that a {\tt List} object can only
be created when supplied with an {\tt int} parameter {\tt n} (intended
to represent the length of the list). {\tt n} is required to be
non-negative. The class has two fields intended to hold the data item
in the head of the list and the rest of the list. The nullary
constructor is defined only if the value of the parameter is {\tt 0}.
The unary constructor is defined only if the value of the parameter is
{\tt 1}. Similarly for the binary constructor. 

The example also illustrates {\em dependent methods}. {\tt append} is
a dependent method that has an {\tt int} parameter {\tt m}; it is
required that {\tt m} be non-negative. {\tt append} returns a value of
type {\tt List(n+m)<Node>} and takes a {\tt List<m>Node} argument. The
parameter {\tt m} is passed to the recursive call to {\tt append}.

\end{example}

A dependent class may also specify additional parameters in the body
of the class. Such parameters are called {\em implicit} parameters.
Syntactically, an implicit parameter is specified using instance field
declaration syntax and prefixing the field declaration with the
keyword {\tt parameter}. The declaration of parameters must include an
initializer which may depend only on parameters defined in the
parameter list for the class or on previously defined parameters in
the class body.  

In what follows by the {\em parameters} of a (dependent) class we will
mean either the explicit parameters enumerated in the parameter list,
or the implicit parameters defined using a {\tt parameter} declaration
in the body of the class. Parameters may be used  in an
expression specifying the value of some other parameter (in a
constructor or method call).

\begin{example}
The class {\tt List} may be extended to define an implicit {\tt
boolean} parameter {\tt isEven} which records whether the length of
the list is even.
{\footnotesize
\begin{verbatim}
  public value class List(int n : n >= 0) <Node> {
    parameter boolean isEven = (n % 2 = 0);
    ...
  }  
\end{verbatim}}
\end{example}

The parameters of a class may be used in any type expression in the
body of the class at which they are visible. Parameters have the
status of final instance fields, hence they are not visible in static
fields, initializers or methods, or when they are overridden.

{}\paragraph{Creating instances of a dependent class.}
An instance of a dependent class is created using a {\tt new} operation
and passing an actual list of parameters (of the right type) after the 
class name and before the generic type parameters (if any).  For instance:
{\footnotesize
\begin{verbatim}
  new List(0)<Node>();
  new String(10) (``abcdefghij'');
\end{verbatim}}

{}\paragraph{Dependent constructors.} Sometimes a constructor for
class {\tt C} can only be invoked if the parameters of the class
satisfy a certain condition. For instance, a nullary constructor for
{\tt List} is only defined when the parameter {\tt n} has the value
{\tt 0}.

Such conditions may be specified in a constructor declaration between
the name of the class and the argument list of the constructor using a
where clause. The where clause can reference only the parameters of
the class.

%% Cannot throw an exception. 

Consider the definition of the nullary constructor for instance:
{\footnotesize
\begin{verbatim}
    /** Returns the empty list. Defined only when the parameter n 
        has the value 0. Invocation: new List(0)<Node>().
     */
    public List (: n=0) () {}
\end{verbatim}}

Any attempt to invoke this constructor with a parameter that cannot be
determined at compile-time to equal {\tt 0} will cause a compiler
error.

{}\paragraph{Dependent methods.}  
Instance method declarations may specify parameters after the
qualifiers and before the return type of the method, through a
parameter list (called the parameter list of the method).  The
parameter list may be followed by a where expression on the parameters
of the method and/or the parameters of the class. (Recall that a where
expression is a boolean expression following a ``{\tt :}''.)  

The parameter list of the method is visible in the declaration of the
method and the body of the method. Thus the types of arguments to the
method, and the return type of the method, may be built from type
expressions referencing these parameters.

For example, consider the definition of the {\tt append} method on lists:
{\footnotesize
\begin{verbatim}
    public (int m : m >= 0) List(n+m)<Node> append(List<m>Node l) {
      return 
          (n == 0) ? l : new List(n+m)<Node>(node, rest.append(m)(l));
    }
\end{verbatim}}

The method specifies an {\tt int} parameter, {\tt m}. This parameter
is referenced in the return type of the method {\tt List(n+m)<Node>}
and the argument type of the method {\tt List<m>Node}.

A method invocation must specify the parameters in an actual parameter
list after the name of the method and before the arguments to the
method, e.g:

{\footnotesize
\begin{verbatim}
  rest.append(m)(l)
\end{verbatim}}

For each such method invocation, the compiler must determine the
corresponding method definition and statically check that its where
clause (if any) is satisfied by the invocation.

{\em MetaNote Need to determine rules for overriding for generic
methods and compare with the rule above.}

\paragraph{Extending dependent classes.}

{\em MetaNote: This should be standard. A class definition may extend
a dependent super class, e.g. class Foo(int i) extends Fum(i*i) \{
\ldots \}. The expressions in the actual parameter list for the super
class may involve only the parameters of the class being defined. The
intuition is that these parameters are analogous to explicit arguments
that must be passed in every super-constructor invocation.}

\subsection{Types defined by a Dependent Class definition}
A dependent class definition introduces many types.

Let {\tt C} be a class declared with explicit parameters {\tt
x1,\ldots, xk} of type {\tt T1,\ldots, Tk}.  A parametric type is an
expression of the form {\tt C (: e)} where {\tt e} is a boolean
expression in the parameters of the class. We permit the syntactic
abbreviation {\tt C(t1,\ldots, tk)}, for values {\tt t1,\ldots, tk} of
type {\tt T1,\ldots, Tk} respectively for the type {\tt C(: x1=t1 \&\&
x2=t2\ldots \&\& xk=tk)}.  If {\tt C} is a generic class with type
parameters {\tt <T1,\ldots, Tm>}, then any type expression based on
{\tt C} must also specify the generic type parameters.

An object is an instance of a parametric type if its an instance of
the base type, and the values of its parameters satisfy the boolean
condition. 

{}\begin{example} The expression {\tt List(3)<Node>} is a type -- the
type of all lists of {\tt Node} of length {\tt 3}. 

The expression {\tt List (: isEven) <Node>} is a type -- the type of all
lists of {\tt Node} of even length.
\end{example}

\subsection{Dependent interfaces}

\section{Type-checking dependent classes}

We first consider an example.
\begin{example}[List, revisisted]
Consider the code in Example~\ref{List}.  In type-checking this code,
the compiler must perform the following inferences. It must verify
that {\tt null} is a permissible value for the type {\tt nullable
Node}. It must also verify that {\tt null} is a permissible value for
the type {\tt nullable List(n-1)<Node>} under the assumption that {\tt
n >= 0}. Both these inferences follow from the inference rule that
{\tt null} is a permissible value for the type {\tt nullable T} for
any {\tt T}.

To type-check the body of the second constructor, the compiler must
establish that the value {\tt 0} for {\tt n} satisfies the condition
{\tt n=0} (the condition guarding the nullary constructor).

To type-check the third constructor, the compiler must establish that
the type {\tt List(n-1)<Node>} is non-empty, given that {\tt n >=
2}.\footnote{The \Xten{} compiler must ensure that all types used as
types of arguments, fields, or local, loop or try/catch variable are
statically known to be non-empty. Without dependent types, all types
other than {\tt void} are non-empty. Dependent types introduce the
possibility that some types may denote empty sets, e.g. {\tt
List(-1)}.}  To establish this, it must show that {\tt n >=2} implies
{\tt n-1 >= 0}. The check for the body of the constructor is
straightforward.

To type-check the method, the compiler must infer that if if {\tt m}
and {\tt n} are non-negative, then so is {\tt m+n}. This establishes
that the return type of the method is not empty. It must also infer
that under the condition {\tt n==0}, {\tt m+n} is identical to {\tt
m}, and hence returning the value {\tt l} is type-correct. It must
also determine that the type of {\tt rest.append(m)(l)} is {\tt
List(m+n-1)<Node>}, and hence the invocation of the binary constructor
is correct.
\end{example}

The general idea is this. In general, the compiler must use
constraints (perform algebraic reasoning). Thus corresponding to each
program is an underlying logical theory in which reasoning will be
performed. 

Each programming language -- such as \Xten{} -- will specify the base
underlying classes (and the operations on them) which can occur as
types in parameter lists. For instance, in the code for {\tt List}
above, the only type that appears in parameter lists is {\tt int}, and
the only operations on {\tt int} are addition, subtraction, {\tt >=},
{\tt ==}, and the only constants are {\tt 0} and {\tt 1}.  (This
language falls within Presburger arithmetic, a decidable fragment of
arithmetic.)  The compiler must come equipped with a constraint solver
(decision procedure) that can answer questions of the form: does one
constraint entail another?  Constraints are atomic formulas built up
from these operations, using variables. For instance, the compiler
must answer each one of:
{\footnotesize
\begin{verbatim}
  n >= 2 |- n-1 >= 0
  n >= 0, m >= 0 |- m+n >= 0
\end{verbatim}}

Ultimately, the only variables that will occur in constraints are
those that correspond to {\tt config} parameters and those that are
defined by implicit parameter definitions. We need to establish that
the verification of any class will generate only a finite number of
constraints, hence only a finite constraint problem for the constraint
solver.

Second, it should be possible for instances of user-defined classes
(and operations on them) to occur as type parameters. For the compiler
to check conditions involving such values, it is necessary that the
underlying constraint solver be extended. 

There are two general ways in which the constraint solver may be
extended.  Both require that the programmer single out some classes
and methods on those classes as {\em pure}. (We shall think of
constants as corresponding to zero-ary methods.) Only instances of
pure classes and expressions involving pure methods on these instances
are allowed in parameter expressions.

How shall constraints be generated for such pure methods? First, the
programmer may explicitly supply with each pure method {\tt T m(T1 x1,
..., Tn xn)} a constraint on {\tt n+2} variables in the constraint
system of the underlying solver that is entailed by {\tt y =
o.m(x1,..., xn)}. Whenever the compiler has to perform reasoning on an
expression involving this method invocation, it uses the constraint
supplied by the programmer. A second more ambitious possibility is
that a symbolic evaluator of the language may be run on the body of
the method to automatically generate the corresponding constraint.

Finally an additional possibility is that the constraint solver itself
be made extensible. In this case, when a user writes a class which is
intended to be used in specifying parameters, he also supplies an
additional program which is used to extend the underlying constraint
solver used by the compiler. This program adds more primitive
constraints and knows how to perform reasoning using these
constraints. This is how I expect we will initially implement the
\Xten{} language. As language designers and implementers we will
provide constraint solvers for finite functions and {\tt Herbrand}
terms on top of arithmetic. 

\section{A more extended example}
Below I discuss the problem of specifying the operations on regions,
distributions and arrays in \Xten. I specify the signatures of these
classes, making extensive use of dependent types.

\subsection{$k$-dimensional regions}
{\footnotesize
\begin{verbatim}
package x10.lang;

/** A region represents a k-dimensional space of points. A region is a
 * dependent class, with the value parameter specifying the dimension
 * of the region.
 * @author vj 
 * @date 12/24/2004
 */

public final value class region( int dimension : dimension >= 0 )  {

    /** Construct a 1-dimensional region, if low =< high. Otherwise
     * through a MalformedRegionException.
     */
    extern public region (: dimension==1) (int low, int high) 
        throws MalformedRegionException;

    /** Construct a region, using the list of region(1)'s passed as
     * arguments to the constructor.
     */
    extern public region( List(dimension)<region(1)> regions );

    /** Throws IndexOutOfBoundException if i > dimension. Returns the
        region(1) associated with the i'th dimension of this otherwise.
     */
    extern public region(1) dimension( int i ) 
        throws IndexOutOfBoundException;


    /** Returns true iff the region contains every point between two
     * points in the region.
     */
    extern public boolean isConvex();

    /** Return the low bound for a 1-dimensional region.
     */
    extern public (:dimension=1) int low();

    /** Return the high bound for a 1-dimensional region.
     */
    extern public (:dimension=1) int high();

    /** Return the next element for a 1-dimensional region, if any.
     */
    extern public (:dimension=1) int next( int current ) 
        throws IndexOutOfBoundException; 

    extern public region(dimension) union( region(dimension) r);
    extern public region(dimension) intersection( region(dimension) r);
    extern public region(dimension) difference( region(dimension) r);
    extern public region(dimension) convexHull();

    /**
       Returns true iff this is a superset of r.
     */
    extern public boolean contains( region(dimension) r);
    /**
       Returns true iff this is disjoint from r.
     */
    extern public boolean disjoint( region(dimension) r);

    /** Returns true iff the set of points in r and this are equal.
     */
    public boolean equal( region(dimension) r) {
        return this.contains(r) && r.contains(this);
    }

    // Static methods follow.

    public static region(2) upperTriangular(int size) {
        return upperTriangular(2)( size );
    }
    public static region(2) lowerTriangular(int size) {
        return lowerTriangular(2)( size );
    }
    public static region(2) banded(int size, int width) {
        return banded(2)( size );
    }

    /** Return an \code{upperTriangular} region for a dim-dimensional
     * space of size \code{size} in each dimension.
     */
    extern public static (int dim) region(dim) upperTriangular(int size);

    /** Return a lowerTriangular region for a dim-dimensional space of
     * size \code{size} in each dimension.
     */
    extern public static (int dim) region(dim) lowerTriangular(int size);

    /** Return a banded region of width {\code width} for a
     * dim-dimensional space of size {\code size} in each dimension.
     */
    extern public static (int dim) region(dim) banded(int size, int width);


}

\end{verbatim}}

\subsection{Point}
{\footnotesize
\begin{verbatim}
package x10.lang;

public final class point( region region ) {
    parameter int dimension = region.dimension;
    // an array of the given size.
    int[dimension] val;

    /** Create a point with the given values in each dimension.
     */
    public point( int[dimension] val ) {
        this.val = val;
    }

    /** Return the value of this point on the i'th dimension.
     */    
    public int valAt( int i) throws IndexOutOfBoundException {
        if (i < 1 || i > dimension) throw new IndexOutOfBoundException();
        return val[i];
    }

    /** Return the next point in the given region on this given
     * dimension, if any.
     */
    public void inc( int i ) 
        throws IndexOutOfBoundException, MalformedRegionException {
        int val = valAt(i);
        val[i] = region.dimension(i).next( val );
    }

    /** Return true iff the point is on the upper boundary of the i'th
     * dimension.
     */
    public boolean onUpperBoundary(int i) 
        throws IndexOutOfBoundException {
        int val = valAt(i);
        return val == region.dimension(i).high();
    }

    /** Return true iff the point is on the lower boundary of the i'th
     * dimension.
     */
    public boolean onLowerBoundary(int i) 
        throws IndexOutOfBoundException {
        int val = valAt(i);
        return val == region.dimension(i).low();
    }
}
\end{verbatim}}

\subsection{Distribution}
{\footnotesize
\begin{verbatim}
package x10.lang;

/** A distribution is a mapping from a given region to a set of
 * places. It takes as parameter the region over which the mapping is
 * defined. The dimensionality of the distribution is the same as the
 * dimensionality of the underlying region.

   @author vj
   @date 12/24/2004
 */

public final value class distribution( region region ) {
    /** The parameter dimension may be used in constructing types derived
     * from the class distribution. For instance,
     * distribution(dimension=k) is the type of all k-dimensional
     * distributions.
     */
    parameter int dimension = region.dimension;

    /** places is the range of the distribution. Guranteed that if a
     * place P is in this set then for some point p in region,
     * this.valueAt(p)==P.
     */
    public final Set<place> places; // consider making this a parameter?

    /** Returns the place to which the point p in region is mapped.
     */
    extern public place valueAt(point(region) p);

    /** Returns the region mapped by this distribution to the place P.
        The value returned is a subset of this.region.
     */
    extern public region(dimension) restriction( place P );

    /** Returns the distribution obtained by range-restricting this to Ps.
        The region of the distribution returned is contained in this.region.
     */
    extern public distribution(:this.region.contains(region))
        restriction( Set<place> Ps );

    /** Returns a new distribution obtained by restricting this to the
     * domain region.intersection(R), where parameter R is a region
     * with the same dimension.
     */
    extern public (region(dimension) R) distribution(region.intersection(R)) 
        restriction();

    /** Returns the restriction of this to the domain region.difference(R),
        where parameter R is a region with the same dimension.
     */
    extern public (region(dimension) R) distribution(region.difference(R)) 
        difference();

    /** Takes as parameter a distribution D defined over a region
        disjoint from this. Returns a distribution defined over a
        region which is the union of this.region and D.region.
        This distribution must assume the value of D over D.region
        and this over this.region.

        @seealso distribution.asymmetricUnion.
     */
    extern public (distribution(:region.disjoint(this.region) &&
                                dimension=this.dimension) D) 
        distribution(region.union(D.region)) union();

    /** Returns a distribution defined on region.union(R): it takes on 
        this.valueAt(p) for all points p in region, and D.valueAt(p) for all
        points in R.difference(region).
     */
    extern public (region(dimension) R) distribution(region.union(R)) 
        asymmetricUnion( distribution(R) D);

    /** Return a distribution on region.setMinus(R) which takes on the
     * same value at each point in its domain as this. R is passed as
     * a parameter; this allows the type of the return value to be
     * parametric in R.
     */
    extern public (region(dimension) R) distribution(region.setMinus(R)) 
        setMinus();

    /** Return true iff the given distribution D, which must be over a
     * region of the same dimension as this, is defined over a subset
     * of this.region and agrees with it at each point.
     */
    extern public (region(dimension) r) 
        boolean subDistribution( distribution(r) D); 

    /** Returns true iff this and d map each point in their common
     * domain to the same place.
     */
    public boolean equal( distribution( region ) d ) {
        return this.subDistribution(region)(d) 
            && d.subDistribution(region)(this);
    }

    /** Returns the unique 1-dimensional distribution U over the region 1..k,
     * (where k is the cardinality of Q) which maps the point [i] to the 
     * i'th element in Q in canonical place-order.
     */
    extern public static distribution(:dimension=1) unique( Set<place> Q );

    /** Returns the constant distribution which maps every point in its
        region to the given place P.
    */
    extern public static (region R) distribution(R) constant( place P );

    /** Returns the block distribution over the given region, and over
     * place.MAX_PLACES places.
     */
    public static (region R) distribution(R) block() {
        return this.block(R)(place.places);
    }

    /** Returns the block distribution over the given region and the
     * given set of places. Chunks of the region are distributed over
     * s, in canonical order.
     */
    extern public static (region R) distribution(R) block( Set<place> s);


    /** Returns the cyclic distribution over the given region, and over
     * all places.
     */
    public static (region R) distribution(R) cyclic() {
        return this.cyclic(R)(place.places);
    }

    extern public static (region R) distribution(R) cyclic( Set<place> s);

    /** Returns the block-cyclic distribution over the given region, and over
     * place.MAX_PLACES places. Exception thrown if blockSize < 1.
     */
    extern public static (region R) 
        distribution(R) blockCyclic( int blockSize) 
        throws MalformedRegionException;

    /** Returns a distribution which assigns a random place in the
     * given set of places to each point in the region.
     */
    extern public static (region R) distribution(R) random();

    /** Returns a distribution which assigns some arbitrary place in
     * the given set of places to each point in the region. There are
     * no guarantees on this assignment, e.g. all points may be
     * assigned to the same place.
     */
    extern public static (region R) distribution(R) arbitrary();

}
\end{verbatim}}
\subsection{Arrays}
Finally we can now define arrays. An array is built over a
distribution and a base type.

{\footnotesize
\begin{verbatim}
package x10.lang;

/** The class of all  multidimensional, distributed arrays in X10.

    <p> I dont yet know how to handle B@current base type for the
    array.

 * @author vj 12/24/2004
 */

public final value class array ( distribution dist )<B@P> {
    parameter int dimension = dist.dimension;
    parameter region(dimension) region = dist.region;

    /** Return an array initialized with the given function which
        maps each point in region to a value in B. 
     */
    extern public array( Fun<point(region),B@P> init);

    /** Return the value of the array at the given point in the
     * region.
     */
    extern public B@P valueAt(point(region) p);

    /** Return the value obtained by reducing the given array with the
        function fun, which is assumed to be associative and
        commutative. unit should satisfy fun(unit,x)=x=fun(x,unit).
     */
    extern public B reduce(Fun<B@?,Fun<B@?,B@?>> fun, B@? unit);


    /** Return an array of B with the same distribution as this, by 
        scanning this with the function fun, and unit unit.
     */
    extern public array(dist)<B> scan(Fun<B@?,Fun<B@?,B@?>> fun, B@? unit);

    /** Return an array of B@P defined on the intersection of the
        region underlying the array and the parameter region R.
     */
    extern public (region(dimension) R) 
        array(dist.restriction(R)())<B@P>  restriction();

    /** Return an array of B@P defined on the intersection of 
        the region underlying this and the parametric distribution.
     */    
    public  (distribution(:dimension=this.dimension) D) 
        array(dist.restriction(D.region)())<B@P> restriction();

    /** Take as parameter a distribution D of the same dimension as *
     * this, and defined over a disjoint region. Take as argument an *
     * array other over D. Return an array whose distribution is the
     * union of this and D and which takes on the value
     * this.atValue(p) for p in this.region and other.atValue(p) for p
     * in other.region.
     */
    extern public (distribution(:region.disjoint(this.region) &&
                                dimension=this.dimension) D)
        array(dist.union(D))<B@P> compose( array(D)<B@P> other);

    /** Return the array obtained by overlaying this array on top of
        other. The method takes as parameter a distribution D over the
        same dimension. It returns an array over the distribution
        dist.asymmetricUnion(D).
     */
    extern public (distribution(:dimension=this.dimension) D)  
        array(dist.asymmetricUnion(D))<B@P> overlay( array(D)<B@P> other);

    extern public array<B> overlay(array<B> other);

    /** Assume given an array a over distribution dist, but with
     * basetype C@P. Assume given a function f: B@P -> C@P -> D@P.
     * Return an array with distribution dist over the type D@P
     * containing fun(this.atValue(p),a.atValue(p)) for each p in
     * dist.region.
     */
    extern public <C@P, D> 
        array(dist)<D@P> lift(Fun<B@P, Fun<C@P, D@P>> fun, array(dist)<C@P> a);

    /**  Return an array of B with distribution d initialized
         with the value b at every point in d.
     */
    extern public static (distribution D) <B@P> array(D)<B@P> constant(B@? b);

}
\end{verbatim}}
\end{document}