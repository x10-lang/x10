% First page

\thispagestyle{empty}

% \todo{"another" report?}

\title{Report on the Experimental Language \Xten \\
\large Version \integerversion}
\author{Please send comments to \\
Vijay Saraswat at \texttt{vsaraswa@us.ibm.com}}
\date\today
\maketitle

\if 0
\topnewpage[{
\begin{center}   
{\huge\bf Report on the Experimental Language \Xten{}}
\vskip 1ex
$$
\begin{tabular}{l@{\extracolsep{.5in}}lll}
\multicolumn{4}{c}{\sc Version \integerversion}\\
\multicolumn{4}{c}{\sc Please send comments to 
Vijay Saraswat at 
{\tt vsaraswa@us.ibm.com}}\\
%\multicolumn{4}{c}{({\sc IBM Confidential})}

%\ldots
\end{tabular}
$$
\vskip 2ex
% {\it Dedicated to the Memory of APL} % vj
{\bf \today}
\vskip 2.6ex
\end{center}


}]
\fi

\newcommand\authorsc[1]{#1}
%\newcommand\authorsc[1]{\textsc{#1}}


\chapter*{Summary}
This draft report provides an initial description of the programming
language \Xten. \Xten{} is a single-inheritance class-based object-oriented
(OO) programming language designed for high-performance, high-productivity
computing on high-end computers supporting $\approx 10^5$ hardware threads
and $\approx 10^{15}$ operations per second. 

{}\Xten{} is based on state-of-the-art object-oriented programming
languages and deviates from them only as necessary to support its
design goals. The language is intended to have a simple and clear
semantics and be readily accessible to mainstream OO programmers. It
is intended to support a wide variety of concurrent programming
idioms.
%, incuding data parallelism, task parallelism, pipelining.
%producer/consumer and divide and conquer.

%We expect to revise this document in the light of experience gained in implementing
%and using this language.

The \Xten{} design team consists of
\authorsc{David Bacon}, 
\authorsc{Raj Barik}, 
\authorsc{Ganesh Bikshandi}, 
\authorsc{Bob Blainey}, 
\authorsc{Philippe Charles}, 
\authorsc{Perry Cheng}, 
\authorsc{Christopher Donawa}, 
\authorsc{Julian Dolby}, 
\authorsc{Kemal Ebcio\u{g}lu},
\authorsc{Robert Fuhrer},
\authorsc{Patrick Gallop}, 
\authorsc{Christian Grothoff}, 
\authorsc{Allan Kielstra}, 
\authorsc{Sreedhar Kodali}, 
\authorsc{Sriram Krishnamoorthy}, 
\authorsc{Nathaniel Nystrom}, 
\authorsc{Igor Peshansky}, 
\authorsc{Vijay Saraswat} (contact author), 
\authorsc{Vivek Sarkar},
\authorsc{Armando Solar-Lezama},  
\authorsc{S. Alexander Spoon}, 
\authorsc{Sayantan Sur}, 
\authorsc{Christoph von Praun},
\authorsc{Pradeep Varma},
\authorsc{Krishna Venkata},
\authorsc{Jan Vitek}, and
\authorsc{Tong Wen}.

For extended discussions and support we would like to thank: 
Robert Callahan, Calin
Cascaval, Norman Cohen, Elmootaz Elnozahy, John Field, Bob Fuhrer,
Orren Krieger, Doug Lea, John McCalpin, Paul McKenney, Ram Rajamony,
R.K.~Shyamasundar, Filip Pizlo, V.T.~Rajan, Frank Tip, and Mandana Vaziri.

We thank Jonathan Rhees and William Clinger with help in obtaining the
\LaTeX{} style file and macros used in producing the Scheme report,
on which this document is based. We acknowledge the influence of
the $\mbox{\Java}^{\mbox{\authorsc{\small tm}}}$ Language
Specification \cite{jls2}.
%document, as evidenced by the numerous citations in the text.

This document revises Version 1.1 of the Report, released in
June 2007. It documents the language corresponding to the second
revision of the first version of the implementation.  This
revision was done by
\authorsc{Raj Barik}, 
\authorsc{Philippe Charles}, 
\authorsc{Christopher Donawa}, 
\authorsc{Robert Fuhrer},
\authorsc{Nathaniel Nystrom},  
\authorsc{Vijay Saraswat},
\authorsc{Vivek Sarkar},
\authorsc{Pradeep Varma}, and
\authorsc{Krishna Venkata}.
(Earlier implementations benefited from significant contributions by
\authorsc{Christian Grothoff} and 
\authorsc{Christoph von Praun}.)
\authorsc{Tong Wen} has written many application programs
in \Xten{}. \authorsc{Guojing Cong} has helped in the
development of many applications.


%\vfill
%\begin{center}
%{\large \bf
%*** DRAFT*** \\
%%August 31, 1989
%\today
%}\end{center}

\vfill
\eject


\chapter*{Contents}
\addvspace{3.5pt}                  % don't shrink this gap
\renewcommand{\tocshrink}{-3.5pt}  % value determined experimentally
{\footnotesize
\tableofcontents
}

\vfill
\eject


