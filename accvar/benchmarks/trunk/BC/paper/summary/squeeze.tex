\makeatletter

\def\@listi{\leftmargin\leftmargini
            \parsep 1.0\p@ \@plus\p@ \@minus\p@
            \topsep 1.25\p@ \@plus2\p@ \@minus2\p@
            \itemsep1.0\p@ \@plus\p@ \@minus\p@}
\let\@listI\@listi
\@listi
\def\@listii {\leftmargin\leftmarginii
              \labelwidth\leftmarginii
              \advance\labelwidth-\labelsep
              \topsep    2.0\p@ \@plus2\p@ \@minus\p@
              \parsep    1.5\p@   \@plus\p@  \@minus\p@
              \itemsep   \parsep}
\def\@listiii{\leftmargin\leftmarginiii
              \labelwidth\leftmarginiii
              \advance\labelwidth-\labelsep
              \topsep    1\p@ \@plus\p@\@minus\p@
              \parsep    \z@
              \partopsep \p@ \@plus\z@ \@minus\p@
              \itemsep   \topsep}

\setlength\abovecaptionskip{2\p@}
\setlength\belowcaptionskip{-2\p@}
\setlength\parskip{0em}
\setlength\parindent{1em}
\setlength\dbltextfloatsep{2pt}
\setlength\dblfloatsep{-5pt}

% Footnote handling
\setlength{\footnotesep}{-8pt}
\setlength{\skip\footins}{10\p@ \@plus 4\p@ \@minus 2\p@}% less 7 pt for rule
%
% footnoterule: let \footins specify the distance between the text
% and the rule (\footins should be at least 7pts), and space a bit 
% before the first footnote so that \footnotesep can be smaller
%
\renewcommand\footnoterule{%
  \kern-7\p@
    \hrule width .4\columnwidth% \hrule is by default .4pt high
      \kern 6.6\p@}
      %

\makeatother
