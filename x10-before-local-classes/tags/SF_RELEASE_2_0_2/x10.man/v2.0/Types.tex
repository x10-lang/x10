\chapter{Types}
\label{XtenTypes}\index{types}

{}\Xten{} is a {\em strongly typed} object-oriented language: every
variable and expression has a type that is known at compile-time.
Types limit the values that variables can hold and specify the places
at which these values lie.


{}\Xten{} supports three kinds of runtime entities, {\em objects},
{\em structs}, and {\em functions}. Objects are instances of {\em
  classes} (\Sref{ReferenceClasses}). They may contain mutable fields
and must stay resident in the place in which they were
created. Objects are said to be {\em boxed} in that variables of a
class type are implemented through a single memory location that
contains a reference to the memory containing the declared state of
the object (and other meta-information such as the list of methods of
the object). Thus objects are represented through an extra level of
indirection. A consequence of this flexibility is that every class
type contains the value \Xcd{null} corresponding to the invalid
reference. \Xcd{null} is often useful as a default value. Further, two
objects may be compared for equality (\Xcd{==}) in constant time by
simply containing references to the memory used to represent the
objects.

Structs are instances of {\em struct classes}
(\Sref{StructClasses}). They are immutable and may be freely copied
from place to place.  Further, they may be allocated inline, i.e.{}
using only as much memory as necessary to hold the fields of the
struct (and any additional memory necessary to satisfy alignment
constraints for data layout. 

Functions are instances of {\em function types|} {\Sref{Functions})
  and are created using function literals
  \Xcd{(x1:T1,...,xn:Tn)\{c\}:T => e}. Functions contain no
  user-visible mutable or immutable state; their representation,
  however, contains enough memory to hold the values of the variables
  in the environment that are referenced in the body of the function
  (\Xcd{e}). Functions may be freely copied from place to place and
  may be repeatedly applied to a set of arguments, provided that the
  precondition \Xcd{c} is satisfied.

These runtime entities are classified into different groups using {\em
  types}.  Types are used in variable declarations, explicit coercions
and conversions, object creation, array creation, class literals,
static state and method accessors, and \xcd"instanceof"
expressions.\footnote{In order to allow this version of the language
  to focus on the core new ideas, \XtenCurrVer{} does not have
  user-definable class loaders, though there is no technical reason
  why they could not have been added.}

\Xten{} has a unified type system. The top of the type hierarchy
is the interface \Xcd{x10.lang.Any}. This specifies the following signature:
\begin{xten}
package x10.lang;
public interface Any {
  property def home():Place;
  property def at(p:Object):Boolean;
  property def at(p:Place):Boolean;
  global safe def toString():String;
  global safe def typeName():String;
  global safe def equals(Any):Boolean;
  global safe def hashCode():Int;
}
\end{xten}

Properties are described in (\Sref{properties}): in brief they are
immutable instance fields of objects and structs that can be used to
construct types through constraints. Property methods (\Sref{properties}) permit
more complex expressions to be defined using properties and other property methods.
A method is \Xcd{safe} if it has certain behavioral characteristics \Sref{SafeAnnotation}.
It is \Xcd{global} if it can be invoked at any place.

Types in \Xten{} are specified through declarations and through
type constructors, described in the remainder of the chapter:

\begin{itemize}
\item A class declaration defines a \emph{class type}
(\Sref{ReferenceTypes}).
\item An interface declaration defines an \emph{interface type}
(\Sref{InterfaceTypes}).
\item Classes and interface have \emph{type parameters}.
A class or interface with one or more type parameters is a
\emph{generic class} or \emph{generic interface}
(\Sref{Generics}).
\item New type constructors may be defined with \emph{type
definitions} (\Sref{TypeDefs}).
\item Methods, constructors, closures, and type definitions may
have \emph{type parameters}, which are instantiated with
concrete types at invocation (\Sref{TypeParameters}).
\item \emph{Function type} constructors are used to define
function types; functions and method selectors have function
type (\Sref{FunctionTypes}).
\item A \emph{constrained type} constrains the properties of
a base type (\Sref{ConstrainedTypes}).
\item Types may be marked with user-defined annotations.
\emph{Annotated types} (\Sref{AnnotatedTypes})
may be processed by compiler plugins.
\end{itemize}

%%TODOvj: Revisit grammer for types.
\begin{grammar}
Type \: FunctionType \\
    \| ConstrainedType  \\

FunctionType \: TypeParameters\opt \xcd"(" Formals\opt \xcd")"
Constraint\opt Throws\opt \xcd"=>" Type \\
TypeParameters \: \xcd"[" TypeParameter ( \xcd"," TypeParameter )\star \xcd"]" \\
TypeParameter \: Identifier \\
Throws \: \xcd"throws" TypeName ( \xcd"," TypeName )\star \\

ConstrainedType \: Annotation\star BaseType Constraint\opt
PlaceConstraint\opt \\

BaseType \: ClassBaseType \\
     \| InterfaceBaseType \\
     \| PathType \\
     \| \xcd"(" Type \xcd")" \\

ClassType \: Annotation\star ClassBaseType Constraint\opt
PlaceConstraint\opt \\
InterfaceType \: Annotation\star InterfaceBaseType Constraint\opt
PlaceConstraint\opt \\

PathType \: Expression \xcd"." Identifier \\

Annotation \: \xcd"@" InterfaceBaseType Constraint\opt \\

ClassOrInterfaceType \: ClassType \\ \| InterfaceType \\
ClassBaseType \: TypeName \\
InterfaceBaseType \: TypeName \\
\end{grammar}

% \section{Type definitions and type constructors}
% 
% Types in \Xten{} are specified through declarations and through
% type constructors:

% \paragraph{Class types.}

\section{Classes and interfaces}
\label{ReferenceTypes}

\subsection{Class types}

\index{types!class types}
\index{class}
\index{class declaration}
\index{declaration!class declaration}
\index{declaration!reference class declaration}

        A {\em class declaration} (\Sref{XtenClasses})
        introduces a 
        {\em class type}
        containing all instances of the class.

        Class instances are created via
        constructor calls.  Class instances have fields and
        methods, type members, and
        value properties bound at construction time.
        In addition, classes have static
        members: constant fields, type definitions, and member
        classes and member interfaces.


        A class with type parameters is {\em generic}.  A class
        type is legal only if all of its parameters are instantiated
        on concrete types.

\Xten{} does not permit mutable static state, so the role of static methods and
initializers is quite limited. Instead programmers should use
singleton classes to carry mutable static state.

\index{\Xcd{Object}}
\index{\Xcd{x10.lang.Object}}

        Classes
        are structured in a
        single-inheritance hierarchy.  All classes
        extend the class \xcd"x10.lang.Object".
        Classes are declared to extend a single
        superclass (except for \xcd"Object", which extends
        no other class).

\index{class!reference class}
\index{reference class type}
\index{\Xcd{Object}}
\index{\Xcd{x10.lang.Object}}

Variables of class type may contain the value \xcd"null". 

\subsection{Interface types}
\label{InterfaceTypes}

\index{types!interface types}
\index{interface}
\index{interface declaration}
\index{declaration!interface declaration}

        An {\em interface declaration} (\Sref{XtenInterfaces})
        defines an {\em interface type}, which specifies a set of
        methods, type members, and properties to be implemented by any class
        declared to implement the interface.  Interfaces can also
        have static members: constant fields, type definitions,
        and member classes and interfaces.

        An interface may extend multiple interfaces.

Classes may be declared to implement multiple interfaces.
Semantically, the interface type is the set of all objects that are
instances of classes that implement the interface. A class implements
an interface if it is declared to and if it implements all the methods
and properties defined in the interface.

\subsection{Properties}
\index{properties}
\label{properties}
        Classes and interfaces may have {\em properties},
        public final instance fields bound on object creation.  
%
        For example, the following code declares a class named
        \xcd"Point" with properties \xcd"x" and \xcd"y"
        and a \xcd"move" method.  The properties are bound
        using the \xcd"property" statement in the constructor.

\begin{xten}
class Point(x: Int, y: Int) {
  def this(x: Int, y: Int) { property(x, y); }
  def move(dx: Int, dy: Int) = new Point(x+dx, y+dy);
}
\end{xten}

        The properties of a class or interface may be
        constrained with a boolean expression.  The type
        \xcd"Point{x==0}" is the set of all points whose \xcd"x"
        property is \xcd"0".

\section{Type parameters}
\label{TypeParameters}

\index{types!type parameters}
\index{methods!parametrized methods}
\index{constructors!parametrized constructors}
\index{closures!parametrized closures}
\index{parametrized methods}
\index{parametrized constructors}
\index{parametrized closures}

        A class, interface, method, or closure may have type parameters
        whose scope is the signature and body of the declaring
        class, interface, method,
        or closure.

        Similarly, a type definition may have
        type parameters that scope over the body of the type
        definition.

        Type parameters may be constrained by a {\em guard} on the declaration
        (\Sref{ClassGuard}, \Sref{TypeDefGuard},
        \Sref{MethodGuard},\Sref{ClosureGuard}).
        The type parameters of classes and interfaces must be
        bound to concrete types (possibly to a type parameter)
        for the type to be legal; thus \xcd"List[int]" and
        \xcd"List[C]" are legal types, but \xcd"List" alone is
        not.
        The type parameters of
        methods and closures
        must be bound to
        concrete types at invocation.
        Parametrized type definitions specify new type
        constructors; the type parameters of a type definition must be bound
        to yield a type.

\subsection{Generic types}
\label{Generics}
\index{types!generic types}
        A {\em generic class} is a class
        declared with one or more type parameters.  Generic
        classes can be instantiated by instantiating the type
        parameters of the base type.

Consider the following declaration of a \xcd"Cell" class.
\begin{xten}
class Cell[X] {
    var x: X;
    def this(x: X) { this.x = x; }
    def get(): X = x;
    def set(x: X) = { this.x = x; }
}
\end{xten}

This declares a class \xcd"Cell" with a type parameter \xcd"X".
\xcd"Cell" may be used as a type by instantiating \xcd"X".

\xcd"Cell[Int]" is the type of all \xcd"Cell" containing an
\xcd"Int".  The \xcd"get" method returns an \xcd"Int"; the
\xcd"set" method takes an \xcd"Int" as argument.  Note that
\xcd"Cell" alone is not a legal type because the parameter is
not bound.

Parameters may be declares as {\em invariant}, {\em covariant}, or
{\em contravariant}.  The \xcd"X" parameter of \xcd"Cell" above is
invariant.  
Consider the following classes:
\begin{xten}
class Get[+X] {
    var x: X;
    def this(x: X) { this.x = x; }
    def get(): X = x;
}

class Set[-X] {
    var x: X;
    def this(x: X) { this.x = x; }
    def set(x: X) = { this.x = x: }
}
\end{xten}
The \xcd"X" parameter of the \xcd"Get" class is covariant;
the \xcd"X" parameter of the \xcd"Set" class is contravariant.

A covariant type parameter is permitted to appear only in covariant type positions,
and a contravariant type parameter in contravariant positions. 
\begin{itemize}
\item The return type of a method is a covariant position.
\item The argument types of a method are contravariant positions.
\item Whether a type argument position of a generic class, interface or struct type \Xcd{C}
is covariant or contravariant is determined by the \Xcd{+} or \Xcd{-} annotation
at that position in the declaration of \Xcd{C}.

\end{itemize}


Given types \xcd"S" and \xcd"T".
\begin{itemize}
\item
If the parameter of \xcd"Get" is covariant, then
\xcd"Get[S]" is a subtype of \xcd"Get[T]" if
\xcd"S" is a {\em subtype} of \xcd"T".

\item
If the parameter of \xcd"Set" is contravariant, then
\xcd"Set[S]" is a subtype of \xcd"Set[T]" if
\xcd"S" is a {\em supertype} of \xcd"T".

\item
If the parameter of \xcd"Cell" is invariant, then
\xcd"Cell[S]" is a subtype of \xcd"Cell[T]" if
\xcd"S" is a {\em equal} to \xcd"T".
\end{itemize}

\section{Type definitions}
\label{TypeDefs}

\index{types!type definitions}
\index{declarations!type definitions}
With value arguments, type arguments, and constraints, the syntax for \Xten{}
types can often be verbose. 
For example, a non-null list of non-null strings is \\
%~~type~~`~~`~~ ~~import x10.util.*;
\xcd`List[String{self!=null}]{self!=null}`.
\Xten{} provides {\em type definitions}
to allow users to give short names to long types, and to commonly-used
combinations of types. 
We could name that type: 
%~~gen
% package TypeDefs.glip.first;
% import x10.util.*;
% class LnSn {
% static
%~~vis
\begin{xten}
type LnSn = List[String{self!=null}]{self!=null};
\end{xten}
%~~siv
%}
%~~neg
Or, we could abstract it somewhat, defining a type constructor
\xcd`Nonnull[T]` for the type of \xcd`T`'s which are not null:
%~~gen
% package TypeDefs.glip.second;
% import x10.util.*;
% class LnSn {
% def example() {
%~~vis
\begin{xten}
type Nonnull[T] = T{self!=null};
type LnSn = Nonnull[List[Nonnull[String]]];
var example : LnSn;
\end{xten}
%~~siv
%}}
%~~neg

Type definitions can also refer to values, in particular, inside of
constraints.  The type of \xcd`n`-element \xcd`Rails[Int]s` is 
%~~type~~`~~`~~n:Int ~~
\xcd`Rail[Int]{self.length == n}`
but it is often convenient to give a shorter name: 
%~~gen
% package TypeDefs.glip.third;
% class Xmpl {
% def example() {
%~~vis
\begin{xten}
type Rail(n:Int) = Rail[Int]{self.length == n};
var example : Rail(78); 
\end{xten}
%~~siv
%}}
%~~neg

Type definitions, like many other X10 abstractions, can have constraints on
their use. 
\bard{Have we seen this -- in Overview?}
\begin{xten}

\end{xten}


Type definitions have the following syntax:

\begin{grammar}
TypeDefinition \: 
                \xcd"type"~Identifier
                           ( \xcd"[" TypeParameters \xcd"]" )\opt \\
                        && ( \xcd"(" Formals \xcd")" )\opt
                            Constraint\opt \xcd"=" Type \\
\end{grammar}

\noindent
A type definition can be thought of as a type-valued function,
mapping type parameters and value parameters to a concrete type.
%
The following examples are legal type definitions, given \xcd`import x10.util.*`:
%~~gen
% import x10.util.*;
% class TypeDefBrow {
% def someTypeDefs () {
%~~vis
\begin{xten}
type StringSet = Set[String];
type MapToList[K,V] = Map[K,List[V]];
type Int(x: Int) = Int{self==x};
type Dist(r: Int) = Dist{self.rank==r};
type Dist(r: Region) = Dist{self.region==r};
type Redund(n:Int, r:Region){r.rank==n} = Dist{rank==n && region==r};
\end{xten}
%~~siv
% }}
%~~neg
\label{TypeDefGuard}
As the two definitions of \xcd"Dist" demonstrate, type definitions may 
be overloaded: two type definitions with different numbers of type
parameters or with different types of value
parameters, according to the method overloading rules
(\Sref{MethodOverload}), define distinct type constructors.

Type definitions may appear as (static) class or interface member or
in a block statement.

Type definitions are applicative, not generative; that is, they
define aliases for types but do not introduce new types.
Thus, the following code is legal:
%~~gen
% import x10.util.*;
% class TypeDefNonGenerative {
% def someTypeDefs () {
%~~vis
\begin{xten}
type A = Int;
type B = String;
type C = String;
a: A = 3;
b: B = new C("Hi");
c: C = b + ", Mom!";
\end{xten}
%~~siv
% }}
%~~neg
% An instance of a defined type with no type parameters and no
% value parameters may 
If a type definition has no type parameters and no value
parameters and is an alias for a class type, a \xcd"new"
expression may be used to create an instance of the class using
the type definition's name.
Given the following type definition:
\begin{xtenmath}
type A = C[T$_1$, $\dots$, T$_k$]{c};
\end{xtenmath}
where 
\xcdmath"C[T$_1$, $\dots$, T$_k$]" is a
class type, a constructor of \xcdmath"C" may be invoked with
\xcdmath"new A(e$_1$, $\dots$, e$_n$)", if the
invocation
\xcdmath"new C[T$_1$, $\dots$, T$_k$](e$_1$, $\dots$, e$_n$)" is
legal and if the constructor return type is a subtype of
\xcd"A".

The collection of type definitions in
\xcdmath"x10.lang.\_" is automatically imported in every compilation unit.
\index{import,type definitions}\label{X10LangUnderscore}

\if 0
All type definitions are members of their enclosing package or
class.  A compilation unit may have one or more type definitions
or class or interface declarations with the same name, as long
as the definitions have distinct parameters according to the
method overloading rules (\Sref{MethodOverload}).
\fi




\section{Constrained types}
\label{ConstrainedTypes}
\label{DepType:DepType}
\label{DepTypes}

\index{dependent types}
\index{constrained types}
\index{generic types}
\index{types!constrained types}
\index{types!dependent types}
\index{types!generic types}

        Given a type \xcd"T", a {\em constrained type}
        \xcd"T{e}" may be
        constructed by constraining its
        properties with a
        boolean expression \xcd"e".

        \xcd"T{e}" is a {\em dependent type}, that is, a type
        dependent on values.
        The type \xcd"T" is called the
        {\em base type} and \xcd"e" is called the {\em constraint}.
        For reference types, the constraint may
        specify the places at which the object resides.

For brevity, the constraint may be omitted and
interpreted as \xcd"true".



Constraints on properties induce a natural subtyping relationship:
\xcd"C{c}" is a subtype of
\xcd"D{d}" if \xcd"C" is a subclass of \xcd"D" and
\xcd"c" entails \xcd"d".

Type parameters cannot be constrained.

\subsection{Constraints}

\def\withmath#1{\relax\ifmmode#1\else{$#1$}\fi}
\def\LL#1{\withmath{\lbrack\!\lbrack #1\rbrack\!\rbrack}}

Expressions used as constraints are restricted by the constraint
system in use to ensure that the constraints can be solved at compile
time.  The \Xten{} compiler allows compiler plugins to be installed to
extend the constraint language and the constraint system.  Constraints
must be of type \xcd"Boolean".  The compiler supports the following
constraint syntax.


\begin{grammar}
Constraint \: ValueArguments     Guard\opt \\
           \| ValueArguments\opt Guard     \\
           \\
ValueArguments   \:  \xcd"(" ArgumentList\opt \xcd")" \\
ArgumentList     \:  Expression ( \xcd"," Expression )\star \\
Guard            \: \xcd"{" DepExpression \xcd"}" \\
DepExpression    \: ( Formal \xcd";" )\star ArgumentList \\
\end{grammar}

In \XtenCurrVer{} value constraints may be equalities (\xcd"=="),
disequalities (\xcd"!=") and conjunctions thereof.  The terms over
which these constraints are specified include literals and
(accessible, immutable) variables and fields, property methods, and the special
constants {\tt here}, {\tt self}, and {\tt this}. Additionally, place
types are permitted (\Sref{PlaceTypes}).

Type constraints may be subtyping and supertyping (\xcd"<:" and
\xcd":>") expressions over types.

% //, and existential quantification over typed variables.

\emph{
Subsequent implementations are intended to support boolean algebra,
arithmetic, relational algebra, etc., to permit types over regions and
distributions. We envision this as a major step towards removing most,
if not all, dynamic array bounds and place checks from \Xten{}.
}


\subsubsection{Acyclicity restriction}

To ensure that type-checking is decidable, we
require that property graphs be acyclic.
That is, it should not be the case at runtime that
a set of objects can be created such that the
graph formed by taking objects as nodes and adding an edge from $m$ to
$n$ if $m$ has a property whose value is $n$ has a cycle in it.

Currently this restriction is not checked by the compiler. Future
versions of the compiler will check this restriction by introducing
rules on escaping of \Xcd{this} (\Sref{protorules}) before the invocation of
\Xcd{property} calls.


\section{Place types and Type reconstruction}\label{PlaceTypes}\index{place types}

\Xten{} distinguishes two kinds of places: {\em shared places}\index{places!shared}\label{SharedPlaces}, 
which correspond to the memory of individual processors which is
shared across multiple activities, and {\em scoped
places}\index{places!scoped}\label{ScopedPlaces} such as threadlocal
memory and method memory which is available only in limited
scope. \XtenCurrVer{} supports only {\tt threadlocal} scoped memory,
i.e.{} memory accessible only to the current activity.  Future
versions may support {\tt methodlocal} and {\tt blocklocal} memory.

\subsection{Place expressions}
The following are place expressions:
\begin{itemize}

\item {\tt here}: the place of the current computation
(in a method) or the place of the current object (in field
declarations).  At the top level of a method, {\tt here} is the place
at which the method invocation's object lives. Inside an {\tt async},
{\tt here} is the place of the nearest enclosing {\tt async}.

{}\item {\tt threadlocal}: the scoped place for the current thread
(only in scope for method activations)

{}\item {\tt D[i]}: where {\tt D} is a distribution
(\S~\ref{XtenDistributions}) and {\tt i} is a point in the region of
the distribution.

{}\item {\tt e.place} where {\tt e} is a variable
(\S~\ref{XtenVariables}) at a reference type. \Xten{} allows the use
of the expression {\tt e} in contexts expecting a {\tt place}
parameter as shorthand for {\tt e.place}.
\end{itemize}
Note that \XtenCurrVer{} does not permit the dynamic creation of a
place. Each \Xten{} computation is initiated with a fixed number of
places, as determined by a configuration parameter. The number is available
from {\tt place.MAX\_PLACES}.  The set of places is available 

Place expressions are used in the following contexts: 
\begin{itemize}
\item As a  place type in a reference type (\S~\ref{ReferenceTypes}).
\item As a target for an {\tt async} activity (\S~\ref{AsyncActivity}).
\item In a class cast expression  (\S~\ref{ClassCast}).
\item In an {\tt instanceof} expression (\S~\ref{instanceOf}).
\item In stable equality comparisons, at type {\tt place}.
\item As a parameter to a method invocation (\S~\ref{MethodInvocation}).
{}\item As a type parameter to a generic class or interface
(\S~\ref{TypeParameter}), or an actual argument to a generic class or interface (\S~\ref{ClassCreation}).
\end{itemize}

An example:
\begin{x10}
<P> void foo(Foo@P foo, Bar@P bar) \{
   int x = async (P) \{ return foo.f + bar.b; \};
   ...
\}
\end{x10}


Places can be passed as parameters to classes. The place where the
object lives is an implicit parameter, accessible within constructors
and instance initializers as here:

\begin{x10}
    class Foo <place P, place Q> \{
        Bar@P bar;
        Baz@Q baz;
        Qux@here qux;
    \}
\end{x10}

A reference type {\tt Foo} (with no place annotation) is always taken
as shorthand for {\tt Foo@here}.

An object can be cast to a particular place. If the object is not at
the right place, a {\tt BadPlaceException} is thrown:

\begin{x10}
    <P> m(Object@P obj) \{
        Object@here h = (Object@here)obj;
        h.blah();
    \}
\end{x10}

Places can be checked for equality. One can view this as the analog of
the {\tt instanceof} operator for places.

\begin{x10}
 <P> m(Object@P obj) \{
     if (here == P) \{
        // will never throw an exception
       Object@here h = (Object@here)obj;
       h.m();
   \}
\}
\end{x10}

A special form of object reference type, {\tt anywhere} or {\tt ?}, 
constrains the object to be at a shared place but does not constrain
the place. This allows collections of objects at heterogenous places.

\begin{x10}
    Object@?[] objects;
\end{x10}

We allow {\tt async} and {\tt future} statements to use {\tt ?} to
stand for any place. For instance:

\begin{x10}
     Object o = !future(?)\{new Object();\}
\end{x10}

\noindent spawns an activity at some arbitrary place and return an object created at that place.

Similarly the object reference type {\tt current} can be used in
(i)~constructing a distribution for a reference array or (ii)
specifying the location of the base type of a reference array. In the
first case points mapped to {\tt current} by the distribution will
reside in the same place as the array object itself, and in the second
case the value of the array at a particular point is an object in the
same place as that array component. Example:
\begin{x10}
    Object@current[] objects;
\end{x10}

\subsubsection{Type reconstruction}\index{type!reconstruction}

{}\Xten{} permits the use of the generic method syntax in variable
declarations. Any variable declaration may be prefixed with a type
parameter list and may use these parameters in type expressions in the
declaration. Such a parameterized variable declaration succeeds at
compile time if it is possible for the compiler to assign unique types
to the parameters in such a way that the declaration type-checks.  The
scope of the parameter is the scope of the variable introduced by the
declaration. Throughout this scope the parameter has the value
inferred by the compiler.

For instance:
\begin{x10}
  // This introduces P as a constant place, the
  // location of objects[0]
 <P> Object@P obj = objects[0];
 async (P) { obj.blah(); };
\end{x10}

Often it is the case that a type parameter is not referenced after it
is introduced. In such cases \Xten{} permits the use of ``\_'' (the
{\em anonymous
parameter})\label{AnonymousParameter}\index{parameter!anonymous@{\tt\_}}
as a parameter. Multiple occurrences of ``\_'' are taken to stand for
``fresh names'' in each occurrence.  For instance:
\begin{x10}
 <P> \_@P obj = objects[0];
 async (P) { obj.blah(); };
\end{x10}
\noindent should be taken as shorthand for
\begin{x10}
 <P,Q> Q@P obj = objects[0];
 async (P) { obj.blah(); };
\end{x10}
{}\noindent where {\tt Q} does not occur anywhere else in the
program. If a declaration only uses the anoymous parameter the angle
brackets may be omitted. Thus for example the often used special case:
\begin{x10}
  \_ obj = new int[D];
\end{x10}
\noindent is shorthand for:
\begin{x10}
  <> \_@\_ obj = new int[D];
\end{x10}

{}\noindent By default, objects are created {\tt here}. An object can
be created in thread local storage by using an {\tt @threadlocal}
suffix to the data type name:
\begin{x10}
   new Foo@threadlocal<here, here>();
\end{x10}


Note that the effect of creating an object at a place {\tt P} and
returning a reference to it may be obtained by:
\begin{x10}
    Foo@P x = future (P) { new Foo();}
\end{x10}

{}\Xten{} imposes the rule that the lifetime of places passed as place
parameters to objects must be no shorter than the lifetime of the
object itself. It also maintains the invariant that all place variables
in scope are guaranteed to last longer than this method. This implies
that given an invocation:
\begin{x10}
    new Foo@P<..., Q, ...>
\end{x10}
\noindent the following constraints must be true
\begin{itemize}
\item {\cf P} is {\cf threadlocal},  or
\item {\cf P} is {\cf here} (and not in an {\cf async}) and {\cf Q} is a parameter to this class, or {\cf Q} is {\cf shared} or {\cf P} is {\cf Q}
\end{itemize}

In a generic object constructor invocation, a type parameter is always
replaced with a type which includes the place. For instance

\begin{x10}
    \_ s = new Stack<Point@P>();
\end{x10}

Constructor calls, method calls and field accesses have the following place constraints.
\begin{x10}
  Foo@P foo = new Foo@P();
  int[]@P data;
  foo.m();
  int i = foo.f;
  foo.f = i;
  int i = data[j];
  data[j] = i;
\end{x10}
Here {\tt P} is scoped or {\tt here} or {\tt accessible}. {\tt f}
must be a final field.

\Xten{} also provides place inference for {\tt asyncs} and {\tt futures} (\S~\ref{AsyncActivity}):

\begin{x10}
  async \{{\cf\em{}Statement;}\} 
  async (\_) \{{\cf\em{}Statement;}\}
  future \{{\cf\em{}Statement;}\} 
  future (\_) \{{\cf\em{}Statement;}\} 
\end{x10}

In the {\tt async} ({\tt future}) statement, the target place of the
{\tt async} ({\tt future}) is the unique place that will satisfy the
place constraints of the body; if there is more than one such place,
an error is thrown at compile-time. We permit the second form so as to
allow the programmer to document that s/he intends the compiler to
infer the location.

Place variables outside an async are accessible inside the async.

\paragraph{Examples}
Consider a {\tt Stack} containing elements of some type {\tt T} which must
all be located at a given place {\tt P}:
\begin{x10}
 class Stack<T@P> \{
     ?T@P[] elements;
     int size;
     Stack() \{
         // filled with nulls.
         elements = new ?T@P[1000]; 
         size = 0;
     \}
     void push(T@P t) \{ elements[size++] = t; \}
     T@P pop() \{ return elements[--size]; \}
     void fill(T@P v) \{ 
         for (int i : elements) \{ 
            elements[i] = v; 
         \} 
     \}
 \}
\end{x10}

Consider the following array initializers (\S~\ref{ArrayInitializer}):
\begin{x10}
  distribution D = block(1000);
  // region 1..1000 treated as 1..1000 here
  \_ data = new int@current[1000](i){ return i*i; }; 
    // gives in P the place to which data[2] was 
    // mapped, i.e. the first place.
  <P> int@P q = data[2]; 
    // initialize the array with 10 times 
    // the index value
  float[D] d = new float[D] (i){ return 10.0*i; };
  float[D] d2 = new float[D] (i){ return i*i; };
  float[D] result = 
    new float[D] (i){ return di] + d2[i]; };
\end{x10}

The code fragments type-check because the compiler may make the
inference that {\tt here} inside the {\tt ateach} is {\tt D[i]}, and
that the place of the elements {\tt d[i]} and {\tt d2[i]} is also 
{\tt D[i]}. The \Xten{} compiler uses only syntactic equality and simple
intraprocedural dataflow identities to determine which places are the
same.  

The place associated with a particular distribution element may be
accessed using array syntax.
\begin{x10}
  place P = D[i];
  Object@P obj = async { new Object@P(); };
  int x = async (D[77]) { return data[77]; };
\end{x10}
Distributions can be passed as parameters much as places can.

\begin{x10}
 <P> void m(Object@P f) \{
    \_ stack = new Stack@local <Object@P>();
    fill(stack);
 \}
\end{x10}

In the following, the type specifier for the argument expands to
{\tt Stack<Object@P>@here}:

\begin{x10}
 <P> void fill(Stack<Object@P> stack) \{
    for (int i : 1000 ) \{
        stack.push( new Object@P(););
    \}
 \}

 <P> void unordered\_fill(Stack<Object@P> stack) \{
    ^Object@P[] futures = new ^Object@P[1000] 
          \{return new Object@P();\}; 
    for (^Object@P f : futures) \{
        stack.push(f);
    \}
 \}
\end{x10}

\paragraph{Examples with type inference}
\begin{x10}
 <P> void m() \{
    \_ stack = new Stack@threadlocal<Object, P>();
    fill(stack);
\}

 <P> void fill(Stack@threadlocal<Object@P> stack) \{
    for (int i = 0; i < 1000; i++) \{
        stack.push(async \{return new Object@P();\});
    \}
\}

<P> void unordered\_fill(Stack@local<Object@P> stack) \{
    \_ values = new \_[1000] \{return new Object@P();\};
    for (\_ f : values) \{
        stack.push(f);
    \}
\}

 // Use of anywhere types and newplace 
 // to create heterogenous collections
 void anywhere\_test() \{
    // create 1000 objects at 
    // 1000 different shared places
   \_ objs = new Object@?[1000]@local 
          (i)\{ async(?) \{ return new Object();\}\};
    for (\_ o : objs) \{
        <P> Object@P v = o;
        async \{v.blah(); \};
    \}
 \}
\end{x10}



\subsection{Constraint semantics}

\begin{staticrule}{Variable occurrence}
In a dependent type \xcd"T" = \xcd"C{c}", the only variables that may
occur in \xcd"c" are (a)
\xcd"self", (b) properties visible at \xcd"T", (c) final local variables, final
method parameters or final constructor parameters visible at \xcd"T", (d)
final fields visible at \xcd"T"'s lexical place in the source program.  
\end{staticrule}

\begin{staticrule}{Restrictions on \xcd"this"}
  The special variable \xcd"this" may be used in a dependent clause for a type \xcd"T"
  only if (a)~\xcd"T" occurs in a property declaration for a
  class, (b)~\xcd"T"
  occurs in an instance method, (c)~\xcd"T" occurs in an
  instance field, (d)~\xcd"T"
  occurs in an instance initializer.

  In particular, \xcd"this" may not be used in types that occur in a static
  context, or in the arguments, body or return type of a constructor or
  in the extends or implements clauses of class and interface
  definitions.  In these contexts, the object that \xcd"this" would
  correspond to is not defined.
\end{staticrule}

\begin{staticrule}{Variable visibility}
  If a type \xcd"T" occurs in a field, method or constructor
  declaration, then all variables used in \xcd"T" must have at least the
  same visibility as the declaration.  The relation ``at least the same
  visibility as'' is given by the transitive closure of:

\begin{xten}
public > protected > package > private
\end{xten}

All inherited properties of a type \xcd"T" are visible in the property
list of \xcd"T", and the body of \xcd"T".

\end{staticrule}

In general, variables (i.e., local variables, parameters,
properties, fields) are visible at
\xcd"T" if they are defined before \xcd"T" in the program. This rule applies to
types in property lists as well as parameter lists (for methods and
constructors).
A formal parameter is visible in the types of all other formal
parameters of the same method, constructor, or type definition,
as well as in the method or constructor body itself.
Properties are accessible via their containing object--\xcd"this"
within the body of their class declaration.  The special
variable \xcd"this" is in scope at each property
declaration, constructor signatures and bodies, instance method signatures
and bodies,
and instance field signatures and initializers, but not in scope
at \xcd"static" method or field declarations or \xcd"static"
initializers.  

We permit variable declarations \xcd"v: T" where \xcd"T" is obtained
from a dependent type \xcd"C{c}" by replacing one or more occurrences
of \xcd"self" in \xcd"c" by \xcd"v". (If such a declaration \xcd"v: T"
is type-correct, it must be the case that the variable \xcd"v" is not
visible at the type \xcd"T". Hence we can always recover the
underlying dependent type \xcd"C{c}" by replacing all occurrences of \xcd"v"
in the constraint of \xcd"T" by \xcd"self".)

For instance, \xcd"v: Int{v == 0}" is shorthand for \xcd"v: Int{self == 0}".

\begin{staticrule}{Constraint type}
The type of a constraint \xcd"c" must be \xcd"Boolean".  
\end{staticrule}

A variable occurring in the constraint \xcd"c" of a dependent type, other than
\xcd"self" or a property of \xcd"self", is said to be a {\em
parameter} of \xcd"c".\label{DepType:Parameter} \index{parameter}

An instance \xcd"o" of \xcd"C" is said to be of type \xcd"C{c}"
(or: {\em belong to}
\xcd"C{c}") if the predicate \xcd"c" evaluates to \xcd"true" in the current lexical
environment, augmented with the binding \xcd"self" $\mapsto$ \xcd"o". We shall
use the function \LL{\mbox{\Xcd{C\{c\}}}} to denote the set of
objects that belong to \xcd"C{c}". 


\subsection{Consistency of dependent types}\label{DepType:Consistency}\index{dependent type,consistency}

A dependent type \xcd"C{c}" may contain zero or more parameters. We require
that a type never be empty---so that it is possible for a variable of
the type to contain a value. This is accomplished by requiring that
the constraint \xcd"c" must be satisfiable {\em regardless} of the value assumed
by parameters to the constraint (if any). Formally, consider a type
\xcd"T" = \xcd"C{c}", with the variables
\xcdmath"f$_1$: F$_1$, $\dots$, f$_k$: F$_k$"
free in \xcd"c".  Let 
\xcdmath"$S$ = {f$_1$: F$_1$, $\dots$, f$_k$: F$_k$, f$_{k+1}$: F$_{k+1}$, $\dots$, f$_n$: F$_n$}"
be the smallest set of
declarations containing
\xcdmath"f$_1$: F$_1$, $\dots$, f$_k$: F$_k$"
and closed under the rule:
\xcd"f: F" in $S$ if a reference to variable \xcd"f" (which
is declared as \xcd"f: F") occurs in a type in $S$.

(NOTE: The syntax rules for the language ensure that $S$ is always
finite. The type for a variable \xcd"v" cannot reference a variable whose
type depends on \xcd"v".)

We say that \xcd"T" = \xcd"C{c}" is {\em parametrically consistent} (in brief:
{\em consistent}) if:

\begin{itemize}
\item Each type \xcdmath"F$_1$, $\dots$, F$_n$" is (recursively) parametrically consistent, and
\item It can be established that
\xcdmath"$\forall$f$_1$: F$_1$, $\dots$, f$_n$: F$_n$. $\exists$self: C. c && $\mathit{inv}$(C)".
\end{itemize}

\noindent
where \xcdmath"$\mathit{inv}$(C)" is the invariant associated
with the type \xcd"C" (\Sref{DepType:TypeInvariant}).  Note by
definition of $S$ the formula above has no free variables.

\begin{staticrule*}
For a declaration \xcd"v: T" to be type-correct, \xcd"T" must be parametrically
consistent. The compiler issues an error if it cannot determine
the type is parametrically consistent.
\end{staticrule*}

\begin{example}

A class that represents a line has two distinct points:

\begin{xten}
class Line(start: Point, 
           end: Point{self != this.start}) {...}      
\end{xten}
\end{example}

One can use dependent type to define other closed geometric figures as well.

\begin{example}
Here is an example:
\begin{xten}
class Point(x: Int, y: Int) {...}
\end{xten}

To see that the declaration \xcd"end: Point{self != start}" is
parametrically consistent, note that the following formula is valid:
\begin{xtenmath}
$\forall$this: Line. $\exists$self: Point. self != this.start  
\end{xtenmath}
\noindent since the set of all \xcd"Point"s has more than one element.
\end{example}

\begin{example}
A triangle has three lines sharing three vertices.
\begin{xten}
class Triangle 
 (a: Line, 
  b: Line{a.end == b.start}, 
  c: Line{b.end == c.start && c.end == a.start}) 
 { ... }
\end{xten}

Given \xcd"a: Line", the type \xcd"b: Line{a.end == b.start}" is consistent,
and
given the two, the type \xcd"c: Line{b.end == c.start, c.end == a.start}"
is consistent.

%%Similarly:
%%
%%   // A class with properties a, b,c,x satisfying the 
%%   // given constraints.
%%   class SolvableQuad(a: Int, b: Int, 
%%                      c: Int{b*b - 4*a*c >= 0},
%%                      x: Int{a*x*x + b*x + c==0}) { 
%%     ...
%%   }
%%
%%  // A class with properties r, x, and y satisfying
%%  // the conditions for (x,y) to lie on a circle with center (0,0)
%%  // and radius r.
%%   class Circle (r: Int{r > 0},
%%                 x: Int{r*r - x*x >= 0},
%%                 y: Int{y*y == r*r -x*x}) { 
%%   ...
%%   }
\end{example}

\section{Function types}
\label{FunctionTypes}
\label{FunctionType}
\index{function!types}
\index{types!function types}

        Function types are defined via the \xcd"=>" type
        constructor.  Closures (\Sref{Closures}) and method
        selectors (\Sref{MethodSelectors}) are of function type.
        The general form of a function type is:
\begin{xtenmath}
(x$_1$: T$_1$, $\dots$, x$_n$: T$_n$){c} => T
        throws S$_1$, $\dots$, S$_k$
\end{xtenmath}
        This
        is the type of functions that take 
        value parameters
        \xcdmath"x$_i$"
        of types
        \xcdmath"T$_i$"
        such that the guard \xcd"c" holds
        and returns a value of type \xcd"T" or throws
        exceptions of 
        types S$_i$.

The value parameters are in scope throughout the function
signature---they may be used in the types of other formal parameters
and in the return type.  Value parameters names  may be
omitted if they are not used.  The guard specifies a condition that 
must hold for an application to be well-typed.

\begin{grammar}
FunctionType \: TypeParameters\opt \xcd"(" Formals\opt \xcd")" Constraint\opt
\xcd"=>" Type Throws\opt \\
TypeParameters \: \xcd"[" TypeParameter ( \xcd"," TypeParameter
)\star \xcd"]" \\
TypeParameter \: Identifier \\
Formals \: Formal ( \xcd"," Formal )\star \\
\end{grammar}


For every sequence of types \xcd"T1,..., Tn,T", and \xcd"n" distinct variables
\xcd"x1,...,xn" and constraint \xcd"c", the expression
\xcd"(x1:T1,...,xn:Tn){c}=>T" is a \emph{function type}. It stands for
 the set of all functions \xcd"f" which can be applied in a place \xcd"p" to a
 list of values \xcd"(v1,...,vn)" provided that the constraint
 \xcd"c[v1,...,vn,p/x1,...,xn,here]" is true, and which returns a value of
 type \xcd"T[v1,...vn/x1,...,xn]". When \xcd"c" is true, the clause \xcd"{c}" can be
 omitted. When \xcd"x1,...,xn" do not occur in \xcd"c" or \xcd"T", they can be
 omitted. Thus the type \xcd"(T1,...,Tn)=>T" is actually shorthand for
 \xcd"(x1:T1,...,xn:Tn){true}=>T", for some variables \xcd"x1,...,xn".


Juxtaposition is used to express function application: the expression
\xcd"f(a1,..,an)" expresses the application of a function \xcd"f" to the argument
list \xcd"a1,...,an".

\index{Exception!unchecked}
Note that function invocation may throw unchecked exceptions. 

A function type is covariant in its result type and contravariant in
each of its argument types. That is, let 
\xcd"S1,...,Sn,S,T1,...Tn,T" be any
types satisfying \xcd"Si <: Ti" and \xcd"S <: T". Then
\xcd"(x1:T1,...,xn:Tn){c}=>S" is a subtype of
\xcd"(x1:S1,...,xn:Sn){c}=>T".


A value \xcd"f" of a function type \xcd"(x1:T1,...,xn:Tn){c}=>T" also
has all the methods of \Xcd{Any} associated with it (see \Sref{FunctionAnyMethods}).


A function type \xcd"F=(x1:T1,...,xn:Tn){c}=>T" can be used as the declared type of local variables, parameters, loop variables, return types of methods and in \xcd"_� instanceof F" and \xcd"_ as F" expressions. 


A class or struct definition may use a function type \xcd"F" in its
implements clause; this declares an abstract method 
\xcd"def apply(x1:T1,...,xn:Tn){c}:T" on that class. Similarly, an interface
definition may specify a function type "F" in its "extends" clause. A
class or struct implementing such an interface implicitly defines an
abstract method \xcd"def apply(x1:T1,..,xn:Tn){c}:T". Expressions of such
a struct, class or interface type can be assigned to variables of type
\xcd"F" and can be applied via juxtaposition to an argument list of the
right type.


Thus, objects and structs in \Xten{} may behave like functions. 

A function type \xcd"F" is not a class type in that it does not extend any
type or implement any interfaces, or support equality tests. \xcd"F" cannot be extended by any type. It
is not an interface type in that it is not a subtype of
\xcd"x10.lang.Object". (Values of type \xcd"F" cannot be assigned to variables of
type \xcd"x10.lang.Object".) It is not a struct type in that it has no
defined fields and hence no notion of structural equality.

\xcd"null" is a legal value for a function type. 


\section{Annotated types}
\label{AnnotatedTypes}

\index{types!annotated types}
\index{annotations!type annotations}

        Any \Xten{} type may be annotated with zero or more
        user-defined \emph{type annotations}
        (\Sref{XtenAnnotations}).  

        Annotations are defined as (constrained) interface types and are
        processed by compiler plugins, which may interpret the
        annotation symbolically.

        A type \xcd"T" is annotated by interface types
        \xcdmath"A$_1$", \dots,
        \xcdmath"A$_n$"
        using the syntax
        \xcdmath"@A$_1$ $\dots$ @A$_n$ T".

\section{Subtyping and type equivalence}\label{DepType:Equivalence}
\index{type equivalence}
\index{subtyping}

Subtyping is relation between types.  It is the
reflexive, transitive 
closure of the {\em direct subtyping} relation, defined as
follows.

\paragraph{Class types.}  A class type is a direct subtype of
any
class it is declared to extend.  A class type is direct subtype
of any interfaces it is declared to implement.

\paragraph{Interface types.}  An interface type is a direct
subtype of any interfaces it is declared to extend.

\paragraph{Function types.}

Function types are covariant on their return type and
contravariant on their argument types.
For instance,
a function type
\xcd"(S1) => T1" 
is a subtype of another function type
\xcd"(S2) => T2" 
if \xcd"S2" is a subtype of \xcd"S1"
and \xcd"T1" is a subtype of \xcd"T2".

\paragraph{Constrained types.}

Two dependent types \xcd"C{c}" and \xcd"C{d}" are said to be {\em equivalent} if 
\xcd"c" is true whenever \xcd"d" is, and vice versa. Thus, 
$\LL{\mbox{\Xcd{C\{c\}}}} = \LL{\mbox{\Xcd{C\{d\}}}}$.

Note that two dependent type that are syntactically different may be
equivalent. For instance, \xcd"Int{self >= 0}" and
\xcd"Int{self == 0 || self > 0}" are equivalent though they are syntactically
distinct. The \Java{} type system is essentially a nominal system---two
types are the same if and only if they have the same name. The \Xten{}
type system extends the nominal type system of \Java{} to permit
constraint-based equivalence.

A dependent type \xcd"C{c}" is a subtype of a type \xcd"C{d}" if
\xcd"c" implies \xcd"d".  When this subtyping relationship holds, 
$\LL{\mbox{\Xcd{C\{c\}}}}$ is a
subset of $\LL{\mbox{\Xcd{C\{d\}}}}$. All dependent types
defined on a class \xcd"C"
refine the unconstrained class type \xcd"C"; \xcd"C" is
equivalent to \xcd"C{true}".

\paragraph{Type parameters.}

A type parameter \xcd"X" of a class or interface \xcd"C"
is a subtype of a type \xcd"T" if
the 
class invariant of \xcd"C" implies that \xcd"X" is a subtype of \xcd"T".
Similarly, \xcd"T" is a subtype of parameter \xcd"X" if the
class invariant implies the relationship.

A type parameter \xcd"X" of a method
\xcd"m"
is a subtype of a type \xcd"T" if
the 
guard of \xcd"m" implies that \xcd"X" is a subtype of \xcd"T".
Similarly, \xcd"T" is a subtype of parameter \xcd"X" if the
guard implies the relationship.


\section{Least common ancestor of types}
\label{LCA}

To compute the type of conditional expressions
(\Sref{Conditional}),
and of rail constructors
(\Sref{RailConstructors}), the least common ancestor of types
must be computed.

The least common ancestor of two  types
\xcdmath"T$_1$" and \xcdmath"T$_2$"
is the
unique most-specific type
that is a supertype of both
\xcdmath"T$_1$" and \xcdmath"T$_2$".

If the most-specific type is not unique (which can happen when
\xcdmath"T$_1$" and \xcdmath"T$_2$" both implement two
or more incomparable interfaces), then
least common ancestor type is \xcd"x10.lang.Any".

\section{Coercions and conversions}
\label{XtenConversions}
\index{conversions}\index{coercions}

\XtenCurrVer{} supports the following coercions and conversions

\subsection{Coercions}

A {\em coercion} does not change object identity;
a coerced object may be explicitly coerced back to its original
type through a cast.

\paragraph{Subsumption coercion.}
A subtype may be implicitly coerced to any supertype.
\index{coercions!subsumption coercion}

\paragraph{Explicit coercion (casting with \xcd"as")}
A reference type may be explicitly coerced to any other
reference type using the \xcd"as" operation.
If the value coerced is not an instance of the target type,
a \xcd"TypeCastException" is thrown.  Casting to a constrained
type may require a run-time check that the constraint is
satisfied.
\index{coercions!explicit coercion}
\index{casting}
\index{\Xcd{as}}

\subsection{Conversions}

A {\em conversion} may change object identity if the type being
converted to is not the same as the type converted from.

\paragraph{Narrowing conversion.}
A value class may be explicitly converted to any superclass
using the \xcd"to" operation.

\index{conversions!narrowing conversions}

\paragraph{Widening numeric conversion.}
A numeric type may be implicitly converted to a wider
numeric type.
In particular, an implicit conversion may be performed between
a numeric type and a type to its right, below:
\begin{xten}
Byte < Short < Int < Long < Float < Double
\end{xten}

\index{conversions!widening conversions}
\index{conversions!numeric conversions}

\paragraph{Boxing conversion.}
A type \xcd"T" may be implicitly converted to the
boxed type \xcd"Box[T]".

\index{conversions!boxing conversions}
\index{autoboxing}
\index{boxing}

\paragraph{Unboxing conversion.}
A value type \xcd"Box[T]" may be implicitly converted to the
type \xcd"T".

\index{conversions!unboxing conversions}
\index{autoboxing}
\index{unboxing}

\paragraph{Explicit conversion (casting with \xcd"to")}
A type may be explicitly converted to any type to which it can be
implicitly converted or implicitly or explicitly coerced.

\index{conversions!explicit conversion}
\index{casting}
\index{\Xcd{to}}

\paragraph{String conversion.}
Any object that is an operand of the binary
\xcd"+" operator may
be converted to \xcd"String" if the other operand is a \xcd"String".
A conversion to \xcd"String" is performed by invoking the \xcd"toString()"
method of the object.

\index{conversions!string conversion}



%\subsection{Syntactic abbreviations}\label{DepType:SyntaxAbbrev}

\section{Built-in types}

The package \xcd"x10.lang" provides a number of built-in class and
interface declarations that can be used to construct types.

\subsection{The class \Xcd{Object}}
\label{Object}
\index{\Xcd{Object}}
\index{\Xcd{x10.lang.Object}}

The class \xcd"x10.lang.Object" is the supertype of all classes.
A variable of this type can hold a reference to any object.
The code for this class (with annotations removed) is:
\begin{xten}
public class Object (home: Place) 
     implements Any
{
    public native def this();
    public property def home() = home;
    public property def at(p:Place) = home==p;
    public property def at(r:Object) = home==r.home;
    public global safe native def toString() : String;
    public global safe native def typeName() : String;
    public global safe def equals(x:Any) = this == x;
    public global safe native def hashCode():Int;
}
\end{xten}

\subsection{The class \Xcd{String}}
\label{String}\index{\Xcd{String}}\index{\Xcd{x10.lang.String}}

All strings in \Xten{} are instances of the class
\xcd"x10.lang.String".
A string object is immutable,
and has a concatenation operator (\xcd"+") available on it.


\subsection{Array types}
\label{ArrayTypeConstructors}\index{array types}
\index{\Xcd{Array}}
\index{\Xcd{ValArray}}
\index{\Xcd{x10.lang.Array}}
\index{\Xcd{x10.lang.ValArray}}

Arrays in \Xten{} are instances of the value classes
\xcd"x10.lang.Array" and \xcd"x10.lang.ValArray".
Because of the importance of arrays in \Xten{}, the language
supports more concise syntax for accessing array elements and
performing operations on arrays.

% The array type constructor takes as argument a type (the {\em base
% type}), an optional distribution (\Sref{XtenDistributions}), and
% optionally the keyword \xcd"value":

The array type \xcd"Array[T]" is the type of all
reference arrays of base type \xcd"T". Such an array can take on any
distribution, over any region. 

The array type \xcd"ValArray[T]" specifies the type of all
values arrays of base type \xcd"T".
The array elements of a \xcd"ValArray" are
all final.\footnote{Note that the base type of a
\xcd"ValArray" can be a value class or a reference class, just as the 
type of a final variable can be a value class or a reference class.}

Both array classes implement the function type
\xcd"(Point) => T"; the element of array \xcd"A" at point
\xcd"p" may be accessed using the 
syntax \xcd"A(p)".  The \xcd"Array" class 
also implements the \xcd"Settable[Point,T]" interface 
permitting assignment to an array element using the syntax
\xcd"A(p) = v".

\XtenCurrVer{} also allows a distribution to be specified 
as a property initializer on the array type.
The distribution must be an expression of type
\xcd"Dist" whose
value does not depend on the value of any mutable variable.

\Xten{} also supports dependent types for arrays,
e.g.,
\xcd"Array[Double]{rank==3}" is the type of all arrays of 
\xcd"Double" of rank \xcd"3".

\subsection{Rails}

A \emph{rail} is a 1-dimensional, zero-based, local array. 
The type \xcd"Rail[T](n)" is an alias of the mutable array type
\xcd"Array[T](Region(0..n-1)->here)".
The type \xcd"ValRail[T](n)" is an alias of the mutable array type
\xcd"ValArray[T](Region(0..n-1)->here)".

\begin{xten}
package x10.lang;
type Rail = Array{rail};
type Rail[T] = Array[T]{rail};
type Rail[T](n: Int) = Array[T]{rail, region==[0..n-1]);
type Rail[T](n: Int) = Array[T]{rail, region==[0..n-1]);
\end{xten}

\Xten{} supports shorthand syntax for rail construction
(\Sref{RailConstructors}).


\subsection{Future types}

For any type \xcd"T", \xcd"future[T]" is a type:
\begin{verbatim}
424   Type ::=   future [ Type ] 
\end{verbatim}
The type  represents a value which when forced will return a value of type
\xcd"T". Thus the type makes available the following methods:

\begin{xten}
  public def force(): T;
\end{xten}

%% Nothing to say about future/nullable yet.


\section{Type inference}
\label{TypeInference}
\index{types!inference}
\index{type inference}

\XtenCurrVer{} supports limited local type inference, permitting
variable types and return types to be elided.
It is a static error if an omitted type cannot be inferred or
uniquely determined.

\subsection{Variable declarations}

The type of a variable declaration can be omitted if the
declaration has an initializer.  The inferred type of the
variable is the computed type of the initializer.

\subsection{Return types}

The return type of a method can be omitted if the method has a
body (i.e., is not \xcd"abstract" or \xcd"extern").  The
inferred return type is the computed type of the body.

The return type of a closure can be omitted.
The inferred return type is the computed type of the body.

The return type of a constructor can be omitted if the
constructor has a body (i.e., is not \xcd"extern").
The inferred return type is the enclosing class type with
properties bound to the arguments in the constructor's \xcd"property"
statement, if any, or to the unconstrained class type.

\index{Void}
The inferred type of a method or closure body is the least common ancestor
of the types of the expressions in \xcd"return" statements
in the body.  If the method does not return a value, the
inferred type is \xcd"Void".

\subsection{Type arguments}

A call to a polymorphic method %, closure, or constructor 
may omit the
explicit type arguments.  If the method has a type parameter
\xcd"T", the type argument corresponding to \xcd"T" is inferred
to be the least common ancestor of the types of any formal
parameters of type \xcd"T".

Consider the following method:
\begin{xten}
def choose[T](a: T, b: T): T { ... }
\end{xten}
%
Given \xcd"Set[T] <: Collection[T]", 
\xcd"List[T] <: Collection[T]",
and \xcd"SubClass <: SuperClass",
in the following snippet, the algorithm will infer the type
\xcd"Collection[Any]" for \xcd"x".
\begin{xten}
def m(intSet: Set[Int], stringList: List[String]) {
  val x = choose(intSet, stringList);
...
}
\end{xten}
%
And in this snippet, the algorithm should infer the type
\xcd"Collection[Int]" for \xcd"y".
\begin{xten}
def m(intSet: Set[Int], intList: List[Int]) {
  val y = choose(intSet, intList);
  ...
}
\end{xten}
%
Finally, in this snippet, the algorithm should infer the type
\xcd"Collection{T <: SuperClass}" for \xcd"z".
\begin{xten}
def m(intSet: Set[SubClass], numList: List{T <: SuperClass}) {
  val z = choose(intSet, numList);
  ...
}
\end{xten}

