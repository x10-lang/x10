%\documentclass[10pt,twoside,twocolumn,notitlepage]{report}
\documentclass[12pt,twoside,notitlepage]{report}
\usepackage{x10}
\usepackage{tenv}
\def\Hat{{\tt \char`\^}}
\usepackage{url}
\usepackage{times}
\usepackage{txtt}
\usepackage{ifpdf}
\usepackage{tocloft}
\usepackage{bcprules}
\usepackage{xspace}

\newif\ifdraft
%\drafttrue
\draftfalse

\pagestyle{headings}
\showboxdepth=0
\makeindex

\usepackage{commands}

\usepackage[
pdfauthor={Vijay Saraswat, Bard Bloom, Igor Peshansky, Olivier Tardieu, and David Grove},
pdftitle={Report on the Language X10},
pdfcreator={pdftex},
pdfkeywords={X10},
linkcolor=blue,
citecolor=blue,
urlcolor=blue
]{hyperref}

\ifpdf
          \pdfinfo {
              /Author   (Vijay Saraswat, Bard Bloom, Igor Peshansky, Olivier Tardieu, and David Grove)
              /Title    (Report on the Language X10)
              /Keywords (X10)
              /Subject  ()
              /Creator  (TeX)
              /Producer (PDFLaTeX)
          }
\fi

\def\headertitle{The \XtenCurrVer{} Report (Draft) }
\def\integerversion{2.0.5}

% Sizes and dimensions

%\topmargin -.375in       %    Nominal distance from top of page to top of
                         %    box containing running head.
%\headsep 15pt            %    Space between running head and text.

%\textheight 9.0in        % Height of text (including footnotes and figures, 
                         % excluding running head and foot).

%\textwidth 5.5in         % Width of text line.
\columnsep 15pt          % Space between columns 
\columnseprule 0pt       % Width of rule between columns.

\parskip 5pt plus 2pt minus 2pt % Extra vertical space between paragraphs.
\parindent 0pt                  % Width of paragraph indentation.
%\topsep 0pt plus 2pt            % Extra vertical space, in addition to 
                                % \parskip, added above and below list and
                                % paragraphing environments.


\newif\iftwocolumn

\makeatletter
\twocolumnfalse
\if@twocolumn
\twocolumntrue
\fi
\makeatother

\iftwocolumn

\oddsidemargin  0in    % Left margin on odd-numbered pages.
\evensidemargin 0in    % Left margin on even-numbered pages.

\else

\oddsidemargin  .5in    % Left margin on odd-numbered pages.
\evensidemargin .5in    % Left margin on even-numbered pages.

\fi


\newtenv{example}{Example}[section]
\newtenv{planned}{Planned}[section]

\begin{document}

% \parindent 0pt %!! 15pt                    % Width of paragraph indentation.

%\hfil {\bf 7 Feb 2005}
%\hfil \today{}

% First page

\thispagestyle{empty}

% \todo{"another" report?}

\title{Report on the Experimental Language \Xten \\
\large Version \integerversion}
\author{Please send comments to \\
Vijay Saraswat at \texttt{vsaraswa@us.ibm.com}}
\date\today
\maketitle

\if 0
\topnewpage[{
\begin{center}   
{\huge\bf Report on the Experimental Language \Xten{}}
\vskip 1ex
$$
\begin{tabular}{l@{\extracolsep{.5in}}lll}
\multicolumn{4}{c}{\sc Version \integerversion}\\
\multicolumn{4}{c}{\sc Please send comments to 
Vijay Saraswat at 
{\tt vsaraswa@us.ibm.com}}\\
%\multicolumn{4}{c}{({\sc IBM Confidential})}

%\ldots
\end{tabular}
$$
\vskip 2ex
% {\it Dedicated to the Memory of APL} % vj
{\bf \today}
\vskip 2.6ex
\end{center}


}]
\fi

\newcommand\authorsc[1]{#1}
%\newcommand\authorsc[1]{\textsc{#1}}


\chapter*{Summary}
This draft report provides an initial description of the programming
language \Xten. \Xten{} is a single-inheritance class-based object-oriented
(OO) programming language designed for high-performance, high-productivity
computing on high-end computers supporting $\approx 10^5$ hardware threads
and $\approx 10^{15}$ operations per second. 

{}\Xten{} is based on state-of-the-art object-oriented programming
languages and deviates from them only as necessary to support its
design goals. The language is intended to have a simple and clear
semantics and be readily accessible to mainstream OO programmers. It
is intended to support a wide variety of concurrent programming
idioms.
%, incuding data parallelism, task parallelism, pipelining.
%producer/consumer and divide and conquer.

%We expect to revise this document in the light of experience gained in implementing
%and using this language.

The \Xten{} design team consists of
\authorsc{David Bacon}, 
\authorsc{Raj Barik}, 
\authorsc{Ganesh Bikshandi}, 
\authorsc{Bob Blainey}, 
\authorsc{Philippe Charles}, 
\authorsc{Perry Cheng}, 
\authorsc{Christopher Donawa}, 
\authorsc{Julian Dolby}, 
\authorsc{Kemal Ebcio\u{g}lu},
\authorsc{Robert Fuhrer},
\authorsc{Patrick Gallop}, 
\authorsc{Christian Grothoff}, 
\authorsc{Allan Kielstra}, 
\authorsc{Sreedhar Kodali}, 
\authorsc{Sriram Krishnamoorthy}, 
\authorsc{Nathaniel Nystrom}, 
\authorsc{Igor Peshansky}, 
\authorsc{Vijay Saraswat} (contact author), 
\authorsc{Vivek Sarkar},
\authorsc{Armando Solar-Lezama},  
\authorsc{S. Alexander Spoon}, 
\authorsc{Sayantan Sur}, 
\authorsc{Christoph von Praun},
\authorsc{Pradeep Varma},
\authorsc{Krishna Venkata},
\authorsc{Jan Vitek}, and
\authorsc{Tong Wen}.

For extended discussions and support we would like to thank: 
Robert Callahan, Calin
Cascaval, Norman Cohen, Elmootaz Elnozahy, John Field, Bob Fuhrer,
Orren Krieger, Doug Lea, John McCalpin, Paul McKenney, Ram Rajamony,
R.K.~Shyamasundar, Filip Pizlo, V.T.~Rajan, Frank Tip, and Mandana Vaziri.

We thank Jonathan Rhees and William Clinger with help in obtaining the
\LaTeX{} style file and macros used in producing the Scheme report,
on which this document is based. We acknowledge the influence of
the $\mbox{\Java}^{\mbox{\authorsc{\small tm}}}$ Language
Specification \cite{jls2}.
%document, as evidenced by the numerous citations in the text.

This document revises Version 1.1 of the Report, released in
June 2007. It documents the language corresponding to the second
revision of the first version of the implementation.  This
revision was done by
\authorsc{Raj Barik}, 
\authorsc{Philippe Charles}, 
\authorsc{Christopher Donawa}, 
\authorsc{Robert Fuhrer},
\authorsc{Nathaniel Nystrom},  
\authorsc{Vijay Saraswat},
\authorsc{Vivek Sarkar},
\authorsc{Pradeep Varma}, and
\authorsc{Krishna Venkata}.
(Earlier implementations benefited from significant contributions by
\authorsc{Christian Grothoff} and 
\authorsc{Christoph von Praun}.)
\authorsc{Tong Wen} has written many application programs
in \Xten{}. \authorsc{Guojing Cong} has helped in the
development of many applications.


%\vfill
%\begin{center}
%{\large \bf
%*** DRAFT*** \\
%%August 31, 1989
%\today
%}\end{center}

\vfill
\eject


\chapter*{Contents}
\addvspace{3.5pt}                  % don't shrink this gap
\renewcommand{\tocshrink}{-3.5pt}  % value determined experimentally
{\footnotesize
\tableofcontents
}

\vfill
\eject


 

\clearpage

{\parskip 0pt
\addtolength{\cftsecnumwidth}{0.5em}
\addtolength{\cftsubsecnumwidth}{0.5em}
%\addtolength{\cftsecindent}{0.5em}
\addtolength{\cftsubsecindent}{0.5em}
\tableofcontents
}

\chapter{Introduction}

\subsection*{Background}
Larger computational problems require more powerful computers capable of
performing a larger number of operations per second. The era of
increasing performance by simply increasing clocking frequency now
seems to be behind us. It has become increasingly difficult
to manage chip power and heat.  Instead, computer
designers are starting to look at {\em scale out} systems in which the
system's computational capacity is increased by adding additional
nodes of comparable power to existing nodes, and connecting nodes with
a high-speed communication network.

A central problem with scale out systems is a definition of the {\em
memory model}, that is, a model of the interaction between shared
memory and  simultaneous (read, write) operations on that
memory by multiple processors. The traditional ``one operation at a
time, to completion'' model that underlies Lamport's notion of {\em
sequential consistency} (SC) proves too expensive to implement in
hardware, at scale. Various models of {\em relaxed consistency} have
proven too difficult for programmers to work with.  

One response to this problem has been to move to a {\em fragmented
memory model}. Multiple processors are made to interact via a
relatively language-neutral message-passing format such as MPI
\cite{mpi}. This model has enjoyed some success: several
high-performance applications have been written in this
style. Unfortunately, this model leads to a {\em loss of programmer
productivity}: the message-passing format is integrated into the host
language by means of an application-programming interface (API), the
programmer must explicitly represent and manage the interaction
between multiple processes and choreograph their data exchange; large
data-structures (such as distributed arrays, graphs, hash-tables) that
are conceptually unitary must be thought of as fragmented across
different nodes; all processors must generally execute the same code
(in an SPMD fashion) etc.

One response to this problem has been the advent of the {\em
partitioned global address space} (PGAS) model underlying languages
such as UPC, Titanium and Co-Array Fortran \cite{pgas,titanium}. These
languages permit the programmer to think of a single computation
running across multiple processors, sharing a common address
space. All data resides at some processors, which is said to have {\em
affinity} to the data.  Each processor may operate directly on the
data it contains but must use some indirect mechanism to access or
update data at other processors. Some kind of global {\em barriers}
are used to ensure that processors remain roughly in lock-step.

\Xten{} is a modern object-oriented programming language
in the PGAS family. The fundamental goal of \Xten{} is to enable
scalable, high-performance, high-productivity transformational
programming for high-end computers---for traditional numerical
computation workloads (such as weather simulation, molecular dynamics,
particle transport problems etc) as well as commercial server
workloads.

\Xten{} is based on state-of-the-art object-oriented
programming ideas primarily to take advantage of their proven
flexibility and ease-of-use for a wide spectrum of programming
problems. \Xten{} takes advantage of several years of research (e.g.,
in the context of the Java Grande forum,
\cite{moreira00java,kava}) on how to adapt such languages to the context of
high-performance numerical computing. Thus \Xten{} provides support
for user-defined {\em struct types} (such as \xcd"Int", \xcd"Float",
\xcd"Complex" etc), supports a very
flexible form of multi-dimensional arrays (based on ideas in ZPL
\cite{zpl}) and supports IEEE-standard floating point arithmetic.
Some capabilities for supporting operator overloading are also provided.

{}\Xten{} introduces a flexible treatment of concurrency, distribution
and locality, within an integrated type system. \Xten{} extends the
PGAS model with {\em asynchrony} (yielding the {\em APGAS} programming
model). {}\Xten{} introduces {\em places} as an abstraction for a
computational context with a locally synchronous view of shared
memory. An \Xten{} computation runs over a large collection of places.
Each place hosts some data and runs one or more {\em
activities}. Activities are extremely lightweight threads of
execution. An activity may synchronously (and {\em atomically}) use
one or more memory locations in the place in which it resides,
leveraging current symmetric multiprocessor (SMP) technology.  
An activity may shift to another place to execute a statement block.
%TODO yoav: ``using ``at'' one can update another place synchronously
\Xten{} provides weaker ordering guarantees for
inter-place data access, enabling applications to scale.  
Multiple memory locations in multiple places cannot be
accessed atomically.  {\em
Immutable} data needs no consistency management and may be freely
copied by the implementation between places.  One or more {\em clocks}
may be used to order activities running in multiple
places.  \xcd`DistArray`s, distributed arrays,  may be distributed across
multiple 
places and  support parallel collective operations. A novel
exception flow model ensures that exceptions thrown by asynchronous
activities can be caught at a suitable parent activity.  The type
system tracks which memory accesses are local. The programmer may
introduce place casts which verify the access is local at run time.
Linking with native code is supported.

%%OBSOLETE?%% \XtenCurrVer builds on v1.7 to support the following features: {\em
%%OBSOLETE?%%   structs} (i.e., ``header-less'', inlinable objects), type rules for
%%OBSOLETE?%% preventing escape of \xcd{this} from a constructor, 
%%OBSOLETE?%% the introduction of a global object model, permitting user-specified
%%OBSOLETE?%% (immutable) fields to be replicated with the object reference.
%%OBSOLETE?%% \xcd{value} classes are no longer supported; their functionality is
%%OBSOLETE?%% accomplished by using structs or global fields and methods.
%%OBSOLETE?%% 

%%OBSOLETE?%% Several representative idioms for concurrency and communication have
%%OBSOLETE?%% already found pleasant expression in \Xten. We intend to develop
%%OBSOLETE?%% several full-scale applications to get better experience with the
%%OBSOLETE?%% language, and revisit the design in the light of this experience.


\chapter{Overview of \Xten}

\Xten{} is a statically typed object-oriented language, extending a
sequential core language with
\emph{places}, \emph{activities}, \emph{clocks},
(distributed, multi-dimensional) \emph{arrays} and \emph{struct}
types. All these changes are motivated by the desire to use the new
language for high-end, high-performance, high-productivity computing.

\section{Object-oriented features}

The sequential core of \Xten{} is a {\em container-based} object-oriented language
similar to \java{} and C++, and more recent language such as Scala.  
Programmers write \Xten{} code by defining containers for data and behavior called
\emph{interfaces}
(\Sref{XtenInterfaces}),
\emph{classes}
(\Sref{XtenClasses}) and
\emph{structs}
(\Sref{XtenStructs}).
X10 provides inheritance and subtyping in fairly traditional ways. 

\begin{example}

\xcd`Normed` describes entities with a \xcd`norm()` method. \xcd`Normed` is
intended to be used for entities with a position in some coordinate system,
and \xcd`norm()` gives the distance between the entity and the origin. A
\xcd`Slider` is an object which can be moved around on a line; a
\xcd`PlanePoint` is a fixed position in a plane. Both \xcd`Slider`s and
\xcd`PlanePoint`s have a sensible \xcd`norm()` method, and implement
\xcd`Normed`.

%~~gen
% package Overview;
%~~vis
\begin{xten}
interface Normed {
  def norm():Double;
}
class Slider implements Normed {
  var x : Double = 0;
  public def norm() = Math.abs(x);
  public def move(dx:Double) { x += dx; }
}
struct PlanePoint implements Normed {
  val x : Double, y:Double;
  public def this(x:Double, y:Double) {
    this.x = x; this.y = y;
  }
  public def norm() = Math.sqrt(x*x+y*y);
}
\end{xten}
%~~siv
%
%~~neg
\end{example}

\paragraph{Interfaces}

An \Xten{} interface specifies a collection of abstract methods; \xcd`Normed`
specifies just \xcd`norm()`. Classes and
structs can be specified to {\em implement} interfaces, as \xcd`Slider` and
\xcd`PlanePoint` implement \xcd`Normed` , and, when they do so, must provide
all the methods that the interface demands.

Interfaces are
purely abstract. Every value of type \xcd`Normed` must be an instance of some
class like \xcd`Slider` or some struct like \xcd`PlanePoint` which implements
\xcd`Normed`; no value can be \xcd`Normed` and nothing else. 


\paragraph{Classes and Structs}

There are two kinds of concrete containers: \emph{classes}
(\Sref{ReferenceClasses}) and \emph{structs} (\Sref{Structs}). Concrete
containers hold data in {\em fields}, and give concrete implementations of
methods, as \xcd`Slider` and \xcd`PlainPoint` above.

Classes are organized in a single-inheritance tree: a class may have only a
single parent class, though it may implement many interfaces and have many
subclasses. Classes may have mutable fields, as \xcd`Slider` does.

In contrast, structs are headerless values, lacking the internal organs
which give objects their intricate behavior.  This makes them less powerful
than objects (\eg, structs cannot inherit methods, though objects can), but also
cheaper (\eg, they can be inlined, and they require less space than objects).  
Structs are immutable, though their fields may be immutably set to objects
which are themselves mutable.  They behave like objects in all ways consistent
with these limitations; \eg, while they cannot {\em inherit} methods, they can
have them -- as \xcd`PlanePoint` does.

\Xten{} has no primitive classes per se. However, the standard library
\xcd"x10.lang" supplies structs \xcd"Boolean", \xcd"Byte", \xcd"Short",
\xcd"Char", \xcd"Int", \xcd"Long", \xcd"Float", \xcd"Double", \xcd"Complex"
and \xcd"String". The user may defined additional arithmetic structs using the
facilities of the language.



\paragraph{Functions.}

X10 provides functions (\Sref{Closures}) to allow code to be used
as values.  Functions are first-class data: they can be stored in lists,
passed between activities, and so on.  \xcd`square`, below, is a function
which squares an \xcd`Int`.  \xcd`of4` takes an \xcd`Int`-to-\xcd`Int`
function and applies it to the number \xcd`4`.  So, \xcd`fourSquared` computes
\xcd`of4(square)`, which is \xcd`square(4)`, which is 16, in a fairly
complicated way.
%~~gen
% package Overview.of.Functions.one;
% class Whatever{
% def chkplz() {
%~~vis
\begin{xten}
  val square = (i:Int) => i*i;
  val of4 = (f: (Int)=>Int) => f(4);
  val fourSquared = of4(square);
\end{xten}
%~~siv
%}}
%~~neg



They are used extensively in X10
programs.  For example, the normal way to construct a \xcd`Rail[Int]` --
that is, a fixed-length array of numbers, like an \xcd`int[]` in Java -- is to
pass two arguments to a factory method: the first argument being the length of
the rail, and the second being a function which computes the initial value of
the \xcd`i`{$^{th}$} element.  The following code constructs a rail
initialized to the squares of 0,1,...,9: \xcd`r(0) == 0`, \xcd`r(5)==25`, etc. 
%~~gen
% package Overview.of.Functions.two;
% class Whatevermore {
%  def plzchk(){
%    val square = (i:Int) => i*i;
%~~vis
\begin{xten}
  val r : Rail[Int] = Rail.make[Int](10, square);
\end{xten}
%~~siv
%}}
%~~neg




%%BARD-HERE



\paragraph{Constrained Types}

X10 containers may declare {\em properties}, which are fields bound immutably
at the creation of the container.  The static analysis system understands
properties, and can work with them logically.   


For example, an implementation of matrices \xcd`Mat` might have the numbers of
rows and columns as properties.  A little bit of care in definitions allows
the definition of a \xcd`+` operation that works on matrices of the same
shape, and \xcd`*` that works on matrices with appropriately matching shapes
The following code typechecks, but an attempt to compute \xcd`axb1 + bxc` or
\xcd`bxc * axb1` would result in a compile-time type error:
%~~gen
%package Overview.Mat1;
%abstract class Mat(rows:Int, cols:Int) {
%  static type Mat(r:Int, c:Int) = Mat{self.rows==r&&self.cols==c};
%  public def this(r:Int, c:Int) : Mat(r,c) = {property(r,c);}
%  static def makeMat(r:Int,c:Int) : Mat(r,c) = null;
%  abstract global operator this + (y:Mat(this.rows,this.cols)):Mat(this.rows, this.cols);
%  abstract global operator this * (y:Mat) {this.cols == y.rows} : Mat(this.rows, y.cols);
%~~vis
\begin{xten}
  static def example(a:Int, b:Int, c:Int) {
    val axb1 : Mat(a,b) = makeMat(a,b);
    val axb2 : Mat(a,b) = makeMat(a,b);
    val bxc  : Mat(b,c) = makeMat(b,c);
    val axc  : Mat(a,c) = (axb1 +axb2) * bxc;
  }

\end{xten}
%~~siv
%}
%~~neg

The ``little bit of care'' shows off many of the features of constrained
types.    
The \xcd`(rows:Int, cols:Int)` in the class definition declares two
properties, \xcd`rows` and \xcd`cols`.\footnote{The class is officially declared
abstract to allow for multiple implementations, like sparse and band matrices,
but in fact is abstract to avoid having to write the actual definitions of
\xcd`+` and \xcd`*`.}  

A constrained type looks like \xcd`Mat{self.rows==r && self.cols==c}`: a type
name, followed by a Boolean expression in braces.  The special variable
\xcd`self` refers to the matrix whose number of rows and columns is being
checked.  The \xcd`type` declaration on the second line makes
\xcd`Mat(2,3)` be a synonym for \xcd`Mat{self.rows==r && self.cols==c}`,
allowing for compact types in many places.

Functions can return constrained types.  
The \xcd`makeMat(r,c)` method returns a \xcd`Mat(r,c)` -- a matrix whose shape
is given by the arguments to the method.  For the sake of brevity in
the example, it returns \xcd`null`; in real code, it would actually produce a
matrix -- which must be statically provable to have the right shape.

The arguments of methods can have type constraints as well.  The 
\xcd`operator this +` line lets \xcd`A+B` add two matrices.  The type of the
second argument \xcd`y` is constrained to have the same number of rows and
columns as the first argument \xcd`this`. Attempts to add mismatched matrices
will be flagged as type errors at compilation.

At times it is more convenient to put the constraint on the method as a whole,
as seen in the \xcd`operator this *` line. Unlike for \xcd`+`, there is no
need to constrain both dimensions; we simply need to check that the columns of
the left factor match the rows of the right. This constraint is written in
\xcd`{...}` after the argument list.  The shape of the result is computed from
the shapes of the arguments.

And that is all that is necessary for a user-defined class of matrices to have
shape-checking for matrix addition and multiplication.  The \xcd`example`
method compiles under those definitions.

%~~gen
%package Overview.Mat2;
%~~vis
\begin{xten}
abstract class Mat(rows:Int, cols:Int) {
 static type Mat(r:Int, c:Int) = Mat{self.rows==r&&self.cols==c};
 static def makeMat(r:Int,c:Int) : Mat(r,c) = null;
 abstract global operator this + (y:Mat(this.rows,this.cols))
                 :Mat(this.rows, this.cols);
 abstract global operator this * (y:Mat) {this.cols == y.rows} 
                 :Mat(this.rows, y.cols);
\end{xten}
%~~siv
%  static def example(a:Int, b:Int, c:Int) {
%    val axb1 : Mat(a,b) = makeMat(a,b);
%    val axb2 : Mat(a,b) = makeMat(a,b);
%    val bxc  : Mat(b,c) = makeMat(b,c);
%    val axc  : Mat(a,c) = (axb1 +axb2) * bxc;
%  }
%}
%~~neg




%%BARD-HERE

\paragraph{Generic types}

Classes and interfaces may have type parameters, permitting the definition of
{\em generic types}.  Type parameters may be instantiated by any type -- a
container type or a function type.

For example,
the following code declares a simple \xcd"List" class with a
type parameter \xcd"T".

%~~gen
%~~vis
\begin{xten}
class List[T] {
    var head: T;
    var tail: List[T]!;
    def this(h: T, t: List[T]!) { head = h; tail = t; }
    def add(x: T) {
        if (this.tail == null)
            this.tail = new List(x, null);
        else
            this.tail.add(x);
    }
}
\end{xten}
%~~siv
%~~neg
The constructor (\xcd"def this") initializes the fields of the new object.
The \xcd"add" method appends an element to the list.
\xcd"List" is a generic type.  When  instances of \xcd"List" are
allocated, the type \param{} \xcd"T" must be bound to a concrete
type.  \xcd"List[Int]" is the type of lists of element type
\xcd"Int", \xcd"List[String]" is the type of lists of element
type \xcd"String".

\section{The sequential core}

\paragraph{Control flow.}  \Xten{} supports standard sequential control flow
constructs: \xcd"if" statements, \xcd"while" loops, \xcd"for" loops,
\xcd"switch" statements, etc.  \Xten{} also supports exceptions: exceptions are
raised by \xcd"throw" statements and are handled by \xcd"try"--\xcd"catch"
statements.

\paragraph{Primitive operations.}  The language provides syntax for performing
binary and unary operations on values. The programmer may specify code
by using \Xcd{operator} definitions.
%% TODOvj: Add link to operator section.


\paragraph{Allocation.}
Objects are allocated with the \xcd"new" operator
(\Sref{ClassCreation}), which takes a class name and type and value
arguments to pass to the class's constructor.  The constructor must
ensure that all properties of the class and its superclasses are
bound. Structs are created through the invocation of a constructor
without using the \Xcd{new} operator and live on the heap only in
the fields of objects.

\paragraph{Coercions and conversions}
\Xten{} supports implicit and explicit coercions and
conversions (\Sref{XtenConversions}).

Values of one type can be converted to another type using the
\xcd"as" operation:
%~~gen
%class AsIsizer {
%~~vis
\begin{xten}
val x: Int = 65535;
val y: Byte = x as Byte; // convert to Byte,
                         // retaining the lower 8 bits
\end{xten}
%~~siv
%}
%~~neg
The \xcd"as" operation does not necessarily preserve equality
and for numeric values may
result in a loss of precision.

References may be coerced to another type, preserving object
identity.  A run-time check is performed to ensure the reference
is to an object of the target type.  If not,
a \xcd"ClassCastException" is thrown.
%TODO -- needs dynamic test case.
%~~gen
% class B{}
% class C extends B{}
% class D extends B{}
% class Whudnut { def plurd() {
%~~vis
\begin{xten}
// C and D are immediate subclasses of B.
val x: B = new C();
val y: C = x as C; // run-time check succeeds
val z: D = x as D; // run-time check fails
\end{xten}
%~~siv
%}}
%~~neg


\section{Places and activities}
An \Xten{} program is intended to run on a wide range of computers,
from uniprocessors to large clusters of parallel processors supporting
millions of concurrent operations. To support this scale, \Xten{}
introduces the central concept of \emph{place} (\Sref{XtenPlaces}).
Conceptually, a place is a ``virtual shared-memory multi-processor'':
a computational unit with a finite (though perhaps changing) number of
hardware threads and a bounded amount of shared memory, uniformly
accessible by all threads.

An \Xten{} computation acts on \emph{data
objects}(\Sref{XtenObjects}) through the execution of lightweight
threads called \emph{activities}(\Sref{XtenActivities}).  Objects are
of two kinds. A \emph{scalar} object has a small, statically fixed set
of fields, each of which has a distinct name. A scalar object is
located at a single place and stays at that place throughout its
lifetime.  An \emph{aggregate} object has many fields (the number may
be known only when the object is created), uniformly accessed through
an index (e.g., an integer) and may be distributed across many
places. The distribution of an aggregate object remains unchanged
throughout the computation. \Xten{} assumes an underlying garbage
collector will dispose of (scalar and aggregate) objects and reclaim
the memory associated with them once it can be determined that these
objects are no longer accessible from the current state of the
computation. (There are no operations in the language to allow a
programmer to explicitly release memory.)

{}\Xten{} has a \emph{unified} or \emph{global address space}. This
means that an activity can reference objects at other places.
However, an activity may synchronously access data items only in the
current place (the place in which the activity is running). It may
atomically update one or more data items, but only in the current
place.  To read a remote location, an activity must spawn another
activity \emph{asynchronously} (\Sref{AsynchronousActivity}). This
operation returns immediately, leaving the spawning activity with a
\emph{future} (\Sref{XtenFutures}) for the result. Similarly, remote
location can be written into only by asynchronously spawning an
activity to run at that location.

Throughout its lifetime an activity executes at the same place. An
activity may dynamically spawn activities in the current or remote
places.

\paragraph{Place casts.}
The programmer may use the standard type cast mechanism
(\Sref{ClassCast}) to cast a value to a located type. A
\xcd"BadPlaceException" is thrown if the value is not of the given
type. This is the only language construct that throws a \xcd"BadPlaceException".

\paragraph{Atomic blocks.}

\Xten{} introduces statements of the form \xcd"atomic S" where \xcd"S"
is a statement.  The type system ensures that such a statement will
dynamically access only local data. (The statement may throw
a \xcd"BadPlaceException"---but only because of a failed place cast.)
Such a statement is executed by the activity as if in a single step
during which all other activities are frozen.

\paragraph{Asynchronous activities.}

An asynchronous activity is created by a statement \xcd"async (P) S"
where \xcd"P" is a place expression and \xcd"S" is a statement.  Such
a statement is executed by spawning an activity at the place
designated by \xcd"P" to execute statement \xcd"S".

An asynchronous expression of type \xcd"Future[T]" has the form
\xcd"future (p) e" where \xcd"e" is an expression of type
\xcd"T".  The expression \xcd"e"
may reference \xcd`val`s and shared \xcd`var`s declared in the lexically
enclosing environment.  It executes the expression \xcd"e" at the
place \xcd"p" as an asynchronous activity, immediately returning with
a future. The future may later be forced causing the activity to be
blocked until the return value has been computed by the asynchronous
activity.

\section{Clocks}
The MPI style of coordinating the activity of multiple processes with
a single barrier is not suitable for the dynamic network of (possibly
diverse) activities in an \Xten{} computation. Instead, it becomes
necessary to allow a computation to use multiple barriers. \Xten{}
\emph{clocks} (\Sref{XtenClocks}) are designed to offer the
functionality of multiple barriers in a dynamic context while still
supporting determinate, deadlock-free parallel computation.

Activities may use clocks to repeatedly detect quiescence of arbitrary
programmer-specified, data-dependent set of activities. Each activity
is spawned with a known set of clocks and may dynamically create new
clocks. At any given time an activity is \emph{registered} with zero or
more clocks. It may register newly created activities with a clock,
un-register itself with a clock, suspend on a clock or require that a
statement (possibly involving execution of new async activities) be
executed to completion before the clock can advance.  At any given
step of the execution a clock is in a given phase. It advances to the
next phase only when all its registered activities have \emph{quiesced}
(by executing a \xcd"next" operation on the clock).
When a clock advances, all its activities may now resume execution.

Thus clocks act as \emph{barriers} for a dynamically varying collection
of activities. They generalize the barriers found in MPI style program
in that an activity may use multiple clocks simultaneously. Yet
programs using clocks are guaranteed not to suffer from
deadlock.

\futureext{In future versions of the language,
clocks will be integrated into the \Xten{} type system,
permitting variables to be declared so that they are \xcd`val` in each
phase of a clock.}

\section{Arrays, regions and distributions}

An \Xten{} array type is a map from a \emph{distribution}
(\Sref{XtenDistributions}) to a type, which may itself be an
array type.

% POINT REGION
A distribution is a map from a \emph{region} (\Sref{XtenRegions}) to
places.  A region is a collection of \emph{points} or
\emph{indices}. For instance, the region \xcd"[0..200,1..100]" specifies
a collection of two-dimensional points \xcd"(i,j)" with
\xcd"i" ranging from \xcd"0" to \xcd"200" and \xcd"j" ranging
from \xcd"1" to \xcd"100". Points are used in array index expressions
to pick out a particular array element.

Operations are provided to construct regions from other regions, and
to iterate over regions. Standard set operations, such as union,
disjunction and set difference are available for regions.

A primitive set of distributions is provided, together with operations
on distributions. A \emph{sub-distribution} of a distribution is one
defined on a smaller region and agrees with the distribution
at all points.  The standard operations on regions are extended to
distributions.

% XXX views

A new array can be created by restricting an existing array to a
sub-distribution, by combining multiple arrays, and by performing
pointwise operations on arrays with the same distribution.

\Xten{} allows array constructors to iterate over the underlying
distribution and specify a value at each item in the underlying
region. Such a constructor may spawn activities at multiple places.

\emph{In future versions of the language, a programmer may specify new
distributions, and new operations on distributions.}

\section{Annotations}

\Xten{} supports annotations on classes and interfaces, methods
and constructors,
variables, types, expressions and statements.
These annotations may be processed by compiler plugins.

\section{Translating MPI programs to \Xten{}}

While \Xten{} permits considerably greater flexibility in writing
distributed programs and data structures than MPI, it is instructive
to examine how to translate MPI programs to \Xten.

Each separate MPI process can be translated into an \Xten{}
place. Async activities may be used to read and write variables
located at different processes. A single clock may be used for barrier
synchronization between multiple MPI processes. \Xten{} collective
operations may be used to implement MPI collective operations.
\Xten{} is more general than MPI in (a)~not requiring synchronization
between two processes in order to enable one to read and write the
other's values, (b)~permitting the use of high-level atomic blocks
within a process to obtain mutual exclusion between multiple
activities running in the same node (c)~permitting the use of multiple
clocks to combine the expression of different physics (e.g.,
computations modeling blood coagulation together with computations
involving the flow of blood), (d)~not requiring an SPMD style of
computation.


%\note{Relaxed exception model}
\section{Summary and future work}
\subsection{Design for scalability}
\Xten{} is designed for scalability. An activity may atomically
access only multiple locations in the current place. Unconditional
atomic blocks are statically guaranteed to be non-blocking, and may
be implemented using non-blocking techniques that avoid mutual
exclusion bottlenecks. Data-flow synchronization permits point-to-point
coordination between reader/writer activities, obviating the need for
barrier-based or lock-based synchronization in many cases.

\subsection{Design for productivity}
\Xten{} is designed for productivity.

\paragraph{Safety and correctness.}
Programs written in \Xten{} are guaranteed to be statically
\emph{type safe}, \emph{memory safe} and \emph{pointer safe}. Static type safety
guarantees that at run time a location contains only those values whose

dynamic type satisfies the constraints imposed by the location's
static type and every run-time operation performed on the value in a
location is permitted by the static type of the location.

Memory safety guarantees that an object may only access memory within
its representation, and other objects it has a reference to. \Xten{}
supports no pointer arithmetic, and bound-checks array accesses
dynamically if necessary. \Xten{} uses dynamic garbage collection to
collect objects no longer referenced by the computation. \Xten{}
guarantees that no object can retain a reference to an object
whose memory has been reclaimed.  Further, \Xten{} guarantees that
every location is initialized at run time before it is read,
and every value read from a location has previously been written into
that location.

%XXX
%Pointer safety guarantees that a null pointer exception cannot be
%thrown by an operation on a value of a non-nullable type.

Because places are reflected in the type system, static type safety
also implies \emph{place safety}: a location may contain references to only
those objects whose location satisfies the restrictions of the static
place type of the location.

\Xten{} programs that use only clocks and unconditional atomic
blocks are guaranteed not to deadlock. Unconditional atomic blocks
are non-blocking, hence cannot introduce deadlocks (assuming the
implementation is correct).

Many concurrent programs can be shown to be determinate (hence
race-free) statically.

\paragraph{Integration.}
A key issue for any new programming language is how well it can be
integrated with existing (external) languages, system environments,
libraries and tools.

We believe that \Xten{}, like \java{}, will be able to support a large
number of libraries and tools. An area where we expect future versions
of \Xten{} to improve on \java{} like languages is \emph{native
integration} (\Sref{extern}). Specifically, \Xten{} will permit
permit multi-dimensional local arrays to be operated on natively by
native code.

%% Portability measures the amount of effort required to move an
%% application across multiple platforms, architectures and system
%% generations. The performance portability of applications across widely
%% different computer architectures (distributed cluster vs. vector
%% processor) depends significantly on inherent properties of the
%% underlying algorithms used in the application.
%%
\subsection{Conclusion}
{}\Xten{} is considerably higher-level than thread-based languages in
that it supports dynamically spawning very lightweight activities, the
use of atomic operations for mutual exclusion, and the use of clocks
for repeated quiescence detection.

Yet it is much more concrete than languages like HPF in that it forces
the programmer to explicitly deal with distribution of data
objects. In this the language reflects the designers' belief that
issues of locality and distribution cannot be hidden from the
programmer of high-performance code in high-end computing.  A
performance model that distinguishes between computation and
communication must be made explicit and transparent.\footnote{In this
\Xten{} is similar to more modern languages such as ZPL \cite{zpl}.}
At the same time we believe that the place-based type system and
support for generic programming will allow the \Xten{} programmer to
be highly productive; many of the tedious details of
distribution-specific code can be handled in a generic fashion.

We expect the next version of the language to be significantly
informed by experience in implementing and using the language. We
expect it to have constructs to support continuous program
optimization, and allow the programmer to provide guidance on
clustering places to (hardware) nodes. For instance, we may introduce
a notion of hierarchical clustering of places.

\chapter{Lexical structure}


Lexically a program consists of a stream of white space, comments,
identifiers, keywords, literals, separators and operators, all of them
composed of ASCII characters. 

\paragraph{Whitespace}
\index{white space}
% Whitespace \index{whitespace} follows \java{} rules \cite[Chapter 3.6]{jls2}.
ASCII space, horizontal tab (HT), form feed (FF) and line
terminators constitute white space.

\paragraph{Comments}
\index{comment}
% Comments \index{comments} follows \java{} rules
% \cite[Chapter 3.7]{jls2}. 
All text included within the ASCII characters ``\xcd"/*"'' and
``\xcd"*/"'' is
considered a comment and ignored; nested comments are not
allowed.  All text from the ASCII characters
``\xcd"//"'' to the end of line is considered a comment and is ignored.

\paragraph{Identifiers}
\index{identifier}
\index{variable name}

Identifiers consist of a single letter followed by zero or more
letters or digits.
The letters are the ASCII characters \xcd`a` through \xcd`z`, \xcd`A` through
\xcd`Z`, and \xcd`_`.
Digits are defined as the ASCII characters \xcd"0" through \xcd"9".

\paragraph{Keywords}
\index{keywords}
\Xten{} reserves the following keywords:
\begin{xten}
abstract       false          offers         transient      
as             final          operator       true           
assert         finally        package        try            
async          finish         private        var            
ateach         for            property       when           
break          goto           protected      while          
case           if             public         at             
catch          implements     return         atomic         
class          import         self           await          
continue       in             static         clocked        
def            instanceof     struct         here           
default        interface      super          next           
do             native         switch         offer          
else           new            this           resume         
extends        null           throw          type           
\end{xten}
Note that the primitive types are not considered keywords.

\paragraph{Literals}\label{Literals}\index{literal}

Briefly, \XtenCurrVer{} uses fairly standard syntax for its literals:
integers, unsigned integers, floating point numbers, booleans, 
characters, strings, and \xcd"null".  The most exotic points are (1) unsigned
numbers are marked by a \xcd`u` and cannot have a sign; (2) \xcd`true` and
\xcd`false` are the literals for the booleans; and (3) floating point numbers
are \xcd`Double` unless marked with an \xcd`f` for \xcd`Float`. 

Less briefly, we use the following abbreviations: 
\begin{displaymath}
\begin{array}{rcll}
d &=& \mbox{one or more decimal digits}\\
d_8 &=& \mbox{one or more octal digits}\\
d_{16} &=& \mbox{one or more hexadecimal digits, using \xcd`a`-\xcd`f`
for 10-15}\\
i &=& d 
        \mathbin{|} {\tt 0} d_8 
        \mathbin{|} {\tt 0x} d_{16}
        \mathbin{|} {\tt 0X} d_{16}
\\
s &=& \mbox{optional \xcd`+` or \xcd`-`}\\
b &=& d 
          \mathbin{|} d {\tt .}
          \mathbin{|} d {\tt .} d
          \mathbin{|}  {\tt .} d \\
x &=& ({\tt e } \mathbin{|} {\tt E})
         s
         d \\
f &=& b x
\end{array}
\end{displaymath}

\begin{itemize}

\item \xcd`true` and \xcd`false` are the \xcd`Boolean` literals. \index{Boolean!literal}\index{literal!Boolean}

\item \xcd`null` is a literal for the null value.  It has type
      \xcd`Any{self==null}`. \index{null} \index{object!literal}

\item \index{Int!literal}\index{literal!integer}
\xcd`Int` literals have the form {$si$}; \eg, \xcd`123`,
      \xcd`-321` are decimal \xcd`Int`s, \xcd`0123` and \xcd`-0321` are octal
      \xcd`Int`s, and \xcd`0x123`, \xcd`-0X321`,  \xcd`0xBED`, and \xcd`0XEBEC` are
      hexadecimal \xcd`Int`s.  

\item \xcd`Long` literals have the form {$si{\tt l}$} or
      {$si{\tt L}$}. \Eg, \xcd`1234567890L`  and \xcd`0xBAGEL` are \xcd`Long` literals. 

\item \xcd`UInt` literals have the form {$i{\tt u}$} or {$i {\tt U}$}.
      \Eg, \xcd`123u`, \xcd`0123u`, and \xcd`0xBEAU` are \xcd`UInt` literals.

\item \xcd`ULong` literals have the form {$i {\tt ul}$} or {$i {\tt
      lu}$}, or capital versions of those.  For example, 
      \xcd`123ul`, \xcd`0124567012ul`,  \xcd`0xFLU`, \xcd`OXba1eful`, and \xcd`0xDecafC0ffeefUL` are \xcd`ULong`
      literals. 

\item \xcd`Float` literals have the form {$s f {\tt f}$} or  {$s
\index{float!literal}
\index{literal!float}
      f {\tt F}$}.  Note that the floating-point marker letter \xcd`f` is
      required: unmarked floating-point-looking literals are \xcd`Double`. 
      \Eg, \xcd`1f`, \xcd`6.023E+32f`, \xcd`6.626068E-34F` are \xcd`Float`
      literals. 

\item \xcd`Double` literals have the form {$s f$}\footnote{Except that
\index{double!literal}
\index{literal!double}
      literals like \xcd`1` 
      which match both {$i$} and {$f$} are counted as
      integers, not \xcd`Double`; \xcd`Double`s require a decimal
      point, an exponent, or the \xcd`d` marker.
      }, {$s f {\tt
      D}$}, and {$s f {\tt d}$}.  
      \Eg, \xcd`0.0`, \xcd`0e100`, \xcd`1.3D`,  \xcd`229792458d`, and \xcd`314159265e-8`
      are \xcd`Double` literals.

\item 
\index{char!literal}
\index{literal!char}
\xcd`Char` literals have one of the following forms: 
      \begin{itemize}
      \item \xcd`'`{\it c}\xcd`'` where {\em c} is any printing ASCII
            character other than 
            \xcd`\` or \xcd`'`, representing the character {\em c} itself; 
            \eg, \xcd`'!'`;
      \item \xcd`'\b'`, representing backspace;
      \item \xcd`'\t'`, representing tab;
      \item \xcd`'\n'`, representing newline;
      \item \xcd`'\f'`, representing form feed;
      \item \xcd`'\r'`, representing return;
      \item \xcd`'\''`, representing single-quote;
      \item \xcd`'\"'`, representing double-quote;
      \item \xcd`'\\'`, representing backslash;
      \item \xcd`'\`{\em dd}\xcd`'`, where {\em dd} is one or more octal
            digits, representing the one-byte character numbered {\em dd}; it
            is an error if {\em dd}{$>0377$}.      
      \end{itemize}

\item
\index{string!literal} 
\index{literal!string}
\xcd`String` literals consist of a double-quote \xcd`"`, followed by
      zero or more of the contents of a \xcd`Char` literal, followed by
      another double quote.  \Eg, \xcd`"hi!"`, \xcd`""`.

\item There are no literals of type \xcd`Byte`, \xcd`UByte`, \xcd`Short`, or
      \xcd`UShort`.  

\end{itemize}



\paragraph{Separators}
\Xten{} has the following separators and delimiters:
\begin{xten}
( )  { }  [ ]  ;  ,  .
\end{xten}

\paragraph{Operators}
\index{operator}
\Xten{} has the following operator,  type constructor, and miscellaneous symbols.  (\xcd`?` and
\xcd`:` comprise a single ternary operator, but are written separately.)
\begin{xten}
==  !=  <   >   <=  >=
&&  ||  &   |   ^
<<  >>  >>>
+   -   *   /   %
++  --  !   ~
&=  |=  ^=
<<= >>= >>>=
+=  -=  *=  /=  %=
=   ?   :  =>  ->
<:  :>  @   ..
\end{xten}





\chapter{Types}
\label{XtenTypes}\index{types}

{}\Xten{} is a {\em strongly typed} object-oriented language: every
variable and expression has a type that is known at compile-time.
Types limit the values that variables can hold.

{}\Xten{} supports three kinds of runtime entities, {\em objects},
{\em structs}, and {\em functions}. Objects are instances of {\em
  classes} (\Sref{ReferenceClasses}). They may contain zero or
more mutable fields, and a reference to the list of methods defined on them.

An object is represented by some (contiguous) memory chunk on the
heap. Entities (such as variables and fields) contain a {\em
  reference} to this chunk. That is, objects are represented through
an extra level of indirection.  A consequence of this flexibility is
that an entity containing a reference to an object \xcd{o} needs only
one word of memory to represent that reference, regardess of the
number of fields in \xcd{o}. An assignment to this entity simply
overwrites the reference with another reference (thus taking constant
time). Another consequence is that every class type contains the value
\Xcd{null} corresponding to the invalid reference. \Xcd{null} is often
useful as a default value. Further, two objects may be compared for
identity (\Xcd{==}) in constant time by simply comparing references to
the memory used to represent the objects. The default hash code for an
object is based on the value of this reference. A downside of this
flexibility is that the operations of accessing a field and invoking a
method are more expensive than simply reading a register and
invoking a static function.


Structs are instances of {\em struct types} (\Sref{StructClasses}).  A
struct is represented without the extra level of indirection, with a
memory chunk of size $N$ words precisely big enough to store the value
of every field of the struct (modulo alignment), plus whatever padding is needed. Thus structs cannot
be shared. Entities (such as variables and fields) refering to the
struct must allocate $N$ words to directly contain the chunk.  An
assignment to this entity must copy the $N$ words representing the
right hand side into the left hand side. Since there are no references
to structs, \Xcd{null} is not a legal value for a struct
type. Comparison for identity (\Xcd{==}) involves examining $N$
words. Additionally, structs do not have any mutable fields, hence
they can be freely copied. The payoff for these restrictions lies in
that fields can be stored in registers or local variables, and 
and method invocation is implemented by invoking a static function.

Functions, called closures or lambda-expressions in other languages, are
instances of {\em function types} (\Sref{Functions}). 
A function has zero or more {\em formal parameters} (or {\em arguments}) and a
{\em body}, which is 
an expression that can reference the formal parameters and also other
variables in the surrounding block. For instance, \xcd`(x:Int)=>x*y`
is a unary integer function which multiplies its argument by the
variable \xcd`y` from the surrounding block.  Functions may be freely
copied from place to place and may be repeatedly applied. 

These runtime entities are classified by {\em types}. Types are used in
variable declarations, coercions and  explicit conversions, object creation,
array creation, static state and method accessors, and
\xcd"instanceof" and \xcd`as` expressions.


The basic relationship between values and types is the {\em is an
element of} relation.  We also often say ``$e$ has type $T$'' to
mean ``$e$ is an element of type $T$''.  For example, \xcd`1` has type
\xcd`Int` (the type of all integers representible in 32 bits). It also
has type \xcd`Any` (since all entitites have type \xcd`Any`), type
\xcd`Int{self != 0}` (the type of nonzero integers), type
\xcd`Int{self == 1}` (the type of integers which are equal to \xcd`1`, which
contains only one element), and many others. 

The basic relationship between types is {\em subtyping}: \xcd`T <: U`
holds if every instance of \xcd`T` is also an instance of \xcd`U`. Two
important kinds of subtyping are {\em subclassing} and {\em
  strengthening}. Subclassing is a familiar notion from
object-oriented programming. Here we use it to refer to the
relationship between a class and another class it extends, and the
relationship between a class and another interface it implements. For
instance, in a class hierarchy with classes \xcd`Animal` and \xcd`Cat`
such that \xcd`Cat` extends \xcd`Mammal` and \xcd`Mammal` extends
\xcd`Animal`, every instance of \xcd`Cat` is by definition an instance
of \xcd`Animal` (and \xcd`Mammal`). We say that \xcd`Cat` is a
subclass of \xcd`Animal`, or \xcd`Cat <: Animal` by subclassing. If
\xcd`Animal` implements \xcd`Thing`, then \xcd`Cat` also implements
\xcd`Thing`, and we say \xcd`Cat <: Thing` by subclassing.
Strengthening is an equally familiar notion from logic.  The instances
of \xcd`Int{self == 1}` are all elements of \xcd`Int{self != 0}` as well,
because \xcd`self == 1` logically implies \xcd`self != 0`; so 
\xcd`Int{self  == 1} <: Int{self !=0}` by strengthening.  X10 uses both notions
of subtyping.  See \Sref{DepType:Equivalence} for the full definition
of subtyping in X10.

\subsection{Type System}
\index{type system}
The types in X10 are as follows.  

These are the {\em elementary} types. Other
syntactic forms for types exist, but they are simply abbreviations for types
in the following system.  For example, \xcd`Array[Int](1)` is the type of
one-dimensional integer-valued arrays; it is an abbreviation for
\xcd`Array[Int]{rank==1}`.\\

% remove \refstepcounter{equation}
% snag the argument of \label{X}
% change the (\arabic{equation}) into (\ref{X})

%##(Type FunctionType ConstrainedType
\begin{bbgrammar}
%(FROM #(prod:Type)#)
                Type \: FunctionType & (\ref{prod:Type}) \\
                    \| ConstrainedType \\
%(FROM #(prod:FunctionType)#)
        FunctionType \: TypeParams\opt \xcd"(" FormalParamList\opt \xcd")" WhereClause\opt Offers\opt \xcd"=>" Type & (\ref{prod:FunctionType}) \\
%(FROM #(prod:ConstrainedType)#)
     ConstrainedType \: NamedType & (\ref{prod:ConstrainedType}) \\
                    \| AnnotatedType \\
                    \| \xcd"(" Type \xcd")" \\
\end{bbgrammar}
%##)


Types may be given by name. 
For example, 
%~~type~~`~~`~~ ~~ ^^^ Types10
\xcd`Int`
is the type of 32-bit integers.
Given a class declaration 
%~~gen ^^^ Types20
%package Types.Core.TypeName; 
%~~vis
\begin{xten}
class Triple { /* ... */ }
\end{xten}
%~~siv
%
%~~neg
the identifier \xcd`Triple` may be used as a type.

The type {\em TypeName \xcd`[` Types{$^?$} \xcd`]`} is an instance of
a {\em generic} (or {\em parameterized}) type. 
 For example,
\xcd`Array[Int]` is the type of arrays of integers. 
\xcd`HashMap[String,Int]` is the type of hash maps from strings to
integers.

The type {\em Type \xcd`{` Constraint \xcd`}`} refers to a constrained type.
{\em Constraint} is a Boolean expression -- written in a {\em very} limited
subset of X10 -- describing the acceptable values of the constrained type.
%~~stmt~~`~~`~~ ~~ ^^^ Types30
For example, \xcd`var n : Int{self != 0};` guarantees that \xcd`n` is always a
non-zero integer. 
%~~stmt~~`~~`~~ ~~class Triple{} ^^^ Types40
Similarly, \xcd`var x : Triple{x != null};` defines a \xcd`Triple`-valued
variable \xcd`x` whose value is never null.

The qualified type {\em Type \xcd`.` Type} refers to an instance of a {\em
nested} type; that is, a class or struct defined inside of another class or
struct, and holding an implicit reference to the outer.  For example, given
the type declaration 
%~~gen ^^^ Types50
% package Types.Core.Hardcore.Qualified;
%~~vis
\begin{xten}
class Outer {
  class Inner { /* ... */ }
}
\end{xten}
%~~siv
%
%~~neg
then 
%~~exp~~`~~`~~ ~~ NOTEST class Outer {class Inner { /* ... */ }} ^^^ Types60
\xcd`(new Outer()).new Inner()` creates a value of type 
%~~type~~`~~`~~ ~~class Outer {class Inner { /* ... */ }} ^^^ Types70
\xcd`Outer.Inner`.

Type variables, {\em TypeVar}, refer to types that are parameters.  For
example, the following class defines a cell in a linked list.  
%~~gen ^^^ Types80
% package Types.Core.Bore.Lore;
%~~vis
\begin{xten}
class LinkedList[X] {
  val head : X;
  val tail : LinkedList[X];
  def this(head:X, tail:LinkedList[X]) {
     this.head = head; this.tail = tail;
  }
}
\end{xten}
%~~siv
%
%~~neg
It doesn't
matter what type the cell is, but it has to have {\em some} type.
\xcd`LinkedList[Int]` is a linked list of integers;
\xcd`LinkedList[LinkedList[String]]` a list of lists of strings.
Note that \xcd`LinkedList` is {\em not} a type -- it is missing a type parameter.



The function type 
{\em \xcd`(` Formals{$^?$} \xcd`) =>`  Type} 
refers to functions taking the
listed formal parameters and returning a result of {\em Type}.  In
\XtenCurrVer, function types may not be generic.
The closely-related void function type 
{\em \xcd`(` Formals{$^?$} \xcd`) =>`  \xcd`void`}  takes the listed
parameters and returns no value.
For example, 
%~~type~~`~~`~~ ~~ ^^^ Types90
\xcd`(x:Int) => Int{self != x}` 
is the type of integer-valued functions which have no fixed points.  
An example of such a function is \xcd`(x:Int) => x+1`.





\section{Classes, Structs,  and interfaces}
\label{ReferenceTypes}

\subsection{Class types}

\index{type!class}
\index{class}
\index{class declaration}
\index{declaration!class declaration}
\index{declaration!reference class declaration}

A {\em class declaration} (\Sref{XtenClasses}) declares a {\em class type},
giving its name, behavior, data, and relationships to other classes and
interfaces. 

\begin{ex}

The \xcd`Position` class below could describe the position of a slider
control

%~~gen ^^^ Types100
% package Types.By.Cripes.Classes;
%~~vis
\begin{xten}
class Position {
  private var x : Int = 0;
  public def move(dx:Int) { x += dx; }
  public def pos() : Int = x;
}
\end{xten}
%~~siv
%
%~~neg
\end{ex}
Class instances, also called objects, are created by constructor calls: 
\xcd`new Position()`. Class
instances have fields and methods, type members, and value properties bound at
construction time. In addition, classes have static members: static \xcd`val` fields,
methods, type definitions, and member classes and member interfaces.

Classes may be {\em generic}, \ie, defined with one or more type
parameters (\Sref{TypeParameters}).  

%~~gen ^^^ Types110
%~~vis
\begin{xten}
class Cell[T] {
  var contents : T;
  public def this(t:T) { contents = t;  }
  public def putIn(t:T) { contents = t; }
  public def get() = contents;
  }
\end{xten}
%~~siv
%~~neg


%TODO: Yoav: ``This reasoning is no longer true in the new object model''
%% Why not?
\Xten{} does not permit mutable static state. A fundamental principle of the
X10 model of computation is that all mutable state be local to some place
(\Sref{XtenPlaces}), and, as static variables are
globally available, they
cannot be mutable. When mutable global state is necessary, programmers should
use singleton classes, putting the state in an object and using place-shifting
commands (\Sref{AtStatement}) and atomicity (\Sref{AtomicBlocks}) as necessary
to mutate it safely.

\index{\Xcd{Object}}
\index{\Xcd{x10.lang.Object}}

Classes are structured in a single-inheritance hierarchy. All classes extend
the class \xcd"x10.lang.Object", directly or indirectly. Each class other than
\xcd`Object` extends a single parent class.  \xcd`Object` provides no behaviors
of its own, beyond those required by \xcd`Any`.

\index{class!reference class}
\index{reference class type}
\index{\Xcd{Object}}
\index{\Xcd{x10.lang.Object}}


\index{null}


The null value, represented by the literal
\xcd"null", is a value of every class type \xcd`C`. The type whose values are
all instances of \xcd`C` but not 
\xcd`null` can be defined as \xcd`C{self != null}`.

\subsection{Struct Types}

A {\em struct declaration} \Sref{XtenStructs} introduces a {\em struct type}
containing all instances of the struct.  The \xcd`Coords` struct below gives
an immutable position in 3-space: 
%~~gen ^^^ Types120
% package Types.Structs.Coords;
%~~vis
\begin{xten}
struct Position {
  public val x:Double, y:Double, z:Double; 
  def this(x:Double, y:Double, z:Double) {
     this.x = x; this.y = y; this.z = z;
  }
}
\end{xten}
%~~siv
%
%~~neg

Structs have many capabilities of classes: they can have methods, implement
interfaces, and be generic. However, they have certain restrictions; for
example, they cannot contain mutable (\xcd`val`) fields, or inherit from
superclasses. There is no \xcd`null` value for structs. Due to these
restrictions, structs may be implemented more efficiently than objects.


\subsection{Interface types}
\label{InterfaceTypes}

\index{type!interface}
\index{interface}
\index{interface declaration}
\index{declaration!interface declaration}

An {\em interface declaration} (\Sref{XtenInterfaces}) defines an {\em
interface type}, specifying a set of methods 
%type members, 
and properties which must be provided by any class declared to implement the
interface. 


Interfaces can also have static members: static fields, type
definitions, and member classes, structs and interfaces.  However,
interfaces cannot specify that implementing classes must provide
static members or constructors.

\begin{ex}
In the following interface, \xcd`PI` is a static field, 
\xcd`Vec` a static type definition, 
\xcd`Pair` a static member class.
It can't insist that implementations provide a static method 
like \xcd`meth`, or a nullary constructor.
%~~gen ^^^ Types2y3i
% NOTEST
% package Types2y3i;
%~~vis
\begin{xten}
interface Stat {
  static val PI = 3.14159; 
  static type R = Double;
  static class Pair(x:R, y:R) {}
  // ERROR: static def meth():Int;
  // ERROR: static def this();
}
class Example {
  static def example() {
     val p : Stat.Pair = new Stat.Pair(Stat.PI, Stat.PI);
  }
}
\end{xten}
%~~siv
%
%~~neg

\end{ex}

An interface may extend multiple interfaces.  
%~~gen ^^^ Types130
%package Types.For.Snipes.Interfaces;
%~~vis
\begin{xten}
interface Named {
  def name():String;
}
interface Mobile {
  def move(howFar:Int):void;
}
interface Person extends Named, Mobile {}
interface NamedPoint extends Named, Mobile {} 
\end{xten}
%~~siv
%
%~~neg


Classes and structs may be declared to implement multiple interfaces. Semantically, the
interface type is the set of all objects that are instances of classes
or structs that
implement the interface. A class or struct implements an interface if it is declared to
and if it concretely or abstractly implements all the methods and properties
defined in the interface. For example, \xcd`KimThePoint` implements
\xcd`Person`, and hence \xcd`Named` and \xcd`Mobile`. It would be a static
error if \xcd`KimThePoint` had no \xcd`name` method, unless \xcd`KimThePoint` were also
declared \xcd`abstract`.

%~~gen ^^^ Types140
%interface Named {
%   def name():String;
% }
% interface Mobile {
%   def move(howFar:Int):void;
% }
% interface Person extends Named, Mobile {}
% interface NamedPoint extends Named, Mobile{} 
%~~vis
\begin{xten}
class KimThePoint implements Person {
   var pos : Int = 0;
   public def name() = "Kim (" + pos + ")";
   public def move(dPos:Int) { pos += dPos; }
}
\end{xten}
%~~siv
%
%~~neg


\subsection{Properties}
\index{property}
\label{properties}

Classes, interfaces, and structs may have {\em properties}, specified in
parentheses after the type name. Properties are much like public \xcd`val`
instance fields. They have certain restrictions on their use, however, which
allows the compiler to understand them much better than other public \xcd`val`
fields. In particular, they can be used in types.  \Eg, the number of elements
in an array is a property of the array, and an X10 program can specify that
two arrays have the same number of elements.

\begin{ex}
The
following code declares a class named \xcd"Coords" with properties
\xcd"x" and \xcd"y" and a \xcd"move" method. The properties are bound
using the \xcd"property" statement in the constructor.

%~~gen ^^^ Types150
%package not.x10.lang;
%~~vis
\begin{xten}
class Coords(x: Int, y: Int) { 
  def this(x: Int, y: Int) :
    Coords{self.x==x, self.y==y} = { 
    property(x, y); 
  } 

  def move(dx: Int, dy: Int) = new Coords(x+dx, y+dy); 
}
\end{xten}
%~~siv
%~~neg
\end{ex}
Properties, unlike other public \xcd`val` fields, can be used  
at compile time in {constraints}. This allows us
to specify subtypes based on properties, by appending a boolean expression to
the type. For example, the type \xcd"Coords{x==0}" is the set of all points
whose \xcd"x" property is \xcd"0".  Details of this substantial topic are
found in \Sref{ConstrainedTypes}.



\section{Type Parameters and Generic Types}
\label{TypeParameters}

\index{type!parameter}
\index{method!parametrized}
\index{constructor!parametrized}
\index{closure!parametrized}
\label{Generics}
\index{type!generic}

A class, interface, method, or type definition  may have type
parameters.  Type parameters can be used as types, and will be bound to types
on instantiation.  For example, a generic stack class may be defined as 
\xcd`Stack[T]{...}`.  Stacks can hold values of any type; \eg, 
%~~type~~`~~`~~ ~~class Stack[T]{} ^^^ Types160
\xcd`Stack[Int]` is a stack of integers, and 
%~~type~~`~~`~~ ~~class Stack[T]{} ^^^ Types170
\xcd`Stack[Point {self!=null}]` is a stack of non-null \xcd`Point`s.
Generics {\em must} be instantiated when they are used: \xcd`Stack`, by
itself, is not a valid type.
Type parameters may be constrained by a guard on the declaration
(\Sref{TypeDefGuard},
\Sref{MethodGuard},\Sref{ClosureGuard}).

\index{type!concrete}
\index{concrete type}
A {\em generic class} (or struct, interface, or type definition) 
is a class (resp. struct, interface, or type definition) 
declared with $k \geq 1$ type parameters. 
A generic class (or struct, interface, or type definition) 
can be used to form a type by supplying $k$ types as type arguments within
\xcd`[` \ldots \xcd`]`.
%%When instantiated,
%%with concrete (\viz, non-generic) types for its parameters, 
%%a generic type becomes a concrete type and can be
%%used like any other type. 
For example,
\xcd`Stack` is a generic class, 
%~~type~~`~~`~~ ~~class Stack[T]{} ^^^ Types180
\xcd`Stack[Int]` is a type, and can be used as one: 
%~~stmt~~`~~`~~ ~~class Stack[T]{} ^^^ Types190
\xcd`var stack : Stack[Int];`

A \xcd`Cell[T]` is a generic object, capable of holding a value of type
\xcd`T`.  For example, a \xcd`Cell[Int]` can hold an \xcd`Int`, and a
\xcd`Cell[Cell[Int{self!=0}]]` can hold a \xcd`Cell` which in turn can
only hold non-zero numbers. 
%% vj: Dont know what this saying: bound immutably... but mutable?
%% \xcd`Cell`s are actually useful in situations
%%where values must be bound immutably for one reason, but need to be mutable.
%~~gen ^^^ Types200
% package ch4;
%~~vis
\begin{xten}
class Cell[T] {
    var x: T;
    def this(x: T) { this.x = x; }
    def get(): T = x;
    def set(x: T) = { this.x = x; }
}
\end{xten}
%~~siv
%~~neg


\xcd"Cell[Int]" is the type of \xcd`Int`-holding cells.  
The \xcd"get" method on a \xcd`Cell[Int]` returns an \xcd"Int"; the
\xcd"set" method takes an \xcd"Int" as argument.  Note that
\xcd"Cell" alone is not a legal type because the parameter is
not bound.


A class (whether generic or not) may have generic methods.
Below,
\xcd`NonGeneric` has a generic method 
\xcd`first[T](x:List[T])`. An invocation of such a method may supply
the type parameters explicitly (\eg, \xcd`first[Int](z)`).
 In certain cases (\eg, \xcd`first(z)`)
type parameters may
be omitted and are inferred by the compiler (\Sref{TypeInference}).

%~~gen ^^^ Types210
% package Types.For.Cripes.Sake.Generic.Methods;
% import x10.util.*;
%~~vis
\begin{xten}
class NonGeneric {
  static def first[T](x:List[T]):T = x(0);
  def m(z:List[Int]) {
    val f = first[Int](z);
    val g = first(z);
    return f == g;
  }
}
\end{xten}
%~~siv
%
%~~neg
\limitation{ \XtenCurrVer{}'s C++ back end requires generic methods to be
static or final; the Java back end can accomodate generic instance methods as well. }

Unlike other kinds of variables, type parameters may {\em not} be shadowed.  
If name \xcd`X` is in scope as a type, \xcd`X` may not be rebound as a type
variable.  
For example, neither \xcd`class B` nor \xcd`class C[B]` are allowed in the
following code, because they both shadow the type variable \xcd`B`.
%~~gen ^^^ TypesNoShadow
% package TypesNoShadow;
%~~vis
\begin{xten}
class A[B] {
  // ILLEGAL: class B{} 
  // ILLEGAL: class C[B]{} 
}
\end{xten}
%~~siv
%
%~~neg


\subsection{Use of Generics}

An unconstrained type variable \Xcd{X} can be instantiated any type. Within a
generic struct or class, all the operations of \Xcd{Any} are available on a
variable of type unconstrained \Xcd{X}. Additionally, variables of type
\Xcd{X} may be used with \Xcd{==, !=}, in \Xcd{instanceof}, and casts.  

If a type variable is constrained, the operations implied by its constraint
are available as well.

\begin{ex}
The interface \xcd`Named` describes entities which know their own name.  The
class \xcd`NameMap[T]` is a specialized map which stores and retrieves
\xcd`Named` entities by name.  The call \xcd`t.name()` in \xcd`put()` is only
valid because the constraint \xcd`{T <: Named}` implies that \xcd`T` is a
subtype of \xcd`Named`, and hence provides all the operations of \xcd`Named`. 
%~~gen ^^^ Types6e6x
% package Types6e6x;
% import x10.util.*;
%~~vis
\begin{xten}
interface Named { def name():String; }
class NameMap[T]{T <: Named} {
   val m = new HashMap[String, T]();
   def put(t:T) { m.put(t.name(), t); }
   def get(s:String):T = m.getOrThrow(s);
}
\end{xten}
%~~siv
%
%~~neg


\end{ex}


%%NO-VARIANCE%% \subsection{Variance of Type Parameters}
%%NO-VARIANCE%% \index{covariant}
%%NO-VARIANCE%% \index{contravariant}
%%NO-VARIANCE%% \index{invariant}
%%NO-VARIANCE%% \index{type parameter!covariant}
%%NO-VARIANCE%% \index{type parameter!contravariant}
%%NO-VARIANCE%% \index{type parameter!invariant}
%%NO-VARIANCE%% 
%%NO-VARIANCE%% % Uncomment this when the language implementation properly supports variance.
%%NO-VARIANCE%% %%%\subsection{Variance of Type Parameters}
%%\index{covariant}
%%\index{contravariant}
%%\index{invariant}
%%\index{type parameter!covariant}
%%\index{type parameter!contravariant}
%%\index{type parameter!invariant}

%TODO - examples courtesy of Nate
% 
% class OutputStream[-A] {
%    def write(a: A) = /* implementation left as an exercise for the reader */
% }
% 
% Also:
% 
% interface Comparator[-A] {
%    def compare(A): Int;
% }
% 
% and:
% 
% class HashMap[-K,+V] { ... }
% 
% 

Type parameters of classes (though not of methods) can be {\em variant}.

Consider classes \xcd`Person :> Child`.  Every child is a person, but there
are people who are not children.  What is the relationship between
\xcd`Cell[Person]` and \xcd`Cell[Child]`?  

\subsubsection{Why Variance Is Necessary}

In this case, \xcd`Cell[Person]` and \xcd`Cell[Child]` should be unrelated.  
If we had \xcd`Cell[Person] :> Cell[Child]`, the following code would let us
assign a \xcd`old` (a \xcd`Person` but not a \xcd`Child`) to a
variable \xcd`young` of type \xcd`Child`, thereby breaking the type system: 
\begin{xten}
// INCORRECTLY assuming Cell[Person] :> Cell[Child]
val cc : Cell[Child] = new Cell[Child]();
val cp : Cell[Person] = cc; // legal upcast
cp.set(old);       // legal since old : Person
val young : Child = cc.get(); 
\end{xten}

Similarly, if \xcd`Cell[Person] <: Cell[Child]`: 
\begin{xten}
// INCORRECTLY assuming Cell[Person] <: Cell[Child]
val cp : Cell[Person] = new Cell[Person];
val cc : Cell[Child] = cp; // legal upcast
val cp.set(old); 
val young : Child = cc.get();
\end{xten}

So, there cannot be a subtyping relationship in either direction between the
two. And indeed, neither of these programs passes the X10 typechecker.


\subsubsection{Legitimate Variance}

The \xcd`Cell[Person]`-vs-\xcd`Cell[Child]` problems occur because it is
possible to both store and retrieve values from the same object. However,
entities with only one of the two capabilities {\em can} sensibly have some
subtyping relations. Furthermore, both sorts of entity are useful. An entity
which can store values but not retrieve them can nonetheless summarize them.
An object which can retrieve values but not store values can be constructed
with an initial value, providing a read-only cell.

So, X10 provides {\em variance} to support these options.  Type parameters
may be defined in one of three forms.  
\begin{enumerate}
\item {\em invariant}: Given a definition \xcd`class C[T]{...}`, \xcd`C[Person]` and
      \xcd`C[Child]` are unrelated classes; neither is a subclass of the
      other.
\item {\em covariant}: Given a definition \xcd`class C[+T]{...}` (the \xcd`+` indicates
      covariance), \xcd`C[Person] :> C[Child]`.  This is appropriate when
      \xcd`C` allows retrieving values but not setting them.
\item {\em contravariant}: Given a definition \xcd`class C[-T]{...}` (the \xcd`-` indicates
      contravariance), \xcd`C[Person] <: C[Child]`.  This is appropriate when
      \xcd`C` allows storing values but not retrieving them.
\end{enumerate}


The \xcd"T" parameter of \xcd"Cell" above is
invariant.  

A typical example of covariance is \xcd`Get`.  As the \xcd`example()` method
shows, a \xcd`Get[T]` must be constructed with its value, and will return that
value whenever desired.  \xcd`Get[T]` is only moderately useful as a class; it
is more useful as an interface for providing a limited (read) access to a more
powerful data structure.
%~~gen ^^^ Variance10
% package Variance_gone;
%~~vis
\begin{xten}
class Get[+T] {
  val x: T;
  def this(x: T) { this.x = x; }
  def get(): T = x;
  static def example() {
     val g : Get[Int] = new Get[Int](31);
     val n : Int = g.get();
     x10.io.Console.OUT.print("It's " + n);
     x10.io.Console.OUT.print("It's still " + g.get());
  }
}
\end{xten}
%~~siv
%~~neg

There are few if any {\em classes} with contravariant type parameters.
(Covariant type parameters are only moderately more common.)  
However, it is frequently useful to have {\em interfaces} with contravariant
type parameters.  For example: 
%~~gen ^^^ Variance20
% package Types_contravariance_a;
%~~vis
\begin{xten}
interface OutputStream[-T] {
   def write(T) : void;
}
interface ComparableTo[-T] {
   def compare(T) : Int;
}
\end{xten}
%~~siv
%
%~~neg
Clearly, \xcd`Int <: Any`. 
An \xcd`OutputStream[Int]` is only capable of writing \xcd`Int`s.  
An \xcd`OutputStream[Any]` is capable of writing anything.  In particular, it
can write \xcd`Int`s. Thus, an \xcd`OutputStream[Any]` can be used in place of
an \xcd`OutputStream[Int]`, and hence \xcd`OutputStream[Any] <: OutputStream[Int]`.
Similarly, a \xcd`ComparableTo[Int]` can be compared to an integer. A
\xcd`ComparableTo[Any]` can be compared to anything, and, in particular, to an
integer.  Thus \xcd`ComparableTo[Any] <: ComparableTo[Int]`.
So, both of these interfaces are contravariant.


Given types \xcd"S" and \xcd"T": 
\begin{itemize}
\item
If the parameter of \xcd"Get" is covariant, then
\xcd"Get[S]" is a subtype of \xcd"Get[T]" if
\xcd"S" is a {\em subtype} of \xcd"T".

\item
If the parameter of \xcd"Set" is contravariant, then
\xcd"OutputStream[S]" is a subtype of \xcd"OutputStream[T]" if
\xcd"S" is a {\em supertype} of \xcd"T".

\item
If the parameter of \xcd"Cell" is invariant, then
\xcd"Cell[S]" is a subtype of \xcd"Cell[T]" if
\xcd"S" is a {\em equal} to \xcd"T".
\end{itemize}


In order to make types marked as covariant and contravariant semantically
sound, X10 performs extra checks.  
A covariant type parameter is permitted to appear only in covariant type positions,
and a contravariant type parameter in contravariant positions. 
\begin{itemize}
\item The return type of a method is a covariant position.
\item The argument types of a method are contravariant positions.
\item Whether a type argument position of a generic class, interface or struct type \Xcd{C}
is covariant or contravariant is determined by the \Xcd{+} or \Xcd{-} annotation
at that position in the declaration of \Xcd{C}.
\end{itemize}


There are similar restrictions on use of covariant and contravariant variables.

\limitationx{} Full checking of covariance and contravariance is not yet
implemented.  Covariant and contravariant classes and structs should be used
with great caution.

%TODO: Yoav says ``There are other rules, not implemented or specified,
%involving fields, inheritance, etc.  There are several JIRAs on it. No idea
%what is the work around -- maybe just say ``limitation''?'''

%%NO-VARIANCE%% 
%%NO-VARIANCE%% Class, struct and interface definitions are permitted to specify a {\em
%%NO-VARIANCE%%   variance} 
%%NO-VARIANCE%% for each type parameter. 
%%NO-VARIANCE%% There are three variance specifications: 
%%NO-VARIANCE%% \xcd`+` indicates {\em co-variance},  \xcd`-` indicates {\em
%%NO-VARIANCE%%   contravariance} and the absence of  \xcd`+` and 
%%NO-VARIANCE%%  \xcd`-` indicates {\em invariance}. For a class (or struct or
%%NO-VARIANCE%%  interface) \xcd`S` specifying that a particular parameter position
%%NO-VARIANCE%%  (say, \xcd`i`) is covariant means that 
%%NO-VARIANCE%% if \xcd`Si <: Ti` then
%%NO-VARIANCE%% \xcdmath"S[S1,$\ldots$,Sn] <: S[S1,$\ldots$, Si-1,Ti,Si+1,$\ldots$ Sn]".
%%NO-VARIANCE%% Similarly, saying that position \xcd`i` is is contravariant means
%%NO-VARIANCE%% that 
%%NO-VARIANCE%% if \xcd`Si <: Ti` then
%%NO-VARIANCE%% \xcdmath"S[S1,$\ldots$, Si-1,Ti,Si+1,$\ldots$ Sn] <: S[S1,$\ldots$,Sn]". If the
%%NO-VARIANCE%% position is invariant, then no such relationship is asserted between
%%NO-VARIANCE%% \xcd`Si <: Ti` 
%%NO-VARIANCE%% and
%%NO-VARIANCE%% \xcdmath"S[S1,$\ldots$, Si-1,Ti,Si+1,$\ldots$ Sn]". The compiler must perform
%%NO-VARIANCE%% several checks on the body of the class (or struct or interface) to
%%NO-VARIANCE%% establish that type parameters with a variance are used in a manner
%%NO-VARIANCE%% that is consistent with their semantics.
%%NO-VARIANCE%% 
%%NO-VARIANCE%% \limitation{} The implementation of variance specifications  suffers from
%%NO-VARIANCE%% various limitations in \XtenCurrVer. Users are strongly encouraged not
%%NO-VARIANCE%% to use variance. (Some classes, structs, and interfaces in the standard
%%NO-VARIANCE%% libraries use variance specifications in a careful way that
%%NO-VARIANCE%% circumvents these limitations.)
%%NO-VARIANCE%% 

\section{Type definitions}
\label{TypeDefs}

\index{type!definitions}
\index{declaration!type}
A type definition can be thought of as a type-valued function,
mapping type parameters and value parameters to a concrete type.

%##(TypeDefDecl TypeParams FormalParams WhereClause
\begin{bbgrammar}
%(FROM #(prod:TypeDefDecl)#)
         TypeDefDecl \: Mods\opt \xcd"type" Id TypeParams\opt FormalParams\opt WhereClause\opt \xcd"=" Type \xcd";" & (\ref{prod:TypeDefDecl}) \\
%(FROM #(prod:TypeParams)#)
          TypeParams \: \xcd"[" TypeParamList \xcd"]" & (\ref{prod:TypeParams}) \\
%(FROM #(prod:FormalParams)#)
        FormalParams \: \xcd"(" FormalParamList\opt \xcd")" & (\ref{prod:FormalParams}) \\
%(FROM #(prod:WhereClause)#)
         WhereClause \: DepParams & (\ref{prod:WhereClause}) \\
\end{bbgrammar}
%##)

\noindent 
During type-checking the compiler replaces the use of such a defined
type with its body, substituting the actual type and value parameters
in the call for the formals. This replacement is performed recursively
until the type no longer contains a defined type or a predetermined
compiler limit is reached (in which case the compiler declares an
error). Thus, recursive type definitions are not permitted.

Thus type definitions are considered applicative and not generative --
they do not define new types, only aliases for existing types.

\label{TypeDefGuard}
Type definitions may have guards: an invocation of a type definition
is illegal unless the guard is satisified when formal types and values
are replaced by the actual parameters.

Type definitions may be overloaded: two type definitions with
the same name are permitted provided that they have a different number
of type parameters or different number or type of value parameters.

Type definitions must appear as static members or in a block statement.

\paragraph{Use of type definitions in constructor invocations}
If a type definition has no type parameters and no value
parameters and is an alias for a class type, a \xcd"new"
expression may be used to create an instance of the class using
the type definition's name.
Given the following type definition:
%TODO: Yoav says ``I just opened a jira on it: [1918].  I don't think you
% should be able to have {c} on the typedef A if you want to use it in a 'new'
% expression. If we do allow it, then we should allow: new
% Array[Int]{rank==1}(0..2) and new Array[Int](1)(0..2).
\begin{xtenmath}
type A = C[T$_1$, $\dots$, T$_k$]{c};
\end{xtenmath}
where 
\xcdmath"C[T$_1$, $\dots$, T$_k$]" is a
class type, a constructor of \xcdmath"C" may be invoked with
\xcdmath"new A(e$_1$, $\dots$, e$_n$)", if the
invocation
\xcdmath"new C[T$_1$, $\dots$, T$_k$](e$_1$, $\dots$, e$_n$)" is
legal and if the constructor return type is a subtype of
\xcd"A".

\paragraph{Automatically imported type definitions}
\index{import,type definitions}
\label{X10LangUnderscore}

The collection of type definitions in
\xcdmath"x10.lang._" is automatically imported in every compilation unit.


\subsection{Motivation and use}
The primary purpose of type definitions is to provide a succinct,
meaningful name for complex types
and combinations of types. 
With value arguments, type arguments, and constraints, the syntax for \Xten{}
types can often be verbose. 
For example, a non-null list of non-null strings is \\
%~~type~~`~~`~~ ~~import x10.util.*; ^^^ Types220
\xcd`List[String{self!=null}]{self!=null}`.

We could name that type: 
%~~gen ^^^ Types230
% package TypeDefs.glip.first;
% import x10.util.*;
% class LnSn {
% 
%~~vis
\begin{xten}
static type LnSn = List[String{self!=null}]{self!=null};
\end{xten}
%~~siv
%}
%~~neg
Or, we could abstract it somewhat, defining a type constructor
\xcd`Nonnull[T]` for the type of \xcd`T`'s which are not null:
%~~gen ^^^ Types240
% package TypeDefs.glip.second;
% import x10.util.*;
% 
%~~vis
\begin{xten}
class Example {
  static type Nonnull[T]{T <: Object}  = T{self!=null};
  var example : Nonnull[Example] = new Example();
}
\end{xten}
%~~siv
%
%~~neg

Type definitions can also refer to values, in particular, inside 
constraints.  The type of \xcd`n`-element \xcd`Array[Int](1)`s  is 
%~~type~~`~~`~~n:Int ~~ ^^^ Types250
\xcd`Array[Int]{self.rank==1 && self.size == n}`
but it is often convenient to give a shorter name: 
%~~gen ^^^ Types260
% package TypeDefs.glip.third;
% class Xmpl {
% def example() {
%~~vis
\begin{xten}
type Vec(n:Int) = Array[Int]{self.rank==1 && self.size == n}; 
var example : Vec(78); 
\end{xten}
%~~siv
%}}
%~~neg

%
The following examples are legal type definitions, given \xcd`import x10.util.*`:
%~~gen ^^^ Types270
% package Types.TypeDef.Examples;
% import x10.util.*;
%~~vis
\begin{xten}
class TypeExamples {
  static type StringSet = Set[String];
  static type MapToList[K,V] = Map[K,List[V]];
  static type Int(x: Int) = Int{self==x};
  static type Dist(r: Int) = Dist{self.rank==r};
  static type Dist(r: Region) = Dist{self.region==r};
  static type Redund(n:Int, r:Region){r.rank==n} 
      = Dist{rank==n && region==r};
}
\end{xten}
%~~siv
% 
%~~neg

The following code illustrates that type definitions are applicative rather
than generative.  \xcd`B` and \xcd`C` are both aliases for \xcd`String`,
rather than new types, and so are interchangeable with each other and with
\xcd`String`. Similarly, \xcd`A` and \xcd`Int` are equivalent.
%~~gen ^^^ Types280
% package Types.TypeDef.Example.NonGenerative;
% import x10.util.*;
% class TypeDefNonGenerative {
%~~vis
\begin{xten}
def someTypeDefs () {
  type A = Int;
  type B = String;
  type C = String;
  a: A = 3;
  b: B = new C("Hi");
  c: C = b + ", Mom!";
  }
\end{xten}
%~~siv
% }
%~~neg
% An instance of a defined type with no type parameters and no
% value parameters may 


%%MEMBERSHIP%% All type definitions are members of their enclosing package or
%%MEMBERSHIP%% class.  A compilation unit may have one or more type definitions
%%MEMBERSHIP%% or class or interface declarations with the same name, as long
%%MEMBERSHIP%% as the definitions have distinct parameters according to the
%%MEMBERSHIP%% method overloading rules (\Sref{MethodOverload}).


\section{Constrained types}
\label{ConstrainedTypes}
\label{DepType:DepType}
\label{DepTypes}

\index{dependent type}
\index{type!dependent}
\index{constrained type}
\index{generic type}
\index{type!constrained}
\index{type!generic}


Basic types, like \xcd`Int` and \xcd`List[String]`, provide useful
descriptions of data.  

However, one frequently wants to say more.  One might want to know
that a \xcd`String` variable is not \xcd`null`, or that a matrix is
square, or that one matrix has the same number of columns as another
has rows (so they can be multiplied).  In the multicore setting, one
might wish to know that two values are located at the same processor,
or that one is located at the same place as the current computation.

In most languages, there is simply no way to say these things
statically.  Programmers must made do with comments, \xcd`assert`
statements, and dynamic tests.  X10 programs can do better, with {\em
  constraints} on types, and guards on class, method and type
definitions,

A constraint is a boolean expression \xcd`e` attached to a basic type \xcd`T`,
written \xcd`T{e}`.  (Only a limited selection of boolean expressions is
available.)  The values of type \xcd`T{e}` are the values of \xcd`T` for which
\xcd`e` is true.  For example: 

\begin{itemize}
%~~type~~`~~`~~ ~~ ^^^ Types290
\item \xcd`String{self != null}` is the type of non-null strings.  \xcd`self`
      is a special variable available only in constraints; it refers to the
      datum being constrained, and its type is the type to which the
      constraint is attached.
\item If \xcd`Matrix` has properties \xcd`rows` and \xcd`cols`, 
%~~type~~`~~`~~ ~~class Matrix(rows:Int,cols:Int){} ^^^ Types300
      \xcd`Matrix{self.rows == self.cols}` is the type of square matrices.
\item One way to say that \xcd`a` has the same number of columns that \xcd`b`
      has rows (so that \xcd`a*b` is a valid matrix product), one could say: 
%~~gen ^^^ Types310
% package Types.cripes.whered.you.get.those.gripes;
% class Matrix(rows:Int, cols:Int){
% public static def someMatrix(): Matrix = null;
% public static def example(){
%~~vis
\begin{xten}
  val a : Matrix = someMatrix() ;
  var b : Matrix{b.rows == a.cols} ;
\end{xten}
%~~siv
%}}
%~~neg



\index{self}When constraining a value of type \xcd`T`, \xcd`self` refers to the object of
type \xcd`T` which is being constrained.  For example, \xcd`Int{self == 4}` is
the type of \xcd`Int`s which are equal to 4 -- the best possible description
of \xcd`4`, and a very difficult type to express without using \xcd`self`.  
\end{itemize}

\xcd"T{e}" is a {\em dependent type}, that is, a type dependent on values. The
type \xcd"T" is called the {\em base type} and \xcd"e" is called the {\em
  constraint}. If the constraint is omitted, it is \xcd`true`---that is, the
  base type is unconstrained.

Constraints may refer to immutable values in the local environment: 
%~~gen ^^^ Types320
% class ConstraintsMayReferToValues {
% def thoseValues() {
%~~vis
\begin{xten}
     val n = 1;
     var p : Point{rank == n};
\end{xten}
%~~siv
%}}
%~~neg
In a variable declaration, the variable itself is in scope in its
type. For example, \xcd`val nz: Int{nz != 0} = 1;` declares a
non-zero variable \xcd`nz`.
\bard{This will need to be explained further once the language issues are
sorted out..}

%%TYPES-CONSTR-EXP%% We permit variable declarations \xcd"v: T" where \xcd"T" is obtained
%%TYPES-CONSTR-EXP%% from a dependent type \xcd"C{c}" by replacing one or more occurrences
%%TYPES-CONSTR-EXP%% of \xcd"self" in \xcd"c" by \xcd"v". (If such a declaration \xcd"v: T"
%%TYPES-CONSTR-EXP%% is type-correct, it must be the case that the variable \xcd"v" is not
%%TYPES-CONSTR-EXP%% visible at the type \xcd"T". Hence we can always recover the
%%TYPES-CONSTR-EXP%% underlying dependent type \xcd"C{c}" by replacing all occurrences of \xcd"v"
%%TYPES-CONSTR-EXP%% in the constraint of \xcd"T" by \xcd"self".)
%%TYPES-CONSTR-EXP%% 
%%TYPES-CONSTR-EXP%% For instance, \xcd"v: Int{v == 0}" is shorthand for \xcd"v: Int{self == 0}".
%%TYPES-CONSTR-EXP%% 
%%TYPES-CONSTR-EXP%% 
%%TYPES-CONSTR-EXP%% A variable occurring in the constraint \xcd"c" of a dependent type, other than
%%TYPES-CONSTR-EXP%% \xcd"self" or a property of \xcd"self", is said to be a {\em
%%TYPES-CONSTR-EXP%% parameter} of \xcd"c".\label{DepType:Parameter} \index{parameter}

A constrained type may be constrained further: the type \xcd`S{c}{d}`
is the same as the type \xcd`S{c,d}`.  Multiple constraints are equivalent to
conjoined constraints: \xcd`S{c,d}` in turn is the same as \xcd`S{c && d}`.

\subsection{Syntax of constraints}
\index{constraint!permitted}
\index{constraint!syntax}
\label{PermittedConstraints}
\index{constraint}
\index{expression!allowed in constraint}
\index{expression!constraint}

Only a few kinds of expressions can appear in constraints.  For fundamental
reasons of mathematical logic, the more kinds of expressions that can appear
in constraints, the harder it is to compute the essential properties of
constrained type -- in particular, the harder it is to compute 
\xcd`A{c} <: B{d}`.  It doesn't take much to make this basic fact undecidable.
In order to make sure that it stays decidable, X10 places stringent restrictions on
constraints.  

Only the following forms of expression are allowed in constraints.  

{\bf Value expressions in constraints} may be: 
\begin{enumerate}
\item Literal constants, like \xcd`3` and \xcd`true`;
\item Expressions computable at compile time, like \Xcd{3*4+5};
\item Accessible and immutable variables and parameters;
\item Accessible and immutable fields of objects;
\item Properties of the type being constrained;
\item \xcd`this`, if the constraint is in a place where \xcd`this` is defined;
\item \xcd`here`, if the constraint is in a place where \xcd`here` is defined;
\item \xcd`self`;
\item Calls to property methods, where the receiver and arguments must be
      value expressions acceptable in constraints;
\item \xcd`T haszero`, if \xcd`T` is any type expression.
\end{enumerate}
For an expression \xcd`self.p` to be legal in a constraint, 
\xcd`p` must be 
a property. However terms \xcd`t.f` may be
used in constraints (where \xcd`t` is a term other than \xcd`self` and
\xcd`f` is an immutable field.

{\bf Constraints}, and {\bf Boolean expressions in constraints}  may be any of
the following, where 
all value expressions are of the forms which may appear in constraints: 
\begin{enumerate}
\item Equalities \xcd`e == f`;
\item Inequalities of the form \xcd`e != f`;\footnote{Currently inequalities
      of the form \xcd`e < f` are not supported.}
\item Conjunctions of Boolean expressions that may appear in constraints (but
      only in top-level constraints, not in Boolean expressions in constraints);
\item Subtyping and supertyping expressions: \xcd`T <: U` and \xcd`T :> U`; 
\item Type equalities and inequalities: \xcd`T == U` and \xcd`T != U`; 
% \item Testing a type for a default: \Xcd{hasZero T}.
\end{enumerate}

All variables appearing in a constraint expression must be visible wherever
that expression can used.  \Eg, properties and public fields of an object are
always permitted, but private fields of an object can only constrain private
members.  (Consider a class \xcd`PriVio` with a private field \xcd`p` and a
public method \xcd`m(x: Int{self != p})`, and a call \xcd`ob.m(10)` made
outside of the class. Since \xcd`p` is only visible inside the class, there is
no way to tell if \xcd`10` is of type \xcd`Int{self != p}` at the call site.)

\limitation{
% Currently \Xcd{hasZero T} is not supported.  
Certain spurious syntactic forms -- such as \xcd`a+b,a*b,(c==a&&b),a <b` --
are accepted by the compiler but treated incorrectly.  
}



\subsubsection{Semantics of constraints}
\index{constraint!semantics}
\label{SemanticsOfConstraints}
An assignment of values to variables is said to be a {\em solution} for a
constraint \xcd`c` if under this assignment \xcd`c` evaluates to
\xcd`true`. For instance, the assignment that maps 
the variables \xcd`a` and \xcd`b` to a value \xcd`t` is a solution for
the constraint \xcd`a==b`. An assignment that maps \xcd`a` to 
\xcd`s` and \xcd`b` to a distinct value \xcd`t` is a solution for 
\xcd`a != b`. 

An instance \xcd"o" of \xcd"C" is said to be of type \xcd"C{c}" (or {\em
belong to} \xcd"C{c}") if the constraint \xcd"c" evaluates to \xcd"true" in
the current lexical environment augmented with the binding \xcd"self"
$\mapsto$ \xcd"o".

A constraint \xcd`c` is said to {\em entail} a
constraint \xcd`d` if every solution for \xcd`c` is also a solution
for \xcd`d`. For instance the constraint
\xcd`x==y && y==z && z !=a` entails \xcd`x != a`.

The constraint solver considers the assignment \xcd`a` to \xcd`null`
to  satisfy any constraint of the form \xcd`a.f==t`. Thus, for
instance, the assignment \xcd`var x:Tree{self.home==p}=null` does not
produce an error, since \xcd`self==null` is considered a solution for \xcd`self.home==p`.

To ensure that type-checking is decidable, we require that property graphs be
acyclic.  The property graph, at an instant in an X10 execution, is the graph
whose nodes are all objects in existence at that instance, with an edge from
{$x$} to {$y$} if {$x$} is an object with a property whose value is {$y$}. 
The rules for constructors guarantee this.

Constraints participate in the subtyping relationship in a natural way:
\xcdmath"S[S1,$\ldots$, Sm]{c}" 
is a subtype of 
\xcdmath"T[T1,$\ldots$, Tn]{d}" 
if \xcdmath"S[S1,$\ldots$,Sm]" is a subtype of \xcdmath"T[T1,$\ldots$,Tn]" and
\xcd"c" entails \xcd"d".

For examples of constraints and entailment, see (\Sref{ConstraintExamples})
%%TYPES-CONSTR-EXP%% 
%%TYPES-CONSTR-EXP%% \begin{grammar}
%%TYPES-CONSTR-EXP%% Constraint \: ValueArguments     Guard\opt \\
%%TYPES-CONSTR-EXP%%            \| ValueArguments\opt Guard     \\
%%TYPES-CONSTR-EXP%%            \\
%%TYPES-CONSTR-EXP%% ValueArguments   \:  \xcd"(" ArgumentList\opt \xcd")" \\
%%TYPES-CONSTR-EXP%% ArgumentList     \:  Expression ( \xcd"," Expression )\star \\
%%TYPES-CONSTR-EXP%% Guard            \: \xcd"{" DepExpression \xcd"}" \\
%%TYPES-CONSTR-EXP%% DepExpression    \: ( Formal \xcd";" )\star ArgumentList \\
%%TYPES-CONSTR-EXP%% \end{grammar}
%%TYPES-CONSTR-EXP%% 
%%TYPES-CONSTR-EXP%% In \XtenCurrVer{} value constraints may be equalities (\xcd"=="),
%%TYPES-CONSTR-EXP%% disequalities (\xcd"!=") and conjunctions thereof.  The terms over
%%TYPES-CONSTR-EXP%% which these constraints are specified include literals and
%%TYPES-CONSTR-EXP%% (accessible, immutable) variables and fields, property methods, and the special
%%TYPES-CONSTR-EXP%% constants {\tt here}, {\tt self}, and {\tt this}. Additionally, place
%%TYPES-CONSTR-EXP%% types are permitted (\Sref{PlaceTypes}).
%%TYPES-CONSTR-EXP%% 
%%TYPES-CONSTR-EXP%% \index{self}

%%TYPES-CONSTR-EXP%% Type constraints may be subtyping and supertyping (\xcd"<:" and
%%TYPES-CONSTR-EXP%% \xcd":>") expressions over types.

\subsection{Constraint solver: incompleteness and approximation}
\index{constraint solver!incompleteness}
\index{constraint!entailment}
\index{constraint!subtyping}



The constraint solver is sound in that if it claims that \xcd`c` entails \xcd`d`
then in fact it is the case that every value that satisfies \xcd`c`
satisfies \xcd`d`. 

\limitation{X10's Entailment Algorithm is Incomplete}
However, X10's constraint solver is incomplete. There are situations
in which \xcd`c` entails \xcd`d` but the solver cannot establish it. For
instance it cannot establish that \xcd`a != b && a != c && b != c`
entails \xcd`false` if \xcd`a`, \xcd`b`, and \xcd`c` are of type
\xcd`Boolean`.


Certain other constraint entailments are prohibitively expensive to calculate.  The issues
concern constraints that connect different levels of recursively-defined
types, such as the following.  
%~~gen ^^^ Types330
% package Types.Entailment.EntailFail;
%~~vis
\begin{xten}
class Listlike(x:Int) {
  val kid : Listlike{self.x == this.x};
  def this(x:Int, kid:Listlike) { 
     property(x); 
     this.kid = kid as Listlike{self.x == this.x};}
}
\end{xten}
%~~siv
%
%~~neg
There is nothing wrong with \xcd`Listlike` itself, or with most uses of it;
however, a sufficiently complicated use of it could, in principle, cause X10's
typechecker to fail. 
It is hard to give a plausible example of when X10's algorithm fails, as we
have not yet observed such a failure in practice for a correct program.  

The entailment algorithm of X10 2.0 imposes a certain limit on the number of
times such types will be unwound.   If this limit is exceeded, the compiler
will print a warning, and type-checking will fail in a situation where it is
semantically allowed.  In this case, insert a dynamic cast at the point where
type-checking failed.  

\limitation{ Support for comparisons of generic type variables is
  limited. This will be fixed in future releases.}
% //, and existential quantification over typed variables.

%%TYPES-CONSTR-EXP%% \emph{
%%TYPES-CONSTR-EXP%% Subsequent implementations are intended to support boolean algebra,
%%TYPES-CONSTR-EXP%% arithmetic, relational algebra, etc., to permit types over regions and
%%TYPES-CONSTR-EXP%% distributions. We envision this as a major step towards removing most,
%%TYPES-CONSTR-EXP%% if not all, dynamic array bounds and place checks from \Xten{}.
%%TYPES-CONSTR-EXP%% }




%%PLACE%%\subsection{Place constraints}
%%PLACE%%\label{PlaceTypes}
%%PLACE%%\label{PlaceType}
%%PLACE%%\index{place types}
%%PLACE%%\label{DepType:PlaceType}\index{placetype}
%%PLACE%%
%%PLACE%%An \Xten{} computation spans multiple places (\Sref{XtenPlaces}). Much data
%%PLACE%%can only be accessed from the proper place, and often it is preferable to
%%PLACE%%determine this statically. So, X10 has special syntax for working with places.
%%PLACE%%\xcd`T!` is a value of type \xcd`T` located at the right place for the current
%%PLACE%%computation, and \xcd`T!p` is one located at place \xcd`p`.
%%PLACE%%
%%PLACE%%\begin{grammar}
%%PLACE%%PlaceConstraint     \: \xcd"!" Place\opt \\
%%PLACE%%Place              \:   Expression \\
%%PLACE%%\end{grammar}
%%PLACE%%
%%PLACE%%More specifically, All \Xten{} classes extend the class \xcd"x10.lang.Object",
%%PLACE%%which defines a property \xcd"home" of type \xcd"Place".  \xcd`T!p`, when
%%PLACE%%\xcd`T` is a class, is \xcd`T{self.home==p}`.  If \xcd`p` is omitted, it
%%PLACE%%defaults to \xcd`here`.   \xcd`T!` is far and away the most common usage of
%%PLACE%%\xcd`!`. 
%%PLACE%%
%%PLACE%%Structs don't have \xcd`home`; they are available everywhere.  For structs, 
%%PLACE%%\xcd`T!` and \xcd`T!p` are synonyms for \xcd`T`. Since \xcd`T` is available
%%PLACE%%everywhere, it is available \xcd`here` and at \xcd`p`. 
%%PLACE%%
%%PLACE%%\xcd`!` may be combined with other constraints.  \xcd`T{c}!` is the type of
%%PLACE%%values of \xcd`T!` which satisfy \xcd`c`; it is \xcd`T{c && self.home==here}`
%%PLACE%%for an object type and \xcd`T{c}` for a struct type.  
%%PLACE%%\xcd`T{c}!p` is the type of
%%PLACE%%values of \xcd`T!p` which satisfy \xcd`c`; it is \xcd`T{c && self.home==p}`
%%PLACE%%for an object type and \xcd`T{c}` for a struct type.  
%%PLACE%%
%%PLACE%%
%%PLACE%%
%%PLACE%%% The place specifier \xcd"any" specifies that the object can be
%%PLACE%%% located anywhere.  Thus, the location is unconstrained; that is,
%%PLACE%%% \xcd"C{c}!any" is equivalent to \xcd"C{c}".
%%PLACE%%
%%PLACE%%% XXX ARRAY
%%PLACE%%%The place specifier \xcd"current" on an array base type
%%PLACE%%%specifies that an object with that type at point \xcd"p"
%%PLACE%%%in the array 
%%PLACE%%%is located at \xcd"dist(p)".  The \xcd"current" specifier can be
%%PLACE%%%used only with array types.
%%PLACE%%
%%PLACE%%



\subsection{Example of Constraints}
\label{ConstraintExamples}

Example of entailment and subtyping involving constraints.
\begin{itemize}
\item \xcd`Int{self == 3} <: Int{self != 14}`.  The only value of
      \xcd`Int{self ==3}` is $3$.  All integers but $14$ are members of
      \xcd`Int{self != 14}`, and in particular $3$ is.  
\item Suppose we have classes \xcd`Child <: Person`, and \xcd`Person` has a
      long \xcd`ssn` property.  If \xcd`rhys : Child{ssn == 123456789}`, then
      \xcd`rhys` is also a \xcd`Person` and still has \xcd`ssn==123456789`, so 
      \xcd`rhys : Person{ssn==123456789}` as well.  
      So, \xcd`Child{ssn == 123456789} <: Person{ssn == 123456789}`.
\item Furthermore, since \xcd`123456789 != 555555555`, 
      \xcd`rhys : Person{ssn != 555555555}`.  
      So, \xcd`Child{ssn == 123456789} <: Person{ssn != 555555555}`.  
\item \xcd`T{e} <: T` for any type \xcd`T`.  That is, if you have a value
      \xcd`v` of some base type \xcd`T` which satisfied \xcd`e`, then \xcd`v`
      is of that base type \xcd`T` (with the constraint ignored).
\item If \xcd`A <: B`, then \xcd`A{c} <: B{c}` for every constraint \xcd`{c}`
      for which \xcd`A{c}` and \xcd`B{c}` are defined.  That is, if every
      \xcd`A` is also a \xcd`B`, and \xcd`a : A{c}`, then 
      \xcd`a` is an \xcd`A` and \xcd`c` is true of it. So \xcd`a` is also a
      \xcd`B` (and \xcd`c` is still true of 
      it), so \xcd`a : B{c}`.  
\end{itemize}

Constraints can be used to express simple relationships between objects,
enforcing some class invariants statically.  For example, in geometry, a line
is determined by two {\em distinct} points; a \xcd`Line` struct can specify the
distinctness in a type constraint:\footnote{We call them
\xcd`Position` to avoid confusion with the built-in class \xcd`Point`. 
Also, \xcd`Position` is a struct rather than a class so that the non-equality
test \xcd`start != end` compares the coordinates.  If \xcd`Position` were a
class, \xcd`start != end` would check for different \xcd`Position` objects,
which might have the same coordinates.
}


%~~gen ^^^ Types340
% package triangleExample.partOne;
%~~vis
\begin{xten}
struct Position(x: Int, y: Int) {}
struct Line(start: Position, end: Position){start != end} {}
\end{xten}

%~~siv
%~~neg

Extending this concept, a \xcd`Triangle` can be defined as a figure with three
line segments which match up end-to-end.  Note that the degenerate case in
which two or three of the triangle's vertices coincide is excluded by the
constraint on \xcd`Line`.  However, not all degenerate cases can be excluded
by the type system; in particular, it is impossible to check that the three
vertices are not collinear. 

%~~gen ^^^ Types350
%package triangleExample.partTwo;
% struct Position(x: Int, y: Int) {
%    def this(x:Int,y:Int){property(x,y);}
%    }
% class Line(start: Position, 
%            end: Position{self != start}) {}
% 
%~~vis
\begin{xten}
struct Triangle 
 (a: Line, 
  b: Line{a.end == b.start}, 
  c: Line{b.end == c.start && c.end == a.start})  
 {}
\end{xten}
%~~siv
%
%~~neg

The \xcd`Triangle` class automatically gets a ternary constructor which takes
suitably constrained \xcd`a`, \xcd`b`, and \xcd`c` and produces a new
triangle. 

\section{Default Values}
\index{default value}
\index{type!default value}
\label{DefaultValues}

Some types have default values, and some do not. Default values are used in
situations where variables can legitimately be used without having been
initialized; types without default values cannot be used in such situations.
For example, a field of an object \xcd`var x:T` can be left uninitialized if
\xcd`T` has a default value; it cannot be if \xcd`T` does not. Similarly, a
transient (\Sref{TransientFields}) field \xcd`transient val x:T` is only
allowed if \xcd`T` has a default value.

Default values, or lack of them, is defined thus:
\begin{itemize}
\item The fundamental numeric types (\xcd`Int`, \xcd`UInt`,
      \xcd`Long`, \xcd`ULong`, 
%%limitation%%       \xcd`Short`, \xcd`UShort`, \xcd`Byte`,
%%limitation%%       \xcd`UByte`, 
      \xcd`Float`, \xcd`Double`) all have default value 0.
\item \xcd`Boolean` has default value \xcd`false`.
\item \xcd`Char` has default value \xcd`'\0'`.
\item Struct types other than those listed above have no default value.
\item A function type has a default value of \xcd`null`.
\item A class type has a default value of \xcd`null`.
\item The constrained type \xcd`T{c}` has the same default value as \xcd`T` if
      that default value satisfies \xcd`c`.  If the default value of \xcd`T`
      doesn't satisfy \xcd`c`, then \xcd`T{c}` has no default value.
\end{itemize}

For example, \xcd`var x: Int{x != 4}` has default value 0, which is allowed
because \xcd`0 != 4` satisfies the constraint on \xcd`x`. 
\xcd`var y : Int{y==4}` has no default value, because \xcd`0` does not satisfy \xcd`y==4`.
The fact that \xcd`Int{y==4}` has precisely one value, \viz{} 4, doesn't
matter; the only candidate for its default value, as for any subtype of
\xcd`Int`, is 0. \xcd`y` must be initialized before it is used.

The predicate \xcd`T haszero` tells if the type \xcd`T` has a default value.
\xcd`haszero` may be used in constraints. 

\begin{ex}
The following code defines a sort of cell holding a single value of type
\xcd`T`. The cell is initially empty -- that is, has \xcd`T`'s zero value --
but may be filled later. 
%~~gen ^^^ TypesHaszero10
% package TypesHaszero10;
%~~vis
\begin{xten}
class Cell0[T]{T haszero} {
  public var contents : T;
  public def put(t:T) { contents = t; }
}
\end{xten}
%~~siv
%
%~~neg
\end{ex}

The built-in type \xcd`Zero` has the method \xcd`get[T]()` which
returns the default value of type \xcd`T`.  

\begin{ex}
A variant \xcd`Cell1[T]` which can be initialized with a value of an arbitrary
type
\xcd`T`, or, if \xcd`T` has a default value, can be created with the default
value, is given below.  Note that \xcd`T haszero` is a constraint on one of
the constructors, not the whole type:  
%~~gen ^^^ TypesHaszero20
% package TypesHaszero20;
%~~vis
\begin{xten}
class Cell1[T] {
  public var contents: T;
  def this(t:T) { contents = t; }
  def this(){T haszero} { contents = Zero.get[T](); }
  public def put(t:T) {contents = t;}
}
\end{xten}
%~~siv
%
%~~neg

\end{ex}

\section{Function types}
\label{FunctionTypes}
\label{FunctionType}
\index{function!types}
\index{type!function}

%##(FunctionType
\begin{bbgrammar}
%(FROM #(prod:FunctionType)#)
        FunctionType \: TypeParams\opt \xcd"(" FormalParamList\opt \xcd")" WhereClause\opt Offers\opt \xcd"=>" Type & (\ref{prod:FunctionType}) \\
\end{bbgrammar}
%##)


For every sequence of types \xcd"T1,..., Tn,T", and \xcd"n" distinct variables
\xcd"x1,...,xn" and constraint \xcd"c", the expression
\xcd"(x1:T1,...,xn:Tn){c}=>T" is a \emph{function type}. It stands for
 the set of all functions \xcd"f" which can be applied to a
 list of values \xcd"(v1,...,vn)" provided that the constraint
 \xcd"c[v1,...,vn,p/x1,...,xn]" is true, and which returns a value of
 type \xcd"T[v1,...vn/x1,...,xn]". When \xcd"c" is true, the clause \xcd"{c}" can be
 omitted. When \xcd"x1,...,xn" do not occur in \xcd"c" or \xcd"T", they can be
 omitted. Thus the type \xcd"(T1,...,Tn)=>T" is actually shorthand for
 \xcd"(x1:T1,...,xn:Tn){true}=>T", for some variables \xcd"x1,...,xn".

\limitationx{}
Constraints on closures are not supported.  They parse, but are not checked.

X10 functions, like mathematical functions, take some arguments and produce a
result.  X10 functions, like other X10 code, can change mutable state and
throw exceptions.  Closures (\Sref{Closures}) and method
selectors (\Sref{MethodSelectors}) are of function type.
Typical functions are the reciprocal function: 
%~~gen ^^^ Types360
% package Types.Functions;
% class RecipEx {
% static 
%~~vis
\begin{xten}
val recip = (x : Double) => 1/x;
\end{xten}
%~~siv
%}
%~~neg
and a function which increments  element \xcd`i` of an array \xcd`r`, or throws an exception
if there is no such element, where, for the sake of example, we constrain the
type of \xcd`i`:  
%~~gen ^^^ Types_constraint_b
% package Types_constraint_b;
% NOTEST
% /*NONSTATIC*/class IncrElEx {
% static def example()  {
%~~vis
\begin{xten}
val inc = (r:Array[Int](1), i: Int{i != r.size}) => {
  if (i < 0 || i >= r.size) throw new DoomExn();
  r(i)++;
};
\end{xten}
%~~siv
%}}
%class DoomExn extends Exception{}
%~~neg

In general, a function type needs to list the types 
\xcdmath"T$_i$"
of all the formal parameters,
and their distinct names \xcdmath"x$_i$" in case other types refer to them; a
constraint 
\xcd"c" on the
function as a whole; a return type \xcd"T".

\begin{xtenmath}
(x$_1$: T$_1$, $\dots$, x$_n$: T$_n$){c} => T
\end{xtenmath}


The names \xcdmath"x$_i$" of the formal parameters are not relevant.  Types
which differ only in the names of formals (following the usual rules for
renaming of variables, as in {$\alpha$}-renaming in the {$\lambda$} calculus
\bard{cite something}) are considered equal.  \Eg, the two function types
%~~type~~`~~`~~ ~~ ^^^ Types370
\xcd`(a:Int, b:Array[String](1){b.size==a}) => Boolean`
and \\
%~~type~~`~~`~~ ~~ ^^^ Types380
\xcd`(b:Int, a:Array[String](1){a.size==b}) => Boolean`
are equivalent.

\limitation{
This is not currently implemented properly; these two types are presently
considered different.
}

The formal parameter names are in scope from the point of definition to the
end of the function type---they may be used in the types of other formal parameters
and in the return type. 
Value parameters names may be
omitted if they are not used; the type of the reciprocal function can be
written as
%~~type~~`~~`~~ ~~ ^^^ Types390
\xcd`(Double)=>Double`. 

A function type is covariant in its result type and contravariant in
each of its argument types. That is, let 
\xcd"S1,...,Sn,S,T1,...Tn,T" be any
types satisfying \xcd"Si <: Ti" and \xcd"S <: T". Then
\xcd"(x1:T1,...,xn:Tn){c}=>S" is a subtype of
\xcd"(x1:S1,...,xn:Sn){c}=>T".

A class or struct definition may use a function type 
\xcd"F = (x1:T1,...,xn:Tn){c}=>T" in its 
implements clause; 
this is equivalent to implementing an interface requiring the single method
\xcd"def apply(x1:T1,...,xn:Tn){c}:T". 
Similarly, an interface
definition may specify a function type \xcd"F" in its \xcd"extends" clause.
Values of a class or struct implementing \xcd`F` 
can be used as functions of type \xcd`F` in all ways.  
In particular, applying one to suitable arguments calls the \xcd`apply`
method. 

\limitationx{} A class or struct may not implement two different
instantiations of a generic interface. In particular, a class or
struct can implement only one function type.


A function type \xcd"F" is not a class type in that it does not extend any
type or implement any interfaces, or support equality tests. 
\xcd`F` may be implemented, but not extended, by a class or function type. 
Nor is it a struct type, for it has no predefined notion of equality.


\section{Annotated types}
\label{AnnotatedTypes}

\index{type!annotated}
\index{annotations!type annotations}

        Any \Xten{} type may be annotated with zero or more
        user-defined \emph{type annotations}
        (\Sref{XtenAnnotations}).  

        Annotations are defined as (constrained) interface types and are
        processed by compiler plugins, which may interpret the
        annotation symbolically.

        A type \xcd"T" is annotated by interface types
        \xcdmath"A$_1$", \dots,
        \xcdmath"A$_n$"
        using the syntax
        \xcdmath"@A$_1$ $\dots$ @A$_n$ T".

\section{Subtyping and type equivalence}\label{DepType:Equivalence}
\index{type equivalence}
\index{subtyping}

Intuitively, type \xcdmath"T$_1$" is a subtype of type \xcdmath"T$_2$", 
written \xcdmath"T$_1$ <: T$_2$", 
if
every instance of \xcdmath"T$_1$" is also an instance of \xcdmath"T$_2$".  For
example, \xcd`Child` is a subtype of \xcd`Person` (assuming a suitably defined
class hierarchy): every child is a person.  Similarly, \xcd`Int{self != 0}`
is a subtype of \xcd`Int` -- every non-zero integer is an integer.  

This section formalizes the concept of subtyping. Subtyping of types depends
on a {\em type context}, \viz. a set of constraints on type parameters
and variables that occur in the type.
For example: 

%~~gen ^^^ Types400
% package Types.subtyping.cons;
% NOCOMPILE
%~~vis
\begin{xten}
class ConsTy[T,U] {
   def upcast(t:T){T <: U} :U = t;
}
\end{xten}
%~~siv
%
%~~neg
\noindent
Inside \xcd`upcast`, \xcd`T` is constrained to be a subtype of \xcd`U`, and so
\xcd`T <: U` is true, and \xcd`t` can be treated as a value of type \xcd`U`.  
Outside of \xcd`upcast`, there is no reason to expect any relationship between
them, and \xcd`T <: U` may be false.
However, subtyping of types that have no free variables does not depend
on the context.    \xcd`Int{self != 0} <: Int` is always
true.

\limitation{Subtyping of type variables does not currently work.}


\begin{itemize}
\item {\bf Reflexivity:} Every type \xcd`T` is a subtype of itself: \xcd`T <: T`.

\item {\bf Transitivity:} If \xcd`T <: U` and \xcd`U <: V`, then \xcd`T <: V`. 

\iffalse
{\bf Class types:}  
Given the definition 
\xcd`class C[$\vec{X}$] extends D[$\vec{Y}$]{d} implements I1, ..., In {...}`
where {$\vec{X}$} is a vector of type variables, and 
{$\vec{Y$} a vector of types possibly involving variables from {$\vec{X}$}, 
and {$\vec{T$} an instantiation of {$\vec{X$} and {$\vec{U$} the corresponding
instantiation of {$\vec{Y$}, 
then 
\xcdmath"C[$\vec{T}$]`"is a subtype of \xcd`D[$\vec{U}$]{d}`, \xcd`I1`, ..., \xcd`In`. 

\item
{\bf Interface types:}  
Given the definition 
\xcdmath"interface I[$\vec{X}$] extends I1, ... In {...}`"
then \xcdmath"I` is a subtype of \xcd`"1`, ..., \xcd`In`.

\item 
{\bf Struct types:} 
Given the definition 
\xcdmath"struct S implements I1, ..., In {...}`"then \xcd`S` is a 
subtype of \xcd`I1`, ..., \xcd`In`. 
\fi

\item {\bf Direct Subclassing:} 
Let {$\vec{X}$} be a (possibly empty) vector of type variables, and
{$\vec{Y}$}, {$\vec{Y_i}$} be vectors of type terms over {$\vec{X}$}.
Let {$\vec{T}$} be an instantiation of {$\vec{X}$}, 
and {$\vec{U}$}, {$\vec{U_i}$} the corresponding instantiation of 
{$\vec{Y}$}, {$\vec{Y_i}$}.  Let \xcd`c` be a constraint, and \xcdmath"c$'$"
be the corresponding instantiation.
We elide properties, and interpret empty vectors as absence of the relevant
clauses. 
Suppose that \xcd`C` is declared by one of the
forms: 
\begin{enumerate}
\item \xcdmath"class C[$\vec{X}$]{c} extends D[$\vec{Y}$]{d} implements I$_1[\vec{Y_1}]${i$_1$},...,I$_n[\vec{Y_n}]${i$_n$}{"
\item \xcdmath"interface C[$\vec{X}$]{c} extends I$_1[\vec{Y_1}]${i$_1$},...,I$_n[\vec{Y_n}]${i$_n$}{"
\item \xcdmath"struct C[$\vec{X}$]{c} implements I$_1[\vec{Y_1}]${i$_1$},...,I$_n[\vec{Y_n}]${i$_n$}{"
\end{enumerate}
Then: 
\begin{enumerate}
\item \xcdmath"C[$\vec{T}$] <: D[$\vec{U}$]{d}" for a class
\item \xcdmath"C[$\vec{T}$] <: I$_i$[$\vec{U_i}$]{i$_i$}" for all cases.
\item \xcdmath"C[$\vec{T}$] <: C[$\vec{T}$]{c$'$}" for all cases.
\end{enumerate}


\item
{\bf Function types:}
\xcdmath"(x$_1$: T$_1$, $\dots$, x$_n$: T$_n$){c} => T"
is a  subtype of 
\xcdmath"(x$'_1$: T$'_1$, $\dots$, x$'_n$: T$'_n$){c$'$} => T$'$ "
if: 
\begin{enumerate}
\item Each \xcdmath"T$_i$ <: T$'_i$";
\item \xcd`c` entails \xcdmath"c$'$";
\item \xcdmath"T$'$ <: T";
\end{enumerate}

\item
{\bf Constrained types:}
\xcd`T{c}` is a subtype of \xcd`T{d}` if \xcd`c` entails \xcd`d`. 

\item {\bf Any:} 
Every type \xcd`T` is a subtype of \xcd`x10.lang.Any`.

\item 
{\bf Type Variables:}
Inside the scope of a constraint \xcd`c` which entails \xcd`A <: B`, we have
\xcd`A <: B`.  \eg, \xcd`upcast` above.


%%NO-VARIANCE%% \item 
%%NO-VARIANCE%% {\bf Covariant Generic Types:} 
%%NO-VARIANCE%% If \xcd`C` is a generic type whose {$i$}th type parameter is covariant, 
%%NO-VARIANCE%% and {\xcdmath"T$'_i$ <: T$_i$"}
%%NO-VARIANCE%% and  {\xcdmath"T$'_j$ == T$_j$"} for all {$j \ne i$}, 
%%NO-VARIANCE%% then {\xcdmath"C[T$'_1$, $\ldots$, T$'_n$] <: C[T$'_1$, $\ldots$, T$'_n$]"}.
%%NO-VARIANCE%% \Eg, \xcd`class C[T1, +T2, T3]` with {$i=2$}, and \xcd"U2 <: T2", then
%%NO-VARIANCE%% \xcd`C[T1,U2,T3] <: C[T1,T2,T3]`.
%%NO-VARIANCE%% 
%%NO-VARIANCE%% \item 
%%NO-VARIANCE%% {\bf Contravariant Generic Types:} 
%%NO-VARIANCE%% If \xcd`C` is a generic type whose {$i$}th type parameter is contravariant, 
%%NO-VARIANCE%% and \xcdmath"T$'_i$ <: T$_i$"
%%NO-VARIANCE%% and  \xcdmath"T$'_j$ == T$_j$" for all {$j \ne i$}, 
%%NO-VARIANCE%% then \xcdmath"C[T$'_1$, $\ldots$, T$'_n$] :> C[T$'_1$, $\ldots$, T$'_n$]".
%%NO-VARIANCE%% \Eg, \xcd`class C[T1, -T2, T3]` with {$i=2$}, and \xcdmath"U2 <: T2", then
%%NO-VARIANCE%% \xcd`C[T1,U2,T3] :> C[T1,T2,T3]`.
%%NO-VARIANCE%% 

\end{itemize}


Two types are {\em equivalent}, \xcd`T == U`, if \xcd`T <: U` and \xcd`U <: T`. 


\section{Common ancestors of types}
\label{LCA}

There are several situations where X10 must find a type \xcd`T` that describes
values of two or more different types.  This arises when X10 is trying to find
a good type to describe: 
\begin{itemize}
%~~exp~~`~~`~~test:Boolean ~~ ^^^ Types410
\item Conditional expressions, like \xcd`test ? 0 : "non-zero"` or even 
%~~exp~~`~~`~~test:Boolean ~~ ^^^ Types420
      \xcd`test ? 0 : 1`;
%~~exp~~`~~`~~ ~~ ^^^ Types430
\item Array construction, like \xcd`[0, "non-zero"]` and 
%~~exp~~`~~`~~ ~~ ^^^ Types440
      \xcd`[0,1]`;
\item Functions with multiple returns, like
%~~gen ^^^ Types450
% package Types_odd_inferred_return_type;
% class Examplerator {
%~~vis
\begin{xten}
def f(a:Int) {
  if (a == 0) return 0;
  else return "non-zero";
}
\end{xten}
%~~siv
%}
%~~neg
\end{itemize}

In some cases, there is a unique best type describing the expression.  For
example, if \xcd`B` and \xcd`C` are direct subclasses of \xcd`A`, \xcd`pick`
will have return type \xcd`A`: 
%~~gen ^^^ Types_uniq
% package Types.For.Gripes.About.Pipes.Full.Of.Wipes;
%  class A {} class B extends A{} class C extends A{}
% class D {
%~~vis
\begin{xten}
static def pick(t:Boolean, b:B, c:C) = t ? b : c;  
\end{xten}
%~~siv
%}
%~~neg

However, in many common cases, there is no unique best type describing the
expression.  For example, consider the expression {$E=$} \xcd`b ? 0 : 1`.  The
best type of \xcd`0` 
is \xcd`Int{self==0}`, and the best type of 1 is \xcd`Int{self==1}`.
Certainly {$E$} could be given the type \xcd`Int`, or even \xcd`Any`, and that
would describe all possible results.  However, we actually know more.
\xcd`Int{self != 2}` is a better description of the type of {$E$}---certainly
the result of {$E$} can never be \xcd`2`.   \xcd`Int{self != 2, self != 3}` is
an even better description; {$E$} can't be \xcd`3` either.  We can continue
this process forever, adding integers which {$E$} will definitely not return
and getting better and better approximations. (If the constraint
sublanguage had \xcd`||`, we could give it the type 
\xcd`Int{self == 0 || self == 1`, which would be nearly perfect.  But 
\xcd`||` makes typechecking far more expensive, so it is excluded.)
No X10 type is the best description of {$E$}; there is always a better one.

Similarly, consider two unrelated interfaces: 
%~~gen ^^^ Types460
% package Types.For.Gripes.About.Snipes;
%~~vis
\begin{xten}
interface I1 {}
interface I2 {}
class A implements I1, I2 {}
class B implements I1, I2 {}
class C {
  static def example(t:Boolean, a:A, b:B) = t ? a : b;
}
\end{xten}
%~~siv
%
%~~neg
\xcd`I1` and \xcd`I2` are both perfectly good descriptions of \xcd`t ? a : b`, 
but neither one is better than the other, and there is no single X10 type
which is better than both. (Some languages have {\em conjunctive
    types}, and could say that the return type of \xcd`example` was 
\xcd`I1 && I2`.  This, too, complicates typechecking.)


So, when confronted with expressions like this, X10 computes {\em some}
satisfactory type for the expression, but not necessarily the {\em best} type.  
X10 provides certain guarantees about the common type \xcd`V{v}` computed for 
\xcd`T{t}` and \xcd`U{u}`: 
\begin{itemize}
\item If \xcd`T{t} == U{u}`, then \xcd`V{v} == T{t} == U{u}`.  So, if X10's
      algorithm produces an utterly untenable type for \xcd`a ? b : c`, and
      you want the result to have type \xcd`T{t}`, you can 
      (in the worst case) rewrite it to 
      \xcd`a ? b as T{t} : c as T{t}`.
\item If \xcd`T == U`, then \xcd`V == T == U`.  For example, 
      X10 will compute the type of \xcd`b ? 0 : 1` as 
      \xcd`Int{c}` for some constraint \xcd`c`---perhaps simply 
      picking \xcd`Int{true}`, \viz, \xcd`Int`. 
\item X10 preserves place information about \xcd`GlobalRef`s, because it is so important. If both
      \xcd`t` and \xcd`u` entail \xcd`self.home==p`, then  
      \xcd`v` will also entail \xcd`self.home==p`.  
\item X10 similarly preserves nullity information.  If \xcd`t` and \xcd`u`
      both entail \xcd`x == null` or \xcd`x != null` for some variable
      \xcd`x`, then \xcd`v` will also entail it as well.

\end{itemize}

%\subsection{Syntactic abbreviations}\label{DepType:SyntaxAbbrev}

\section{Fundamental types}

Certain types are used in fundamental ways by X10.  

\subsection{The interface {\tt Any}}

It is quite convenient to have a type which all values are instances of; that
is, a supertype of all types.\footnote{Java, for one, suffers a number of
  inconveniences because some built-in types like \xcd`int` and \xcd`char`
  aren't subtypes of anything else.}  X10's universal supertype is the
  interface \xcd`Any`. 

\begin{xten}
package x10.lang;
public interface Any {
  def toString():String;
  def typeName():String;
  def equals(Any):Boolean;
  def hashCode():Int;
}
\end{xten}

\xcd`Any` provides a handful of essential methods that make sense and are
useful for everything. \xcd`a.toString()` produces a
string representation of \xcd`a`, and \xcd`a.typeName()` the string
representation of its type; both are useful for debugging.  \xcd`a.equals(b)`
is the programmer-overridable equality test, and \xcd`a.hashCode()` an integer
useful for hashing.  


\subsection{The class {\tt Object}}
\label{Object}
\index{\Xcd{Object}}
\index{\Xcd{x10.lang.Object}}

The class \xcd"x10.lang.Object" is the supertype of all classes.
A variable of this type can hold a reference to any object.
\xcd`Object` implements \xcd`Any`.



\section{Type inference}
\label{TypeInference}
\index{type!inference}
\index{type inference}

\XtenCurrVer{} supports limited local type inference, permitting
certain variable types and return types to be elided.
It is a static error if an omitted type cannot be inferred or
uniquely determined. Type inference does not consider coercions.

\subsection{Variable declarations}

The type of a \xcd`val` variable declaration can be omitted if the
declaration has an initializer.  The inferred type of the
variable is the computed type of the initializer.
For example, 
%~~stmt~~`~~`~~ ~~ ^^^ Types470
\xcd`val seven = 7;`
is identical to 
%~~stmt~~`~~`~~ ~~ ^^^ Types480
\xcd`val seven: Int{self==7} = 7;`
Note that type inference gives the most precise X10 type, which might be more
specific than the type that a programmer would write.



\limitation{At the moment,  \xcd`var` declarations may not have their types
elided in this way.  
}

\subsection{Return types}

The return type of a method can be omitted if the method has a body (\ie, is
not \xcd"abstract" or \xcd"native"). The inferred return type is the computed
type of the body.  In the following example, the return type inferred for
\xcd`isTriangle` is 
%~~type~~`~~`~~ ~~ ^^^ Types490
\xcd`Boolean{self==false}`
%~~gen ^^^ Types500
% package Types.Inferred.Return;
%~~vis
\begin{xten}
class Shape {
  def isTriangle() = false; 
}  
\end{xten}
%~~siv
%
%~~neg
Note that, as with other type inference, methods are given the most specific
type.  In many cases, this interferes with subtyping.  For example, if one
tried to write: 
\begin{xten}
class Triangle extends Shape {
  def isTriangle() = true;
}
\end{xten}
\noindent
the X10 compiler would reject this program for attempting to override
\xcd`isTriangle()` by a method with the wrong type, \viz,
\xcd`Boolean{self==true}`.  In this case, supply the type that is actually
intended for \xcd`isTriangle`, such as 
\xcd`def isTriangle() :Boolean =false;`. 

The return type of a closure can be omitted.
The inferred return type is the computed type of the body.

The return type of a constructor can be omitted if the
constructor has a body.
The inferred return type is the enclosing class type with
properties bound to the arguments in the constructor's \xcd"property"
statement, if any, or to the unconstrained class type.
For example, the \xcd`Spot` class has two constructors, the first of which has
inferred return type \xcd`Spot{x==0}` and the second of which has 
inferred return type \xcd`Spot{x==xx}`. 
%~~gen ^^^ Types510
% package Types.Inferred.By.Phone;
%~~vis
\begin{xten}
class Spot(x:Int) {
  def this() {property(0);}
  def this(xx: Int) { property(xx); }
}
\end{xten}
%~~siv
%class Confirm{ 
% static val s0 : Spot{x==0} = new Spot();
% static val s1 : Spot{x==1} = new Spot(1);
%}
%~~neg


\index{void}

A method or closure that has expression-free \xcd`return` statements
(\xcd`return;` rather than \xcd`return e;`) is said to return \xcd`void`.
\xcd`void` is not a type; there are no \xcd`void` values, nor can \xcd`void`
be used as the argument of a generic type. However, \xcd`void` takes the
syntactic place of a type. A method returning \xcd`void` can be specified by
\xcd`def m():void`: 

%~~gen ^^^ Types520
% package Types.voidd;
% class voidddd {
% static 
%~~vis
\begin{xten}
val f : () => void = () => {return;};
\end{xten}
%~~siv
%}
%~~neg

By a convenient abuse of language, \xcd`void` is sometimes
lumped in with types; \eg, we may say ``return type of a method'' rather than
the formally correct but rather more awkward ``return type of a method, or
\xcd`void`''.   Despite this informal usage, \xcd`void` is not a type.  For
example, given 
%~~gen ^^^ Types530
% package Types.void_is_not_a_type;
% class EEEEVil {
%~~vis
\begin{xten}
  static def eval[T] (f:()=>T):T = f();
\end{xten}
%~~siv
% }
%~~neg
\noindent
The call \xcd`eval[void](f)` does {\em not} typecheck; \xcd`void` is not a
type and thus cannot be used as a type argument.  There is no way in X10 to
write a generic function which works with both functions which return a value
and functions which do not.  In most cases, functions which have no sensible
return value can be provided with a dummy return value.

\subsection{Inferring Type Arguments}
\label{TypeParamInfer}


A call to a polymorphic method %, closure, or constructor 
may omit the
explicit type arguments.  If the method has a type parameter
\xcd"T", the type argument corresponding to \xcd"T" is inferred
to be a common ancestor of the types of any formal
parameters of type \xcd"T".

(Exception: it is an error if the method call provides no information about
a type parameter that must be inferred.  For example, given the method
definition \xcd`def m[T](){...}`, an invocation \xcd`m()` is considered a
static error.  The compiler has no idea what \xcd`T` the programmer intends.)


%TODO--check this!
Consider the following method, which chooses one of its arguments.  (A more
sophisticated one might sometimes choose the second argument, but that does
not matter for the sake of this example.)
\begin{xten}
static def choose[T](a: T, b: T): T = a; 
\end{xten}

The type argument \xcd`T` can always be supplied: 
\xcd`choose[Int](1, 2)` picks an integer, 
and \xcd`choose[Any](1, "yes")` picks a value that might be an integer or a
string.  
However, the type argument can be elided.  Suppose that \xcd`Sub <: Super`;
then the following compiles: 

%~~gen ^^^ Types540
% package Types.GenericInference;
% class Exampllll{ 
%~~vis
\begin{xten}
  static def choose[T](a: T, b: T): T = a; 
  static val j : Any = choose("string", 1);
  static val k : Super = choose(new Sub(), new Super());
\end{xten}
%~~siv
%}
% class Super {}
% class Sub extends Super {}
%~~neg


\subsubsection{Sketch of X10 Type Inference for Method Calls}

When the X10 compiler sees a method call 
\xcdmath"a.m(b$_1$, $\ldots$,b$_n$)", and attempts to infer type parameters to see if it could be use of a
method \xcdmath"def m[X$_1$, $\ldots$, X$_t$](y$_1$: S$_1$, $\ldots$, y$_n$:S$_n$)", 
it reasons as follows. 



Suppose that \xcdmath"b$_i$" has type \xcdmath"T$_i$".  Then, X10 is seeking a
set of type {$B$} bindings \xcdmath"X$_j$ = U$_j$", for $1 \le j \le t$, such that 
\xcdmath"T$_i$ <: S$^*_i$" for {$1 \le i \le n$}, where \xcdmath"S$^*$" is
\xcd`S` with each type variable \xcdmath"X$_j$" replaced by the corresponding
\xcdmath"U$_j$".  If it can find such a {$B$}, it has a usable choice of type
arguments and can do the type inference.  If it cannot find {$B$}, then it
cannot do type inference.    (Note that X10's type inference algorithm is
incomplete -- there may {\em be} such a {$B$} that X10 cannot find.  If this
occurs in your program, you will have to write down the type arguments
explicitly.) 

Let $B_0$ be the set {$\{ T_i \subtype S_i | 1 \le i \le n\}$}.  Let
{$B_{n+1}$} be {$B_n$} with one element {$F \subtype G$} or 
{$F \typeeq G$} removed, and
{$C(F \subtype G)$} 
or {$C(F \typeeq G)$} (defined below) added.  Repeat this until 
{$B_n$} consists entirely of comparisons with type variables (\viz, 
\xcdmath"Y$_j$ == U", 
\xcdmath"Y$_j$ <: U", and
\xcdmath"Y$_j$ :> U"), 
or until some {$n$} exceeds a predefined compiler limit. 

The candidate inferred types may be read off of {$B_n$}.  The guessed binding
for \xcdmath"X$_j$" is: 
\begin{itemize}
\item If there is an equality \xcdmath"X$_j$==W" in {$B_n$}, then guess the
      binding \xcdmath"X$_j$=W".  Note that there may be several such
      equalities with different choices of \xcd`W`, but, if the inference is
      to work, all the choices of \xcd`W` must be equal types anyways.
\item Otherwise, if there is one or more upper bounds 
\xcdmath"X$_j$ <: V$_k$" in {$B_n$}, guess the binding 
\xcdmath"X$_j$ = V$_+$", where 
\xcdmath"V$_+$" is the computed lower bound of all the \xcdmath"V$_k$"'s.
\item Otherwise, if there is one or more lower bounds 
\xcdmath"R$_k$ <: X$_j$", guess that
\xcdmath"X$_j$ = R$_+$", where 
\xcdmath"R$_+$" is the computed upper bound of all the \xcdmath"R$_k$"'s.
\end{itemize}
If this does not yield a binding for some variable \xcdmath"X$_j$", then type
inference fails.  Furthermore, if every variable \xcdmath"X$_j$" is given a
binding \xcdmath"U$_j$", but the 
bindings do not work --- 
that is, if 
\xcdmath"a.m[U$_1$, $\ldots$, U$_t$](b$_1$, $\ldots$,b$_n$)"
is not a call of 
the original method 
\xcdmath"def m[X$_1$, $\ldots$, X$_t$](y$_1$: S$_1$, $\ldots$, y$_n$:S$_n$)"
--- then type inference also fails.

\paragraph{Computation of the Replacement Elements}

Given a type relation
{$r$} of the form {$F \subtype G$}
or {$F \typeeq G$}, we compute the set {$C(r)$} of
replacement constraints.  There are a number of cases; we present only the
interesting ones. 

\begin{itemize}
\item If $F$ has the form \xcdmath"$F'${c}", then  
\xcdmath"$C(r)$" is defined to be
 \xcdmath"$F'$ == $G$" if $r$ is an equality, or 
 \xcdmath"$F'$ <: $G$" if {$r$} is a subtyping.
That is, we erase type constraints.  
Validity is not an issue at this point in the algorithm, as 
we check at the end that the result is valid.
However, in important cases, the replacement is valid, in the sense that the 
solutions of {$B_{k+1}$} are precisely the solutions of {$B_k$}.
Specifically, if the equation had the form \xcdmath"Z{c} == A", it could be
solved by \xcd`Z==A` or by \xcd`Z = A{c}`.  By dropping constraints in this
rule, we choose the former solution. 

\item Similarly, we drop constraints on {$G$} as well.

\item If {$F$} has the form \xcdmath"K[F$_1$, $\ldots$, F$_k$]"
and 
{$G$}
has the form \xcdmath"K[G$_1$, $\ldots$, G$_k$]", 
then {$C(r)$} has one type relation comparing each parameter of 
{$F$} with the corresponding one of {$G$}. 

If {$r$} is a type equality {$F \typeeq G$}, then 
{$C(r) = \{ F_l \typeeq G_l | 1 \le l \le k$}.

If {$r$} is a type comparison, and the {$l^{th}$} type parameter of \xcd`K` is
invariant,
%%NO-VARIANCE%%  (resp. covariant or contravariant)
then 
{$C(r)$} has {$F_l \typeeq G_l$}
(resp. {$F_l \subtype G_l$} or {$G_l \subtype F_l $}). 


For example, the constraint \xcdmath"List[X] == List[Y]" produces the
constraint \xcd`X==Y`.

\item Other cases are fairly routine.  \Eg, if {$F$} is a \xcd`type`-defined
      abbreviation, it is expanded.

\end{itemize}

\begin{ex}

%~~gen ^^^ Types1s4y
% package Types1s4y;
%~~vis
\begin{xten}
import x10.util.*;
class Cl[C1, C2, C3]{}
class Example {
  static def me[X1, X2](Cl[Int, X1, X2]) = 
     new Cl[X1, X2, Point]();
  static def example() {
    val a = new Cl[Int, Boolean, String]();
    val b : Cl[Boolean, String, Point] = me[Boolean, String](a);
  }
}
\end{xten}
%~~siv
%
%~~neg

\end{ex}

\section{Type Dependencies}

Type definitions may not be circular, in the sense that no type may be its own
supertype, nor may it be a container for a supertype. This forbids interfaces
like \xcd`interface Loop extends Loop`, and indirect self-references such as
\xcd`interface A extends B.C` where \xcd`interface B extends A`.  

The formal definition of this is based on Java's.  

An {\em entity type} is a class, interface, or struct type.   

Entity type $E$ {\em directly depends on} entity type $F$ if $F$ is mentioned
in the \xcd`extends` or \xcd`implements` clause of $E$, either by itself or as
a qualifier within a super-entity-type name.  
\begin{eg}
In the following, \xcd`A` directly depends on \xcd`B`, \xcd`C`, \xcd`D`, 
\xcd`E`, and \xcd`F`.    It does not directly depend on \xcd`G`.
%~~gen ^^^ Types6a9m
% package Types6a9m;
% NOTEST
% class B{ static class C{}}
% class D{ static interface E{}}
% interface F[X]{}
% class G{}
%~~vis
\begin{xten}
class A extends B.C implements D.E, F[G] {}
\end{xten}
%~~siv
%
%~~neg
It is an ordinary programming idiom to use \xcd`A` as an argument to a generic
interface that \xcd`A` implements.  For example, \xcd`ComparableTo[T]`
describes things which can be compared to a value of type \xcd`T`. Saying that
\xcd`A` implements \xcd`ComparableTo[A]` means that one \xcd`A` can be
compared to another, which is reasonable and useful: 
%~~gen ^^^ Types2x6d
% package Types2x6d;
%~~vis
\begin{xten}
interface ComparableTo[T] {
  def eq(T):Boolean;
}
class A implements ComparableTo[A] {
  public def eq(other:A) = this.equals(other);
}
\end{xten}
%~~siv
%
%~~neg

\end{eg}

Entity type $E$ {\em depends on} entity type $F$ if
either $E$ directly depends on $F$, or $E$ directly depends on an entity type
that depends on $F$.   That is, the relation ``depends on'' is the transitive
closure of the relation ``directly depends on''.  

It is a static error if any entity type $E$ depends on itself.
	
\chapter{Variables}\label{XtenVariables}\index{variables}

%%OLDA variable is a storage location.  \Xten{} supports seven kinds of
%%OLDvariables: constant {\em class variables} (static variables), {\em
%%OLD  instance variables} (the instance fields of a class), {\em array
%%OLD  components}, {\em method parameters}, {\em constructor parameters},
%%OLD{\em exception-handler parameters} and {\em local variables}.

A {\em variable} is an X10 identifier associated with a value within some
context. Variable bindings have these essential properties:
\begin{itemize}
\item {\bf Type:} What sorts of values can be bound to the identifier;
\item {\bf Scope:} The region of code in which the identifier is associated
      with the entity;
\item {\bf Lifetime:} The interval of time in which the identifier is
      associated with the entity.
\item {\bf Visibility:} Which parts of the program can read or manipulate the
      value through the variable.
\end{itemize}



X10 has many varieties of variables, used for a number of purposes. They will
be described in more detail in this chapter.  
\begin{itemize}
\item Class variables, also known as the static fields of a class, which hold
      their values for the lifetime of the class.  
\item Instance variables, which hold their values for the lifetime of an
      object;
\item Array elements, which are not individually named and hold their values
      for the lifetime of an array;
\item Formal parameters to methods, functions, and constructors, which hold
      their values for the duration of method (etc.) invocation;
\item Local variables, which hold their values for the duration of execution
      of a block.
\item Exception-handler parameters, which hold their values for the execution
      of the exception being handled. 
\end{itemize}
A few other kinds of things are called variables for historical reasons; \eg,
type parameters are often called type variables, despite not being variables
in this sense because they do not refer to X10 values.  Other named entities,
such as classes and methods, are not called variables.  However, all
name-to-whatever bindings enjoy similar concepts of scope and visibility.  

In the following example, \xcd`n` is an instance variable, and \xcd`next` is a
local variable defined within the method \xcd`bump`.\footnote{This code is
unnecessarily turgid for the sake of the example.  One would generally write
\xcd`public def bump() = ++n;`.   }
%~~gen
% package Vars.For.Squares;
%~~vis
\begin{xten}
class Counter {
  private var n : Int = 0;
  public def bump() : Int {
    val next = n+1;
    n = next;
    return next;
    }
}
\end{xten}
%~~siv
%
%~~neg
Both variables have type \xcd`Int` (or
perhaps something more specific).    The scope of \xcd`n` is the body of
\xcd`Counter`; the scope of \xcd`next` is the body of \xcd`bump`.  The
lifetime of \xcd`n` is the lifetime of the \xcd`Counter` object holding it;
the lifetime of \xcd`next` is the duration of the call to \xcd`bump`. Neither
variable can be seen from outside of its scope.

\label{exploded-syntax}
\label{VariableDeclarations}
\index{variable declaration}


Variables whose value may not be changed after initialization are said to be
{\em immutable}, or {\em constants} (\Sref{FinalVariables}), or simply
\xcd`val` variables. Variables whose value may change are {\em mutable} or
simply \xcd`var` variables. \xcd`var` variables are declared by the \xcd`var`
keyword. \xcd`val` variables may be declared by the \xcd`val` keyword; when a
variable declaration does not include either \xcd`var` or \xcd`val`, it is
considered \xcd`val`. 


%~~gen
%package Vars.For.Bears.In.Chairs;
%class VarExample{
%static def example() {
%~~vis
\begin{xten}
val a : Int = 0;               // Full 'val' syntax
b : Int = 0;                   // 'val' implied
val c = 0;                     // Type inferred
var d : Int = 0;               // Full 'var' syntax
var e : Int;                   // Not initialized
var f : Int{self != 100} = 0;  // Constrained type
\end{xten}
%~~siv
%}}
%~~neg







\section{Immutable variables}
\label{FinalVariables}
\index{variable!immutable}
\index{immutable variable}

Immutable variables can be given values (by initialization or assignment) at
most once, and must be given values before they are used.  Usually this is
achieved by declaring and initializing the variable in a single statement.
%~~gen
% package Vars.In.Snares;
% class ABitTedious{
% def example() {
%~~vis
\begin{xten}
val a : Int = 10;
val b = (a+1)*(a-1);
\end{xten}
%~~siv
%}}
%~~neg
\xcd`a` and \xcd`b` cannot be assigned to further.

In other cases, the declaration and assignment are separate.  One such
case is how constructors give values to \xcd`val` fields of objects.  The
\xcd`Example` class has an immutable field \xcd`n`, which is given different
values depending on which constructor was called. \xcd`n` can't be given its
value by initialization when it is declared, since it is not knowable which
constructor is called at that point.  
%~~gen
% package Vars.For.Cares;
%~~vis
\begin{xten}
class Example {
  val n : Int; // not initialized here
  def this() { n = 1; }
  def this(dummy:Boolean) { n = 2;}
}
\end{xten}
%~~siv
%
%~~neg

Another common case of separating declaration and assignment is in function
and method call.  The formal parameters are bound to the corresponding actual
parameters, but the binding does not happen until the function is called.  In
the code below, \xcd`x` is initialized to \xcd`3` in the first call and
\xcd`4` in the second.
%~~gen
%package Vars.For.Swears;
%class Examplement {
%static def whatever() {
%~~vis
\begin{xten}
val sq = (x:Int) => x*x;
x10.io.Console.OUT.println("3 squared = " + sq(3));
x10.io.Console.OUT.println("4 squared = " + sq(4));
\end{xten}
%~~siv
%}}
%~~neg





%%IMMUTABLE%% An immutable variable satisfies two conditions: 
%%IMMUTABLE%% \begin{itemize}
%%IMMUTABLE%% \item it can be assigned to at most once, 
%%IMMUTABLE%% \item it must be assigned to before use. 
%%IMMUTABLE%% \end{itemize}
%%IMMUTABLE%% 
%%IMMUTABLE%% \Xten{} follows \java{} language rules in this respect \cite[\S
%%IMMUTABLE%% 4.5.4,8.3.1.2,16]{jls2}. Briefly, the compiler must undertake a
%%IMMUTABLE%% specific analysis to statically guarantee the two properties above.
%%IMMUTABLE%% 
%%IMMUTABLE%% Immutable local variables and fields are defined by the \xcd"val"
%%IMMUTABLE%% keyword.  Elements of value arrays are also immutable.
%%IMMUTABLE%% 
%%IMMUTABLE%% \oldtodo{Check if this analysis needs to be revisited.}

\section{Initial values of variables}
\label{NullaryConstructor}\index{nullary constructor}

Every assignment, binding, or initialization to a variable of type \xcd`T{c}`
must be an instance of type \xcd`T` satisfying the constraint \xcd`{c}`.
Variables must be given a value before they are used. This may be done by
initialization, which is the only way for immutable (\xcd`val`) variables and
one option for mutable (\xcd`var`) ones: 

%~~gen
%package Vars.For.Bears;
%class VarsForBears{
%def check() {
%~~vis
\begin{xten}
  val immut : Int = 3;
  var mutab : Int = immut;
  val use = immut + mutab;
\end{xten}
%~~siv
%}}
%~~neg
Or, for mutable variables, it may be done by a later assignment.  

%~~gen
%package Vars.For.Stars;
%class VarsForStars{
%def check() {
%~~vis
\begin{xten}
  var muta2 : Int;
  muta2 = 4;
  val use = muta2 * 10;
\end{xten}
%~~siv
%}}
%~~neg


Every class variable must be initialized before it is read, through
the execution of an explicit initializer or a static block. Every
instance variable must be initialized before it is read, through the
execution of an explicit initializer or a constructor.
\bard{Revise this in light of initial values}
Mutable instance variables of class type are initialized to 
to \xcd"null".
Mutable instance variables of struct type are 
assumed to have an initializer that sets the value to the
result of invoking the nullary constructor on the class. 
An initializer is required if the default initial value of the variable's type
is not
assignable to the variable's type, \eg, \xcd`Int` variables are initialized to
zero, but that doesn't work for \xcd`val x:Int{x!=0}`.

Each method and constructor parameter is initialized to the
corresponding argument value provided by the invoker of the method. An
exception-handling parameter is initialized to the object thrown by
the exception. A local variable must be explicitly given a value by
initialization or assignment, in a way that the compiler can verify
using the rules for definite assignment \cite[\S~16]{jls2}.


\section{Destructuring syntax}
\index{variable declarator!destructuring}
\index{destructuring}
\Xten{} permits a \emph{destructuring} syntax for local variable
declarations and formal parameters of type \xcd`Point`, \Sref{point-syntax}.
(Future versions of X10 may allow destructuring of other types as well.) 
A point is a sequence of {$r \ge 0$} \xcd`Int`-valued coordinates.  
It is often useful to get at the coordinates directly, in variables. 

The following code makes an anonymous point with one coordinate \xcd`11`, and
binds \xcd`i` to \xcd`11`.  Then it makes a point with coordinates \xcd`22`
and \xcd`33`, binds \xcd`p` to that point, and \xcd`j` and \xcd`k` to \xcd`22`
and \xcd`33` respectively.
%~~gen
% package Vars.For.Glares;
% class DestructuringEx1 {
% def whyJustForLocals() {
%~~vis
\begin{xten}
val (i) : Point = Point.make(11);
val p(j,k) = Point.make(22,33);
\end{xten}
%~~siv
%}}
%~~neg

A useful idiom for iterating over a range of numbers is: 
%~~gen
%package Vars.For.Bears;
% class ForBear {
% def forbear() {
%~~vis
\begin{xten}
var sum : Int = 0;
for ((i) in 1..100) sum += i;
\end{xten}
%~~siv
% ; } } 
%~~neg
\noindent
The parentheses in \xcd`(i)` introduce destructuring, making X10 treat \xcd`i`
as an \xcd`Int`; without them, it would be a \xcd`Point`.  

In general, a pattern of the form \xcdmath"(i$_1$,$\ldots$,i$_n$)" matches a
point with {$n$} coordinates, binding \xcdmath"i$_j$" to coordinate {$j$}.  
A pattern of the form \xcdmath"p(i$_1$,$\ldots$,i$_n$)" does the same, , but
also binds \xcd`p` to the point.

\section{Formal parameters}
Formal parameters are always declared with a type.
The variable name can be omitted if it is not to be used in the
scope of the declaration.

\begin{grammar}
Formal
        \: FormalModifier\star \xcd"var" VarDeclaratorWithType \\
        \| FormalModifier\star \xcd"val" VarDeclaratorWithType \\
        \| FormalModifier\star VarDeclaratorWithType \\
        \| Type \\
FormalModifier \: Annotation \\
              \| \xcd"shared" \\
\end{grammar}

\xcd`var`, \xcd`val`, and \xcd`shared` behave just as they do for local
variables, \Sref{local-variables}.

\section{Local variables}\label{local-variables}
Local variable declarations may have
initializer expressions: \xcd`var i:Int = 1;` introduces 
a variable \xcd`i` and initializes it to 1.
The initializer must be a subtype of
the declared type of the variable.  If the variable is immutable
(\xcd"val")
the type may be omitted and
inferred from the initializer type (\Sref{TypeInference}).
Variables marked \xcd`shared` can be used by many activities at once; see
\Sref{Shared}.

\begin{grammar}
LocalDeclaration
        \: LocalModifier\star \xcd"var" LocalDeclaratorsWithType \\&& ( \xcd"," LocalDeclaratorsWithType )\star \\
        \| LocalModifier\star \xcd"val" LocalDeclarators \\&& ( \xcd"," LocalDeclarators )\star \\
        \| LocalModifier\star LocalDeclaratorsWithType \\&& ( \xcd"," LocalDeclaratorsWithType )\star \\
LocalDeclarators
        \: LocalDeclaratorsWithType \\
        \: LocalDeclaratorWithInit \\
LocalDeclaratorWithInit
        \: VarDeclarator Init \\
LocalDeclaratorsWithType
        \: VarDeclaratorId
                ( \xcd"," VarDeclaratorId )\star ResultType \\
LocalModifier \: Annotation \\
              \| \xcd"shared" \\
Init \: \xcd"=" Expression \\
\end{grammar}

\section{Fields}
Fields are declared either \xcd"var" (mutable, non-static),
\xcd"val" (immutable, non-static), or \xcd"const" (immutable, static);
the default is \xcd"val".
Field declarations may have optional
initializer expressions.  The initializer must be a subtype of
the declared type of the variable.
For \xcd"var" fields,
if the initializer is omitted, the constructor must initialize
the field, or else the field is initialized with
\xcd"null" if a reference type, \xcd"0" if an \xcd"Int", \xcd"0L"
if a \xcd"Long",
\xcd"0.0F" if a \xcd"Float", \xcd"0.0" if a \xcd"Double", or
\xcd"false" if a \xcd"Boolean".  It is a static error if the
default value is not a member of the type (e.g., it is a static
error to elide the initializer for \xcd"Int{self==1}").

If the variable is immutable,
the type may be omitted and
inferred from the initializer type (\Sref{TypeInference}).
Mutable fields must be declared with a type.

\begin{grammar}
FieldDeclaration
        \: FieldModifier\star \xcd"var" FieldDeclaratorsWithType \\&& ( \xcd"," FieldDeclaratorsWithType )\star \\
        \| FieldModifier\star \xcd"const" FieldDeclarators \\&& ( \xcd"," FieldDeclarators )\star \\
        \| FieldModifier\star \xcd"val" FieldDeclarators \\&& ( \xcd"," FieldDeclarators )\star \\
        \| FieldModifier\star FieldDeclaratorsWithType \\&& ( \xcd"," FieldDeclaratorsWithType )\star \\
FieldDeclarators
        \: FieldDeclaratorsWithType \\
        \: FieldDeclaratorWithInit \\
FieldDeclaratorId
        \: Identifier  \\
FieldDeclaratorWithInit
        \: FieldDeclaratorId Init \\
        \| FieldDeclaratorId ResultType Init \\
FieldDeclaratorsWithType
        \: FieldDeclaratorId ( \xcd"," FieldDeclaratorId )\star ResultType \\
FieldModifier \: Annotation \\
                \| \xcd"static" \\
\end{grammar}



\section{Properties}
Property declarations are always declared with a type and are
always immutable (either explicitly declared \xcd"val" or implicitly by default).

\begin{grammar}
Property
        \: PropertyModifier\star \xcd"val" Identifier ResultType \\
        \| PropertyModifier\star Identifier ResultType \\
PropertyModifier \: Annotation \\
\end{grammar}

\chapter{Names and packages}
\label{packages} \index{names}\index{packages}\index{public}\index{protected}\index{private}

\Xten{} supports mechanisms for names and packages in the style of Java
\cite[\S 6,\S 7]{jls2}, including \xcd"public", \xcd"protected", \xcd"private"
and package-specific access control.

\section{Packages}

A package is a named collection of top-level declarations, \viz, class,
interface, and struct declarations.  Package names are sequences of
identifiers, like \xcd`x10.lang` and \xcd`com.ibm.museum`. The multiple
names are simply a convenience.  Packages \xcd`a`, \xcd`a.b`, and \xcd`a.c`
have only a very tenuous relationship, despite the similarity of their names. 

Packages and protection modifiers determine which top-level names can be used
where. Only the \xcd`public` members of package \xcd`pack.age` can be accessed
outside of \xcd`pack.age` itself.  
%~~gen~~Stimulus
%
%~~vis
\begin{xten}
package pack.age;
class Deal {
  public def make() {}
}
public class Stimulus {
  private def taxCut() = true;
  protected def benefits() = true;
  public def jobCreation() = true;
  /*package*/ def jumpstart() = true;
}
\end{xten}
%~~siv
%
%~~neg

The class \xcd`Stimulus` can be referred to from anywhere outside of
\xcd`pack.age` by its full name of \xcd`pack.age.Stimulus`, or can be imported
and referred to simply as \xcd`Stimulus`.  The public \xcd`jobCreation()`
method of a \xcd`Stimulus` can be referred to from anywhere as well; the other
methods have smaller visibility.  The non-\xcd`public` class \xcd`Deal` cannot
be used from outside of \xcd`pack.age`.  



\subsection{Name Collisions}

It is a static error for a package to have two members, or apparent members,
with the same name.  For example, package \xcd`pack.age` cannot define two
classes both named \xcd`Crash`, nor a class and an interface with that name.

Furthermore, \xcd`pack.age` cannot define a member \xcd`Crash` if there is
another package named \xcd`pack.age.Crash`, nor vice-versa. (This prohibition
is the only actual relationship between the two packages.)  This prevents the
ambiguity of whether \xcd`pack.age.Crash` refers to the class or the package.  
Note that the naming convention that package names are lower-case and package
members are capitalized prevents such collisions.


\section{\xcd`import` Declarations}

Any public member of a package can be referred to from anywhere through a
fully-qualified name: \xcd`pack.age.Stimulus`.    

Often, this is too awkward.  X10 has two ways to allow code outside of a class
to refer to the class by its short name (\xcd`Stimulus`): single-type imports
and on-demand imports.   

Imports of either kind appear at the start of the file, immediately after the
\xcd`package` directive if there is one; their scope is the whole file.

\subsection{Single-Type Import}

The declaration \xcd`import ` {\em TypeName} \xcd`;` imports a single type
into the current namespace.  

%%BARD-HERE



\section{Type Names}

\begin{grammar}
TypeName   \: Identifier \\
        \| TypeName \xcd"." Identifier \\
        \| PackageName \xcd"." Identifier \\
PackageName   \: Identifier \\
        \| PackageName \xcd"." Identifier \\
\end{grammar}


While not enforced by the compiler, classes and interfaces
in the \Xten{} library support the following naming conventions.
Names of types---including classes,
type
parameters, and types specified by type definitions---are in
CamelCase and begin with an uppercase letter.  For backward
compatibility with languages such as C and \java{}, type
definitions are provided to allow primitive types
such as \xcd"int" and \xcd"boolean" to be written in lowercase.
Names of methods, fields, value properties, and packages are in camelCase and begin with a lowercase letter.
Names of \xcd"const" fields are in all uppercase with words
separated by an ``\xcd"_"''.



\chapter{Interfaces}
\label{XtenInterfaces}\index{interfaces}

{}\XtenCurrVer{} interfaces are generally modelled on Java interfaces \cite[\S
  9]{jls2}. An interface specifies signatures for public methods, properties,
\xcd`static val`s, and an invariant. It may extend several interfaces, giving
X10 a large fraction of the power of multiple inheritance at a tiny fraction
of the cost.

The following puny example illustrates all these features: 
%~~gen
%
%~~vis
\begin{xten}
interface Pushable(text:String, prio:Int) {
  def push(): Void;
  static val MAX_PRIO = 100;
}
class MessageButton(text:String, prio:Int) 
  implements Pushable{self.prio==Pushable.MAX_PRIO} {
  public def push() { 
    x10.io.Console.OUT.println(text + " pushed");
  }
}
\end{xten}
%~~siv
%
%~~neg
\noindent
\xcd`Pushable` defines two properties, a method, and a static value.  
\xcd`MessageButton` implements a constrained version of \xcd`Pushable`,
\viz\ one with maximum priority.  It also has \xcd`Pushable`'s properties.  It
defines the \xcd`push()` method given in the interface, as a \xcd`public`
method---interface methods are implicitly \xcd`public`.

A concrete type---a class or struct---can {\em implement} an interface,
typically by having all the methods and properties that the interface
requires.

A variable may be declared to be of interface type.  Such a variable has all
the fields and methods declared (directly or indirectly) by the interface;
nothing else is statically available.  Values of a concrete type which
implement the interface may be stored in the variable.  


\label{DepType:Interface}


\begin{grammar}
NormalInterfaceDeclaration \:
      InterfaceModifiers\opt \xcd"interface" Identifier  \\
   && TypePropertyList\opt PropertyList\opt Constraint\opt \\
   && ExtendsInterfaces\opt InterfaceBody \\
\end{grammar}
\noindent
The invariant associated with an interface is the conjunction of the
invariants associated with its superinterfaces and the invariant
defined at the interface. 

\begin{staticrule*}
The compiler declares an error if this constraint
is not consistent.  
\end{staticrule*}

Each interface implicitly defines a nullary getter method
\xcd"def p(): T" for each property \xcd"p: T". The interface may not have
another definition of a method \xcd`p()`. 



A class \xcd"C" is said to implement an interface \xcd"I" if
\begin{itemize}
\item \xcd`I`, or a subtype of \xcd`I`, appears in the \xcd`implements` list
      of \xcd`C`, 
\item \xcd`C`'s properties include all the properties of \xcd"I",
\item \xcd`C`'s class invariant $\mathit{inv}($\xcd"C"$)$ implies
$\mathit{inv}($\xcd"I"$)$.
\item Each method \xcd`m` defined by \xcd`I` is also a method of \xcd`C` --
      {\em with the \xcd`public`} modifier added.   These methods may be
      \xcd`abstract` if \xcd`C` is \xcd`abstract`.
\end{itemize}

\section{Field Definitions}

An interface may declare a \xcd`val` field, with a value.  This field is implicitly
\xcd`public static val`: 
%~~gen
% package Interface.Field;
%~~vis
\begin{xten}
interface KnowsPi {
  PI = 3.14159265358;
}
\end{xten}
%~~siv
%
%~~neg

Classes and structs implementing such an interface get the interface's fields as
\xcd`public static` fields.  Unlike properties and methods, there is no need
for the implementing class to declare them. 
%~~gen
% package Interface.Field.Two;
% interface KnowsPi {PI = 3.14159265358;}
%~~vis
\begin{xten}
class Circle implements KnowsPi {
  static def area(r:Double) = PI * r * r;
}
\end{xten}
%~~siv
%
%~~neg

\subsection{Fine Points of Fields}

It can happen that two parent interfaces give fields of the same name.  In
that case, those fields must be referred to by qualified names.
%~fails~gen
%
%~fails~vis
\begin{xten}
interface E1 {static val a = 1;}
interface E2 {static val a = 2;}
interface E3 extends E1, E2{}
class Example implements E3 {
  def example() = E1.a + E2.a;
}
\end{xten}
%~fails~siv
%
%~fails~neg

If the {\em same} field \xcd`a` is inherited through many paths, there is no need to
disambiguate it:
%~fails~gen
% package Interfaces.Mult.Inher.Field;
%~fails~vis
\begin{xten}
interface I1 { static val a = 1;} 
interface I2 extends I1 {}
interface I3 extends I1 {}
interface I4 extends I2,I3 {}
class Example implements I4 {
  def example() = a;
}
\end{xten}
%~fails~siv
%
%~fails~neg

\section{Interfaces Specifying Properties}

Interfaces may specify properties.  


\bard{Methods}
\bard{Properties}
\bard{Invariannt}



\chapter{Classes}
\label{XtenClasses}\index{class}

The {\em class declaration} has a list of type \params, properties, a
constraint (the {\em class invariant}), a single superclass, one or
more interfaces, and a class body containing the the definition of
fields, properties, methods, and member types.  Each such declaration introduces a
class type (\Sref{ReferenceTypes}).


\begin{grammar}
NormalClassDeclaration \:
      ClassModifiers\opt \xcd"class" Identifier  \\
   && TypeParameterList\opt PropertyList\opt Guard\opt \\
   && Super\opt Interfaces\opt ClassBody \\
\\
TypeParameterList     \:  \xcd"[" TypeParameters \xcd"]" \\
TypeParameters        \:  TypeParameter ( \xcd"," TypeParameter )\star \\
TypeParameter         \:  Variance\opt Annotation\star Identifier     \\
Variance \: \xcd"+" \\
         && \xcd"-" \\
\\
PropertyList     \:  \xcd"(" Properties \xcd")" \\
Properties       \:  Property ( \xcd"," Property )\star \\
Property         \:  Annotation\star \xcd"val"\opt Identifier \xcd":" Type \\
\\
Super \: \xcd"extends" ClassType \\
Interfaces \: \xcd"implements" InterfaceType ( \xcd"," InterfaceType)\star \\
\\
ClassBody \: ClassMember\star \\
ClassMember \: ClassDeclaration \\
            \| InterfaceDeclaration \\
            \| FieldDeclaration \\
            \| MethodDeclaration \\
            \| ConstructorDeclaration \\
\end{grammar}

A type parameter declaration is given by an optional variance tag and an identifier.
A type parameter must be bound to a concrete type when an instance of the class is created.


A property has a name and a type.   Properties
are accessible in the same way as \xcd"public" \xcd"final"
fields.

\begin{staticrule*}
It is a compile-time error for a class
defining a property \xcd"x: T" to have an ancestor class that defines
a property with the name \xcd"x".  
\end{staticrule*}

Each class \xcd"C" defining a property \xcd"x: T" implicitly has a field

\begin{xten}
public val x : T;
\end{xten} 

\noindent and a getter method

\begin{xten}
public final def x()=x;
\end{xten}

\noindent Each interface \xcd"I" defining a property \xcd"x: T"
implicitly has a getter method

\begin{xten}
public def x(): T;
\end{xten}

\begin{staticrule*}
It is a compile-time error for a class or
interface defining a property \xcd"x :T" to have an existing method with
the signature \xcd"x(): T".
\end{staticrule*}

Properties are used to build dependent types from classes, as
described in \Sref{DepType:DepType}.

Properties are initialized by the invocation of a special \Xcd{property} call in each constructor
of the class:
\begin{xten}
property(e1,..., en);
\end{xten}
The number and type of arguments to the \Xcd{property} call must match the number
and type of properties in the class declaration, in left to right lexical order. Each constructor is required to initialize its properties before normal termination.
\index{property!call}
\index{property!initialization}
\label{PropertyCall}

\label{ClassGuard}

The \grammarrule{Guard} in a class or interface declaration specifies an
explicit condition on the properties of the type, and is discussed further
in \Sref{DepType:ClassGuard}.

\begin{staticrule*}
     Every constructor for a class defining
   properties \xcdmath"x$_1$: T$_1$, $\ldots$, x$_n$: T$_n$" must ensure that each of the fields
   corresponding to the properties is definitely initialized
   (cf. requirement on initialization of final fields in Java) before the
   constructor returns.
\end{staticrule*}

Type \params{} are used to define generic classes and
interfaces, as described in \Sref{Generics}.

Classes are structured in a single-inheritance code
hierarchy, may implement multiple interfaces, may have static and
instance fields, may have static and instance methods, may have
constructors, may have static and instance initializers, may have
static and instance inner classes and interfaces. \Xten{} does not
permit mutable static state.

Method signatures may specify checked exceptions. Method definitions
may be overridden by subclasses; the overriding definition may have a
declared return type that is a subclass of the return type of the
definition being overridden. Multiple methods with the same name but
different signatures may be provided on a class (ad hoc
polymorphism). The \Xcd{public}/\Xcd{private}/\Xcd{protected}/default-protected access
modification framework may be used.


\oldtodo{Figure out class modifiers. Figure out which new ones need to be added to support IEEE FP.}

\index{class}\label{ReferenceClasses}

Class declarations may
be used to construct class types (\Sref{ReferenceTypes}). Classes may
have mutable fields. Instances of a class are always created in a
fixed place and in \XtenCurrVer{} stay there for the lifetime of the
object.  Variables declared at a class type always store a reference
to the object, regardless of whether the object is local or remote.


\section{Type invariants}\label{DepType:ClassGuard}
\index{type invariants}
\index{class invariants}
\index{guards}

There is a general recipe for constructing a list of parameters or
properties \xcdmath"x$_1$: T$_1${c$_1$}, $\dots$, x$_k$: T$_k${c$_k$}" that must satisfy a given
(satisfiable) constraint \xcd"c". 

\begin{xtenmath}
class Foo(x$_1$: T1{x$_2$: T$_2$; $\dots$; x$_k$: T$_k$; c},
          x$_2$: T2{x$_3$: T$_3$; $\dots$; x$_k$: T$_k$; c},
          $\dots$
          x$_k$: T$_k${c}) {
  $\dots$
}
\end{xtenmath}

The first type \xcdmath"x$_1$: T$_1${x$_2$: T$_2$; $\dots$; x$_k$: T$_k$; c}" is consistent iff
\xcdmath"$\exists$x$_1$: T$_1$, x$_2$: T$_2$, $\dots$, x$_k$: T$_k$. c" is consistent. The second is
consistent iff
\begin{xtenmath}
$\forall$x$_1$: T$_1${x$_2$: T$_2$; $\dots$; x$_k$: T$_k$; c}
$\exists$x$_2$: T$_2$. $\exists$x$_3$: T$_3$, $\dots$, x$_k$: T$_k$. c
\end{xtenmath}
\noindent But this is always true. Similarly for the conditions for the other
properties.

Thus logically every satisfiable constraint \xcd"c" on a list of parameters
\xcdmath"x$_1$", \dots, \xcdmath"x$_k$"
can be expressed using the dependent types of 
\xcdmath"x$_i$", provided
that the constraint language is rich enough to permit existential
quantifiers.

Nevertheless we will find it convenient to permit the programmer to
explicitly specify a depclause after the list of properties, thus:
\begin{xten}
class Point(i: Int, j: Int) { ... }
class Line(start: Point, end: Point){end != start} { ... }
class Triangle (a: Line, b: Line, c: Line)
   {a.end == b.start,  b.end == c.start,
    c.end == a.start} { ... }
\end{xten}

Consider the definition of the class \xcd"Line". This may be thought of as
saying: the class \xcd"Line" has two fields, \xcd"start: Point" and
\xcd"end: Point".
Further, every instance of \xcd"Line" must satisfy the constraint that
\xcd"end != start". Similarly for the other class definitions. 

In the general case, the production for \grammarrule{NormalClassDeclaration}
specifies that the list of properties may be followed by a
\grammarrule{Guard}.

\begin{grammar}
NormalClassDeclaration \:
      ClassModifiers\opt \xcd"class" Identifier  \\
   && TypeParameterList\opt PropertyList\opt Guard\opt \\
   && Extends\opt Interfaces\opt ClassBody \\
\\
NormalInterfaceDeclaration \:
      InterfaceModifiers\opt \xcd"interface" Identifier  \\
   && TypeParameterList\opt PropertyList\opt Guard\opt \\
   && ExtendsInterfaces\opt InterfaceBody \\
\end{grammar}

All the properties in the list, together with inherited properties,
may appear in the \grammarrule{Guard}. A guard \xcd"c" with
property list \xcdmath"x$_1$: T$_1$, $\dots$, x$_n$: T$_n$"
for a class \xcd"C" is said to be consistent if each of the \xcdmath"T$_i$" are
consistent and the constraint
\begin{xtenmath}
$\exists$x$_1$: T$_1$, $\dots$, x$_n$: T$_n$, self: C. c
\end{xtenmath}
\noindent is valid (always true).

\label{DepType:TypeInvariant}
\index{Class invariant}
\label{DepType:ClassGuardDef}

The guard is an invariant on all instances of the class or interface.

With every defined class or interface \xcd"T" we associate a {\em type
invariant} $\mathit{inv}($\xcd"T"$)$ as follows. The type
invariant associated with \xcd"x10.lang.Object" is 
\xcd"true".

The type invariant associated with any interface \xcd"I" that extends
interfaces \xcdmath"I$_1$, $\dots$, I$_k$" and defines properties
\xcdmath"x$_1$: P$_1$, $\dots$, x$_n$: P$_n$" and
specifies a guard \xcd"c" is given by:

\begin{xtenmath}
$\mathit{inv}$(I$_1$), $\dots$, $\mathit{inv}$(I$_k$),
    self.x$_1$: P$_1$,  $\dots$,  self.x$_n$: P$_n$, c  
\end{xtenmath}

Similarly the type invariant associated with any class \xcd"C" that
implements interfaces \xcdmath"I$_1$, $\dots$, I$_k$",
extends class \xcd"D" and defines properties
\xcdmath"x$_1$: P$_1$, $\dots$, x$_n$: P$_n$" and
specifies a guard \xcd"c" is
given by:
\begin{xtenmath}
$\mathit{inv}$(D), $\mathit{inv}$(I$_1$),  $\dots$, $\mathit{inv}$(I$_k$),
    self.x$_1$: P$_1$,  $\dots$, self.x$_n$: P$_n$,  c  
\end{xtenmath}

It is required that the type invariant associated with a class entail
the type invariants of each interface that it implements.

It is guaranteed that for any variable \xcd"v" of
type \xcd"T{c}" (where \xcd"T" is an interface name or a class name) the only
objects \xcd"o" that may be stored in \xcd"v" are such that \xcd"o" satisfies
$\mathit{inv}(\mbox{\xcd"T"}[\mbox{\xcd"o"}/\mbox{\xcd"this"}])
\wedge \mbox{\xcd"c"}[\mbox{\xcd"o"}/\mbox{\xcd"self"}]$.

\section{\Xcd{implements} and \Xcd{extends} clauses}\label{DepType:Implements}
\label{DepType:Extends}
\index{type-checking!implements clause}
\index{type-checking!extends clause}
\index{implements clause}
\index{extends  clause}
Consider a class definition
\begin{xtenmath}
$\mbox{\emph{ClassModifiers}}^{\mbox{?}}$
class C(x$_1$: P$_1$, $\dots$, x$_n$: P$_n$) extends D{d}
   implements I$_1${c$_1$}, $\dots$, I$_k${c$_k$}
$\mbox{\emph{ClassBody}}$
\end{xtenmath}

Each of the following static semantics rules must be satisfied:

\begin{staticrule}{Int-implements}
The type invariant \xcdmath"$\mathit{inv}$(C)" of \xcd"C" must entail
\xcdmath"c$_i$[this/self]" for each $i$ in $\{1, \dots, k\}$
\end{staticrule}

\begin{staticrule}{Super-extends}
The return type \xcd"c" of each constructor in \grammarrule{ClassBody}
must entail \xcd"d".
\end{staticrule}

\section{Constructor definitions}

A constructor for a class \xcd"C" is guaranteed to return an object of the
class on successful termination. This object must satisfy  \xcdmath"$\mathit{inv}$(C)", the
class invariant associated with \xcd"C" (\Sref{DepType:TypeInvariant}).
However,
often the objects returned by a constructor may satisfy {\em stronger}
properties than the class invariant. \Xten{}'s dependent type system
permits these extra properties to be asserted with the constructor in
the form of a constrained type (the ``return type'' of the constructor):

\begin{grammar}
ConstructorDeclarator \:
  \xcd"def" \xcd"this" TypeParameterList\opt \xcd"(" FormalParameterList\opt \xcd")" \\
  && ReturnType\opt Guard\opt Throws\opt \\
ReturnType    \: \xcd":" Type \\
Guard   \: "\{" DepExpression "\}" \\
Throws    \: \xcd"throws" ExceptionType  ( \xcd"," ExceptionType )\star \\
ExceptionType \: ClassBaseType Annotation\star \\
\end{grammar}

\label{ConstructorGuard}

The parameter list for the constructor
may specify a \emph{guard} that is to be satisfied by the parameters
to the list.

\begin{example}
Here is another example, constructed as a simplified 
version of \Xcd{x10.lang.Region}.
\begin{xten}
type MyRegion(n:Int)=MyRegion{self.rank==n};
class MyRegion(rank:Int) {
  def this(r:Int):MyRegion(r) {
    property(r);
  }
  def this(diag:ValRail[Int]):MyRegion(diag.length){ 
   ...
  }
  def union(r:MyRegion(n)):MyRegion(n) { ...}
  ...
}
}
\end{xten}
The first constructor returns the empty region of rank \Xcd{r}.  The
second constructor takes a \Xcd{ValRail[Int]} of arbitrary length
\Xcd{n} and returns a \Xcd{MyRegion(n)} (intended to represent the set
of points in the rectangular parallelopiped between the origin and the
\Xcd{diag}.)

Now the following code type checks:
\begin{xten}
val R1 = new MyRegion([4,4,4]); // R1's type is MyRegion(3)
val R2 = new MyRegion([5,4,1]); // R2's type is MyRegion(3)
\end{xten}
Hence the following code type checks and infers that \Xcd{R3}'s type is
\Xcd{MyRegion(3)}:
\begin{xten}
val R3 = R1.union(R2);          // R3's type is MyRegion(3)
\end{xten}
\end{example}

\begin{staticrule}{Super-invoke}
   Let \xcd"C" be a class with properties
   \xcdmath"p$_1$: P$_1$, $\dots$, p$_n$: P$_n$", invariant \xcd"c"
   extending the constrained type \xcd"D{d}" (where \xcd"D" is the name of a class).

   For every constructor in \xcd"C" the compiler checks that the call to
   super invokes a constructor for \xcd"D" whose return type is strong enough
   to entail \xcd"d". Specifically, if the call to super is of the form 
     \xcdmath"super(e$_1$, $\dots$, e$_k$)"
   and the static type of each expression \xcdmath"e$_i$" is
   \xcdmath"S$_i$", and the invocation
   is statically resolved to a constructor
\xcdmath"def this(x$_1$: T$_1$, $\dots$, x$_k$: T$_k$){c}: D{d$_1$}"
   then it must be the case that 
\begin{xtenmath}
x$_1$: S$_1$, $\dots$, x$_i$: S$_i$ $\vdash$ x$_i$: T$_i$  (for $i \in \{1, \dots, k\}$)
x$_1$: S$_1$, $\dots$, x$_k$: S$_k$ $\vdash$ c  
d$_1$[a/self], x$_1$: S$_1$, ..., x$_k$: S$_k$ $\vdash$ d[a/self]      
\end{xtenmath}
\noindent where \xcd"a" is a constant that does not appear in 
\xcdmath"x$_1$: S$_1$ $\wedge$ ... $\wedge$ x$_k$: S$_k$".
\end{staticrule}

\begin{staticrule}{Constructor return}
   The compiler checks that every constructor for \xcd"C" ensures that
   the properties \xcdmath"p$_1$,..., p$_n$" are initialized with values which satisfy
   \xcdmath"t(C)", and its own return type \xcd"c'" as follows.  In each constructor, the
   compiler checks that the static types \xcdmath"T$_i$" of the expressions \xcdmath"e$_i$"
   assigned to \xcdmath"p$_i$" are such that the following is
   true:
\begin{xtenmath}
p$_1$: T$_1$, $\dots$, p$_n$: T$_n$ $\vdash$ t(C) $\wedge$ c'     
\end{xtenmath}
\end{staticrule}
(Note that for the assignment of \xcdmath"e$_i$" to \xcdmath"p$_i$"
to be type-correct it must be the
    case that \xcdmath"p$_i$: T$_i$ $\wedge$ p$_i$: P$_i$".) 


\begin{staticrule}{Constructor invocation}
The compiler must check that every invocation \xcdmath"C(e$_1$, $\dots$, e$_n$)" to a
constructor is type correct: each argument \xcdmath"e$_i$" must have a static type
that is a subtype of the declared type \xcdmath"T$_i$" for the $i$th
argument of the
constructor, and the conjunction of static types of the argument must
entail the \grammarrule{Guard} in the parameter list of the constructor.
\end{staticrule}

\section{\Xcd{ proto} qualifier on types}
\label{Prototypes}
\index{proto}
\Xten{} ensures that every variable must have a value consistent with its type
before it is read.

For local variables, this is ensured by using a pre-specified static
analysis to ensure that every local variable is written into before it
is read. Type-checking of assignment ensures the value written is
consistent with the static type of the variable.

For fields, this is ensured by introducing a form of ownership types
called {\em incomplete types} to address the {\em escaping-this}
problem.  To permit flexibility in writing constructors, \Xten{} v1.7
permits \Xcd{this} to be used in a constructor as a reference to the
object currently being constructed. Unfortunately there are no
restrictions on the usage of \Xcd{this}. In particular, this reference
can be permitted to escape: it may be stored in variables on the heap
(thereby permitting concurrently executing activities to read the
value of fields that may not yet have been initialized), passed as an
argument to method invocations, or used as the target for a method
invocation. Indeed, the method may be invoked in a super constructor,
and may have been overridden at a subclass, guaranteeing that accesses
to fields defined in the subclass are accesses to uninitialized
variables. For instance an immutable field may be observed containing
a value (the value the field was initialized with) which may be
different from the value it will contain once the constructor has
returned.

Incomplete types are designed with the following goals:
 \begin{itemize}

\item Guarantee that fields are not read before they are initialized.

\item  Allow the creation of immutable cyclic object graphs.\footnote{(Mutable
       graphs can be created without escaping \Xcd{this} 
    by initializing the backpointer to \Xcd{null} and then
    changing it later.}  This requires that it be possible to pass an
  object under construction into a constructor invocation.

\item Allow appropriate user-defined methods can be called during object
  creation (so that the transformation between the values supplied as
  parameters to a constructor and the values actually placed in fields
  is determined by arbitrary user-defined code).

\item Keep the design minimally invasive. Most programmers should
  not have to be concerned about this problem.

\item Ensure that there is no runtime overhead. 
\end{itemize}

These goals are met by introducing incomplete types through the type
qualifier \Xcd{proto}. Types of the form \Xcd{proto T} are said to be
{\em incomplete types}; types that do not have the qualifier are said to be
{\em complete}.
 Say that an object \Xcd{o} is {\em confined}
to a given activity \Xcd{A} if it can be reached only from stack
frames of \Xcd{A} or from objects which are, recursively, confined to
\Xcd{A}. Thus confined objects cannot be accessed by activities other
than \Xcd{A}.

Incomplete types ensure that objects whose construtors have not exited
are confined. Further, all references to such objects on the stack
are contained in variables of incomplete types. The compiler does not
permit the fields of variables of incomplete types to be read. 
Thus incomplete types permit the construction of graphs of objects
while ensuring that these objects are confined and their fields are not
read during construction. 

The return value of a constructor for class \Xcd{C} that takes no
incomplete arguments is (a subtype of) \Xcd{C}, that is, a complete
type. It will point only to completed objects. It can now be
assigned to any (type-consistent) field of any object, that is, it is
now allowed to escape.

\subsubsection{\Xcd{proto} Rules}
\label{protorules}
\label{ProtoRules}
\index{proto!rules}

For every type \Xcd{T} (where \Xcd{T} is not a type variable), we
introduce the type \Xcd{proto T}. 

There is no relationship between types \Xcd{T} and \Xcd{proto T} --
neither is a subtype of the other.\footnote{Clearly, a value of type
  \Xcd{proto T} cannot be used anywhere that a \Xcd{T} is needed,
  since its fields cannot be read.  As discussed below, an incomplete
  value \Xcd{v} can be assigned to a field {\tt f} of an object {\tt
    o} only if {\tt o} is incomplete. This ensures that \Xcd{v} cannot
  escape through this assignment. A completed value \Xcd{p} cannot be
  substituted for \Xcd{o} -- it may permit \Xcd{v} to escape through
  an assignment to its field. Therefore \Xcd{T} cannot be a subtype of
  \Xcd{proto T}.}

Incomplete types are permitted to occur only as types of method
parameters or local variables or as return types for methods and
constructors. They may not occur in (the source or target of) cast
statements, \Xcd{extends} or \Xcd{implements} clauses, \Xcd{catch}
clauses, or as types of class fields.

Within the body of a class \Xcd{C} the type of \Xcd{this} in
constructors, instance initializers and instance variable initializers
is \Xcd{proto C}.

Let \Xcd{v} be a value of type \Xcd{proto C}, for some class \Xcd{C}. 

No fields of \Xcd{v} can be read.  (This is the defining property of
\Xcd{proto} types.)  However, \Xcd{v}'s (accessible) instance fields
can be assigned.

\Xcd{v} can be assigned to an instance field \Xcd{o.f} only if \Xcd{f}
is of some type \Xcd{S} such that \Xcd{T <: S} and \Xcd{o} has an
incomplete type.

\Xcd{v} can be assigned to local variables  only if they are of some type
\Xcd{proto S} (such that \Xcd{T <: S}).

Instance methods of class \Xcd{C} may be qualified with \Xcd{proto}
(these methods are called {\em incomplete methods}). The type of \Xcd{this}
in incomplete methods is \Xcd{proto C}. Incomplete methods can be
overridden only by incomplete methods.  Only incomplete methods can be
invoked on \Xcd{v}. Incomplete methods which do not take an argument of incomplete
type can be invoked on completed values. 


\Xcd{v} can be passed as argument into a constructor or method call,
or returned from a method.  The return type of a method taking an
argument at an incomplete type must be \Xcd{void} or incomplete.  The
return type of a constructor taking an argument at a \Xcd{proto} type
must be incomplete.
 
A generic class (method) type parameter \Xcd{T} can be
  instantiated with the type \Xcd{proto S} (where \Xcd{S} is not a type
  parameter itself), provided that the class (method) body satisfies
  the conditions above for \Xcd{proto S}.

During code generation, the type \Xcd{proto T} is treated as if it were
\Xcd{T}. That is, there is no run-time cost to \Xcd{proto} types.

The invariants maintained by the design are as follows.  Say that an
object field or stack variable (local variable) contains an incomplete
value if a value of type \Xcd{proto T} (for some \Xcd{T}) was written
into it.

\begin{itemize}
\item If an object \Xcd{o} has a field containing an incomplete value \Xcd{v},
then either \Xcd{v}'s constructor has exited or \Xcd{o} is confined.
Further, every reference to \Xcd{o} on the stack is held at an incomplete type.

\item If a stack variable contains an incomplete value, then
  the variable's type is incomplete.

\end{itemize}

Say that a constructor invocation for a class \Xcd{C} on the call
stack is a {\em root} if it takes no incomplete arguments. Such a
constructor invocation will return an object of type \Xcd{C} whose
fields may point to an arbitrary graph of newly created objects
(objects created by the activity after the constructor
invocation). Since the object returned is at type \Xcd{C} -- and not
\Xcd{proto C} -- It may be assigned to any field of any object on the
heap of type \Xcd{D} such that \Xcd{C <: D}.  It is no longer
confined. Thus the ``magic moment'' when an incomplete value becomes
complete is when the last constructor for any incomplete value it
references (including itself) returns.

\subsubsection{Example}

\begin{example}
This example shows how to create a fixed-size circular buffer.
(Its pointer structure is immutable, though the contents of each
field are mutable.)
{\footnotesize
\begin{xten}
class  CircularBuffer[A] {
  var a: A;
  val next: CircularBuffer[A];
  private def this(x: proto CircularBuffer[A]): proto CircularBuffer[A] {
    next = x;
  }
  def this(var n:Int) {
    var temp: proto CircularBuffer[A] = this;
    while (--n > 0) 
    temp = new CircularBuffer[A](temp);
    next = temp;
  }
}
\end{xten}}

\end{example}
\section{Field definitions}

A class may have zero or more mutable or immutable fields. 
No two fields declared in a class may have the same name. 

Fields may be marked \xcd{static}. Only one instance of such a field
exists, and it may be accessed through the name of the class in which
it is defined (\Sref{StaticInitialization}).  Fields not marked
\xcd{static} are said to be {\em instance} fields. One copy of such a
field exists for every instance of the class.

To avoid an ambiguity, it is a static error for a class to
declare a field with a function type (\Sref{FunctionTypes}) with
the same name and signature  as a method of the same class.

\subsection{Field hiding}

A subclass that defines a field \xcd"f" hides any field \xcd"f"
declared in a superclass, regardless of their types.  The
superclass field \xcd"f" may be accessed within the body of
the subclass via the reference \xcd"super.f".

\subsection{Field qualifiers}
\label{FieldQualifier}
\index{Qualifier!field}

\subsubsection{\Xcd{global} qualifier}
\index{global!field}
\label{GlobalField}

A field may be declared \xcd"global".

\begin{grammar}
  FieldModifier \: \xcd"global"  
\end{grammar}

A \Xcd{global} field must be immutable. It may be read from any place.
Properties and static fields are implicitly marked \Xcd{global}.
Fields not marked \Xcd{global} cannot be overridden by fields marked
\Xcd{global}.

\section{Method definitions}

\Xten{} permits guarded method definitions. 

\begin{grammar}
MethodDeclaration \: MethodHeader \xcd";" \\
                  \| MethodHeader \xcd"=" ClosureBody \\
MethodHeader \:  
  MethodModifiers\opt \xcd"def" Identifier TypeParameters\opt \\
&& \xcd"(" 
  FormalParameterList\opt \xcd")" Guard\opt \\
  && ReturnType\opt Throws\opt \\
\end{grammar}

A formal parameter may optionally have a \xcd"val" or \xcd"var"
modifier (default: \xcd"val").  
The body of the method is executed in an environment in which 
each formal parameter corresponds to a local variable
and is initialized with the value of the actual parameter.
The local variable  is mutable if and only if the
parameter is a \xcd"var" parameter.

\label{MethodGuard}

The guard (specified by \grammarrule{Guard})
specifies a constraint \xcd"c" on the
properties of the class \xcd"C" on which the method is being defined. The
method exists only for those instances of \xcd"C" which satisfy \xcd"c".  It is
illegal for code to invoke the method on objects whose static type is
not a subtype of \xcd"C{c}".

\begin{staticrule*}
    The compiler checks that every method invocation
    \xcdmath"o.m(e$_1$, $\dots$, e$_n$)"
    for a method is type correct. Each argument
    \xcdmath"e$_i$" must have a
    static type \xcdmath"S$_i$" that is a subtype of the declared type
    \xcdmath"T$_i$" for the $i$th
    argument of the method, and the conjunction of static types
    of the arguments must entail the guard in the parameter list
    of the method.

    The compiler checks that in every method invocation
    \xcdmath"o.m(e$_1$, $\dots$, e$_n$)"
    the static type of \xcd"o", \xcd"S", is a subtype of \xcd"C{c}", where the method
    is defined in class \xcd"C" and the guard for \xcd"m" is equivalent to
    \xcd"c".

    Finally, if the declared return type of the method is
    \xcd"D{d}", the
    return type computed for the call is
    \xcdmath"D{a: S; x$_1$: S$_1$; $\dots$; x$_n$: S$_n$; d[a/this]}",
    where \xcd"a" is a new
    variable that does not occur in
    \xcdmath"d, S, S$_1$, $\dots$, S$_n$", and
    \xcdmath"x$_1$, $\dots$, x$_n$" are the formal
    parameters of the method.
\end{staticrule*}
The method body is either an expression, a block of statements,
or a block ending with an expression.

\begin{example}
Consider the program:
\begin{xten}
type Point(r:Int)=Point{self.rank==r};
final public class Point(rank: Int) implements (Int) => Int {
    public global val coords: ValRail[Int](rank);
    public global safe def apply(i: Int) = coords(i);
    public global safe def coords() = coords;
    public global safe operator - this: Point(rank) 
       = Point.make(rank, (i:Int)=>-this.coords(i));
    public global safe operator this + (that: Point(rank)): Point(rank) 
       = Point.make(rank, (i:Int)=> this.coords(i) + that.coords(i));
    ...
}
\end{xten}

The following code fragment will typecheck:
\begin{xten}
s: Point(3) = new Point([1,2,3]);
t: Point(3) = new Point([-1,-1,-1]);
u: Point(3) = s + u;
\end{xten}
\end{example}

\subsection{Property methods}

A method declared with the modifier \xcd"property" may be used
in constraints.  A property method declared in a class must have
a body and must not be \xcd"void".  The body of the method must
consist of only a single \xcd"return" statement or a single
expression.  It is a static error if the expression cannot be
represented in the constraint system. 

The expression may contain invocations of other properties. It is the
responsibility of the programmer to ensure that the evaluation of 
a property terminates at compile-time, otherwise the type-checker
will not terminate and the program will fail to compile.

Property methods in classes are implicitly \xcd"final"; they cannot be
overridden.

A property method definition may omit the formal parameters and
the \xcd"def" keyword.  That is, the following are equivalent:

{\footnotesize
\begin{xten}
property def rail(): boolean = rect && onePlace == here && zeroBased;
property rail: boolean = rect && onePlace == here && zeroBased;
\end{xten}}

\subsection{Method overloading, overriding, hiding, shadowing and obscuring}
\label{MethodOverload}

The definitions of method overloading, overriding, hiding, shadowing
and obscuring in \Xten{} are the same as in \Java, modulo the following
considerations motivated by type parameters and dependent types.

Two or more methods of a class or interface may have the same
name if they have a different number of type parameters, or
they have value parameters of different types.

\XtenCurrVer{} does not permit overloading based on constraints. 


The definition of a method declaration \xcdmath"m$_1$" ``having the same signature
as'' a method declaration \xcdmath"m$_2$" involves identity of types. 

The {\em constraint erasure} of a type \xcdmath"T" is defined as follows.
The constraint erasure of  (a)~a class, interface or struct type \xcdmath"T" is 
\xcdmath"T"; (b)~a type \xcdmath"T{c}" is the constraint erasure of 
\xcdmath"T"; (b)~a type \xcdmath"T[S$_1$,\ldots,S$_n$]" 
is \xcdmath"T'[S$_1$',\ldots,S$_n$']" where each primed type is the erasure of 
the corresponding unprimed type.
 Two methods are said to have {\em the
  same signature} if (a) they have the same number of type parameters,
(b) they have the same number of formal (value) parameters, and (c)
for each formal parameter the constraint erasure of its types are equivalent. It is a
compile-time error for there to be two methods with the same name and
same signature in a class (either defined in that class or in a
superclass).

\begin{staticrule*}
  A class \xcd"C" may not have two declarations for a method named \xcd"m"---either
  defined at \xcd"C" or inherited:
\begin{xtenmath}
def m[X$_1$, $\dots$, X$_m$](v$_1$: T$_1$, $\dots$, v$_n$: T$_n$){tc}: T {...}
def m[X$_1$, $\dots$, X$_m$](v$_1$: S$_1$, $\dots$, v$_n$: S$_n$){sc}: S {...}
\end{xtenmath}
\noindent
if it is the case that the constraint erasures of the types \xcdmath"T$_1$",
\dots, \xcdmath"T$_n$" are
equivalent to the constraint erasures of the types \xcdmath"S$_1$, $\dots$, T$_n$"
respectively.
\end{staticrule*}

In addition, the guard of a overriding method must be 
no stronger than the guard of the overridden method.   This
ensures that any virtual call to the method
satisfies the guard of the callee.

\begin{staticrule*}
  If a class \xcd"C" overrides a method of a class or interface
  \xcd"B", the guard of the method in \xcd"B" must entail
  the guard of the method in \xcd"C".
\end{staticrule*}

A class \xcd"C" inherits from its direct superclass and superinterfaces all
their methods visible according to the access modifiers
of the superclass/superinterfaces that are not hidden or overridden. A method \xcdmath"M$_1$" in a class
\xcd"C" overrides
a method \xcdmath"M$_2$" in a superclass \xcd"D" if
\xcdmath"M$_1$" and \xcdmath"M$_2$" have the same signature.
Methods are overriden on a signature-by-signature basis.

A method invocation \xcdmath"o.m(e$_1$, $\dots$, e$_n$)"
is said to have the {\em static signature}
\xcdmath"<T, T$_1$, $\dots$, T$_n$>" where \xcd"T" is the static type of
\xcd"o", and
\xcdmath"T$_1$",
\dots,
\xcdmath"T$_n$"
are the static types of \xcdmath"e$_1$", \dots, \xcdmath"e$_n$",
respectively.  As in
\Java, it must be the case that the compiler can determine a single
method defined on \xcd"T" with argument type
\xcdmath"T$_1$", \dots \xcdmath"T$_n$"; otherwise, a
compile-time error is declared. However, unlike \Java, the \Xten{} type \xcd"T"
may be a dependent type \xcd"C{c}". Therefore, given a class definition for
\xcd"C" we must determine which methods of \xcd"C" are available at a type
\xcd"C{c}". But the answer to this question is clear: exactly those methods
defined on \xcd"C" are available at the type \xcd"C{c}"
whose guard \xcd"d" is implied by \xcd"c".

\subsection{Method qualifiers}
\label{MethodQualifier}
\index{Qualifier!method}

\subsubsection{\Xcd{atomic} qualifier}
\label{AtomicAnnotation}
\index{atomic}

A method may be declared \xcd"atomic".

\begin{grammar}
  MethodModifier \: \xcd"atomic"  
\end{grammar}

Such a method is treated as if the statement in its body is wrapped 
implicitly in an \xcd"atomic" statement.

\subsubsection{\Xcd{global} qualifier}
\label{LocalAnnotation}
\label{GlobalMethod}
\index{global!method}

A method may be declared \xcd"global".

\begin{grammar}
  MethodModifier \: \xcd"global"  
\end{grammar}

A \xcd"global" method can be invoked on an object \Xcd{o} in any place. The
body of such a method is type-checked without assuming that
\Xcd{here==this.home}. This permits \Xcd{global} fields of \Xcd{o} to
be accessed, but not local fields. The programmer must insert an explicit
\Xcd{at(this)...} to get to the place where the object lives and access
the field.

\Xcd{global} methods can be overridden only by methods also marked \Xcd{global}.

\subsubsection{\Xcd{pinned} qualifier}
\label{PinnedAnnotation}
\label{PinnedMethod}
\index{pinned!method}

A method may be declared \xcd"pinned".

\begin{grammar}
  MethodModifier \: \xcd"pinned"  
\end{grammar}

A \Xcd{pinned} method may not
contain any \Xcd{at} statement or expression whose place argument
is not statically equivalent to \Xcd{here}. It must call only
\Xcd{pinned} methods. That is, a \Xcd{pinned} method does not cause
any communication.

\Xcd{pinned} methods can be overridden only by methods marked \Xcd{pinned}.

\subsubsection{\Xcd{nonblocking} qualifier}
\label{NonblockingAnnotation}
\label{NonblockingMethod}
\index{nonblocking!method}

A method may be declared \xcd"nonblocking".

\begin{grammar}
  MethodModifier \: \xcd"nonblocking"  
\end{grammar}

A \Xcd{nonblocking} method may not
contain any \Xcd{when} statement whose condition
is not statically equivalent to \Xcd{true}. It must call only
\Xcd{nonblocking} methods. That is, a \Xcd{nonblocking} method does not block.

\Xcd{nonblocking} methods can be overridden only by methods marked \Xcd{nonblocking}.

\subsubsection{\Xcd{sequential} qualifier}
\label{SequentialAnnotation}
\label{SequentialMethod}
\index{sequential!method}

A method may be declared \xcd"sequential".

\begin{grammar}
  MethodModifier \: \xcd"sequential"  
\end{grammar}

A \Xcd{sequential} method may not contain any \Xcd{async}
statement. It must call only \Xcd{sequential} methods. That is, a
\Xcd{sequential} method does not spawn any activity.

\Xcd{sequential} methods can be overridden only by methods marked \Xcd{sequential}.

\subsubsection{\Xcd{safe} qualifier}
\label{SafeAnnotation}
\label{SafeMethod}
\index{safe!method}

A method may be declared \xcd"safe".

\begin{grammar}
  MethodModifier \: \xcd"safe"  
\end{grammar}

The \Xcd{safe} annotation is considered shorthand for \Xcd{pinned
  nonblocking sequential}.


\section{Static initialization}
\label{StaticInitialization}
\index{initialization!static}
The \Xten{} runtime implements the following procedure to ensure
reliable initialization of the static state of classes.


Execution commences with a single thread executing the
\emph{initialization} phase of an \Xten{} computation at place \Xcd{0}. This
phase must complete successfully before the body of the \Xcd{main} method is
executed.

The initialization phase must be thought of as if it is implemented in
the following fashion: (The implementation may do something more
efficient as long as it is faithful to this semantics.)

\begin{xten}
Within the scope of a new finish
for every static field f of every class C 
   (with type T and initializer e):
async {
  val l = e; 
  ateach (Dist.makeUnique()) {
     assign l to the static f field of 
         the local C class object;
     mark the f field of the local C 
         class object as initialized;
  }
}
\end{xten}

During this phase, any read of a static field \Xcd{C.f} (where \Xcd{f} is of type \Xcd{T})
is replaced by a call to the method \Xcd{C.read\_f():T} defined on class \Xcd{C}
as follows

\begin{xten}
def read_f():T {
   await (initialized(C.f));
   return C.f;
}
\end{xten}
 

If all these activities terminate normally, all static field have a
legal value (per their type), and the finish terminates normally. If
any activity throws an exception, the finish throws an
exception. Since no user code is executing which can catch exceptions
thrown by the finish, the exceptions are printed on the console, and
computation aborts.


If the activities deadlock, the implementation deadlocks.

In all cases, the main method is executed only once all static fields
have been initialized correctly.

Since static state is immutable, it can be accessed from any
place.



\chapter{Structs}
\label{XtenStructs}
\label{StructClasses}
\label{Structs}
\index{structs}

An instance of a class \Xcd{C} (an \emph{object} ) is represented in \Xten{} as
a contiguously allocated chunk of words in the heap, containing the
fields of the object as well as one or more words used in method
lookup (itable/vtable). Variables with base type \Xcd{C} (or a supertype of
\Xcd{C}) are implemented as cells with enough memory to hold a
\emph{reference} to the object. The size of a reference (32 bits or 64
bits) depends on the underlying operating system.


For many high-performance programming idioms, the overhead of one
extra level of indirection represented by an object is not
acceptable. For instance, a programmer may wish to define a type
\Xcd{Complex} (consisting of two double fields) and require that instances
of this type be represented precisely as these two fields. A variable
or field of type complex should, therefore, contain enough space to
store two doubles. An array of complex of size \Xcd{N} should store \Xcd{2*N}
doubles. Method invocations should be resolved statically so that
there is no need to store vtable/itable words with each
instance. Parameters of type complex should be passed inline to a
method as two doubles. If a method's return type is complex the method
should return two doubles on the stack. Two values of this type should
be equal precisely when the two doubles are equal (structural
equality).


\Xten{} supports the notion of \emph{structs} which are precisely
objects that can be implemented inline with a contiguous chunk of
memory representing their fields, without any vtable/itable. Structs
are introduced by struct definitions. struct definitions look very
similar to class definitions, but have additional restrictions.

\subsection{Structs}

\Xten{} supports user-defined primitives (called
\emph{structs}). Like classes, structs define zero or more fields and
zero or more methods, and may implement zero or more interfaces. A
struct has the same modifiers as a class. However, structs are
implicitly final and do \emph{not} participate in any code
inheritance relation. (This makes structs very easy to implement,
without vtables.)

\begin{xtenmath}
$\mbox{\emph{StructModifiers}}^{\mbox{?}}$
struct C[X$_1$, $\ldots$, X$_n$](p$_1$:T$_1$, $\ldots$, p$_n$:T$_n$){c} 
   implements I$_1$, $\ldots$, I$_k$ {
$\mbox{\emph{StructBody}}$
}
\end{xtenmath}

Each field and method in a struct is implicitly marked \Xcd{global}.  

The size of a variable of struct type \Xcd{C} is the size of the fields
defined at \Xcd{C} (up to alignment considerations). No extra space is
allocated for a vtable or an itable. This means that unlike classes,
structs cannot be defined recursively. That is, a struct \Xcd{S} cannot
contain a field of type \Xcd{S}, or a field of struct type \Xcd{T} which,
recursively, contains a field of type \Xcd{S}.

\begin{itemize}
\item More precisely, we require that the set of \emph{size equations}
  for all structs and classes must have a unique solution. A size
  equation for a struct \Xcd{S} is defined as follows. Assume \Xcd{S} has $m$ fields
  of type \Xcd{S}$_i$ (for $i$ in $0,\ldots,m-1$), and $n$ fields of type (class) \Xcd{C}$_j$
  (for $j$ in $0,\ldots,n-1$). Then the size equation for \Xcd{S} is 
\begin{xtenmath}
size(S) = size(S$_0$)+$\ldots$+size(S$_{m-1}$)+size(C$_0$)+$\ldots$+size(C$_{n-1}$) 
\end{xtenmath}
The size
equation for a class \Xcd{C} is just \Xcd{size(C) = AddressSize}, where
\Xcd{AddressSize} is a compile-time parameter.
\end{itemize}

Values of a struct \Xcd{C} type can be created by invoking a constructor
defined in \Xcd{C}, but without prefixing it with \Xcd{new}.

Constrained types can be built on top of the base \Xcd{C} in the same way as
they can be built on top of a class \Xcd{D}. In struct \Xcd{C[T1,..., Tn]\{c\}},
the type of \Xcd{self} in \Xcd{c} is \Xcd{C[T1,..., Tn]}.


Even if struct \Xcd{C} implements an interface \Xcd{I}, a value of
type \Xcd{C} cannot be assigned to a variable of type
\Xcd{I}.\footnote{ The size of a \Xcd{C} is the size of the fields
  defined at \Xcd{C} (as discussed above), whereas the size of a
  variable of type \Xcd{I} is always \Xcd{AddressSize} (i.e.{}
  variables of interface types can only contain objects, not structs).
} The programmer may wish to define coercions
(\Sref{User-definedCoercions}) to coerce a struct to an interface,
typically by boxing.

However,
if \Xcd{C} implements an interface \Xcd{I}, all the methods and properties defined
on \Xcd{I} must be implemented by \Xcd{C} and can hence be invoked/accessed on a
value of type \Xcd{C}. 

\subsubsection{Struct equality}

Unlike objects, structs do not have global identity. Instead, two
structs are equal (\Xcd{==}) if and only if their corresponding fields are
equal (\Xcd{==}). This is the central property of structs.

\subsubsection{Interfaces implemented by structs}
Structs are not required to implement any interface. Structs intended
to be used in collections such as hashtables should implement
\Xcd{Equality}:
\begin{xten}
package x10.lang;  

public interface Equality {
    def equals(Top):boolean;
    def hashCode():Int;
}
\end{xten}
Structs are required to implement the following methods: 
\begin{xten}
  global def typeName():String;
  global def toString():String;  
\end{xten}
These methods are defined automatically if they are not supplied by the programmer.

Structs have the following methods implicitly defined on them:
\begin{xten}
  global def loc()=here;
  global def loc(p:Place)=true;
  global def loc(O:Object)=true;
\end{xten}

The operations \Xcd{==} and \Xcd{!=} are available on structs, and
correspond to performing these operations componentwise.

Expressions  of a struct type may be used in \Xcd{instanceof} and \Xcd{class-cast} tests.

\subsection{``Primitives''}

The package x10.lang provides the following structs. Most of the functionality of these structs is implemented natively. 
\begin{xten}
boolean, char, 
byte, short, int, long
float, double
ubyte, ushort, uint, ulong
\end{xten}
 
  
\subsection{Generic programming with structs}

An unconstrained type variable \Xcd{X} can be instantiated with \Xcd{Object} or
its subclasses or structs.

Within a generic class, all pre-defined operations
are available on a variable of type
\Xcd{X}. For instance, variables of \Xcd{X} may be used with \Xcd{==, !=},
\Xcd{instanceof}, casts etc

The programmer must be aware of the different interpretations of
equality for structs and classes and ensure that the code is correctly
written for both cases. If necessary the programmer can write code
that distinguishes between the two cases (a type parameter \Xcd{X} is
instantiated to a struct or not) as follows:


\begin{xten}
val x:X = ...;
if (x instanceof Object) { // x is a real object
   val x2 = x as Object; // this cast will always succeed.
   ...
} else { // x is a struct
   ...
}
\end{xten}
 
  
\subsubsection{The class Box[T]}

This class is defined as follows: 

\begin{xten}
package x10.lang;
class Box[T](value:T) implements () => T {
   public def this(v:T) { property(v);}
   public def apply() = v;
   public static def get[T](x:Box[T], default:T) =
              x == null ? default : x();
   public static operator (v:T) = new Box(v);
}
\end{xten}


Thus if \Xcd{x:Box[T]}, then \Xcd{x():T}. Note the evaluation of \Xcd{x()} may
throw a \Xcd{NullPointerException}. The programmer may wish to use
\Xcd{Box.get(x, t)}, providing a default value to use if \Xcd{x == null}.

Notice that \Xcd{Box[T]} does not implement any additional interfaces, even
though \Xcd{T} might.
  
\section{Programming Methodology}

 A programmer should by default organize his/her code in a class
 hierarchy, providing structs only in those well-thought situations
 where concrete types are appropriate.

\section{Compatibility Note}

A value class in \Xten{} v1.7 can often be translated into a struct in \Xten{} 2.0. The crucial conditions to be checked manually are: \begin{itemize}
\item  A struct is of bounded size. 
\item  Each method is global. 
\item  The class is final.
\end{itemize}
 

If these conditions are not met, the value class should be converted
into a class with \Xcd{global} fields and methods.

\subsection{Examples}

An example illustrating pairing:
\begin{xten}
struct Pair[S,T] implements Equality {
  val x: S;
  val y: T;
  def this(x: S, y: T) {
    this.x=x;
    this.y=y;
  }
  def x()=x;
  def y()=y;  
  final def hashCode() = x.hashCode() + y.hashCode();
  final def equalsX[U](o:Pair[S,U]) = x==o.x;
  final def equalsY[U](o:Pair[U,Y]) = y==o.y;
  final def equals(o:Pair[S,T]) = this==o;
}
\end{xten}

The following types all make sense: 
\begin{itemize}
\item \Xcd{Pair[Complex, String]}: A struct with two fields, one inlined field of type \Xcd{Complex} and another of type \Xcd{String}. 
\item \Xcd{Pair[Complex, Int]}: A class whose objects have size
  \Xcd{sizeof(Complex)+sizeof(Ref)} (the state of complex is
  ``inlined''). \emph\bf{FIXME}
\end{itemize}
\emph{EndOfExample.}

The definition of \Xcd{x10.lang.Complex}:
\begin{xten}
public struct Complex {
    public val re:Double;
    public val im:Double;
    public const ZERO = Complex(0.0, 0.0);
    public const ONE = Complex(1.0, 0.0);
    public const I = Complex(0.0, 1.0);
    public const INF = Complex(Double.POSITIVE_INFINITY,
                               Double.POSITIVE_INFINITY);
    public const NaN = Complex(Double.NaN, Double.NaN);
    public def this(real:Double, imaginary:Double) {
        this.re = real;
        this.im = imaginary;
    }
    public operator this + (that:Complex) =
               Complex(re + that.re, im + that.im);
    public operator this + (that:Double)  =
               Complex(re + that, im);
    public operator this - (that:Complex) =
               Complex(re - that.re, im - that.im);
    public operator this - (that:Double)  =
               Complex(re - that, im);
    public operator this * (that:Complex):Complex =
               Complex(re * that.re - im * that.im,
                       re * that.im + im * that.re);
    public operator this * (that:Double)  =
               Complex(re * that, im * that);

    /**
     * Gets the quotient of this complex number and the given
     * complex number.
     * Uses Smith's algorithm
     * <a href="http://doi.acm.org/10.1145/368637.368661"/>
     * TODO: consider using Priest's algorithm
     * <a href="http://doi.acm.org/10.1145/1039813.1039814"/>
     * @return the quotient of this complex number and the
     * given complex number
     */
    public operator this / (that:Complex):Complex {
        if (isNaN() || that.isNaN()) {
            return Complex.NaN;
        }
        val c:Double = that.re;
        val d:Double = that.im;
        if (c == 0.0 && d == 0.0) {
            return Complex.NaN;
        }
        if (that.isInfinite() && !isInfinite()) {
            return Complex.ZERO;
        }
        if (Math.abs(d) <= Math.abs(c)) {
            if (c == 0.0) {
                return Complex(im/d, -re/c);
            }
            val r:Double =  d / c;
            val denominator:Double = c + d * r;
            return Complex((re + im * r) / denominator,
                           (im - re * r) / denominator);
        } else {
            if (d == 0.0) {
                return Complex(re/c, im/c);
            }
            val r:Double = c / d;
            val denominator:Double = c * r + d;
            return Complex((re * r + im) / denominator,
                           (im * r - re) / denominator);
        }
    }
    public operator this / (that:Double) =
               Complex(re / that, im / that);
    public def conjugate() =
               isNaN() ? Complex.NaN : Complex(re, -im);
    public operator - this:Complex =
               isNaN() ? Complex.NaN : Complex(-re, -im);
    public def abs():Double {
        if (isNaN()) {
            return Double.NaN;
        }
        if (isInfinite()) {
            return Double.POSITIVE_INFINITY;
        }
        if (im == 0.0) {
            return Math.abs(re);
        } else if (re == 0.0) {
            return Math.abs(im);
        } else {
            // use hypot to avoid unnecessary under/overflow
            return Math.hypot(re, im);
        }
    }
    public def isNaN()= re.isNaN() || im.isNaN();        
    public def isInfinite() =
               !isNaN() && (re.isInfinite() || im.isInfinite();
    public def toString() = (re + " + " + im + "i");

}
\end{xten}



\chapter{Functions}
\label{Functions}
\label{functions}
\index{functions}
\label{Closures}

\section{Overview}
Functions, the last of the three kinds of values in X10, encapsulate pieces of
code which can be applied to a vector of arguments to produce a value.
Functions, when applied, can do nearly anything that any other code could do:
fail to terminate, throw an exception, offer values, modify variables, spawn activities,
execute in several places, and so on. X10 functions are not mathematical
functions: the \xcd`f(1)` may return \xcd`true` on one call and \xcd`false` on
an immediately following call.

It is a limitation of \XtenCurrVer{} that functions do not support
type arguments. This limitation may be removed in future versions of
the language.

A \emph{function literal} \xcd"(x1:T1,..,xn:Tn){c}:T=>e" creates a function of
type\\ \xcd"(x1:T1,...,xn:Tn){c}=>T" (\Sref{FunctionType}).  For example, 
\xcd`(x:Int) => x*x` is a function literal describing the squaring function on
integers.   
\xcd`null` is also a function value.

Function application is written \xcd`f(a,b,c)`, following common mathematical
usage. 
\index{Exception!unchecked}
Function invocation may throw unchecked exceptions. 

The function body may be a block.  To compute integer squares by repeated
addition (inefficiently), one may write: 
%~~gen
% package Functions.Are.For.Spunctions;
% class Examplllll {
% static 
%~~vis
\begin{xten}
val sq: (Int) => Int 
      = (n:Int) => {
           var s : Int = 0;
           val abs_n = n < 0 ? -n : n;
           for ((i) in 1..abs_n) s += abs_n;
           s
        };
\end{xten}
%~~siv
%}
%~~neg




A function literal evaluates to a function entity {$\phi$}. When {$\phi$} is
applied to a suitable list of actual parameters \xcd`a1`-\xcd`an`, it
evaluates \xcd`e` with the formal parameters bound to the actual parameters.
So, the following are equivalent, where \xcd`e` is an expression involving
\xcd`x1` and \xcd`x2`\footnote{Strictly, there are a few other requirements;
  \eg, \xcd`result` must be a \xcd`var` of type \xcd`T` defined outside the
  outer block, the variables \xcd`a1` and \xcd`a2` had better not appear in
  \xcd`e`, and everything in sight had better typecheck properly.}

%~~gen
% package functions2.why.is.there.a.two;
% abstract class FunctionsTooManyFlippingFunctions[T, T1, T2]{
% abstract def arg1():T1;
% abstract def arg2():T2;
% def thing1(e:T) {var result:T;
%~~vis
\begin{xten}
{
  val f = (x1:T1,x2:T2){true}:T => e;
  val a1 : T1 = arg1();
  val a2 : T2 = arg2();
  result = f(a1,a2);
}
\end{xten}
%~~siv
%}}
%~~neg
and 
%~~gen
% package functions2.why.is.there.a.two.but.here.is.the.other.one;
% abstract class FunctionsTooManyFlippingFunctions[T, T1, T2]{
% abstract def arg1():T1;
% abstract def arg2():T2;
% def thing1(e:T) {var result:T;
%~~vis
\begin{xten}
{
  val a1 : T1 = arg1();
  val a2 : T2 = arg2();
  {
     val x1 : T1 = a1;
     val x2 : T2 = a2;
     result = e;
  }  
}
\end{xten}
%~~siv
%}}
%~~neg
\noindent
This doesn't quite work if the body is a statement rather than an expression.
A few language features are forbidden (\xcd`break` or \xcd`continue` of a loop
that surrounds the function literal) or mean something different (\xcd`return`
inside a function returns from the function). 





The \emph{method selector expression} \Xcd{e.m.(x1:T1,...,xn:Tn)} (\Sref{MethodSelectors})
permits the specification of the function underlying
the method \Xcd{m}, which takes arguments of type \Xcd{(x1:T1,..., xn:Tn)}.
Within this function, \Xcd{this} is bound to the result of evaluating \Xcd{e}.

Function types may be used in \Xcd{implements} clauses of class
definitions. Instances of such classes may be used as functions of the
given type.  Indeed, an object may behave like any (fixed) number of
functions, since the class it is an instance of may implement any
(fixed) number of function types.

%\section{Implementation Notes}
%\begin{itemize}
%
%\item Note that e.m.(T1,...,Tn) will evaluate e to create a
%  function. This function will be applied later to given
%  arguments. Thus this syntax can be used to evaluate the receiver of
%  a method call ahead of the actual invocation. The resulting function
%  can be used multiple times, of course.
%\end{itemize}


\section{Function Literals}
\index{literal!function}
\label{FunctionLiteral}

\Xten{} provides first-class, typed functions, including
\emph{closures}, \emph{operator functions}, and \emph{method
  selectors}.

\begin{grammar}
ClosureExpression \:
        \xcd"("
        Formals\opt
        \xcd")"
\\ &&
        Guard\opt
        ReturnType\opt
        Throws\opt
        Offers\opt
        \xcd"=>" ClosureBody \\
ClosureBody \:
        Expression \\
        \| \xcd"{" Statement\star \xcd"}" \\
        \| \xcd"{" Statement\star Expression \xcd"}" \\
\end{grammar}

Functions have zero or more formal parameters, an optional return type, an
optional set of exceptions throws by the body, and an optional type offered by
the body.  The body has the
same syntax as a method body; it may be either an expression, a block
of statements, or a block terminated by an expression to return. In
particular, a value may be returned from the body of the function
using a return statement (\Sref{ReturnStatement}). 

The type of a
function is a function type (\Sref{FunctionType}).  In some cases the
return type \Xcd{T} is also optional and defaults to the type of the
body. If a formal \Xcd{xi} does not occur in any
\Xcd{Tj}, \Xcd{c}, \Xcd{T} or \Xcd{e}, the declaration \Xcd{xi:Ti} may
be replaced by just \Xcd{Ti}: \xcd`(Int)=>7` is the integer function returning
7 for all inputs.

\label{ClosureGuard}

As with methods, a function may declare a guard to
constrain the actual parameters with which it may be invoked.
The guard may refer to the type parameters, formal parameters,
and any \xcd`val`s in scope at the function expression.

The body of the function is evaluated when the function is
invoked by a call expression (\Sref{Call}), not at the function's
place in the program text.

As with methods, a function with return type \xcd"Void" cannot
have a terminating expression. 
If the return type is omitted, it is inferred, as described in
\Sref{TypeInference}.
It is a static error if the return type cannot be inferred.  \Eg,
\xcd`(Int)=>null` is not well-defined; X10 does not know which type of
\xcd`null` is intended.  
%~~exp~~`~~`~~ ~~
But \xcd`(Int):Rail[Double] => null` is legal.


\begin{example}
The following method takes a function parameter and uses it to
test each element of the list, returning the first matching
element.  It returns \xcd`otherwise` if no element matches.

%~~gen
% package functions2.oh.no;
% import x10.util.*;
% class Finder {
% static 
%~~vis
\begin{xten}

def find[T](f: (T) => Boolean, xs: List[T]!, absent:T): T = {
  for (x: T in xs)
    if (f(x)) return x;
  absent
  }
\end{xten}
%~~siv
% }
%~~neg

The method may be invoked thus:
%~~gen
% package functions2.oh.no.my.ears;
% import x10.util.*;
% class Finderator {
% static def find[T](f: (T) => Boolean, xs: x10.util.List[T]!, absent:T): T = {
%  for (x: T in xs)
%    if (f(x)) return x;
%  absent
%}
% static def checkery() {
%~~vis
\begin{xten}
xs: List[Int]! = new ArrayList[Int]();
x: Int = find((x: Int) => x>0, xs, 0);
\end{xten}
%~~siv
%}}
%~~neg

\end{example}

As with a normal method, the function may have a \xcd"throws"
clause. It is a static error if the body of the function throws a
checked exception that is not declared in the function's \xcd"throws"
clause.

Similarly, it may have an \Xcd{offers T} clause; it is a static error if the
body offers any value not of type \Xcd{T}.

\subsection{Outer variable access}

In a function
\xcdmath"(x$_1$: T$_1$, $\dots$, x$_n$: T$_n$){c} => { s }"
the types \xcdmath"T$_i$", the guard \xcd"c" and the body \xcd"s"
may access many, though not all, sorts of variables from outer scopes.  
Specifically, they can access: 
\begin{itemize}
\item All fields of the enclosing object and class;
\item All type parameters;
\item All \xcd`val` variables;
\item \xcd`var` variables with the \xcd`shared` annotation. 
\end{itemize}


\limitation{\xcd`shared` is not currently supported.}

The function body may refer to instances of enclosing classes using
the syntax \xcd"C.this", where \xcd"C" is the name of the
enclosing class.  \xcd`this` refers to the instance of the immediately
enclosing class, as usual.

For example, the following is legal.  However, it would not be legal to add
\xcd`e` or \xcd`h` to the sum; they are non-\xcd`shared` \xcd`var`s from the
surrounding scope.

%%TODO -- this example uses 'shared', which is not currently available.
\begin{xten}
class Lambda {
   var a : Int = 0;
   val b = 0;
   def m(var c : Int, shared var d : Int,  val e : Int) {
      var f : Int = 0;
      shared var g : Int = 0;
      val h : Int = 0;
      val closure = (var i: Int, val j: Int) => {
    	  return a + b + d + g + i + j + this.a + Lambda.this.a;
      };
      return closure;
   }
}
\end{xten}


{\bf Rationale:} Non-\xcd`shared` \xcd`var`s like \xcd`e` and \xcd`h` are
excluded in X10, as in many other languages, for practical implementation
reasons. They are allocated on the stack, which is desirable for efficiency.
However, the closure may exist for long after the stack frame containing
\xcd`e` and \xcd`h` has been freed, so those storage locations are no longer
valid for those variables. \xcd`shared var`s are heap-allocated, which is less
efficient but allows them to exist after \xcd`m` returns. 


\xcd`shared` does not guarantee {\bf atomic} access to the shared variable. As
with any code that might mutate shared data concurrently, be sure to protect
references to mutable shared state with \xcd`atomic`. For example, the
following code returns a pair of closures which operate on the same shared
variable \xcd`a`, which are concurrency-safe---even if invoked many times
simultaneously. Without \xcd`atomic`, it would no longer be concurrency-safe.


%~fails~gen
% package Functions2.Are.All.Too.Much;
% class Fun2Frivols {
%~fails~vis
\begin{xten}
  def counters() {
      shared var a : Int = 0;
       return [
          () => {atomic a ++;},
          () => {atomic return a;}
          ];
   }
\end{xten}
%~fails~siv
%}
%
%~fails~neg


%SHARED% \begin{note}
%SHARED% The main activity may run in parallel with any
%SHARED% functions it creates. Hence even the read of an outer variable by the
%SHARED% body of a function may result in a race condition. Since functions are
%SHARED% first-class, the analysis of whether a function may execute in parallel
%SHARED% with the activity that created it may be difficult.
%SHARED% \end{note}

%% vj: This should be verified.
%\begin{note}
%The rule for accessing outer variables from function bodies
%should be the same as the rule for accessing outer variables from local
%or anonymous classes.
%\end{note}

\section{Method selectors}
\label{MethodSelectors}
\index{function!method selector}
\index{method!underlying function}

A method selector expression allows a method to be used as a
first-class function, without writing a function expression for it.
For example, consider a class \xcd`Span` defining ranges of integers.  

%~~gen
% package Functions2.Span;
%~~vis
\begin{xten}
class Span(low:Int, high:Int) {
  def this(low:Int, high:Int) {property(low,high);}
  def between(n:Int) = low <= n && n <= high;
  def example() {
    val digit = new Span(0,9);
    val isDigit : (Int) => Boolean = digit.between.(Int);
    if (isDigit(8)) x10.io.Console.OUT.println("8 is!");
  }
}
\end{xten}
%~~siv
%
%~~neg
\noindent


In \xcd`example()`, 
%~~exp~~`~~`~~ digit:Span!~~class Span(low:Int, high:Int) {def this(low:Int, high:Int) {property(low,high);} def between(n:Int) = low <= n && n <= high;}
\xcd`digit.between.(Int)` 
is a unary function testing whether its argument is between zero
and nine.  It could also be written 
%~~exp~~`~~`~~ digit:Span!~~class Span(low:Int, high:Int) {def this(low:Int, high:Int) {property(low,high);} def between(n:Int) = low <= n && n <= high;}
\xcd`(n:Int) => digit.between(n)`.

This is formalized thus:

\begin{grammar}
MethodSelector \:
        Primary \xcd"."
        MethodName \xcd"."
                TypeParameters\opt \xcd"(" Formals\opt \xcd")" \\
      \|
        TypeName \xcd"."
        MethodName \xcd"."
                TypeParameters\opt \xcd"(" Formals\opt \xcd")" \\
\end{grammar}

The \emph{method selector expression} \Xcd{e.m.(T1,...,Tn)} is type
correct only if  the static type of \Xcd{e} is a
class or struct or interface \xcd`V` with a method
\Xcd{m(x1:T1,...xn:Tn)\{c\}:T} defined on it (for some
\Xcd{x1,...,xn,c,T)}. At runtime the evaluation of this expression
evaluates \Xcd{e} to a value \Xcd{v} and creates a function \Xcd{f}
which, when applied to an argument list \Xcd{(a1,...,an)} (of the right
type) yields the value obtained by evaluating \Xcd{v.m(a1,...,an)}.

Thus, the method selector

\begin{xtenmath}
e.m.[X$_1$, $\dots$, X$_m$](T$_1$, $\dots$, T$_n$)
\end{xtenmath}
\noindent behaves as if it were the function
\begin{xtenmath}
((v:V)=>
  [X$_1$, $\dots$, X$_m$](x$_1$: T$_1$, $\dots$, x$_n$: T$_n$){c} 
  => v.m[X$_1$, $\dots$, X$_m$](x$_1$, $\dots$, x$_n$))
(e)
\end{xtenmath}


\limitation{X10 functions, including method selectors, do not currently accept
generic arguments.}

Because of overloading, a method name is not sufficient to
uniquely identify a function for a given class (in Java-like languages).
One needs the argument type information as well.
The selector syntax (dot) is used to distinguish \xcd"e.m()" (a
method invocation on \xcd"e" of method named \xcd"m" with no arguments)
from \xcd"e.m.()"
(the function bound to the method). 

A static method provides a binding from a name to a function that is
independent of any instance of a class; rather it is associated with the
class itself. The static function selector
\xcdmath"T.m.(T$_1$, $\dots$, T$_n$)" denotes the
function bound to the static method named \xcd"m", with argument types
\xcdmath"(T$_1$, $\dots$, T$_n$)" for the type \xcd"T". The return type
of the function is specified by the declaration of \xcd"T.m".

Users of a function type do not care whether a function was defined
directly (using the function syntax), or obtained via (static or
instance) function selectors.


\section{Operator functions}
\label{OperatorFunction}
\index{function!operator}
Every operator (e.g.,
\xcd"+",
\xcd"-",
\xcd"*",
\xcd"/",
\dots) has a family of functions, one for
each type on which the operator is defined. The function can be
selected using the "." syntax:

\begin{grammar}
OperatorFunction
        \: TypeName \xcd"." Operator \xcd"(" Formals\opt \xcd")" \\
        \| TypeName \xcd"." Operator \\
\end{grammar}

If an operator has more than one arity (\eg, unary and binary
\xcd"-"), the unary version may be selected by giving the
formal parameter types.  The binary version is selected by
default, or the types may be specified for clarity.
For example, the following equivalences hold:

\begin{xtenmath}
String.+             $\equiv$ (x: String, y: String): String => x + y
Long.-               $\equiv$ (x: Long, y: Long): Long => x - y
Float.-(Float,Float) $\equiv$ (x: Float, y: Float): Float => x - y
Int.-(Int)           $\equiv$ (x: Int): Int => -x
Boolean.&            $\equiv$ (x: Boolean, y: Boolean): Boolean => x & y
Boolean.!            $\equiv$ (x: Boolean): Boolean => !x
Int.<(Int,Int)       $\equiv$ (x: Int, y: Int): Boolean => x < y
Dist.|(Place)        $\equiv$ (d: Dist, p: Place): Dist => d | p
\end{xtenmath}


%%TODO -- fix commented-out lines!

%~~gen
% package Functions2.For.The.Lose;
% class TypecheckThatSillyExample {
%   def checker() {
%    val l1 : (String, String) => String = String.+;
%    val r1 : (String, String) => String = (x: String, y: String): String => x + y;
%    val l2 : (Long,Long) => Long = Long.-;
%    val r2 : (Long,Long) => Long = (x: Long, y: Long): Long => x - y;
%//var v1 : (Float,Float) => Float = Float.-(Float,Float) ;
%var v2 : (Float,Float) => Float = (x: Float, y: Float): Float => x - y;
%//var v3 : (Int) => Int =  Int.-(Int)     ;      ;
%var v4  : (Int) => Int  =  (x: Int): Int => -x;
%var v5 : (Boolean,Boolean) => Boolean = Boolean.&            ;
%var v6 : (Boolean,Boolean) => Boolean =  (x: Boolean, y: Boolean): Boolean => x & y;
%//var v7 : (Boolean) => Boolean = Boolean.!            ;
%var v8 : (Boolean) => Boolean =  (x: Boolean): Boolean => !x;
%//var v9 : (Int,Int) => Boolean = Int.<(Int,Int)       ;
%var v10: (Int,Int) => Boolean =  (x: Int, y: Int): Boolean => x < y;
%//var v11 : (Dist,Place)=>Dist = Dist.|(Place)        ;
%var v12 : (Dist,Place)=>Dist=  (d: Dist, p: Place): Dist => d | p;
%}
% }
%~~vis
%~~siv
%
%~~neg

Unary and binary promotion (\Sref{XtenPromotions}) is not performed
when invoking these
operations; instead, the operands are coerced individually via implicit
coercions (\Sref{XtenConversions}), as appropriate.


\begin{planned}

{\bf The following is not implemented in version 2.0.3:}

Additionally, for every expression \xcd"e" of a type \xcd"T" at which a binary
operator \xcd"OP" is defined, the expression \xcd"e.OP" or
\xcd"e.OP(T)" represents the function
defined by:

\begin{xten}
(x: T): T => { e OP x }
\end{xten}

\begin{grammar}
Primary \: Expr \xcd"." Operator \xcd"(" Formals\opt \xcd")" \\
        \| Expr \xcd"." Operator \\
\end{grammar}

%% For every expression \xcd"e" of a type \xcd"T" at which a unary
%%operator \xcd"OP" is defined, the expression \xcd"e.OP()"
%% represents the function defined by:

%% \begin{xten}
%% (): T => { OP e }
%% \end{xten}

For example,
one may write an expression that adds one to each member of a
list \xcd"xs" by:

%%TODO -- when this topic works, make the example wwork too.
%~x~gen
% package Functions2.Wants.A.Dinner.Reservation;
% import x10.util.*;
% class Reservation {
% def smerp() {
%   val xs = new ArrayList[Int]();
%~x~vis
\begin{xten}
xs.map(1.+);
\end{xten}
%~x~siv
% }
% }
%
%~x~neg
\end{planned}


\section{Functions as objects of type \Xcd{Any}}
\label{FunctionAnyMethods}

\label{FunctionEquality}
\index{function!equality} \index{equality!function} Two functions \Xcd{f} and
\Xcd{g} are equal (``\Xcd{==}'') if both are instances of classes and the same
object, or if both were obtained by the same evaluation of a function
literal.\footnote{A literal may occur in program text within a loop, and hence
  may be evaluated multiple times.} Further, it is guaranteed that if two
functions are equal then they refer to the same locations in the environment
and represent the same code, so their executions in an identical situation are
indistinguishable. (Specifically, if \xcd`f == g`, then \xcd`f(1)` can be
substituted for \xcd`g(1)` and the result will be identical. However, there is
no guarantee that \xcd`f(1)==g(1)` will evaluate to true. Indeed, there is no
guarantee that \xcd`f(1)==f(1)` will evaluate to true either, as \xcd`f` might
be a function which returns {$n$} on its {$n^{th}$} invocation. However,
\xcd`f(1)==f(1)` and \xcd`f(1)==g(1)` are interchangeable.)
\index{function!==}


Every function type implements all the methods of \Xcd{Any}.
b\xcd`f.equals(g)` is equivalent to \xcd`f==g`.  \xcd`f.hashCode()`, 
\xcd`f.toString()`, and \xcd`f.typeName()` are implementation-dependent, but
respect \xcd`equals` and the basic contracts of \xcd`Any`. 
\xcd`f.home()` returns \xcd`here` and \xcd`f.at(x)`
always returns true, as for structs.

\index{function!equals}
\index{function!hashCode}
\index{function!toString}
\index{function!typeName}
\index{function!home}
\index{function!at(Place)}
\index{function!at(Object)}



\chapter{Expressions}\label{XtenExpressions}\index{expressions}

\Xten{} has a rich expression language.
Evaluating an expression produces a value, or, in a few cases, no value. 
Expression evaluation may have side effects, such as change of the value of a 
\xcd`var` variable or a data structure, allocation of new values, or throwing
an exception. 

Evaluation is performed left to right, wherever possible.  For example, in 
%~~exp~~`~~`~~f:() => ((Int,Int)=>Int), a:()=>Int, b:()=>Int ~~
\xcd`f()(a(),b())`, \xcd`f()` is evaluated, then \xcd`a()`, then \xcd`b()`,
andd then the application.  

\section{Literals}

Literals denote fixed values of built-in types. 
The syntax for literals is given in \Sref{Literals}. 

The type that \Xten{} gives a literal often includes its value. \Eg, \xcd`1`
is of type \xcd`Int{self==1}`, and \xcd`true` is of type
\xcd`Boolean{self==true}`.

\section{\Xcd{this}}

\begin{grammar}
ThisExpression \: \xcd"this" \\
\| ClassName \xcd"." \xcd"this" \\
\end{grammar}

The expression \xcd"this" is a  local \xcd`val` containing a reference
to an instance of the lexically enclosing class.
It may be used only within the body of an instance method, a
constructor, or in the initializer of a instance field -- that is, the places
where there is an instance of the class under consideration.

Within an inner class, \xcd"this" may be qualified with the
name of a lexically enclosing class.  In this case, it
represents an instance of that enclosing class.  
For example, \xcd`Outer` is a class containing \xcd`Inner`.  Each instance of
\xcd`Inner` has a reference \xcd`outer` to the \xcd`Outer` involved in its
creation, which is acquired by use of \xcd`Outer.this`.
%~~gen
% package exp.vexp.pexp.lexp.shexp;
%~~vis
\begin{xten}
class Outer {
  val inner : Inner! = new Inner();
  class Inner {
    val outer : Outer = Outer.this;
  }
  def alwaysTrue() = (this == inner.outer);
}
\end{xten}
%~~siv
%
%~~neg

The type of a \xcd"this" expression is the
innermost enclosing class, or the qualifying class,
constrained by the class invariant and the
method guard, if any.

The \xcd"this" expression may also be used within constraints in
a class or interface header (the class invariant and
\xcd"extends" and \xcd"implements" clauses).  Here, the type of
\xcd"this" is restricted so that only properties declared in the
class header itself, and specifically not any members declared in the class
body or in supertypes, are accessible through \xcd"this".

\section{Local variables}

\begin{grammar}
LocalExpression \: Identifier \\
\end{grammar}

A local variable expression consists simply of the name of the local variable,
field of the current object, formal parameter in scope, etc. It evaluates to
the value of the local variable. \xcd`n` in the second line below is a local
variable expression. 
%~~gen 
% package exp.loc.al.varia.ble; 
% class Example {
% def example() { 
%~~vis
\begin{xten}
val n = 22;
val m = n + 56;
\end{xten}
%~~siv
%} }
%~~neg



\section{Field access}
\label{FieldAccess}


\begin{grammar}
FieldExpression \: Expression \xcd"." Identifier \\
                \| \xcd"super" \xcd"." Identifier \\
                \| ClassName \xcd"." Identifier \\
                \| ClassName \xcd"." \xcd"super" \xcd"." Identifier \\
\end{grammar}

A field of an object instance may be  accessed
with a field access expression.

The type of the access is the declared type of the field with the
actual target substituted for \xcd"this" in the type. 
% If the actual
%target is not a final access path (\Sref{FinalAccessPath}),
%an anonymous path is substituted for \xcd"this".

The field accessed is selected from the fields and value properties
of the static type of the target and its superclasses.

If the field target is given by the keyword \xcd"super", the target's type is
the superclass of the enclosing class.  This form is used to access fields of
the parent class shadowed by same-named fields of the current class.

If the field target is \xcd`Cls.super`, then the target's type is \xcd`Cls`,
which must be an ancestor class of the enclosing class.  This (admittedly
obscure) form is used to access fields of an ancestor class which are shadowed
by same-named fields of some more recent ancestor.  The following example
illustrates all four cases:

%~~gen
% package exp.re.ssio.ns.fiel.dacc.ess;
%~~vis
\begin{xten}
class Uncle {
  public static global val f = 1;
}
class Parent {
  public global val f = 2;
}
class Ego extends Parent {
  public global val f = 3;
  class Child extends Ego {
     public global val f = 4;
     def expDotId() = this.f; // 4
     def superDotId() = super.f; // 3
     def classNameDotId() = Uncle.f; // 1;
     def cnDotSuperDotId() = Ego.super.f; // 2
  }
}
\end{xten}
%~~siv
%
%~~neg


If the field target is \xcd"null", a \xcd"NullPointerException"
is thrown.

If the field target is a class name, a static field is selected.

It is illegal to access  a field that is not visible from
the current context.
It is illegal to access a non-static field
through a static field access expression.

\section{Function Literals}
Function literals are described in \Sref{Functions}.

\section{Calls}
\label{Call}
\label{MethodInvocation}
\label{MethodInvocationSubstitution}

\begin{grammar}
MethodCall \: TypeName \xcd"." Identifier TypeArguments\opt \xcd"(" ArgumentList\opt \xcd")" \\
           \| \xcd"super" \xcd"." Identifier TypeArguments\opt \xcd"(" ArgumentList\opt \xcd")" \\
           \| ClassName \xcd"." \xcd"super" \xcd"." Identifier TypeArguments\opt \xcd"(" ArgumentList\opt \xcd")" \\
Call \: Primary TypeArguments\opt \xcd"(" ArgumentList\opt \xcd")" \\
TypeArguments \: \xcd"[" Type ( \xcd"," Type )\star \xcd"]" \\
\end{grammar}

A \grammarrule{MethodCall} may be to either a \xcd"static" or an 
instance method.  A \grammarrule{Call} may invoke either a method
or a closure.  

The syntax is ambiguous; the target must be type-checked to determine if it is
the name of a method or if it refers to a field containing a closure. It is a
static error if a call may resolve to both a closure call or to a method call.
%~~gen
% package expres.sio.nsca.lls;
%~~vis
\begin{xten}
class Callsome {
  static val closure = () => 1;
  static def method () = 2;
  static val closureEvaluated = Callsome.closure();
  static val methodEvaluated = Callsome.method();
}
\end{xten}
%~~siv
%
%~~neg
However, adding a static method called \xcd`closure` makes \xcd`Callsome.closure()`
ambiguous: it could be a call to the closure, or to the static method: 

%~~gen
% package expres.sio.nsca.lls.twoooo;
% class Callsome {static val closure = () => 1; static def method () = 2; static val methodEvaluated = Callsome.method();
%~~vis
\begin{xten}
  static def closure () = 3;
  // ERROR: static errory = Callsome.closure();
\end{xten}
%~~siv
% }
%~~neg

A closure call \xcdmath"e($\dots$)" is shorthand for a method call
\xcdmath"e.apply($\dots$)". 

Method selection rules are similar to that of \java{}.\bard{Explain}
For a call with no explicit type arguments, a method with 
no parameters is considered more specific than a method with one or more
type parameters that would have to be inferred.

It is a static error if a method's \grammarrule{Guard} is not satisfied by the
caller. 

\section{Assignment}\index{assignment}\label{AssignmentStatement}

\begin{grammar}
Expression \: Assignment \\
Assignment \: SimpleAssignment \\
           \| OpAssignment \\
SimpleAssignment \: LeftHandSide \xcd"=" Expression \\
OpAssignment \: LeftHandSide \xcd"+=" Expression \\
             \: LeftHandSide \xcd"-=" Expression \\
             \: LeftHandSide \xcd"*=" Expression \\
             \: LeftHandSide \xcd"/=" Expression \\
             \: LeftHandSide \xcd"%=" Expression \\
             \: LeftHandSide \xcd"&=" Expression \\
             \: LeftHandSide \xcd"|=" Expression \\
             \: LeftHandSide \xcd"^=" Expression \\
             \: LeftHandSide \xcd"<<=" Expression \\
             \: LeftHandSide \xcd">>=" Expression \\
             \: LeftHandSide \xcd">>>=" Expression \\
LeftHandSide \: Identifier \\
             \| Primary \xcd"." Identifier \\
             \| Primary \xcd"(" Expression \xcd")" \\
\end{grammar}



The assignment expression \xcd"x = e" assigns a value given by
expression \xcd"e"
to a variable \xcd"x".  
Most often, \xcd`x` is a mutable (\xcd`var` variable).  The same syntax is
used for delayed initialization of a \xcd`val`, but \xcd`val`s can only be
initialized once.
%~~gen
% package express.ions.ass.ignment;
% class Example {
% static def exasmple() {
%~~vis
\begin{xten}
  var x : Int;
  val y : Int;
  x = 1;
  y = 2; // Correct; initializes y
  x = 3; 
  // Incorrect: y = 4;
\end{xten}
%~~siv
% } } 
%~~neg


There are three syntactic forms of
assignment: 
\begin{enumerate}
\item \xcd`x = e;`, assigning to a local variable, formal parameter, field of
      \xcd`this`, etc. 
\item \xcd`x.f = e;`, assigning to a field of an object.
\item \xcdmath`a(i$_1$,$\ldots$,i$_n$) = v;`, where {$n \ge 0$}, assigning to
      an element of an array or some other such structure. This is syntactic
      sugar for a method call: \xcdmath`a.set(v,i$_1$,$\ldots$,i$_n$)`.
      Naturally, it is a static error if no suitable \xcd`set` method exists
      for \xcd`a`.
\end{enumerate}

For a binary operator $\diamond$, the $\diamond$-assignment expression
\xcdmath"x $\diamond$= e" combines the current value of \xcd`x` with the value
of \xcd`e` by {$\diamond$}, and stores the result back into \xcd`x`.  
\xcd`i += 2`, for example, adds 2 to \xcd`i`. For variables and fields, 
\xcdmath"x $\diamond$= e" behaves just like \xcdmath"x = x $\diamond$ e". 

The subscripting forms of \xcdmath"a(i) $\diamond$= b" are slightly subtle.
Subexpressions of \xcd`a` and \xcd`i` are only evaluated once.  However,
\xcd`a(i)` and \xcd`a(i)=c` are each executed once---in particular, there is
one call to \xcd`a.apply(i)` and one to \xcd`a.set(i,c)`, the desugared forms
of \xcd`a(i)` and \xcd`a(i)=c`.  If subscripting is implemented strangely for
the class of \xcd`a`, the behavior is {\em not} necessarily updating a single
storage location. Specifically, \xcd`A()(I()) += B()` is tantamount to: 
%~~gen
% package expressions.stupid.addab;
% class Example {
% def example(A:()=>Rail[Int]!, I: () => Int, B: () => Int ) {
%~~vis
\begin{xten}
{
  val aa = A();  // Evaluate A() once
  val ii = I();  // Evaluate I() once
  val bb = B();  // Evaluate B() once
  val tmp = aa(ii) + bb; // read aa(ii)
  aa(ii) = tmp;  // write sum back to aa(ii)
}
\end{xten}
%~~siv
%}}
%~~neg

\limitation{+= does not currently meet this specification.}




\section{Increment and decrement}


The operators \xcd"++" and \xcd"--" increment and decrement
a variable, respectively.  
\xcd`x++` and \xcd`++x` both increment \xcd`x`, just as the statement 
\xcd`x += 1` would, and similarly for \xcd`--`.  

The difference between the two is the return value.  
\xcd`++x` returns the {\em new} value of \xcd`x`, after incrementing.
\xcd`x++` returns the {\em old} value of \xcd`, before incrementing.`

\limitation{This currently only works for numeric types.}

\section{Numeric Operations}
\label{XtenPromotions}
\index{promotion}
\index{numeric promotion}

Numeric types (\xcd`Byte`, \xcd`Short`, \xcd`Int`, \xcd`Long`, \xcd`Float`,
\xcd`Double`, and unsigned variants of fixed-point types) are normal X10
structs, though most of their methods are implemented via native code. They
obey the same general rules as other X10 structs. For example, numeric
operations are defined by \xcd`operator` definitions, the same way you could
for any struct.

Promoting a numeric value to a longer numeric type preserves the sign of the
value.  For example, \xcd`(255 as UByte) as UInt` is 255.  

\subsection{Conversions and coercions}

Specifically, each numeric type can be converted or coerced into each other
numeric type, perhaps with loss of accuracy.
%~~gen
% package exp.ress.io.ns.numeric.conversions;
% class ExampleOfConversionAndStuff {
% def example() {
%~~vis
\begin{xten}
val f : (Int)=>Boolean = (Int) => true; 
val n : Byte = 123 as Byte; // explicit 
val ok = f(n); // implicit
\end{xten}
%~~siv
% } }
%~~neg



\subsection{Unary plus and unary minus}

The unary \xcd`+` operation on numbers is an identity function.
The unary \xcd`-` operation on numbers is a negation function.
On unsigned numbers, these are two's-complement.  For example, 
\xcd`-(0x0F as UByte)` is 
\xcd`(0xF1 as UByte)`.
\bard{UInts and such are closed under negation -- the negative of a UInt is
done binarily.  }



\section{Bitwise complement}

The unary \xcd"~" operator, only defined on integral types, complements each
bit in its operand.  

\section{Binary arithmetic operations} 

The binary arithmetic operators perform the familiar binary arithmetic
operations: \xcd`+` adds, \xcd`-` subtracts, \xcd`*` multiplies, 
\xcd`/` divides, and \xcd`%`
computes remainder.

On integers, the operands are coerced to the longer of their two types, and
then operated upon.  
Floating point operations are determined by the IEEE 754
standard. 
The integer \xcd"/" and \xcd"%" throw a \xcd"DivideByZeroException"
if the right operand is zero.

\section{Binary shift operations}

The operands of the binary shift operations must be of integral type.
The type of the result is the type of the left operand.

If the promoted type of the left operand is \xcd"Int",
the right operand is masked with \xcd"0x1f" using the bitwise
AND (\xcd"&") operator, giving a number {$\le$} the number of bits in an
\xcd`Int`. 
If the promoted type of the left operand is \xcd"Long",
the right operand is masked with \xcd"0x3f" using the bitwise
AND (\xcd"&") operator, giving a number {$\le$} the number of bits in a
\xcd`Long`. 

The \xcd"<<" operator left-shifts the left operand by the number of
bits given by the right operand.
The \xcd">>" operator right-shifts the left operand by the number of
bits given by the right operand.  The result is sign extended;
that is, if the right operand is $k$,
the most significant $k$ bits of the result are set to the most
significant bit of the operand.

The \xcd">>>" operator right-shifts the left operand by the number of
bits given by the right operand.  The result is not sign extended;
that is, if the right operand is $k$,
the most significant $k$ bits of the result are set to \xcd"0".
This operation is deprecated, and may be removed in a later version of the
language. 

\section{Binary bitwise operations}

The binary bitwise operations operate on integral types, which are promoted to
the longer of the two types.
The \xcd"&" operator  performs the bitwise AND of the promoted operands.
The \xcd"|" operator  performs the bitwise inclusive OR of the promoted operands.
The \xcd"^" operator  performs the bitwise exclusive OR of the promoted operands.

\section{String concatenation}

The \xcd"+"  operator is used for string concatenation 
 as well as addition.
If either operand is of static type \xcd"x10.lang.String",
 the other operand is converted to a \xcd"String" , if needed,
  and  the two strings  are concatenated.
 String conversion of a non-\xcd"null" value is  performed by invoking the
 \xcd"toString()" method of the value.
  If the value is \xcd"null", the value is converted to 
  \xcd'"null"'.

The type of the result is \xcd"String".

 For example, 
%~~exp~~`~~`~~ ~~
      \xcd`"one " + 2 + here` 
      evaluates to something like \xcd`one 2(Place 0)`.  

\section{Logical negation}

The operand of the  unary \xcd"!" operator 
must be of type \xcd"x10.lang.Boolean".
The type of the result is \xcd"Boolean".
If the value of the operand is \xcd"true", the result is \xcd"false"; if
if the value of the operand  is \xcd"false", the result is \xcd"true".

\section{Boolean logical operations}

Operands of the binary boolean logical operators must be of type \xcd"Boolean".
The type of the result is \xcd"Boolean"

The \xcd"&" operator  evaluates to \xcd"true" if both of its
operands evaluate to \xcd"true"; otherwise, the operator
evaluates to \xcd"false".

The \xcd"|" operator  evaluates to \xcd"false" if both of its
operands evaluate to \xcd"false"; otherwise, the operator
evaluates to \xcd"true".

\section{Boolean conditional operations}

Operands of the binary boolean conditional operators must be of type
\xcd"Boolean". 
The type of the result is \xcd"Boolean"

The \xcd"&&" operator  evaluates to \xcd"true" if both of its
operands evaluate to \xcd"true"; otherwise, the operator
evaluates to \xcd"false".
Unlike the logical operator \xcd"&",
if the first operand is \xcd"false",
the second operand is not evaluated.

The \xcd"||" operator  evaluates to \xcd"false" if both of its
operands evaluate to \xcd"false"; otherwise, the operator
evaluates to \xcd"true".
Unlike the logical operator \xcd"||",
if the first operand is \xcd"true",
the second operand is not evaluated.

\section{Relational operations} 

The relational operations compare numbers, producing \xcd`Boolean` results.  

The \xcd"<" operator evaluates to \xcd"true" if the left operand is
less than the right.
The \xcd"<=" operator evaluates to \xcd"true" if the left operand is
less than or equal to the right.
The \xcd">" operator evaluates to \xcd"true" if the left operand is
greater than the right.
The \xcd">=" operator evaluates to \xcd"true" if the left operand is
greater than or equal to the right.

Floating point comparison is determined by the IEEE 754
standard.  Thus,
if either operand is NaN, the result is \xcd"false".
Negative zero and positive zero are considered to be equal.
All finite values are less than positive infinity and greater
than negative infinity.

\section{Conditional expressions}
\label{Conditional}

\begin{grammar}
ConditionalExpression \: Expression
                \xcd"?" Expression
                \xcd":" Expression 
\end{grammar}

A conditional expression evaluates its first subexpression (the
condition); if \xcd"true"
the second subexpression (the consequent) is evaluated; otherwise,
the third subexpression (the alternative) is evaluated.

The type of the condition must be \xcd"Boolean".
The type of the conditional expression is some common 
ancestor (as constrained by \Sref{LCA}) of the types of the consequent and the
alternative. 

For example, 
%~~exp~~`~~`~~a:Int,b:Int ~~
\xcd`a == b ? 1 : 2`
evaluates to \xcd`1` if \xcd`a` and \xcd`b` are the same, and \xcd`2` if they
are different.   As the type of \xcd`1` is \xcd`Int{self==1}` and of \xcd`2`
is \xcd`Int{self==2}`, the type of the conditional expression has the form
\xcd`Int{c}`, where \xcd`self==1` and \xcd`self==2` both imply \xcd`c`.  For
example, it might be \xcd`Int{true}` -- or perhaps it might be 
\xcd`Int{self != 8}`. Note that this term has no most accurate type in the X10
type system.

\section{Stable equality}
\label{StableEquality}
\index{==}\index{!=}

\begin{grammar}
EqualityExpression \: Expression \xcd"==" Expression \\
\| Expression \xcd"!=" Expression \\
\end{grammar}

The \xcd"==" and \xcd"!=" operators provide a fundamental, though
non-abstract, notion of equality.  \xcd`a==b` is true if the values of \xcd`a`
and \xcd`b` are extremely identical, in a sense defined shortly.  \xcd`a != b`
is true iff \xcd`a==b` is false.

\begin{itemize}
\item If \xcd`a` and \xcd`b` are values of object type, then \xcd`a==b` holds
      if \xcd`a` and \xcd`b` are the same object.
\item If one operand is \xcd`null`, then \xcd`a==b` holds iff the other is
      also \xcd`null`.
\item If the operands both have struct type, then they must be structurally equal;
that is, they must be instances of the same struct
and all their fields or components must be \xcd"==". 
\item The definition of equality for function types is specified in
      \Sref{FunctionEquality}.
\item If the operands have numeric types, they are coerced into the larger of
      the two types and then compared for numeric equality.
\end{itemize}

The predicates \xcd"==" and \xcd"!=" may not be overridden by the programmer.
Note that \xcd`a==b` is a form of \emph{stable equality}; that is, the result of
the equality operation is not affected by the mutable state of the program,
after evaluation of \xcd`a` and \xcd`b`. 


\section{Allocation}
\label{ClassCreation}

\begin{grammar}
NewExpression \: \xcd"new" ClassName TypeArguments\opt \xcd"(" ArgumentList\opt \xcd")"
        ClassBody\opt \\
  \| \xcd"new" InterfaceName TypeArguments\opt \xcd"(" ArgumentList\opt \xcd")"
        ClassBody
\end{grammar}

An allocation expression creates a new instance of a class and
invokes a constructor of the class.
The expression designates the class name and passes
type and value arguments to the constructor.

The allocation expression may have an optional class body.
In this case, an anonymous subclass of the given class is
allocated.   An anonymous class allocation may also specify a
single super-interface rather than a superclass; the superclass
of the anonymous class is \xcd"x10.lang.Object".

If the class is anonymous---that is, if a class body is
provided---then the constructor is selected from the superclass.
The constructor to invoke is selected using the same rules as
for method invocation (\Sref{MethodInvocation}).

The type of an allocation expression
is the return type of the constructor invoked, with appropriate
substitutions  of actual arguments for formal parameters, as
specified in \Sref{MethodInvocationSubstitution}.

It is illegal to allocate an instance of an \xcd"abstract" class.
It is illegal to allocate an instance of a class or to invoke a
constructor that is not visible at
the allocation expression.

Note that instantiating a struct type uses function application syntax, not
\xcd`new`.  As structs do not have subclassing, there is no need or
possibility of a {\em ClassBody}.


\section{Casts}\label{ClassCast}\index{classcast}

The cast operation may be used to cast an expression to a given type:

\begin{grammar}
UnaryExpression \: CastExpression \\
CastExpression \: UnaryExpression \xcd"as" Type \\
\end{grammar}

The result of this operation is a value of the given type if the cast
is permissible at run time, and either a compile-time error or a runtime
exception 
(\xcd`x10.lang.TypeCastException`) if it is not.  

When evaluating \xcd`E as T{c}`, first the value of \xcd`E` is converted to
type \xcd`T` (which may fail), and then the constraint \xcd`{c}` is checked. 



\begin{itemize}
\item If \xcd`T` is a primitive type, then \xcd`E`'s value is converted to type
      \xcd`T` according to the rules of
      \Sref{sec:effects-of-explicit-numeric-coercions}. 
      
\item If \xcd`T` is a class, then the first half of the cast succeeds if the
      run-time value of \xcd`E` is an instance of class \xcd`T`, or of a
      subclass 

\item If \xcd`T` is an interface, then the first half of the cast succeeds if
      the run-time value of \xcd`E` is an instance of a class implementing
      \xcd`T`. 

\item If \xcd`T` is a struct type, then the first half of the cast succeeds if
      the run-time value of \xcd`E` is an instance of \xcd`T`.  

\item If \xcd`T` is a function type, then the first half of the cast succeeds
      if the run-time value of \xcd`X` is a function of precisely that type.
\end{itemize}

If the first half of the cast succeeds, the second half -- the constraint
\xcd`{c}` -- must be checked.  In general this will be done at runtime, though
in special cases it can be checked at compile time.   For example, 
\xcd`n as Int{self != w}` succeeds if \xcd`n != w` --- even if \xcd`w` is a value
read from input, and thus not determined at compile time.

The compiler may forbid casts that it knows cannot possibly work. If there is
no way for the value of \xcd`E` to be of type \xcd`T{c}`, then 
\xcd`E as T{c}` can result in a static error, rather than a runtime error.  
For example, \xcd`1 as Int{self==2}` may fail to compile, because the compiler
knows that \xcd`1`, which has type \xcd`Int{self==1}`, cannot possibly be of
type \xcd`Int{self==2}`. 


%BB% \bard{This section need serious whomping.  The Java mention needs to go.  The
%BB% rules for coercions are given in \Sref{sec:effects-of-explicit-numeric-coercions}.
%BB% If the \xcd`Type` has a constraint, the constraint will be checked at runtime. 
%BB% We need to give examples. 
%BB% }
%BB% 
%BB% Type conversion is checked according to the
%BB% rules of the \java{} language (e.g., \cite[\S 5.5]{jls2}).
%BB% For constrained types, both the base
%BB% type and the constraint are checked.
%BB% If the
%BB% value cannot be cast to the appropriate type, a
%BB% \xcd"ClassCastException"
%BB% is thrown. 



% {\bf Conversions of numeric values}
% {\bf Can't do (a as T) if a can't be a T.}


%If the value cannot be cast to the
%appropriate place type a \xcd"BadPlaceException" is thrown. 

% Any attempt to cast an expression of a reference type to a value type
% (or vice versa) results in a compile-time error. Some casts---such as
% those that seek to cast a value of a subtype to a supertype---are
% known to succeed at compile-time. Such casts should not cause extra
% computational overhead at run time.

\section{\Xcd{instanceof}}
\label{instanceOf}
\index{\Xcd{instanceof}}

\Xten{} permits types to be used in an in instanceof expression
to determine whether an object is an instance of the given type:

\begin{grammar}
RelationalExpression \: RelationalExpression \xcd"instanceof" Type
\end{grammar}

In the above expression, \grammarrule{Type} is any type. At run time, the
result of this operator is \xcd"true" if the
\grammarrule{RelationalExpression} can be coerced to \grammarrule{Type}
without a \xcd"TypeCastException" being thrown or static error occurring.
Otherwise the result is \xcd"false". This determination may involve checking
that the constraint, if any, associated with the type is true for the given
expression.

%~~exp~~`~~`~~x:Int~~
For example, \xcd`3 instanceof Int{self==x}` is an overly-complicated way of
saying \xcd`3==x`.

\section{Subtyping expressions}

\begin{grammar}
SubtypingExpression \: Expression \xcd"<:" Expression \\
                    \| Expression \xcd":>" Expression \\
                    \| Expression \xcd"==" Expression \\
\end{grammar}

The subtyping expression \xcdmath"T$_1$ <: T$_2$" evaluates to \xcd"true"
\xcdmath"T$_1$" is a subtype of \xcdmath"T$_2$".

The expression \xcdmath"T$_1$ :> T$_2$" evaluates to \xcd"true"
\xcdmath"T$_2$" is a subtype of \xcdmath"T$_1$".

The expression \xcdmath"T$_1$ == T$_2$"
evaluates to  \xcd"true" \xcdmath"T$_1$" is a subtype of \xcdmath"T$_2$" and
if \xcdmath"T$_2$" is a subtype of \xcdmath"T$_1$".

Subtyping expressions are particularly useful in giving constraints on generic
types.  \xcd`x10.util.Ordered[T]` is an interface whose values can be compared
with values of type \xcd`T`. 
In particular, \xcd`T <: x10.util.Ordered[T]` is
true if values of type \xcd`T` can be compared to other values of type
\xcd`T`.  So, if we wish to define a generic class \xcd`OrderedList[T]`, of
lists whose elements are kept in the right order, we need the elements to be
ordered.  This is phrased as a constraint on \xcd`T`: 
%~~gen
% package expre.ssi.onsfgua.rde.dq.uantification;
%~~vis
\begin{xten}
class OrderedList[T]{T <: x10.util.Ordered[T]} {
  // ...
}
\end{xten}
%~~siv
%
%~~neg




\section{Contains expressions}

\begin{grammar}
ContainsExpression \: Expression \xcd"in" Expression \\
\end{grammar}

The expression \xcd"p in r" tests if a value \xcd"p" is in a collection
\xcd"r"; it evaluates to \xcd"r.contains(p)".
The collection \xcd"r"
must be of type \xcd"Collection[T]" and the value \xcd"p" must
be of type \xcd"T".

\section{Rail constructors}
\label{RailConstructors}

\begin{grammar}
RailConstructor \: \xcd"[" Expressions \xcd"]" \\
Expressions \: Expression ( \xcd"," Expression )\star \\
\end{grammar}

The rail constructor \xcdmath"[a$_0$, $\dots$, a$_{k-1}$]"
creates an instance of \xcd"ValRail" with length $k$, 
whose $i$th element is
\xcdmath"a$_i$".  The element type of the rail is a common ancestor of the
types of the \xcdmath"a$_i$"'s, as per \Sref{LCA}.
%~~gen
% package ex.pre.ssio.nsandrailconstructors;
% class Example {
% def example() {
%~~vis
\begin{xten}
val a : ValRail[Int] = [1,3,5];
val b : ValRail[Any] = [1, a, "please"];
\end{xten}
%~~siv
% } } 
%~~neg

Since arrays are subtypes of \xcd"(Point) => T",
rail constructors can be passed into the \xcd"Array" and
\xcd"ValArray" constructors as initializer functions.

Rail constructors of type \xcd"ValRail[Int]" and length \xcd"n" 
may be implicitly converted to type \xcd"Point{rank==n}".
Rail constructors of type \xcd"ValRail[Region]" and length \xcd"n" 
may be implicitly converted to type \xcd"Region{rank==n}".

%~~gen
% package ex.pre.ssio.nsandrailconstructors;
% class Exympyl {
% def example() {
%~~vis
\begin{xten}
val a : Point{rank==4} = [1,2,3,4];
val b : Region{rank==2} = [ -1 .. 1, -2 .. 2];
\end{xten}
%~~siv
% } } 
%~~neg


\section{Coercions and conversions}
\label{XtenConversions}
\label{User-definedCoercions}
\index{conversions}\index{coercions}

\XtenCurrVer{} supports the following coercions and conversions.

\subsection{Coercions}

A {\em coercion} does not change object identity; a coerced object may
be explicitly coerced back to its original type through a cast. A {\em
  conversion} may change object identity if the type being converted
to is not the same as the type converted from. \Xten{} permits
user-defined conversions (\Sref{sec:user-defined-conversions}).

\paragraph{Subsumption coercion.}
A subtype may be implicitly coerced to any supertype.
\index{coercions!subsumption coercion}

\paragraph{Explicit coercion (casting with \xcd"as")}
An object of any class may be explicitly coerced to any other
class type using the \xcd"as" operation.  If \xcd`Child <: Person` and
\xcd`rhys:Child`, then 
%~~gen
% package Types.Coercions;
%  class Person {}
%  class Child extends Person{} 
%  class Exampllllle { 
%    def example(rhys:Child) =
%~~vis
\begin{xten}
  rhys as Person
\end{xten}
%~~siv
%;}
%~~neg
is an expression of type \xcd`Person`.  

If the value coerced is not an instance of the target type,
a \xcd"ClassCastException" is thrown.  Casting to a constrained
type may require a run-time check that the constraint is
satisfied.
\index{coercions!explicit coercion}
\index{casting}
\index{\Xcd{as}}

\limitation{It is currently a static error, rather than the specified
\xcd`ClassCastException`, when the cast is statically determinable to be
impossible.}

\paragraph{Effects of explicit numeric coercion}
\label{sec:effects-of-explicit-numeric-coercions}

Coercing a number of one type to another type gives the best approximation of
the number in the result type, or a suitable disaster value if no
approximation is good enough.  

\begin{itemize}
\item Casting a number to a {\em wider} numeric type is safe and effective,
      and can be done by an implicit conversion as well as an explicit
%~~exp~~`~~`~~ ~~
      coercion.  For example, \xcd`4 as Long` produces the \xcd`Long` value of
      4. 
\item Casting a floating-point value to an integer value truncates the digits
      after the decimal point, thereby rounding the number towards zero.  
%~~exp~~`~~`~~
      \xcd`54.321 as Int` is \xcd`54`, and 
%~~exp~~`~~`~~ ~~
      \xcd`-54.321 as Int` is \xcd`-54`.
      If the floating-point value is too large to represent as that kind of
      integer, the coercion returns the largest or smallest value of that type
      instead: \xcd`1e110 as Int` is 
      \xcd`Int.MAX_VALUE`, \xcd`2147483647`. 

\item Casting a \xcd`Double` to a \xcd`Float` normally truncates digits: 
%~~exp~~`~~`~~ ~~
      \xcd`0.12345678901234567890 as Float` is \xcd`0.12345679f`.  This can
      turn a nonzero \xcd`Double` into \xcd`0.0f`, the zero of type
      \xcd`Float`: 
%~~exp~~`~~`~~ ~~
      \xcd`1e-100 as Float` is \xcd`0.0f`.  Since 
      \xcd`Double`s can be as large as about \xcd`1.79E308` and \xcd`Float`s
      can only be as large as about \xcd`3.4E38f`, a large \xcd`Double` will
      be converted to the special \xcd`Float` value of \xcd`Infinity`: 
%~~exp~~`~~`~~ ~~
      \xcd`1e100 as Float` is \xcd`Infinity`.
\item Integers are coerced to smaller integer types by truncating the
      high-order bits. If the value of the large integer fits into the smaller
      integer's range, this gives the same number in the smaller type: 
%~~exp~~`~~`~~ ~~
      \xcd`12 as Byte` is the \xcd`Byte`-sized 12, 
%~~exp~~`~~`~~ ~~
      \xcd`-12 as Byte` is -12. 
      However, if the larger integer {\em doesn't} fit in the smaller type,
%~~exp~~`~~`~~ ~~
      the numeric value and even the sign can change: \xcd`254 as Byte` is
      \xcd`Byte`-sized \xcd`-2`.  


\end{itemize}

\subsection{Conversions}

\paragraph{Widening numeric conversion.}
A numeric type may be implicitly converted to a wider numeric type. In
particular, an implicit conversion may be performed between a numeric
type and a type to its right, below:

\begin{xten}
Byte < Short < Int < Long < Float < Double
\end{xten}

\index{conversions!widening conversions}
\index{conversions!numeric conversions}

\paragraph{String conversion.}
Any object that is an operand of the binary
\xcd"+" operator may
be converted to \xcd"String" if the other operand is a \xcd"String".
A conversion to \xcd"String" is performed by invoking the \xcd"toString()"
method of the object.

\index{conversions!string conversion}

\paragraph{User defined conversions.}\label{sec:user-defined-conversions}
\index{conversions!user defined}

The user may define conversion operators from type \Xcd{A} {\em to} a
container type \Xcd{B} by specifying a method on \Xcd{B} as follows:

\begin{xten}
  public static operator (r: A): T = ... 
\end{xten}

The return type \Xcd{T} should be a subtype of \Xcd{B}. The return
type need not be specified explicitly; it will be computed in the
usual fashion if it is not. However, it is good practice for the
programmer to specify the return type for such operators explicitly.

For instance, the code for \Xcd{x10.lang.Point} contains:

\begin{xten}
  public static global safe operator (r: Rail[int])
     : Point(r.length) = make(r);
\end{xten}

The compiler looks for such operators on the container type \Xcd{B}
when it encounters an expression of the form \Xcd{r as B} (where
\Xcd{r} is of type \Xcd{A}). If it finds such a method, it sets the
type of the expression \Xcd{r as B} to be the return type of the
method. Thus the type of \Xcd{r as B} is guaranteed to be some subtype
of \Xcd{B}.

\begin{example}
Consider the following code:  



%~~stmt~~\begin{xten}~~\end{xten}~~ ~~
\begin{xten}
val p  = [2, 2, 2, 2, 2] as Point;
val q = [1, 1, 1, 1, 1] as Point;
val a = p - q;    
\end{xten}
This code fragment compiles successfully, given the above operator definition. 
The type of \Xcd{p} is inferred to be \Xcd{Point(5)} (i.e.{} the type 
%~~type~~`~~`~~ ~~
\xcd`Point{self.rank==5}`.
Similarly for \Xcd{q}. Hence the application of the operator ``\Xcd{-}'' is legal (it requires both arguments to have the same rank). The type of \Xcd{a} is computed as \Xcd{Point(5)}.
\end{example}
	
\chapter{Statements}\label{XtenStatements}\index{statements}

This chapter describes the statements in the sequential core of
\Xten{}.  Statements involving concurrency and distribution
are described in \Sref{XtenActivities}.

\section{Empty statement}

\begin{grammar}
Statement \: \xcd";" \\
\end{grammar}

The empty statement \xcd";" does nothing.  It is useful when a
loop header is evaluated for its side effects.  For example,
the following code sums the elements of a rail.

%~~gen
% package statements.should.be.called.commands.but.nobody.ever.does.that;
% class EmptyStatementExample {
% def summizmo (a:Rail[Int]!){
%~~vis
\begin{xten}
var sum: Int = 0;
for (var i: Int = 0; i < a.length; i++, sum += a(i))
    ;
\end{xten}
%~~siv
%}}
%~~neg

\section{Local variable declaration}

Short-lived variables are introduced by local variables declarations, as
described in \Sref{VariableDeclarations}. Local variables may be declared only
within a block statement (\Sref{Blocks}). The scope of a local variable
declaration is the statement itself and the subsequent statements in the
block.
%~~gen
% package statements.should.have.locals;
% class LocalExample {
% def example(a:Int) {
%~~vis
\begin{xten}
  if (a > 1) {
     val b = a/2;
     var c : Int = 0;
     // b and c are defined here
  }
  // b and c are not defined here.
\end{xten}
%~~siv
%} }
%~~neg


\section{Block statement}
\label{Blocks}

\begin{grammar}
Statement \: BlockStatement \\
BlockStatement \: \xcd"{" Statement\star \xcd"}" \\
\end{grammar}

A block statement consists of a sequence of statements delimited by
``\xcd"{"'' and ``\xcd"}"''. When a block is evaluated, the statements inside
of it are evaluated in order.  Blocks are useful for putting several
statements in a place where X10 asks for a single one, such as the consequent
of an \xcd`if`, and for limiting the scope of local variables.
%~~gen
% package statements.FOR.block.heads;
% class Example {
% def example(b:Boolean, S1:(Int)=>Void, S2:(Int)=>Void ) {
%~~vis
\begin{xten}
if (b) {
  // This is a block
  val v = 1;
  S1(v); 
  S2(v);
}
\end{xten}
%~~siv
%  } } 
%~~neg


\section{Expression statement}

\begin{grammar}
Statement \: ExpressionStatement \\
ExpressionStatement \: StatementExpression \xcd";" \\
StatementExpression \: Assignment \\
          \| NewExpression \\
          \| Call \\
\end{grammar}

The expression statement evaluates an expression.  The value of the expression
is not used.
Side effects of the expression occur.  
%~~gen
% package Sta.tem.ent.s.expressions;
% import x10.util.*;
%~~vis
\begin{xten}
class StmtEx {
  def this() { x10.io.Console.OUT.println("New StmtEx made");  }
  static def call() { x10.io.Console.OUT.println("call!");  }
  def example() {
     var a : Int = 0;
     a = 1; // assignment
     new StmtEx(); // allocation
     call(); // call
  }
}
\end{xten}
%~~siv
%
%~~neg
Note that only selected forms of expression can appear in expression
statements.  


\section{Labeled statement}

\begin{grammar}
Statement \: LabeledStatement \\
LabeledStatement \: Identifier \xcd":" LoopStatement \\
\end{grammar}

Loop statements (\xcd`for`, \xcd`while`, \xcd`do`, \xcd`ateach`,
\xcd`foreach`) may be labeled. The label may be used as the target of a break
or continue statement. The scope of a label is the statement labeled.
%~~gen
% package state.meant.labe.L;
% class Example {
% def example(a:(Int,Int) => Int, do_things_to:(Int)=>Int) {
%~~vis
\begin{xten}
lbl : for ((i) in 1..10) {
   for ((j) in i..10) {  
      if (a(i,j) == 0) break lbl;
      if (a(i,j) == 1) continue lbl;
      do_things_to(a(i,j)); 
   }
}
\end{xten}
%~~siv
%} } 
%~~neg


\section{Break statement}

\begin{grammar}
Statement \: BreakStatement \\
BreakStatement \: \xcd"break" Identifier\opt \\
\end{grammar}

An unlabeled break statement exits the currently enclosing loop or switch
statement. An labeled break statement exits the enclosing loop or switch
statement with the given label.
It is illegal to break out of a loop not defined in the current
method, constructor, initializer, or closure.  

The following code searches for an element of a two-dimensional
array and breaks out of the loop when it is found:

%~~gen
% package statements.come.from.banks.and.cranks;
% class LabelledBreakeyBreakyHeart {
% def findy(a:ValRail[ValRail[Int]], v:Int): Boolean {
%~~vis
\begin{xten}
var found: Boolean = false;
outer: for (var i: Int = 0; i < a.length; i++)
    for (var j: Int = 0; j < a(i).length; j++)
        if (a(i)(j) == v) {
            found = true;
            break outer;
        }
\end{xten}
%~~siv
% return found;
%}}
%~~neg

\section{Continue statement}

\begin{grammar}
Statement \: ContinueStatement \\
ContinueStatement \: \xcd"continue" Identifier\opt \\
\end{grammar}

An unlabeled \xcd`continue` skips the rest of the current iteration of the
innermost enclosing loop, and proceeds on to the next.  A labeled
\xcd`continue` does the same to the enclosing loop with that label.
It is illegal to continue a loop not defined in the current
method, constructor, initializer, or closure.

\section{If statement}

\begin{grammar}
Statement \: IfThenStatement \\
          \| IfThenElseStatement \\
IfThenStatement \: \xcd"if" \xcd"(" Expression \xcd")" Statement \\
IfThenElseStatement \: \xcd"if" \xcd"(" Expression \xcd")" Statement \xcd"else" Statement \\
\end{grammar}

An if statement comes in two forms: with and without an else
clause.

The if-then statement evaluates a condition expression, which must be of type
\xcd`Boolean`. If the condition is \xcd`true`, it evaluates the then-clause.
If the condition is \xcd"false", the if-then statement completes normally.

The if-then-else statement evaluates a condition expression and 
evaluates the then-clause if the condition is
\xcd"true"; otherwise, the \xcd`else`-clause is evaluated.

As is traditional in languages derived from Algol, the if-statement is syntactically
ambiguous.  That is, 
\begin{xten}
if (B1) if (B2) S1 else S2
\end{xten}
could be intended to mean either 
\begin{xten}
if (B1) { if (B2) S1 else S2 }
\end{xten} 
or 
\begin{xten}
if (B1) {if (B2) S1} else S2
\end{xten}
X10, as is traditional, attaches an \xcd`else` clause to the most recent
\xcd`if` that doesn't have one.
This example is interpreted as 
\xcd`if (B1) { if (B2) S1 else S2 }`. 



\section{Switch statement}

\begin{grammar}
Statement \: SwitchStatement \\
SwitchStatement \: \xcd"switch" \xcd"(" Expression \xcd")" \xcd"{" Case\plus \xcd"}" \\
Case \: \xcd"case" Expression \xcd":" Statement\star \\
     \| \xcd"default" \xcd":" Statement\star \\
\end{grammar}

A switch statement evaluates an index expression and then branches to
a case whose value equal to the value of the index expression.
If no such case exists, the switch branches to the 
\xcd"default" case, if any.

Statements in each case branch evaluated in sequence.  At the
end of the branch, normal control-flow falls through to the next case, if
any.  To prevent fall-through, a case branch may be exited using
a \xcd"break" statement.

The index expression must be of type \xcd"Int".
Case labels must be of type \xcd"Int", \xcd`Byte`, \xcd`Short`, or \xcd`Char`
and must be compile-time 
constants.  Case labels cannot be duplicated within the
\xcd"switch" statement.

In the following example, case \xcd`1` falls through to case3 \xcd`2`.  The
other cases are separated by \xcd`break`s.
%~~gen
% package Statement.Case;
% class Example {
% def example(i : Int, println: (String)=>Void) {
%~~vis
\begin{xten}
switch (i) {
  case 1: println("one, and ");
  case 2: println("two"); 
          break;
  case 3: println("three");
          break;
  default: println("Something else");
           break;
}
\end{xten}
%~~siv
% } } 
%~~neg


\section{While statement}
\index{\Xcd{while}}

\begin{grammar}
Statement \: WhileStatement \\
WhileStatement \: \xcd"while" \xcd"(" Expression \xcd")" Statement \\
\end{grammar}

A while statement evaluates a \xcd`Boolean`-valued condition and executes a
loop body if \xcd"true". If the loop body completes normally (either by
reaching the end or via a \xcd"continue" statement with the loop header as
target), the condition is reevaluated and the loop repeats if \xcd"true". If
the condition is \xcd"false", the loop exits.

%~~gen
% package Statements.AreFor.Flatements;
% class Example {
% def example(var n:Int) {
%~~vis
\begin{xten}
  while (n > 1) {
     n = (n % 2 == 1) ? 3*n+1 : n/2;
  }
\end{xten}
%~~siv
% } } 
%~~neg

\section{Do--while statement}
\index{\Xcd{do}}

\begin{grammar}
Statement \: DoWhileStatement \\
DoWhileStatement \: \xcd"do" Statement \xcd"while" \xcd"(" Expression \xcd")" \xcd";" \\
\end{grammar}


A do-while statement executes a loop body, and then evaluates a
\xcd`Boolean`-valued condition expression. If \xcd"true", the loop repeats.
Otherwise, the loop exits.



\section{For statement}
\index{\Xcd{for}}

\begin{grammar}
Statement \: ForStatement \\
          \| EnhancedForStatement \\
ForStatement \: \xcd"for" \xcd"("
        ForInit\opt \xcd";" Expression\opt \xcd";" ForUpdate\opt
        \xcd")" Statement \\
ForInit \:
        StatementExpression ( \xcd"," StatementExpression )\star
        \\
      \| LocalVariableDeclaration \\
ForUpdate \:
        StatementExpression ( \xcd"," StatementExpression )\star\\
EnhancedForStatement \: \xcd"for" \xcd"("
        Formal \xcd"in" Expression 
        \xcd")" Statement \\
\end{grammar}

\xcd`for` statements provide bounded iteration, such as looping over a list.
It has two forms: a basic form allowing near-arbitrary iteration, {\em a la}
C, and an enhanced form designed to iterate over a collection.

A basic \xcd`for` statement provides for arbitrary iteration in a somewhat
more organized fashion than a \xcd`while`.  \xcd`for(init; test; step)body` is
equivalent to: 
\begin{xten}
{
   init;
   while(test) {
      body;
      step;
   }
}
\end{xten}

\xcd`init` is performed before the loop, and is traditionally used to declare
and/or initialize the loop variables. It may be a single variable binding
statement, such as \xcd`var i:Int = 0` or \xcd`var i:Int=0, j:Int=100`. (Note
that a single variable binding statement may bind multiple variables.)
Variables introduced by \xcd`init` may appear anywhere in the \xcd`for`
statement, but not outside of it.  Or, it may be a sequence of expression
statements, such as \xcd`i=0, j=100`, operating on already-defined variables.
If omitted, \xcd`init` does nothing.

\xcd`test` is a Boolean-valued expression; an iteration of the loop will only
proceed if \xcd`test` is true at the beginning of the loop, after \xcd`init`
on the first iteration or or \xcd`step` on later ones. If omitted, \xcd`test`
defaults to \xcd`true`, giving a loop that will run until stopped by some
other means such as \xcd`break`, \xcd`return`, or \xcd`throw`.

\xcd`step` is performed after the loop body, between one iteration and the
next. It traditionally updates the loop variables from one iteration to the
next: \eg, \xcd`i++` and \xcd`i++,j--`.  If omitted, \xcd`step` does nothing.

\xcd`body` is a statement, often a code block, which is performed whenever
\xcd`test` is true.  If omitted, \xcd`body` does nothing.




\label{ForAllLoop}


An enhanced for statement is used to iterate over a collection, or other
structure designed to support iteration by implementing the interface
\xcd`Iterable[T]`.    The loop variable must be of type \xcd`T`, 
or destructurable from a value of type \xcd`T`
(\Sref{exploded-syntax}; in practice, this means that 
\xcd`for ((i) in 1..10)` iterates over numbers from 1 to 10, while 
\xcd`for (i in 1..10` iterates over \xcd`Point`s from 1 to 10).
Each iteration of the loop
binds the iteration variable to another element of the collection.

%~~gen
% package statements.for_for;
% class Example {
% def example(n:Int) {
%~~vis
\begin{xten}
  var sum : Int = 0;
  for ((i) in 1..n) sum += i;
\end{xten}
%~~siv
% } } 
%~~neg



Certain common variant cases are accepted.  If collection is of type
\xcd"Region", the iteration variable may be of type \xcd"Point". 
If the iteratable \xcd`e` is of type \xcd`Dist` or \xcd`Array`, it is treated
as if it were \xcd`e.region`.  





\section{Throw statement}
\index{throw}

\begin{grammar}
Statement \: ThrowStatement \\
ThrowStatement \: \xcd"throw" Expression \xcd";"
\end{grammar}

\index{Exception}
The \xcd"throw" statement throws an exception, which 
must be an instance of \xcd"x10.lang.Throwable". 

For example, the following code checks if an index is in range and
throws an exception if not.

%~~gen
% package statements.are.from.mars.expressions.are.from.venus;
% class ThrowStatementExample {
% def thingie(i:Int, x:ValRail[Boolean]) throws MyIndexOutOfBoundsException {
%~~vis
\begin{xten}
if (i < 0 || i >= x.length)
    throw new MyIndexOutOfBoundsException();
\end{xten}
%~~siv
%} }
% class MyIndexOutOfBoundsException extends Exception {}
%~~neg

\section{Try--catch statement}

\begin{grammar}
Statement \: TryStatement \\
TryStatement \: \xcd"try" BlockStatement Catch\plus  \\
             \| \xcd"try" BlockStatement Catch\star Finally \\
Catch \: \xcd"catch" \xcd"(" Formal \xcd")" BlockStatement \\
Finally \: \xcd"finally" BlockStatement \\
\end{grammar}

Exceptions are handled with a \xcd"try" statement.
A \xcd"try" statement consists of a \xcd"try" block, zero or more
\xcd"catch" blocks, and an optional \xcd"finally" block.

First, the \xcd"try" block is evaluated.  If the block throws an
exception, control transfers to the first matching \xcd"catch"
block, if any.  A \xcd"catch" matches if the value of the
exception thrown is a subclass of the \xcd"catch" block's formal
parameter type.

The \xcd"finally" block, if present, is evaluated on all normal
and exceptional control-flow paths from the \xcd"try" block.
If the \xcd"try" block completes normally
or via a \xcd"return", a \xcd"break", or a
\xcd"continue" statement, 
the \xcd"finally"
block is evaluated, and then control resumes at
the statement following the \xcd"try" statement, at the branch target, or at
the caller as appropriate.
If the \xcd"try" block completes
exceptionally, the \xcd"finally" block is evaluated after the
matching \xcd"catch" block, if any, and then the
exception is rethrown.

It is a static error to attempt to catch an exception type which is not
throwable by the block.  

\section{Return statement}
\label{ReturnStatement}
\index{ReturnStatement}
\begin{grammar}
Statement \: ReturnStatement \\
ReturnStatement \: \xcd"return" Expression \xcd";" \\
             \| \xcd"return" \xcd";" \\
\end{grammar}

Methods and closures may return values using a return statement.
If the method's return type is expliclty declared \xcd"Void",
the method must return without a value; otherwise, it must return
a value of the appropriate type.
	
\chapter{Places}\label{XtenPlaces}\index{places}

An \Xten{} place is a repository for data and activities, corresponding
loosely to a process or a processor. Places induce a concept of ``local''. The
activities running in a place may access data items located at that place with
the efficiency of on-chip access. Accesses to remote places may take orders of
magnitude longer. X10's system of places is designed to make this obvious.
Programmers are aware of the places of their data, and know when they are
incurring communication costs, but the actual operation to do so is easy. It's
not hard to use non-local data; it's simply hard to to do so accidentally.

The set of places available to a computation is determined at the time that
the program is started, and remains fixed through the run of the program. See
the {\tt README} documentation on how to set command line and configuration
options to set the number of places.)

Places are first-class values in X10, as instances of the built-in struct,
\xcd"x10.lang.Place".   \xcd`Place` provides a number of useful ways to
query places, such as \xcd`Place.places`, a \xcd`ValRail` of all the places
available to the current run of the program.

Every object \xcd`ob` created by the program has a {\em home place}
\xcd`ob.home`. Accesses to \xcd`ob` from activities located at \xcd`ob.home`
are privileged.  Mutable fields can only be changed from \xcd`ob.home`, and
normal fields and methods can only be accessed from there. 

However, objects can be referred to from anywhere.  Places other than
\xcd`ob.home` may have {\em remote references} to \xcd`ob`.  Remote references
convey fewer privileges than local ones, but they are far from useless.
Fields and methods can be defined to be \xcd`global`, usable from anywhere. 


\section{The Structure of Places}

Places are numbered 0 through \xcd`Places.MAX_PLACES`, stored in the field
\xcd`pl.id`.  The \xcd`ValRail` \xcd`Place.places` contains the places of the
program, in numeric order. 
The program starts by executing a \xcd`main` method at
\xcd`Place.FIRST_PLACE`, which is \xcd`Place.places(0)`; see
\Sref{initial-computation}. 

Operations on places include \xcd`pl.next()`, which gives the next entry
(looping around) in \xcd`Place.places` and its opposite \xcd`pl.prev()`. In
particular, \xcd`here.next()` means ``a place other than \xcd`here`'', except
in single-place executions.
%~~exp~~`~~`~~pl:Place~~
There are also a number of tests, like \xcd`pl.isSPE()` and 
%~~exp~~`~~`~~pl:Place ~~
\xcd`pl.isCUDA()`, which test for particular kinds of processors.

Future versions of the language may permit user-definable
places, and the ability to dynamically create places.

Place expressions \index{place expression} (\viz, expressions of type
\xcd`Place`), such as \xcd`here` 
%~~exp~~`~~`~~ ob:Object~~
and \xcd`ob.home`, are used in \xcd`at` and
\xcd`asynch` statements.  



\section{\Xcd{here}}\index{here}\label{Here}

The variable \xcd"here" is always bound to the place at which the current
computation is running, in the same way that \xcd`this` is always bound to the
instance of the current class (for non-static code), or \xcd`self` is bound to
the instance of the type currently being constrained.  
\xcd`here` may denote different places in the same method body, due to
place-shifting operations. In the following code, \xcd`here` has one value at
\xcd`h0`, and a different one at \xcd`h1`. 
%~~gen
% package places.are.For.Graces;
% class Example {
% def example() {
%~~vis
\begin{xten}
val h0 = here;
at (here.next()) {
  val h1 = here; 
  assert (h0 != h1);
}
\end{xten}
%~~siv
%} } 
%~~neg
\noindent
(Similar examples show that \xcd`self` and \xcd`this` have the same behavior:
\xcd`self` can be modified by constrained types appearing inside of type
constraints, and \xcd`this` by inner classes.)


\begin{grammar}
Expression \: \xcd"here" \\
\end{grammar}

\begin{example}
For example, the following method looks through a collection of \xcd`Thing`s
for ones which belong in the current place \xcd`here`, and deals with the
things which do.  Note that every object \xcd`thing` has a property
\xcd`thing.home` giving its home location.
%~~gen
%package Places.Are.For.Graces;;
%abstract class Thing {}
%class DoMine {
%  static def dealWith(Thing) {}	
%~~vis
\begin{xten}
  public static def dealWithLocal(things: Rail[Thing]!) {
     for(thing in things) {
    	 if (thing.home == here) 
            dealWith(thing);
     }	  
  }
\end{xten}
%~~siv
%}
%~~neg



\end{example}

\xcd`here` is frequently used in constraints, quite often of the form
\xcd`ob.home == here`. Such constraints are necessary to check that a
non-global method can be called on \xcd`ob`: 


%TODO~~gen
% package Places.Are.For.Aces.Of.Spaces;
% class Thing {
% def nonGlobalMethod():Void{}
% static def example() {
%TODO~~vis
\begin{xten}
val ob : Thing{self.home == here} = new Thing();
ob.nonGlobalMethod();
\end{xten}
%TODO~~siv
%}}
%TODO~~neg

This idiom is so common and useful that the constraint
\xcd`T{self.home==here}` can be abbreviated as \xcd`T!`: 

%~~gen
% package Places.Are.For.Aces.Of.Graces;
% class Thing {
% def nonGlobalMethod():Void{}
% static def example() {
%~~vis
\begin{xten}
val ob : Thing! = new Thing();
ob.nonGlobalMethod();
\end{xten}
%~~siv
%}}
%~~neg


\limitationx {\em In the current implementation, sometimes \xcd^T{self.home==here}^
does not work, though \Xcd{T!} does.}

	
\chapter{Activities}\label{XtenActivities}
\index{activity}


An {\em activity} is a statement being executed, independently, with its own
local variables; it may be thought of as a very light-weight thread. An
\Xten{} computation may have many concurrent {activities} executing at any
give time.  All X10 code runs as part of an activity; when an X10 program is
started, the \xcd`main` method is invoked in an activity, called the {\em root
activity}.\index{root
activity}


Activities coordinate their execution by various control and data structures.
For example, 
%~~stmt~~`~~`~~x:Int, var y:Int ~~ ^^^ Activities10
\xcd`when(x==0);` blocks the current activity until some other activity
sets \xcd`x` to zero.  However, activities determine the places at which they
may be blocked and resumed, by \xcd`when` and similar constructs.  There are
no means by which one activity can arbitrarily interrupt, block, or resume
another, no method  \xcd`activity.interrupt()`.


\index{activity!running}
\index{activity!blocked}
\index{activity!terminated}
An activity may be {\em running}, {\em blocked} on some condition or {\em
terminated}. If it is terminated, it is terminated in the same way that its
statement is: in particular, if the statement terminates abruptly, the
activity terminates abruptly for the same reason.
(\Sref{ExceptionModel}).

Activities can be long-running entities with a good deal of local state.  In
particular they can involve recursive method calls (and therefore have runtime
stacks).  However, activities can also be short-running light-weight entities,
\eg, it is reasonable to have an activity that simply increments a variable.

An activity may asynchronously and in parallel launch activities at
other places.  Every activity except the initial \xcd`main` activity is spawned
by another.  Thus, at any instant, the activities in a program form a tree.

\index{termination}
\index{termination!local}
\index{termination!global}
X10 uses this tree in crucial ways.  
First is the distinction 
between {\em local} termination and {\em global}
termination of a statement. The execution of a statement by an
activity is said to terminate locally when the activity has finished
all its computation. (For instance the
creation of an asynchronous activity terminates locally when the
activity has been created.)  It is said to terminate globally when it
has terminated locally and all activities that it may have spawned at
any place have, recursively, terminated globally.
For example, consider: 
%~~gen ^^^ Activities20
% package Activites.Are.For.Whacktivities;
% class Example {
% def example( s1:() => void, s2 : () => void ) {
%~~vis
\begin{xten}
async {s1();}
async {s2();}
\end{xten}
%~~siv
% } } 
%~~neg
The primary activity spawns two child activities and then terminates locally,
very quickly.  The child activities may take arbitrary amounts of time to
terminate (and may spawn grandchildren).  When \xcd`s1()`, \xcd`s2()`, and
all their descendants terminate locally, then the primary activity terminates
globally. 

The program as a whole terminates when the root activity terminates globally.
In particular, X10 does not permit the creation of 
daemon threads---threads that outlive the lifetime of the root
activity.  We say that an \Xten{} computation is {\em rooted}
(\Sref{initial-computation}).

\futureext{ We may permit the initial activity to be a daemon activity
to permit reactive computations, such as webservers, that may not
terminate.}

\section{The \Xten{} rooted exception model}
\label{ExceptionModel}
\index{exception!model}
\index{exception!rooted}
\index{exception}


The rooted nature of \Xten{} computations permits the definition of a
{\em rooted exception model.} In multi-threaded programming languages
there is a natural parent-child relationship between a thread and a
thread that it spawns. Typically the parent thread continues execution
in parallel with the child thread. Therefore the parent thread cannot
serve to catch any exceptions thrown by the child thread. 

The presence of a root activity and the concept of global termination permits
\Xten{} to adopt a more powerful exception model. In any state of the
computation, say that an activity $A$ is {\em a root of} an activity $B$ if
$A$ is an ancestor of $B$ and $A$ is blocked at a statement (such as the
\xcd"finish" statement \Sref{finish}) awaiting the termination of $B$ (and
possibly other activities). For every \Xten{} computation, the \emph{root-of}
relation is guaranteed to be a tree. The root of the tree is the root activity
of the entire computation. If $A$ is the nearest root of $B$, the path from
$A$ to $B$ is called the {\em activation path} for the activity.\footnote{Note
  that depending on the state of the computation the activation path may
  traverse activities that are running, blocked or terminated.}

We may now state the exception model for \Xten.  An uncaught exception
propagates up the activation path to its nearest root activity, where
it may be handled locally or propagated up the \emph{root-of} tree when
the activity terminates (based on the semantics of the statement being
executed by the activity).\footnote{In \XtenCurrVer{} the \xcd"finish"
statement is the only statement that marks its activity as a root
activity. Future versions of the language may introduce more such
statements.}  
There is always a good place to put a \xcd`try`-\xcd`catch` block to catch
exceptions thrown by an asynchronous activity.

\section{{\tt async}: Spawning an activity}\label{AsynchronousActivity}\label{AsyncActivity}
\index{async}
\index{activity!creating}

Asynchronous activities serve as a single abstraction for supporting a
wide range of concurrency constructs such as message passing, threads,
DMA, streaming, data prefetching. (In general, asynchronous operations
are better suited for supporting scalability than synchronous
operations.)

An activity is created by executing the \xcd`async` statement: 

%##(AsyncStatement ClockedClause
\begin{bbgrammar}
%(FROM #(prod:AsyncStatement)#)
      AsyncStatement \: \xcd"async" ClockedClause\opt Statement & (\ref{prod:AsyncStatement}) \\
                    \| \xcd"clocked" \xcd"async" Statement \\
%(FROM #(prod:ClockedClause)#)
       ClockedClause \: \xcd"clocked" \xcd"(" ClockList \xcd")" & (\ref{prod:ClockedClause}) \\
\end{bbgrammar}
%##)


The basic form of \xcd`async` is \xcd`async S`, which starts a new activity
located \xcd`here` executing \xcd`S`.   (For the clocked form, see
\Sref{ClockedFinish}.)  

Multiple activities launched by a single activity at another place are not
ordered in any way. They are added to the set of activities at the target
place and will be executed based on the local scheduler's decisions.
If some particular sequencing of events is needed, \xcd`when`, \xcd`atomic`,
\xcd`finish`, clocks, and other X10 constructs can be used.
\Xten{} implementations are not required to have fair schedulers,
though every implementation should make a best faith effort to ensure
that every activity eventually gets a chance to make forward progress.

\begin{staticrule*}
The statement in the body of an \xcd"async" is subject to the
restriction that it must be acceptable as the body of a \xcd"void"
method for an anonymous inner class declared at that point in the code. As
such, it may reference variables in lexically enclosing scopes.
\end{staticrule*}

\section{Finish}\index{finish}\label{finish}
The statement \xcd"finish S" converts global termination to local
termination.

%##(FinishStatement
\begin{bbgrammar}
%(FROM #(prod:FinishStatement)#)
     FinishStatement \: \xcd"finish" Statement & (\ref{prod:FinishStatement}) \\
                    \| \xcd"clocked" \xcd"finish" Statement \\
\end{bbgrammar}
%##)

An activity $A$ executes \xcd"finish S" by executing \xcd"S" and
then waiting for all activities spawned by \xcd`S` (directly or
indirectly, here or at other places) to terminate. An activity may
terminate normally, or abruptly, i.e. by throwing an exception.
All exceptions thrown by spawned activities are caught and
accumulated. 

\xcd"finish S" terminates locally when all activities spawned by
\xcd"S" terminate globally (either abruptly or normally). If \xcd"S"
terminates normally, then \xcd"finish S" terminates normally and $A$
continues execution with the next statement after \xcd"finish S".  If
\xcd"S" or one of the activities spawned by it terminate abruptly,
then \xcd"finish S" terminates abruptly and throws a single exception,
\Xcd{x10.lang.MultipleExceptions} formed from the collection of
exceptions accumulated at \xcd"finish S".

Thus \xcd"finish S" statement serves as a collection point for
uncaught exceptions generated during the execution of \xcd"S".

Note that repeatedly \xcd"finish"ing a statement has little effect after
the first \xcd"finish": \xcd"finish finish S" is indistinguishable
from \xcd"finish S" if \xcd`S` throws no exceptions.  (If \xcd`S` throws
exceptions, \xcd`finish S` wraps them in one layer of 
\xcd`MultipleExceptions` and \xcd`finish finish S` in two layers.)

%%OLIVIER-DENIES%% \paragraph{Interaction with clocks.}\label{sec:finish:clock-rule}
%%OLIVIER-DENIES%% 
%%OLIVIER-DENIES%% \xcd"finish S" interacts with clocks (\Sref{XtenClocks}). 
%%OLIVIER-DENIES%% While executing \xcd"S", an activity must not spawn any \xcd"clocked"
%%OLIVIER-DENIES%% asyncs. (Asyncs spawned during the execution of \xcd"S" may spawn
%%OLIVIER-DENIES%% clocked asyncs.) A
%%OLIVIER-DENIES%% \xcd"ClockUseException"\index{clock!ClockUseException} is thrown
%%OLIVIER-DENIES%% if (and when) this condition is violated.
%%OLIVIER-DENIES%% 
%%OLIVIER-DENIES%% This is necessary to prevent deadlocks.  In the following invalid code 
%%OLIVIER-DENIES%% %~s~gen
%%OLIVIER-DENIES%% % package Activities.Finish.Hates.Clocks;
%%OLIVIER-DENIES%% % class Example{
%%OLIVIER-DENIES%% % def example() {
%%OLIVIER-DENIES%% %~s~vis
%%OLIVIER-DENIES%% \begin{xten}
%%OLIVIER-DENIES%% val c:Clock = Clock.make();
%%OLIVIER-DENIES%% async clocked(c) {                // (A) 
%%OLIVIER-DENIES%%       finish async clocked(c) {   // (B) INVALID
%%OLIVIER-DENIES%%             next;                 // (Bnext)
%%OLIVIER-DENIES%%       }
%%OLIVIER-DENIES%%       next;                       // (Anext)
%%OLIVIER-DENIES%% }
%%OLIVIER-DENIES%% \end{xten}
%%OLIVIER-DENIES%% %~s~siv
%%OLIVIER-DENIES%% % } } 
%%OLIVIER-DENIES%% %~s~neg
%%OLIVIER-DENIES%% \xcd`(A)`, first of all, waits for the \xcd`finish` containing \xcd`(B)` to
%%OLIVIER-DENIES%% finish.  
%%OLIVIER-DENIES%% \xcd`(B)` will execute its \xcd`next` at \xcd`(Bnext)`, and then wait for all
%%OLIVIER-DENIES%% other activities registered on \xcd`c` to execute their \xcd`next`s.
%%OLIVIER-DENIES%% However, \xcd`(A)` is registered on \xcd`c`.  So, \xcd`(B)` cannot finish
%%OLIVIER-DENIES%% until \xcd`(A)` has proceeded to \xcd`(Anext)`, and \xcd`(A)` cannot proceed
%%OLIVIER-DENIES%% until \xcd`(B)` finishes. Thus, this causes deadlock.
%%OLIVIER-DENIES%% 
%%OLIVIER-DENIES%% 
%%OLIVIER-DENIES%% 
%%OLIVIER-DENIES%% In \XtenCurrVer{} this condition is checked dynamically; future
%%OLIVIER-DENIES%% versions of the language will introduce type qualifiers which permit
%%OLIVIER-DENIES%% this condition to be checked statically.
%%OLIVIER-DENIES%% 
%%OLIVIER-DENIES%% \futureext{
%%OLIVIER-DENIES%% The semantics of \xcd"finish S" is conjunctive; it terminates when all
%%OLIVIER-DENIES%% the activities created during the execution of \xcd"S" (recursively)
%%OLIVIER-DENIES%% terminate. In many situations (e.g., nondeterministic search) it is
%%OLIVIER-DENIES%% natural to require a statement to terminate when any {\em one} of the
%%OLIVIER-DENIES%% activities it has spawned succeeds. The other activities may then be
%%OLIVIER-DENIES%% safely aborted. Future versions of the language may introduce a
%%OLIVIER-DENIES%% \xcd"finishone S" construct to support such speculative or nondeterministic
%%OLIVIER-DENIES%% computation.
%%OLIVIER-DENIES%% }
%%OLIVIER-DENIES%% 

\section{Initial activity}\label{initial-computation}
\index{initial activity}
\index{activity!initial}

An \Xten{} computation is initiated from the command line on the
presentation of a class or struct name \xcd"C". The class or struct must have a
\xcd"public static def main(a: Array[String](1)):void" method, 
or a \xcd"public static def main(a: Array[String]):void" method, 
otherwise an exception is thrown
and the computation terminates.  The single statement
\begin{xten}
finish async (Place.FIRST_PLACE) {
  C.main(s);
}
\end{xten} 
\noindent is executed where \xcd"s" is a one-dimensional \xcd`Array` of
strings created 
from the command line arguments. This single activity is the root activity
for the entire computation. (See \Sref{XtenPlaces} for a discussion of
places.)

%% Say something about configuration information? 

\section{Ateach statements}\index{\Xcd{ateach}}\label{ateach-section}
\index{ateach}
\deprecated{} The \xcd`ateach` construct is deprecated.

%##(AtEachStatement LoopIndexDeclarator LoopIndex 
\begin{bbgrammar}
%(FROM #(prod:AtEachStatement)#)
     AtEachStatement \: \xcd"ateach" \xcd"(" LoopIndex \xcd"in" Exp \xcd")" ClockedClause\opt Statement & (\ref{prod:AtEachStatement}) \\
                    \| \xcd"ateach" \xcd"(" Exp \xcd")" Statement \\
%(FROM #(prod:LoopIndexDeclarator)#)
 LoopIndexDeclarator \: Id HasResultType\opt & (\ref{prod:LoopIndexDeclarator}) \\
                    \| \xcd"[" IdList \xcd"]" HasResultType\opt \\
                    \| Id \xcd"[" IdList \xcd"]" HasResultType\opt \\
%(FROM #(prod:LoopIndex)#)
           LoopIndex \: Mods\opt LoopIndexDeclarator & (\ref{prod:LoopIndex}) \\
                    \| Mods\opt VarKeyword LoopIndexDeclarator \\
\end{bbgrammar}
%##)
In \xcd`ateach(p in D) S`, \xcd`D` must be either of type \xcd"Dist"
(see \Sref{XtenDistributions}) or of type \xcd`DistArray[T]` (see
\Sref{XtenArrays}), and \xcd`p` will be of type \xcd"Point" (see
\Sref{point-syntax}). If \xcd`D` is an \xcd`DistArray[T]`, then
\xcd`ateach (p in D)S` is identical to 
\xcd`ateach(p in D.dist)S`; the iteration is over the array's underlying
distribution.   

Instead of writing \xcd`ateach (p in D) S` the programmer should write 
\xcd`for(p in D) at(D(p);F) async S` to get the same effect. 
For each point \xcd`p` in \xcd`D`, at place \xcd`D(p)`, transmitting
information as specified by \xcd`F`, 
\xcd`S` is
executed simultaneously.

However, this often results in excessive communication and parallelism. Instead the
programmer may want to write: 
%~~gen ^^^ Activities80
% package Activities.Activities.Activities;
% class EquivCode {
% static def S(pt:Point) {}
% static def example(D:Dist) {
% KNOWNFAIL-at
%~~vis
\begin{xten}
for (place in D.places()) async at (place; p, D, *) {
    for (p in D|here) {
        S(p);
    }
}
\end{xten}
%~~siv
%}} 
%~~neg

If the programmer wishes to execute \xcd`S` in parallel at each place,
\xcd`S(p)` may be replaced by 
\xcd`async S(p)`
.




\section{Atomic blocks}\label{AtomicBlocks}
\index{atomic}

X10's \xcd`atomic` blocks provide a high-level construct for coordinating
the mutation of shared data. 
A programmer may use atomic blocks to guarantee that invariants of
shared data-structures are maintained even as they are being accessed
simultaneously by multiple activities running in the same place.  

An X10 program in which all accesses (both reads and writes) of shared
variables appear in \xcd`atomic` or \xcd`when` blocks is guaranteed to use all
shared variables atomically.  Equivalently, 
if two accesses to some shared variable \xcd`v` could collide at runtime, and
one is in an atomic block, then the other must be in an atomic block as well
to guarantee atomicity of the accesses to \xcd`v`. 
If some accesses to shared variables are not
protected by \xcd`atomic` or \xcd`when`, then race conditions or deadlocks may
occur.  

In particular, atomic sections are atomic with respect to each other. They may
not be atomic with respect to non-atomic code.  

X10 guarantees that atomic sections at the same place are mutually exclusive.
That is, if one activity $A$ at a given place $p$ is executing an atomic
section, then no other activity $B$ at $p$ will also be executing an atomic
section. If such a $B$ attempts to execute an \xcd`atomic` or \xcd`when`
command, it will be blocked until $A$ finishes executing its atomic section.  



%##(AtomicStatement WhenStatement
\begin{bbgrammar}
%(FROM #(prod:AtomicStatement)#)
     AtomicStatement \: \xcd"atomic" Statement & (\ref{prod:AtomicStatement}) \\
%(FROM #(prod:WhenStatement)#)
       WhenStatement \: \xcd"when" \xcd"(" Exp \xcd")" Statement & (\ref{prod:WhenStatement}) \\
\end{bbgrammar}
%##)

For example, consider a class \xcd`Redund[T]`, which encapsulates a list
\xcd`list` and, (redundantly) keeps the size of the list in a second field
\xcd`size`.  Then \xcd`r:Redund[T]` has the invariant 
\xcd`r.list.size() == r.size`, which must be true at any point at which
no method calls on \xcd`r` are active.

If the \xcd`add` method on \xcd`Redund` (which adds an element to the list) 
were defined as: 
%~~gen ^^^ Activities90
% package Activities.Atomic.Redund.One;
% import x10.util.*;
% class Redund[T] {
%   val list = new ArrayList[T]();
%   var size : Int = 0;
%~~vis
\begin{xten}
def add(x:T) { // Incorrect
  this.list.add(x);
  this.size = this.size + 1;
}
\end{xten}
%~~siv
%}
%~~neg
Then two activities simultaneously adding elements to the same \xcd`r` could break the
invariant.  Suppose that \xcd`r` starts out empty.  Let the first activity
perform the \xcd`list.add`, and compute \xcd`this.size+1`, which is 1, but not store it
back into \xcd`this.size` yet.  
(At this point, \xcd`r.list.size()==1` and \xcd`r.size==0`; the invariant
expression is false, but, as the first call to \xcd`r.add()` is active, the
invariant does not need to be true -- it only needs to be true when the
call finishes.)
Now, let the second activity do its call to
\xcd`add` to completion, which finishes with \xcd`r.size==1`.  
(As before, the invariant expression is false, but a call to \xcd`r.add()` is
still active, so the invariant need not be true.)
Finally, let
the first activity finish, which assigns the \xcd`1` computed before back into
\xcd`this.size`.  At the end, there are two elements in \xcd`r.list`, but
\xcd`r.size==1`. Since there are no calls to \xcd`r.add()` active, the
invariant must be true, but it is not.

In this case, the invariant can be maintained by making the increment atomic.
Doing so forbids that sequence of events; the \xcd`atomic` block cannot be
stopped partway.  
%~~gen ^^^ Activities100
% package Activities.Atomic.Redund.Two;
% import x10.util.*;
% class Redund[T] {
%   val list = new ArrayList[T]();
%   var size : Int = 0;
%~~vis
\begin{xten}
def add(x:T) { 
  atomic {
    this.list.add(x);
    this.size = this.size + 1; 
  }
}
\end{xten}
%~~siv
%}
%~~neg

\subsection{Unconditional atomic blocks}
The simplest form of an atomic block is the {\em unconditional
atomic block}: \xcd`atomic S`.
When \xcd`atomic S` is executing at some place \xcd`p`, no other activity at
\xcd`p` may enter an atomic block.  
So, other activities may continue, even at the same place, but code protected
by atomic blocks is not subject to interference from other code in atomic
blocks. 

If execution of the statement may throw an exception, it is
the programmer's responsibility to wrap the atomic block within a
\xcd"try"/\xcd"finally" clause and include undo code in the finally
clause. Thus the \xcd"atomic" statement only guarantees atomicity on
successful execution, not on a faulty execution.



%%NOW-DYN%% For the sake of efficient implementation \XtenCurrVer{} requires
%%NOW-DYN%% that the atomic block be {\em analyzable}, that is, the set of
%%NOW-DYN%% locations that are read and written by the \grammarrule{BlockStatement} are
%%NOW-DYN%% bounded and determined statically.\footnote{A static bound is a constant
%%NOW-DYN%% that depends only on the program text, and is independent 
%%NOW-DYN%% of any runtime parameters. }
%%NOW-DYN%% The exact algorithm to be used by
%%NOW-DYN%% the compiler to perform this analysis will be specified in future
%%NOW-DYN%% versions of the language.
%%NOW-DYN%% \tbd{}

Atomic blocks are closely related to non-blocking synchronization
constructs \cite{herlihy91waitfree}, and can be used to implement 
non-blocking concurrent algorithms.

Code executed inside of \xcd`atomic S` and \xcd`when S` is subject
to certain restrictions. A violation of these restrictions causes an 
\xcd`IllegalOperationException` to be thrown at the point of the violation.

\begin{itemize}
\item \xcd`S` may not spawn another activity.
\item \xcd`S` may not use any blocking statements; \xcd`when`, \xcd`next`,
      \xcd`finish`.  (The use of a nested \xcd`atomic` is permitted.)
\item \xcd`S` may not \xcd`force()` a \xcd`Future`. 
\item \xcd`S` may not use \xcd`at` expressions.
\end{itemize}



\paragraph{Consequences.}
Note an important property of an (unconditional) atomic block:

\begin{eqnarray}
 \mbox{\xcd"atomic \{s1; atomic s2\}"} &=& \mbox{\xcd"atomic \{s1; s2\}"}
\end{eqnarray}

Atomic blocks do not introduce deadlocks.    They may exhibit all the bad
behavior of sequential programs, including throwing exceptions and running
forever, but they are guaranteed not to deadlock.


\subsubsection{Example}

The following class method implements a (generic) compare and swap (CAS) operation:


%~~gen ^^^ Activities110
% package Activities.And.Protein;
% class CASSizer{
%~~vis
\begin{xten}
var target:Object = null;
public atomic def CAS(old1: Object, new1: Object): Boolean {
   if (target.equals(old1)) {
     target = new1;
     return true;
   }
   return false;
}
\end{xten}
%~~siv
%}
%~~neg

\subsection{Conditional atomic blocks}
\index{atomic!conditional}
\index{when}



Conditional atomic blocks allow the activity to wait for some condition to be
satisfied before executing an atomic block. For example, consider a
\xcd`Redund` class holding a list \xcd`r.list` and, redundantly, its length
\xcd`r.size`.  A \xcd`pop` operation will delay until the \xcd`Redund` is
nonempty, and then remove an element and update the length.  
%~~gen ^^^ Activities120
% package Activities.Condato.Example.Not.A.Tree;
% import x10.util.*;
% class Redund[T] {
% val list = new ArrayList[T]();
% var size : Int = 0;
%~~vis
\begin{xten}
def pop():T {
  var ret : T;
  when(size>0) {
    ret = list.removeAt(0);
    size --;
    }
  return ret;
}
\end{xten}
%~~siv
% }
%~~neg


The execution of the test is atomic with the execution of the block.  This is
important; it means that no other activity can sneak in and make the condition
be false before the block is executed.  In this example, two \xcd`pop`s
executing on a list with one element would work properly. Without the
conditional atomic block -- even doing the decrement atomically -- one call to
\xcd`pop` could pass the \xcd`size>0` guard; then the other call could run to
completion (removing the only element of the list); then, when the first call
proceeds, its \xcd`removeAt` will fail.  

Note that \xcd`if` would not work here.  
\xcd`if(size>0) atomic{size--; return list.removeAt(0);}` allows another
activity to act between the test and the atomic block.  
And 
\xcd`atomic{ if(size>0) {size--; ret = list.removeAt(0);}}` 
does not wait for \xcd`size>0` to become true.


Conditional atomic blocks are of the form \xcd`when(b)S`; 
\xcd`b` is called the {\em guard}, and \xcd`S` the {\em body}.

An activity executing such a statement suspends until such time as the  guard
is true in the current state. In that state, the 
body is executed. 
The checking of the guards and the execution of the corresponding
guarded statement is done atomically. 

\Xten{} does not guarantee that a conditional atomic block
will execute if its condition holds only intermittently. For, based on
the vagaries of the scheduler, the precise instant at which a
condition holds may be missed. Therefore the programmer is advised to
ensure that conditions being tested by conditional atomic blocks are
eventually stable, \ie, they will continue to hold until the block
executes (the action in the body of the block may cause the condition
to not hold any more).

%%Fourth, \Xten{} does not guarantees only {\em weak fairness} when executing
%%conditional atomic blocks. Let $c$ be the guard of some conditional
%%atomic block $A$. $A$ is required to make forward progress only if
%%$c$ is {\em eventually stable}. That is, any execution $s_1, s_2,
%%\ldots$ of the program is considered illegal only if there is a $j$
%%such that $c$ holds in all states $s_k$ for $k > j$ and in which $A$
%%does not execute. Specifically, if the system executes in such a way
%%that $c$ holds only intermmitently (that is, for some state in which
%%$c$ holds there is always a later state in which $c$ does not hold),
%%$A$ is not required to be executed (though it may be executed).



The statement \xcd"when (true) S" is
behaviorally identical to \xcd"atomic S": it never suspends.

The body \xcd`S` of \xcd`when(b)S` is subject to the same restrictions that
the body of \xcd`atomic S` is.  The guard is subject to the same restrictions
as well.  Furthermore, guards should not have side effects.


Note that this implies that guarded statements are required to be {\em
flat}, that is, they may not contain conditional atomic blocks. (The
implementation of nested conditional atomic blocks may require
sophisticated operational techniques such as rollbacks.)


\begin{example}
The following class shows how to implement a bounded buffer of size
$1$ in \Xten{} for repeated communication between a sender and a
receiver.  The call \xcd`buf.send(ob)` waits until the buffer has space, and
then puts \xcd`ob` into it.  Dually, \xcd`buf.receive()` waits until the
buffer has something in it, and then returns that thing.


%~~gen ^^^ Activities130
% package Activities;
%~~vis
\begin{xten}
class OneBuffer[T] {
  var datum: T;
  def this(t:T) { this.datum = t; this.filled = true; }
  var filled: Boolean;
  public def send(v: T) {
    when (!filled) {
      this.datum = v;
      this.filled = true;
    }
  }
  public def receive(): T {
    when (filled) {
      v: T = datum;
      filled = false;
      return v;
    }
  }
}
\end{xten}
%~~siv
%
%~~neg
\end{example}

\section{Use of Atomic Blocks}

The semantics of atomicity is chosen as a compromise between programming
simplicity and efficient implementation.  Unlike some possible definitions of
``atomic'', atomic blocks do not provide absolute atomicity.  

Atomic blocks are atomic with respect to {\em each other}.
%~~gen ^^^ Activities4c2r
% package Activities4c2r;
% class Example {
% def example() {
%~~vis
\begin{xten}
var n : Int = 0;
finish {
  async atomic n = n + 1; //(a)
  async atomic n = n + 2; //(b)
}
\end{xten}
%~~siv
%}}
%~~neg
This program has only two possible interleavings: either \xcd`(a)` entirely
precedes \xcd`(b)` or \xcd`(b)` entirely precedes \xcd`(a)`.  Both end up with
\xcd`n==3`. 


However, atomic blocks are not atomic with respect to non-atomic code.  It we
remove the  \xcd`atomic`s on \xcd`(a)`, we get far messier semantics.
%~~gen ^^^ Activities5u4q
% package Activities5u4q;
% class Example {
% def example() {
%~~vis
\begin{xten}
var n : Int = 0;
finish {
  // LEGAL BUT UNWISE 
  async n = n + 1;          //(a)
  async atomic n = n + 2;   //(b)
}
\end{xten}
%~~siv
%}}
%~~neg

If X10 had absolute atomic semantics, this program would be guaranteed to
treat the atomic increment as a single statement.  This would permit three
interleavings: the two possible from the fully atomic program, or a third one
with the events:  \xcd`(a)`'s read of \xcd`0` from \xcd`n`, the entirety of
\xcd`(b)`, and then \xcd`(a)`'s write of \xcd`0+1` back to \xcd`n`.  This
interleaving results in \xcd`n==1`. So, with absolute atomic semantics,
\xcd`n==1` or \xcd`n==3` are the possible results.

However, X10's semantics are weaker than that.  Atomic statements are atomic
with respect to each other --- but there is no guarantee about how they
interact with non-atomic statements at all.  In particular, the following
fourth interleaving is possible: \xcd`(a)`'s read of \xcd`0` from \xcd`n`, 
\xcd`(b)`'s read of \xcd`0` from \xcd`n`, \xcd`(a)`'s write of \xcd`1` to
\xcd`n`, and \xcd`(b)`'s write of \xcd`2` to \xcd`n`.   Thus, \xcd`n==2` is
permissible as a result in X10.

X10's semantics permit more efficient implementation than absolute atomicity.
Absolute atomicity would, in principle, require all activities at place
\xcd`p` to stop whenever one of them enters an atomic section, which would
seriously curtail concurrency.  X10 simply requires that, when one activity is
in an atomic section, that other activities stop {\em when they are trying to
enter an atomic section} --- which is to say, they can continue computing on
their own all they like.  The difference can be substantial.

However, X10's semantics do impose a certain burden on the programmer.  A
sufficient rule of thumb is that, if {\em any} access to a
variable is done in an atomic section, then {\em all} accesses to it must be
in atomic sections.  
	
\chapter{Clocks}\label{XtenClocks}\index{clocks}

Many concurrent algorithms proceed in phases: in phase {$k$}, several
activities work independently, but synchronize together before proceeding on
to phase {$k+1$}. X10 supports this communication structure (and many
variations on it) with a generalization of barriers \bard{cite something}
called {\em clocks}. Clocks are designed so that programs which follow a
simple syntactic discipline will not have either deadlocks or race conditions.


The following minimalist example of clocked code has two worker activities A
and B, and three phases. In the first phase, each worker activity says its
name followed by 1; in the second phase, by a 2, and in the third, by a 3.  
So, if \xcd`say` prints its argument, 
\xcd`A-1 B-1 A-2 B-2 A-3 B-3`
would be a legitimate run of the program, but
\xcd`A-1 A-2 B-1 B-2 A-3 B-3`
(with \xcd`A-2` before \xcd`B-1`) would not.

The program creates a clock \xcd`cl` to manage the phases.  Each worker does
the work of its first phase, and then executes \xcd`next;` to signal that it
is finished with that work. \xcd`next;` is blocking, and causes the worker to
wait until all workers have finished with the phase -- as measured by the
clock \xcd`cl` to which they are both registered.  
Then they do the second phase, and another \xcd`next;` to make sure that
neither proceeds to the third phase until both are ready.  This example uses
\xcd`finish` to wait for both workers to finish.  The parent thread is also
registered on the clock just as the workers are, and executes \xcd`next;next;`
to run through the phases.


%%TODO -- put the 'atomic' back in when that's legal.

%~~gen
%package Clocks.For.Spock;
%class ClockEx {
%  static def say(s:String) = 
% { /*atomic{x10.io.Console.OUT.println(s);}*/ }
%  public static def main(argv:Rail[String]!) {
%~~vis
\begin{xten}
    val cl = Clock.make();
    finish{
      async clocked(cl) {// Activity A
        say("A-1");
        next;
        say("A-2");
        next;
        say("A-3"); 
      }// Activity A

      async clocked(cl) {// Activity B
        say("B-1");
        next;
        say("B-2");
        next;
        say("B-3"); 
      }// Activity B
      next;next;       
    }
    say("All done");

\end{xten}
%~~siv
%  }
% }
%~~neg

This chapter describes the syntax and semantics of clocks and
statements in the language that have parameters of type \xcd"Clock". 

The key invariants associated with clocks are as follows.  At any
stage of the computation, a clock has zero or more {\em registered}
activities. An activity may perform operations only on those clocks it
is registered with (these clocks constitute its {\em clock set}).  An
activity is registered with one or more clocks when it is created.
During its lifetime the only additional clocks it is registered with
are exactly those that it creates. In particular it is not possible
for an activity to register itself with a clock it discovers by
reading a data-structure.

The primary operations that an activity \xcd`a` may perform on a clock \xcd`c`
that it is registered upon are: 
\begin{itemize}
\item It may {\em register} a newly-created activity on \xcd`c`: 
      \xcd`async clocked(c){S}`.
\item It may {\em unregister} itself from \xcd`c`, with \xcd`c.drop()`.  After
      doing so, it can no longer use most primary operations on \xcd`c`.
\item It may {\em resume} the clock, with \xcd`c.resume()`, indicating that it
      has finished with the current phase associated with \xcd`c` and is ready
      to move on to the next one.
\item It may {\em wait} on the clock, with \xcd`c.next()`.  This first does
      \xcd`c.resume()`, and then blocks the current activity until the start
      of the next phase, \viz, until all other activities registered on that
      clock have called \xcd`c.resume()`.
\item It may {\em block} on all the clocks it is registered with
      simultaneously, by the command \xcd`next;`.  This calls \xcd`c.next()`
      on all clocks \xcd`c` that the current activity is registered with.
\item Other miscellaneous operations are available as well; see the
      \xcd`Clock` API.
\end{itemize}

%%CLOCK%% An activity may perform the following operations on a clock \xcd"c".
%%CLOCK%% It may {\em unregister} with \xcd"c" by executing \xcd"c.drop();".
%%CLOCK%% After this, it may perform no further actions on \xcd"c"
%%CLOCK%% for its lifetime. It may {\em check} to see if it is unregistered on a
%%CLOCK%% clock. It may {\em register} a newly forked activity with \xcd"c".
%%CLOCK%% %% It may {\em post} a statement \xcd"S" for completion in the current phase
%%CLOCK%% %% of \xcd"c" by executing the statement \xcd"now(c) S". 
%%CLOCK%% Once registered and "active" (see below), it may also perform the following operations.
%%CLOCK%% It may {\em resume} the clock by executing \xcd"c.resume();". This
%%CLOCK%% indicates to \xcd"c" that it has finished posting all statements it
%%CLOCK%% wishes to perform in the current phase. Finally, it may {\em block}
%%CLOCK%% (by executing \xcd"next;") on all the clocks that it is registered
%%CLOCK%% with. (This operation implicitly \xcd"resume"'s all clocks for the
%%CLOCK%% activity.) It will resume from this statement only when all these
%%CLOCK%% clocks are ready to advance to the next phase.

%%CLOCK%% A clock becomes ready to advance to the next phase when every activity
%%CLOCK%% registered with the clock has executed at least one \xcd"resume"
%%CLOCK%% operation on that clock and all statements posted for completion in
%%CLOCK%% the current phase have been completed.

Though clocks introduce a blocking statement (\xcd"next") an important
property of \Xten{} is that clocks -- when used with the \xcd`next;` {\em
  statement} only, without the \xcd`c.next()` method call -- cannot introduce
deadlocks. That is, the system cannot reach a quiescent state (in which no
activity is progressing) from which it is unable to progress. For, before
blocking each activity resumes all clocks it is registered with. Thus if a
configuration were to be stuck (that is, no activity can progress) all clocks
will have been resumed. But this implies that all activities blocked on
\xcd"next" may continue and the configuration is not stuck. The only other
possibility is that an activity may be stuck on \xcd"finish". But the
interaction rule between \xcd"finish" and clocks
(\Sref{sec:finish:clock-rule}) guarantees that this cannot cause a cycle in
the wait-for graph. A more rigorous proof may be found in \cite{X10-concur05}.

\section{Clock operations}\label{sec:clock}
There are two language constructs for working with clocks. 
\xcd`async clocked(cl) S` starts a new activity registered on one or more
clocks.  \xcd`next;` blocks the current activity until all the activities
sharing clocks with it are ready to proceed. 
Clocks are objects, and have a number of useful methods on them as well.

\subsection{Creating new clocks}\index{clock!creation}\label{sec:clock:create}

Clocks are created using a factory method on \xcd"x10.lang.Clock":


%~~gen
% package Clocks.For.Spocks;
%class Clockuser {
% def example() {
%~~vis
\begin{xten}
val timeSynchronizer: Clock = Clock.make();
\end{xten}
%~~siv
%}}
%~~neg

%%CLOCKVAR%% \eat{All clocked variables are implicitly \xcd`val`. The initializer for a
%%CLOCKVAR%% local variable declaration of type \xcd"Clock" must be a new clock
%%CLOCKVAR%% expression. Thus \Xten{} does not permit aliasing of clocks.
%%CLOCKVAR%% Clocks are created in the place global heap and hence outlive the
%%CLOCKVAR%% lifetime of the creating activity.  Clocks are structs, hence may be freely
%%CLOCKVAR%% copied from place to 
%%CLOCKVAR%% place. (Clock instances typically contain references to mutable state
%%CLOCKVAR%% that maintains the current state of the clock.)
%%CLOCKVAR%% }

The current activity is automatically registered with the newly
created clock.  It may deregister using the \xcd"drop" method on
clocks (see the documentation of \xcd"x10.lang.Clock"). All activities
are automatically deregistered from all clocks they are registered
with on termination (normal or abrupt).

\subsection{Registering new activities on clocks}
\index{clock!clocked statements}\label{sec:clock:register}

The statement 

%~~gen
%package Clocks.For.Jocks;
%class Qlocked{
%static def S():Void{}
%static def flock() { 
% val c1 = Clock.make(), c2 = Clock.make(), c3 = Clock.make();
%~~vis
\begin{xten}
  async clocked (c1, c2, c3) S
\end{xten}
%~~siv
%();
%}}
%~~neg
starts a new activity, initially registered with
clocks \xcd`c1`, \xcd`c2`, and \xcd`c3`, and  running \xcd`S`. The activity running this code must
be registered on those clocks. Furthermore, it cannot be quiescent on any of
them (see \Sref{resume}), because that would introduce a race condition.
Violations of these conditions are punished by the throwing of a
\xcd"ClockUseException"\index{clock!ClockUseException}. 

% An activity may transmit only those clocks that are registered with and
% has not quiesced on (\Sref{resume}). 
% A \xcd"ClockUseException"\index{clock!ClockUseException} is
%thrown if (and when) this condition is violated.

An activity may check that it is registered on a clock \xcd"c" by
%~~exp~~`~~`~~c:Clock ~~
the predicate \xcd`c.registered()`


\begin{note}
\Xten{} does not contain a ``register'' operation that would allow an activity
to discover a clock in a datastructure and register itself on it. Therefore,
while a clock \xcd`c` may be stored in a data structure by one activity
\xcd`a` and read from it by another activity \xcd`b`, \xcd`b` cannot do much
with \xcd`c` unless it is already registered with it.  In particular, it
cannot register itself on \xcd`c`, and, lacking that registration, cannot
register a sub-activity on it with \xcd`clocked(c) async S`.
\end{note}

\oldtodo{Add text on arrays of clocks.}

\subsection{Resuming clocks}\index{clock!resume}\label{resume}\label{sec:clock:resume}
\Xten{} permits {\em split phase} clocks. An activity may wish
to indicate that it has completed whatever work it wishes to perform
in the current phase of a  clock \xcd"c" it is registered with, without
suspending all activity. It may do so  by executing 
%~~exp~~`~~`~~c:Clock ~~
\xcd`c.resume()`.



An activity may invoke \xcd`resume()` only on a clock it is registered with,
and has not yet dropped (\Sref{sec:clock:drop}). A
\xcd"ClockUseException"\index{clock!ClockUseException} is thrown if this
condition is violated. Nothing happens if the activity has already invoked a
\xcd"resume" on this clock in the current phase. Otherwise, \xcd`c.resume()`
indicates that the activity will not transmit \xcd"c" to an 
\xcd"async" (through a \xcd"clocked" clause), 
until it terminates, drops \xcd"c" or executes a \xcd"next".

\bard{The following is under investigation}
\begin{staticrule*}
It is a static error if any activity has a potentially
live execution path from a \xcd"resume" statement on a clock \xcd"c"
to a
%\xcd"now" or
async spawn statement (which registers the new
activity on \xcd"c") unless the path goes through a \xcd"next"
statement. (A \xcd"c.drop()" following a \xcd"c.resume()" is legal,
as is \xcd"c.resume()" following a \xcd"c.resume()".)
\end{staticrule*}

\subsection{Advancing clocks}\index{clock!next}\label{sec:clock:next}
An activity may execute the statement
\begin{xten}
next;
\end{xten}

\noindent 
Execution of this statement blocks until all the clocks that the
activity is registered with (if any) have advanced. (The activity
implicitly issues a \xcd"resume" on all clocks it is registered
with before suspending.)

An \Xten{} computation is said to be {\em quiescent} on a clock
\xcd"c" if each activity registered with \xcd"c" has resumed \xcd"c".
Note that once a computation is quiescent on \xcd"c", it will remain
quiescent on \xcd"c" forever (unless the system takes some action),
since no other activity can become registered with \xcd"c".  That is,
quiescence on a clock is a {\em stable property}.

Once the implementation has detected quiescence on \xcd"c", the system
marks all activities registered with \xcd"c" as being able to progress
on \xcd"c". An activity blocked on \xcd"next" resumes execution once
it is marked for progress by all the clocks it is registered with.

\subsection{Dropping clocks}\index{clock!drop}\label{sec:clock:drop}
%~~exp~~`~~`~~ c:Clock~~
An activity may drop a clock by executing \xcd`c.drop()`.



\noindent{} The activity is no longer considered registered with this
clock.  A \xcd"ClockUseException" is thrown if the activity has
already dropped \xcd"c".


%\subsection{Posting statements on a clock}\index{clock!now}\label{sec:clock:now}
\Xten{} provides syntactic support for a common idiom. Often it may be
necessary for an activity $A$ to require that a certain set of
statements be executed to completion before a clock $c$ can move
forward, without $A$ actually waiting for the completion
of the statement. We introduce the syntax:
\begin{x10}
461 Statement ::= NowStatement
471 StatementNoShortIf ::= 
       NowStatementNoShortIf
479 NowStatement ::= 
       now ( Clock ) Statement
489 NowStatementNoShortIf ::= 
       now ( Clock ) StatementNoShortIf
\end{x10}
\noindent 

A statement {\tt now (c) s} may be considered as shorthand for
\begin{x10}
  async clocked(c) \{ 
     finish async s; 
  \}
\end{x10}

\paragraph{Note.} Because of the static semantics of {\tt finish}
it is not possible to nest {\cf now} statements. Instead if it proves
useful, we may introduce a multi-clocked {\tt now} statement,
which permits the statement to be posted on multiple clocks
simultaneously.
\begin{x10}
479' NowStatement ::= 
       now ( ClockList ) Statement
489' NowStatementNoShortIf ::= 
       now ( ClockList ) StatementNoShortIf  
\end{x10}


\section{Program equivalences}
From the discussion above it should be clear that the following
equivalences hold:

\begin{eqnarray}
 \mbox{\xcd"c.resume(); next;"}       &=& \mbox{\xcd"next;"}\\
 \mbox{\xcd"c.resume(); d.resume();"} &=& \mbox{\xcd"d.resume(); c.resume();"}\\
 \mbox{\xcd"c.resume(); c.resume();"} &=& \mbox{\xcd"c.resume();"}
\end{eqnarray}

Note that \xcd"next; next;" is not the same as \xcd"next;". The
first will wait for clocks to advance twice, and the second
once.  

%\notinfouro{\subsection{Implementation Notes}
Clocks may be implemented efficiently with message passing, e.g.{} by
using short-circuit ideas in \cite{SaraswatPODC88}.  Recall that every
activity is spawned with references to a fixed number of clocks. Each
reference should be thought of as a global pointer to a location in
some place representing the clock. (We shall discuss a further
optimization below.) Each clock keeps two counters: the total number
of outstanding references to the clock, and the number of activities
that are currently suspended on the clock.

When an activity $A$ spawns another activity $B$ that will reference a
clock $c$ referenced by $A$, $A$ {\em splits} the reference by sending
a message to the clock. Whenever an activity drops a reference to a
clock, or suspends on it, it sends a message to the clock. 

The optimization is that the clock can be represented in a distributed
fashion. Each place keeps a local counter for each clock that is
referenced by an activity in that place. The global location for the
clock simply keeps track of the places that have references and that
are quiescent. This can reduce the inter-place message traffic
significantly.
}
%\notinfouro{\section{Clocked types}\index{types!clocked}

We allow types to specify clocks, via a {\cf clocked(c)} modifier,
e.g.{}

\begin{x10}
  clocked(c) int r;
\end{x10}

This declaration asserts that {\cf r} is accessible
(readable/writable) only by those statements that are clocked on {\cf
c}. Thus propagation of the clock provides some control over the
``visibility'' of {\cf r}.

The declaration 

\begin{x10}
  clocked(c) final int l = r;
\end{x10}

\noindent asserts additionally that in each clock instant {\cf l} is final, 
i.e.{} the value of {\cf l} may be reset at the beginning of each phase
of {\tt c} but stays constant during the phase.

This statement terminates when the computation of {\tt r} has
terminated and the assignment has been performed.

\todo{Generalize the syntax so that aggregate variables can be clocked with an aggregate clock of the same shape.}

\subsection{Clocked assignment}\index{assignment!clocked}
We expect that most arrays containing application data will be
declared to be {\cf clocked final}. We support this very powerful type
declaration by providing a new statement:
{\footnotesize
\begin{verbatim}
  next(c) l = r; 
\end{verbatim}}


\noindent 
for a variable $l$ declared to be clocked on $c$. The statement
assigns $r$ to the {\em next} value of $l$. There may be multiple such
assignments before the clock advances. The last such assignment
specifies the value of the variable that will be visible after the
clock has advanced.  If the variable is {\cf clocked final} it is
guaranteed that {\em all} readers of the variable throughout this
phase will see the value $r$.

The expression {\tt r} is implicitly treated as {\tt now(c) r}. That
is, the clock {\tt c} will not advance until the computation of {\tt r} has
terminated.

}
%\notinfouro{\section{Examples}
\todo{Bring in other examples from Concur paper.}
Consider the core of the ASCI Benchmark Sweep3D program for computing
solutions to mass transport problems.

In a nutshell the core computation is a triply nested sequential loop
in which the value of a variable in the current iteration is dependent
on the values of neighboring variables in a past iteration. Such a
problem can be parallelized through pipelining. One visualizes a
diagonal wavefront sweeping through the array. An MPI version of the
program may be described as follows. There is a two dimensional grid
of processors which performs the following computation
repeatedly. Each processor synchronously receives a value from the
processor to its west, then to its north, then computes some function
of these values and computes a new value to be sent to the processor
to its east and then to its south.  Ignoring the behavior of the
boundary processors for the moment such a computation may be described
by the following \Xten{} program:

\begin{x10}
region R = [1..n0,1..m0];
clock[R] W,N;
clock(W) final double [cyclic(R)] A; 
for (int t : 1..TMax) \{
  ateach( i,j:A) 
    clock (W[i-1,j],N[i,j-1],W[i,j],N[i,j]) \{
      double west = now (W[i-1,j]) future\{A[i-1,j]\}; 
      W[i-1,j].continue();           
      double north = now (N[i,j-1]) future\{A[i,j-1]\}; 
      N[i,j-1].continue();
      next(W[i,j]) A[i,j] = compute(west, north);
      next W[i-1,j],N[i,j-1],W[i,j],N[i,j];
  \}
\}
\end{x10}
}

	
\chapter{Local and Distributed Arrays}\label{XtenArrays}\index{arrays}

\Xcd{Array}s provide indexed access to data at a single \Xcd{Place}, {\em via}
\Xcd{Point}s---indices of any dimensionality. \Xcd{DistArray}s is similar, but
spreads the data across multiple \xcd`Place`s, {\em via} \Xcd{Dist}s.  
We refer to arrays either sort as ``general arrays''.  


This chapter provides an overview of the \Xcd{x10.array} classes \Xcd{Array}
and \Xcd{DistArray}, and their supporting classes \Xcd{Point}, \Xcd{Region}
and \Xcd{Dist}.  



\section{Points}\label{point-syntax}\index{point syntax}

General arrays are indexed by \xcd`Point`s--$n$-dimensional tuples of
integers.  The \xcd`rank`
property of a point gives its dimensionality.  Points can be constructed from
integers or \xcd`ValRail`s by
the \xcd`Point.make` factory methods:
%~~gen
% package Arrays.Points.Example1;
% class Example1 {
% def example1() {
%~~vis
\begin{xten}
val origin_1 : Point{self.rank==1} = Point.make(0);
val origin_2 : Point{self.rank==2} = Point.make(0,0);
val origin_5 : Point{self.rank==5} = Point.make([0,0,0,0,0]);
\end{xten}
%~~siv
% } } 
%~~neg

There is an implicit conversion from \xcd`ValRail[Int]` to \xcd`Point`, giving
a convenient syntax for constructing points: 

%~~gen
% package Arrays.Points.Example2;
% class Example{
% def example() {
%~~vis
\begin{xten}
val p : Point = [1,2,3];
val q : Point{rank==5} = [1,2,3,4,5];
\end{xten}
%~~siv
% } } 
%~~neg

The coordinates of a point are available by subscripting; \xcd`p(i)` is the
\xcd"i"th coordinate of the point \xcd"p".

\section{Regions}\label{XtenRegions}\index{region}

A region is a set of points.  {}\Xten{}
provides a built-in class, \xcd"x10.lang.Region", to allow the
creation of new regions and to perform operations on regions. 

Each region \xcd"R" has a constant integer rank, \xcd"R.rank".
%% TODO: Should be uint.

Here are several examples of region declarations:
\begin{xten}
val MAX_HEIGHT=20;
val Null = Region.makeUnit();  // Empty 0-dimensional region          
val N = 10;
val K = 2;
val R1 = 1..100; // 1-dim region with extent 1..100
val R2 = [1..100] as Region(1); // same as R1
val R3 = (0..99) * (-1..MAX_HEIGHT);   
val R4 = [0..99, -1..MAX_HEIGHT] as Region(2); // same as R3  
val R5 = Region.makeUpperTriangular(N);
val R7 = R4 && R5; // intersection of two regions
val R8 = R4 || R5; // union of two regions
\end{xten}

The expression \xcdmath"a$_1$..a$_2$"
is shorthand for the rectangular, rank-1 region
consisting of the points
$\{$\xcdmath"[a$_1$]", \dots, \xcdmath"[a$_2$]"$\}$.
Each subexpression of \xcdmath"a$_i$" must be of type \xcd"Int".
If \xcdmath"a$_1$"
is greater than \xcdmath"a$_2$", the region is empty.

A region may be constructed by converting from a rail of
regions or a rail of points, typically using a rail constructor
(\Sref{RailConstructors})
(e.g., \xcd"R4" above).
The region constructed from a rail of points represents the
region containing just those points.
The region constructed from a rail of regions
represents
the Cartesian product of each of the arguments.
\XtenCurrVer{} does not (yet) support hierarchical regions.

\index{region!upperTriangular}
\index{region!lowerTriangular}\index{region!banded}

Various built-in regions are provided through  factory
methods on \xcd"Region".  For instance:
\begin{itemize}
\item \xcd"Region.makeUpperTriangular(N)" returns a region corresponding
to the non-zero indices in an upper-triangular \xcd"N x N" matrix.
\item \xcd"Region.makeLowerTriangular(N)" returns a region corresponding
to the non-zero indices in a lower-triangular \xcd"N x N" matrix.
\end{itemize}

All the points in a region are ordered canonically by the
lexicographic total order. Thus the points of a region \xcd"R=(1..2)*(1..2)"
are ordered as 
\begin{xten}
(1,1), (1,2), (2,1), (2,2)
\end{xten}
Sequential iteration statements such as \xcd"for" (\Sref{ForAllLoop})
iterate over the points in a region in the canonical order.

A region is said to be {\em rectangular}\index{region!convex} if it is of
the form \xcdmath"(T$_1$ * $\cdots$ * T$_k$)" for some set of regions
\xcdmath"T$_i$". Such a
region satisfies the property that if two points $p_1$ and $p_3$ are
in the region, then so is every point $p_2$ between them (that is, it is {\em convex}). 
(Note that \xcd"||" may produce non-convex regions from convex regions, e.g.,
\xcd"[1,1] || [3,3]" is a non-convex region.  The operation
\xcd`R.boundingBox()` gives the smallest rectangular region containing
\xcd`R`.)  



%%RECT.CLOSURE  For each region \xcd"R", the {\em rectangular closure} of \xcd"R" is the
%%RECT.CLOSURE  smallest rectangular region enclosing \xcd"R".  For each integer \xcd"i"
%%RECT.CLOSURE  less than \xcd"R.rank", the term \xcd"R(i)" represents the enumeration
%%RECT.CLOSURE  in the \xcd"i"th dimension of the rectangular closure of \xcd"R". It may be
%%RECT.CLOSURE  used in a type expression wherever an enumeration may be used.



\subsection{Operations on regions}
Various non side-effecting operators (i.e., pure functions) are
provided on regions. These allow the programmer to express sparse as
well as dense regions.

Let \xcd"R" be a region. A subset of \xcd"R" is also called a
{\em sub-region}.\index{region!sub-region}

Let \xcdmath"R$_1$" and \xcdmath"R$_2$" be two regions whose type
establishes that they are of the same rank. Let 
\xcdmath"S" be a region of unrelated rank.

\xcdmath"R$_1$ && R$_2$" is the intersection of \xcdmath"R$_1$" and
\xcdmath"R$_2$". 

\index{region!intersection}

\xcdmath"R$_1$ || R$_2$" is the union of the \xcdmath"R$_1$" and
\xcdmath"R$_2$".\index{region!union}

\xcdmath"R$_1$ - R$_2$" is the set difference of \xcdmath"R$_1$" and
\xcdmath"R$_2$".\index{region!set difference}

\xcdmath"R$_1$ * S" is the Cartesian product of \xcdmath"R$_1$" and
\xcdmath"S",  formed by pairing each point in \xcdmath"R$_1$" with every the point in \xcdmath"S".
\index{region!product}
Thus, \xcd"([1..2,3..4] as Region 2) * (5..6)"
is the region of rank \Xcd{3} containing the points \Xcd{(x,y,z)}
where \Xcd{x} is \Xcd{1} or \Xcd{2}, 
\Xcd{y} is \Xcd{3} or \Xcd{4}, and
\Xcd{z} is \Xcd{5} or \Xcd{6}. 


For a region \xcdmath"R" and point \xcdmath"p" of the same rank 
\xcdmath"R+p" and \xcdmath"R-p" represent the translation of the region
with \xcdmath"p". That is, point \xcdmath"q" is in 
\xcdmath"R" if and only if point \xcdmath"q+p" is in \xcdmath"R+p". (And similarly
for \xcdmath"R-p".)

%%TODO: Determine how equality is actually implemented. This should not be the definition of ==. 
%%  This could be the definition of .equals(..).

%% Two regions are equal (\xcd"==") if they represent the same set of
%% points.\index{region!==}

For more details on the available methods on \xcdmath"Region", please
consult the API documentation.


\section{Distributions}\label{XtenDistributions}
\index{distribution}

A {\em distribution} is a mapping from a region to a set of places.
{}\Xten{} provides a built-in class, \xcd"x10.lang.Dist", to allow the creation of new distributions and
to perform operations on distributions. This class is \xcd"final" in
{}\XtenCurrVer; future versions of the language may permit
user-definable distributions. 
%DIST_VAR% Since distributions play a dual role
%DIST_VAR% (values as well as types), variables of type \xcd"Dist" must
%DIST_VAR% be initialized and are implicitly \xcd"val".
%DIST_VAR% \bard{Is this true?}

The {\em rank} of a distribution is the rank of the underlying region.

%Recall that each program runs in a fixed number of places, determined
%by runtime parameters. The static constant Place.MAX_PLACES specifies
%the maximum number of places. The collection of places is assumed to
%be totally ordered.


\begin{xten}
R: Region = 1..100;
D: Dist = Dist.makeBlock(R);
D: Dist = Dist.makeCyclic(R);
D: Dist = R -> here;
D: Dist = Dist.random(R);
\end{xten}

Let \xcd"D" be a distribution. \xcd"D.region" denotes the underlying
region. \xcd"D.places" is the set of places constituting the range of
\xcd"D" (viewed as a function). Given a point \xcd"p", the expression
\xcd"D(p)" represents the application of \xcd"D" to \xcd"p", that is,
the place that \xcd"p" is mapped to by \xcd"D". The evaluation of the
expression \xcd"D(p)" throws an \xcd"ArrayIndexOutofBoundsException"
if \xcd"p" does not lie in the underlying region.

When operated on as a distribution, a region \xcd"R" implicitly
behaves as the distribution mapping each item in \xcd"R" to \xcd"here"
(i.e., \xcd"R->here", see below). Conversely, when used in a context
expecting a region, a distribution \xcd"D" should be thought of as
standing for \xcd"D.region".

{}\oldtodo{Allan: We do not specify how the values of an array at a place
are stored, e.g. in row-major or column major order. Need to work this
out.}

\subsection{Operations returning distributions}

Let \xcd"R" be a region, \xcd"Q" a set of places \{\xcd"p1", \dots, \xcd"pk"\}
(enumerated in canonical order), and \xcd"P" a place.  

\paragraph{Unique distribution} \index{distribution!unique}
The distribution \xcd"Dist.makeUnique(Q)" is the unique distribution from the
region \xcd"1..k" to \xcd"Q" mapping each point \xcd"i" to \xcd"pi".

\paragraph{Constant distributions.} \index{distribution!constant}
The distribution \xcd"R->P" maps every point in \xcd"R" to \xcd"P", as does
\xcd`Dist.makeConstant(R,P)`. 

\paragraph{Block distributions.}\index{distribution!block}
The distribution \xcd"Dist.makeBlock(R, Q)" distributes the elements of \xcd"R"
(in order) over the set of places \xcd"Q" in blocks  as
follows. Let $p$ equal \xcd"|R| div N" and $q$ equal \xcd"|R| mod N",
where \xcd"N" is the size of \xcd"Q", and 
\xcd"|R|" is the size of \xcd"R".  The first $q$ places get
successive blocks of size $(p+1)$ and the remaining places get blocks of
size $p$.

The distribution \xcd"Dist.makeBlock(R)" is the same distribution as {\cf
Dist.makeBlock(R, Place.places)}.

\oldtodo{Check into block distributions per dimension.}
\paragraph{Cyclic distributions.} \index{distribution!cyclic}
The distribution \xcd"Dist.makeCyclic(R, Q)" distributes the points in \xcd"R"
cyclically across places in \xcd"Q" in order.

The distribution \xcd"Dist.makeCyclic(R)" is the same distribution as
\xcd"Dist.makeCyclic(R, Place.places)". 

Thus the distribution \xcd"Dist.makeCyclic(Place.MAX_PLACES)" provides a 1--1
mapping from the region \xcd"Place.MAX_PLACES" to the set of all
places and is the same as the distribution \xcd"Dist.makeCyclic(Place.places)".

\paragraph{Block cyclic distributions.}\index{distribution!block cyclic}
The distribution \xcd"Dist.makeBlockCyclic(R, N, Q)" distributes the elements
of \xcd"R" cyclically over the set of places \xcd"Q" in blocks of size
\xcd"N".

\paragraph{Arbitrary distributions.} \index{distribution!arbitrary}
The distribution \xcd"Dist.makeArbitrary(R,Q)" arbitrarily allocates points in {\cf
R} to \xcd"Q". As above, \xcd"Dist.makeArbitrary(R)" is the same distribution as
\xcd"Dist.makeArbitrary(R, Place.places)".

\oldtodo{Determine which other built-in distributions to provide.}

\paragraph{Domain Restriction.} \index{distribution!restriction!region}

If \xcd"D" is a distribution and \xcd"R" is a sub-region of {\cf
D.region}, then \xcd"D | R" represents the restriction of \xcd"D" to
\xcd"R".  The compiler throws an error if it cannot determine that
\xcd"R" is a sub-region of \xcd"D.region".

\paragraph{Range Restriction.}\index{distribution!restriction!range}

If \xcd"D" is a distribution and \xcd"P" a place expression, the term
\xcd"D | P" denotes the sub-distribution of \xcd"D" defined over all the
points in the region of \xcd"D" mapped to \xcd"P".

Note that \xcd"D | here" does not necessarily contain adjacent points
in \xcd"D.region". For instance, if \xcd"D" is a cyclic distribution,
\xcd"D | here" will typically contain points that are \xcd"P" apart,
where \xcd"P" is the number of places. An implementation may find a
way to still represent them in contiguous memory, e.g., using a
complex arithmetic function to map from the region index to an index
into the array.

\subsection{User-defined distributions}\index{distribution!user-defined}

Future versions of \Xten{} may provide user-defined distributions, in
a way that supports static reasoning.

\subsection{Operations on distributions}

A {\em sub-distribution}\index{sub-distribution} of \xcd"D" is
any distribution \xcd"E" defined on some subset of the region of
\xcd"D", which agrees with \xcd"D" on all points in its region.
We also say that \xcd"D" is a {\em super-distribution} of
\xcd"E". A distribution \xcdmath"D$_1$" {\em is larger than}
\xcdmath"D$_2$" if \xcdmath"D$_1$" is a super-distribution of
\xcdmath"D$_2$".

Let \xcdmath"D$_1$" and \xcdmath"D$_2$" be two distributions.  


\paragraph{Intersection of distributions.}\index{distribution!intersection}
\xcdmath"D$_1$ && D$_2$", the intersection of \xcdmath"D$_1$"
and \xcdmath"D$_2$", is the largest common sub-distribution of
\xcdmath"D$_1$" and \xcdmath"D$_2$".

\paragraph{Asymmetric union of distributions.}\index{distribution!union!asymmetric}
\xcdmath"D$_1$.overlay(D$_2$)", the asymmetric union of
\xcdmath"D$_1$" and \xcdmath"D$_2$", is the distribution whose
region is the union of the regions of \xcdmath"D$_1$" and
\xcdmath"D$_2$", and whose value at each point \xcd"p" in its
region is \xcdmath"D$_2$(p)" if \xcdmath"p" lies in
\xcdmath"D$_2$.region" otherwise it is \xcdmath"D$_1$(p)".
(\xcdmath"D$_1$" provides the defaults.)

\paragraph{Disjoint union of distributions.}\index{distribution!union!disjoint}
\xcdmath"D$_1$ || D$_2$", the disjoint union of \xcdmath"D$_1$"
and \xcdmath"D$_2$", is defined only if the regions of
\xcdmath"D$_1$" and \xcdmath"D$_2$" are disjoint. Its value is
\xcdmath"D$_1$.overlay(D$_2$)" (or equivalently
\xcdmath"D$_2$.overlay(D$_1$)".  (It is the least
super-distribution of \xcdmath"D$_1$" and \xcdmath"D$_2$".)

\paragraph{Difference of distributions.}\index{distribution!difference}
\xcdmath"D$_1$ - D$_2$" is the largest sub-distribution of
\xcdmath"D$_1$" whose region is disjoint from that of
\xcdmath"D$_2$".


\subsection{Example}
\begin{xten}
def dotProduct(a: Array[T](D), b: Array[T](D)): Array[Double](D) =
  (new Array[T]([1:D.places],
      (Point) => (new Array[T](D | here,
                    (i): Point) => a(i)*b(i)).sum())).sum();
\end{xten}

This code returns the inner product of two \xcd"T" vectors defined
over the same (otherwise unknown) distribution. The result is the sum
reduction of an array of \xcd"T" with one element at each place in the
range of \xcd"D". The value of this array at each point is the sum
reduction of the array formed by multiplying the corresponding
elements of \xcd"a" and \xcd"b" in the local sub-array at the current
place.




\section{Array initializer}\label{ArrayInitializer}\label{array!creation}

Arrays are instantiated by invoking one of the \xcd"make" factory
methods of the \xcd"Array" class.

An array creation 
must take either an \xcd"Int" as an argument or a \xcd"Dist". In the first
case an array is created over the distribution \xcd"[0:N-1]->here";
in the second over the given distribution. 

An array creation operation may also specify an initializer
function.
The function is applied in parallel
at all points in the domain of the distribution. The array
construction operation terminates locally only when the array has been
fully created and initialized (at all places in the range of the
distribution).

For instance:
\begin{xten}
val data : Array[Int]
    = Array.make[Int](1..1000->here, ((i):Point) => i);
val data2 : Array[Int]
    = Array.make[Int]([1..1000,1..1000]->here, ((i,j):Point) => i*j);
\end{xten}

{}\noindent 
The first declaration stores in \xcd"data" a reference to a mutable
array with \xcd"1000" elements each of which is located in the
same place as the array. Each array component is initialized to \xcd"i".

The second declaration stores in \xcd"data2" a reference to a mutable
2-d array over \xcd"[1..1000, 1..1000]" initialized with \xcd"i*j"
at point \xcd"[i,j]".

Other examples:
\begin{xten}
val D1:Dist(1) = ...; /* An expression that creates a Dist */
val D2:Dist(2) = ...; /* An expression that creates a Dist */

val data : Array[Int]
    = Array.make[Int](1000, ((i):Point) => i*i);

val data2 : Array[Float]
    = Array.make[Float](D1, ((i):Point) => i*i as Float);

val result : Array[Float]
   = Array.make[Float](D2, ((i,j):Point) => i+j as Float);;
\end{xten}

\section{Operations on arrays}
In the following let \xcd"a" be an array with distribution \xcd"D" and
base type \xcd"T". 

\subsection{Element operations}\index{array!access}
The value of \xcd"a" at a point \xcd"p" in its region of definition is
obtained by using the indexing operation \xcd"a(p)". This operation
may be used on the left hand side of an assignment operation to update
the value. The operator assignments \xcd"a(i) op= e" are also available
in \Xten{}.

For array variables, the right-hand-side of an assignment must
have the same distribution \xcd"D" as an array being assigned. This
assignment involves
control communication between the sites hosting \xcd"D". Each
site performs the assignment(s) of array components locally. The
assignment terminates when assignment has terminated at all
sites hosting \xcd"D".

\subsection{Constant promotion}\label{ConstantArray}\index{arrays!constant promotion}

For a distribution \xcd"D" and a val \xcd"v" of
type \xcd"T" the expression \xcd"new Array[T](D, (p: Point) => v)"
denotes the mutable array with
distribution \xcd"D" and base type \xcd"T" initialized with \xcd"v"
at every point.

\subsection{Restriction of an array}\index{array!restriction}

Let \xcd"D1" be a sub-distribution of \xcd"D". Then \xcd"a | D1"
represents the sub-array of \xcd"a" with the distribution \xcd"D1".

Recall that a rich set of operators are available on distributions
(\Sref{XtenDistributions}) to obtain sub-distributions
(e.g. restricting to a sub-region, to a specific place etc).

\subsection{Assembling an array}
Let \xcd"a1,a2" be arrays of the same base type \xcd"T" defined over
distributions \xcd"D1" and \xcd"D2" respectively. Assume that both
arrays are value or reference arrays. 
\paragraph{Assembling arrays over disjoint regions}\index{array!union!disjoint}

If \xcd"D1" and \xcd"D2" are disjoint then the expression \xcd"a1 || a2" denotes the unique array of base type \xcd"T" defined over the
distribution \xcd"D1 || D2" such that its value at point \xcd"p" is
\xcd"a1(p)" if \xcd"p" lies in \xcd"D1" and \xcd"a2(p)"
otherwise. This array is a reference (value) array if \xcd"a1" is.

\paragraph{Overlaying an array on another}\index{array!union!asymmetric}
The expression
\xcd"a1.overlay(a2)" (read: the array \xcd"a1" {\em overlaid with} \xcd"a2")
represents an array whose underlying region is the union of that of
\xcd"a1" and \xcd"a2" and whose distribution maps each point \xcd"p"
in this region to \xcd"D2(p)" if that is defined and to \xcd"D1(p)"
otherwise. The value \xcd"a1.overlay(a2)(p)" is \xcd"a2(p)" if it is defined and \xcd"a1(p)" otherwise.

This array is a reference (value) array if \xcd"a1" is.

The expression \xcd"a1.update(a2)" updates the array \xcd"a1" in place
with the result of \xcd"a1.overlay(a2)".

\oldtodo{Define Flooding of arrays}

\oldtodo{Wrapping an array}

\oldtodo{Extending an array in a given direction.}

\subsection{Global operations }

\paragraph{Pointwise operations}\label{ArrayPointwise}\index{array!pointwise operations}
The unary \xcd"lift" operation applies a function to each element of
an array, returning a new array with the same distribution.
The \xcd"lift" operation is implemented by the following method
in \xcd"Array[T]":
\begin{xten}
def lift[S](f: (T) => S): Array[S](dist);
\end{xten}

The binary \xcd"lift" operation takes a binary function and
another
array over the same distribution and applies the function
pointwise to corresponding elements of the two arrays, returning
a new array with the same distribution.
The \xcd"lift" operation is implemented by the following method
in \xcd"Array[T]":
\begin{xten}
def lift[S,R](f: (T,S) => R, Array[S](dist)): Array[R](dist);
\end{xten}

\paragraph{Reductions}\label{ArrayReductions}\index{array!reductions}

Let \xcd"f" be a function of type \xcd"(T,T)=>T".  Let
\xcd"a" be a value or reference array over base type \xcd"T".
Let \xcd"unit" be a value of type \xcd"T".
Then the
operation \xcd"a.reduce(f, unit)" returns a value of type \xcd"T" obtained
by performing \xcd"f" on all points in \xcd"a" in some order, and in
parallel.  The function \xcd"f" must be associative and
commutative.  The value \xcd"unit" should satisfy
\xcd"f(unit,x)" \xcd"==" \xcd"x" \xcd"==" \xcd"f(x,unit)".

This operation involves communication between the places over which
the array is distributed. The \Xten{} implementation guarantees that
only one value of type \xcd"T" is communicated from a place as part of
this reduction process.

\paragraph{Scans}\label{ArrayScans}\index{array!scans}

Let \xcd"f" be a reduction operator defined on type \xcd"T". Let
\xcd"a" be a value or reference array over base type \xcd"T" and
distribution \xcd"D". Then the operation \xcd"a||f()" returns an array
of base type \xcd"T" and distribution \xcd"D" whose $i$th element
(in canonical order) is obtained by performing the reduction \xcd"f"
on the first $i$ elements of \xcd"a" (in canonical order).

This operation involves communication between the places over which
the array is distributed. The \Xten{} implementation will endeavour to
minimize the communication between places to implement this operation.

Other operations on arrays may be found in \xcd"x10.lang.Array" and
other related classes.
	
\chapter{Annotations}\label{XtenAnnotations}\index{annotations}


\Xten{} provides an 
an annotation system  system for to allow the
compiler to be extended with new static analyses and new
transformations.

Annotations are interface types that decorate the abstract syntax tree
of an \Xten{} program.  The \Xten{} type-checker ensures that an annotation
is a legal interface type.
In \Xten{}, interfaces may declare
both methods and properties.  Therefore, like any interface type, an
annotation may instantiate
one or more of its interface's properties.
%%PLUGINNERY%%  Unlike with Java
%%PLUGINNERY%%  annotations,
%%PLUGINNERY%%  property initializers need not be
%%PLUGINNERY%%  compile-time constants;
%%PLUGINNERY%%  however, a given compiler plugin
%%PLUGINNERY%%  may do additional checks to constrain the allowable
%%PLUGINNERY%%  initializer expressions.
%%PLUGINNERY%%  The \Xten{} type-checker does not check that
%%PLUGINNERY%%  all properties of an annotation are initialized,
%%PLUGINNERY%%  although this could be enforced by
%%PLUGINNERY%%  a compiler plugin.

\section{Annotation syntax}

The annotation syntax consists of an ``\texttt{@}'' followed by an interface type.

%##(Annotations Annotation
\begin{bbgrammar}
%(FROM #(prod:Annotations)#)
         Annotations \: Annotation & (\ref{prod:Annotations}) \\
                    \| Annotations Annotation \\
%(FROM #(prod:Annotation)#)
          Annotation \: \xcd"@" NamedType & (\ref{prod:Annotation}) \\
\end{bbgrammar}
%##)

Annotations can be applied to most syntactic constructs in the language
including class declarations, constructors, methods, field declarations,
local variable declarations and formal parameters, statements,
expressions, and types.
Multiple occurrences of the same annotation (i.e., multiple
annotations with the same interface type) on the same entity are permitted.

%%OBSOLETE%% \begin{grammar}
%%OBSOLETE%% ClassModifier \: Annotation \\
%%OBSOLETE%% InterfaceModifier \: Annotation \\
%%OBSOLETE%% FieldModifier \: Annotation \\
%%OBSOLETE%% MethodModifier \: Annotation \\
%%OBSOLETE%% VariableModifier \: Annotation \\
%%OBSOLETE%% ConstructorModifier \: Annotation \\
%%OBSOLETE%% AbstractMethodModifier \: Annotation \\
%%OBSOLETE%% ConstantModifier \: Annotation \\
%%OBSOLETE%% Type \: AnnotatedType \\
%%OBSOLETE%% AnnotatedType \: Annotation\plus Type \\
%%OBSOLETE%% Statement \: AnnotatedStatement \\
%%OBSOLETE%% AnnotatedStatement \: Annotation\plus Statement \\
%%OBSOLETE%% Expression \: AnnotatedExpression \\
%%OBSOLETE%% AnnotatedExpression \: Annotation\plus Expression \\
%%OBSOLETE%% \end{grammar}

\noindent
Recall that interface types may have dependent parameters.

\noindent
The following examples illustrate the syntax:

\begin{itemize}
\item Declaration annotations:
\begin{xtennoindent}
  // class annotation
  @Value
  class Cons { ... }

  // method annotation
  @PreCondition(0 <= i && i < this.size)
  public def get(i: Int): Object { ... }

  // constructor annotation
  @Where(x != null)
  def this(x: T) { ... }

  // constructor return type annotation
  def this(x: T): C@Initialized { ... }

  // variable annotation
  @Unique x: A;
\end{xtennoindent}
\item Type annotations:
\begin{xtennoindent}
  List@Nonempty

  Int@Range(1,4)

  Array[Array[Double]]@Size(n * n)
\end{xtennoindent}
\item Expression annotations:
\begin{xtennoindent}
  m() : @RemoteCall
\end{xtennoindent}
\item Statement annotations:
\begin{xtennoindent}
  @Atomic { ... }

  @MinIterations(0)
  @MaxIterations(n)
  for (var i: Int = 0; i < n; i++) { ... }

  // An annotated empty statement ;
  @Assert(x < y);
\end{xtennoindent}
\end{itemize}

\section{Annotation declarations}

Annotations are declared as interfaces.  They must be
subtypes of the interface \texttt{x10.lang.annotation.Annotation}.
Annotations on particular static entities must extend the corresponding
\xcd`Annotation` subclasses, as follows: 
\begin{itemize}
\item Expressions---\xcd`ExpressionAnnotation`
\item Statements---\xcd`StatementAnnotation`
\item Classes---\xcd`ClassAnnotation`
\item Fields---\xcd`FieldAnnotation`
\item Methods---\xcd`MethodAnnotation`
\item Imports---\xcd`ImportAnnotation`
\item Packages---\xcd`PackageAnnotation`
\end{itemize}


%%PLUGINNERY%%  \section{Compiler plugins}
%%PLUGINNERY%%  \index{plugins}
%%PLUGINNERY%%  
%%PLUGINNERY%%  After the base \Xten{} semantic checking is completed, 
%%PLUGINNERY%%  compiler plugins are loaded and run.  Plugins may perform
%%PLUGINNERY%%  any number of compiler passes to implement
%%PLUGINNERY%%  additional semantic checking and code transformations, including
%%PLUGINNERY%%  transformations using the abstract syntax of the annotations
%%PLUGINNERY%%  themselves.  Plugins should output valid \Xten{} abstract
%%PLUGINNERY%%  syntax trees.
%%PLUGINNERY%%  
%%PLUGINNERY%%  Plugins are implemented in \Java{} as
%%PLUGINNERY%%  Polyglot~\cite{ncm03} passes applied to the AST
%%PLUGINNERY%%  after normal base \Xten{} type checking.
%%PLUGINNERY%%  Plugins to run are specified on the command-line.  The order of
%%PLUGINNERY%%  execution is determined by the Polyglot pass scheduler.
%%PLUGINNERY%%  \index{Polyglot}
%%PLUGINNERY%%  
%%PLUGINNERY%%  To run compiler plugins, add the command-line option:
%%PLUGINNERY%%  \begin{verbatim}
%%PLUGINNERY%%    -PLUGINS=P1,P2,...,Pn
%%PLUGINNERY%%  \end{verbatim}
%%PLUGINNERY%%  where \texttt{P1}, \texttt{P2}, \dots, \texttt{Pn} are classes that implement the
%%PLUGINNERY%%  \texttt{CompilerPlugin} interface:
%%PLUGINNERY%%  \index{CompilerPlugin}
%%PLUGINNERY%%  
%%PLUGINNERY%%  \begin{xten}
%%PLUGINNERY%%  package polyglot.ext.x10.plugin;
%%PLUGINNERY%%  
%%PLUGINNERY%%  import polyglot.ext.x10.ExtensionInfo;
%%PLUGINNERY%%  import polyglot.frontend.Job;
%%PLUGINNERY%%  import polyglot.frontend.goals.Goal;
%%PLUGINNERY%%  
%%PLUGINNERY%%  public interface CompilerPlugin {
%%PLUGINNERY%%    public Goal
%%PLUGINNERY%%      register(ExtensionInfo extInfo, Job job);
%%PLUGINNERY%%  }
%%PLUGINNERY%%  \end{xten}
%%PLUGINNERY%%  
%%PLUGINNERY%%  \index{Goal}
%%PLUGINNERY%%  The \texttt{Goal} object returned by the \texttt{register} method specifies dependencies on other passes.
%%PLUGINNERY%%  Documentation for Polyglot can be found at:
%%PLUGINNERY%%  \begin{verbatim}
%%PLUGINNERY%%  http://www.cs.cornell.edu/Projects/polyglot
%%PLUGINNERY%%  \end{verbatim}
%%PLUGINNERY%%  Most plugins should implement either \texttt{SimpleOnePassPlugin} or
%%PLUGINNERY%%  \texttt{SimpleVisitorPlugin}.
%%PLUGINNERY%%  
%%PLUGINNERY%%  The compiler loads plugin classes from the x10c classpath.
%%PLUGINNERY%%  
%%PLUGINNERY%%  Plugins are given access to a Polyglot AST and type system.  Annotations are
%%PLUGINNERY%%  represented in the AST as \texttt{Node}s with the following interface:
%%PLUGINNERY%%  \index{Node}
%%PLUGINNERY%%  
%%PLUGINNERY%%  \begin{verbatim}
%%PLUGINNERY%%  package polyglot.ext.x10.ast;
%%PLUGINNERY%%  
%%PLUGINNERY%%  public interface AnnotationNode extends Node {
%%PLUGINNERY%%    X10ClassType annotation();
%%PLUGINNERY%%  }
%%PLUGINNERY%%  \end{verbatim}
%%PLUGINNERY%%  
%%PLUGINNERY%%  Annotations for a \texttt{Node} object \texttt{n} can be accessed through the
%%PLUGINNERY%%  node's extension object as follows:
%%PLUGINNERY%%  \index{AnnotationNode}
%%PLUGINNERY%%  
%%PLUGINNERY%%  \begin{verbatim}
%%PLUGINNERY%%  List<AnnotationNode> annotations =
%%PLUGINNERY%%    ((X10Ext) n.ext()).annotations();
%%PLUGINNERY%%  List<X10ClassType> annotationTypes =
%%PLUGINNERY%%    ((X10Ext) n.ext()).annotationInterfaces();
%%PLUGINNERY%%  \end{verbatim}
%%PLUGINNERY%%  In the type system, \texttt{X10TypeObject} has the following
%%PLUGINNERY%%  method for accessing annotations:
%%PLUGINNERY%%  \begin{verbatim}
%%PLUGINNERY%%  List<X10ClassType> annotations();
%%PLUGINNERY%%  \end{verbatim}
%%PLUGINNERY%%  
%%PLUGINNERY%%  
%%PLUGINNERY%%  %\balance
%%PLUGINNERY%%  
%%PLUGINNERY%%  % \clearpage
%%PLUGINNERY%%  

\section{Linking with native code}\label{extern}\index{extern}
\XtenCurrVer{} supports a simple facility to permit the efficient
intra-thread communication of an array of primitive type to code
written in the language {\tt C}.  The array must be a ``local''
array. The primary intent of this design is to permit the reuse of
native code that efficiently implements some numeric array/matrix
calculation.

Future language releases are expected to support similar bindings to
{\sc Fortran}, and to support parallel native processing of
distributed \Xten{} arrays. 

The interface consists of two parts. First, an array intended to be
communicated to native code must be created as an {\tt unsafe} array:
\begin{x10}
450 ArrayCreationExpression ::= 
      new ArrayBaseType Unsafeopt [ ] 
        ArrayInitializer
451   | new ArrayBaseType Unsafeopt [ Expression ]
452   | new ArrayBaseType Unsafeopt 
          [ Expression ] Expression
453   | new ArrayBaseType Unsafeopt [ Expression ] 
          ( FormalParameter ) MethodBody
454   | new ArrayBaseType value 
           Unsafeopt [ Expression ]
455   | new ArrayBaseType value 
           Unsafeopt [ Expression ] Expression
456   | new ArrayBaseType value 
        Unsafeopt [ Expression ] 
          ( FormalParameter ) MethodBody
530   Unsafeopt ::=
531     | unsafe
\end{x10}
Unsafe arrays can be of any dimension. However, \XtenCurrVer{}
requires that unsafe arrays be of a primitive type, and local (i.e.{}
with an underlying distribution that maps all elements in its region
to {\tt here}).

Unsafe arrays are allocated in a special array of memory that permits
their efficient transmission to natively linked code.
%% Comment about when this memory is freed.

Second, the \Xten{} programmer may specify that certain methods are to
be implemented natively by using the keyword {\tt extern}:
\begin{x10}
446   MethodModifier ::= extern
\end{x10}
Such a method must have the statement ``{\tt ;}'' as its body.
\XtenCurrVer{} requires that the method be {\tt static}; this
restriction is likely to be lifted in the future.  Primitive types in
the method argument are translated to their corresponding JNI type
(e.g.{} {\tt float} is translated to {\tt jfloat}, {\tt double} to
{\tt jdouble} etc).  The only non-primitive type permitted in an {\tt
extern} method is an (unsafe) array. This is passed at type {\tt
jlong} as an eight byte address into the unsafe region which contains
the data for the array. ({\tt jlong} is not the same as {\tt long} on
32-bit machines.)


Since only the starting address of an array is passed, if the array is
multidimensional, the user must explicitly communicate (or have a
guarantee of) the rank of the passed array, and must either typecast
or explicitly code the address calculation.  Note that all \Xten{}
arrays are created in row-major order, and so any native routine must
also access them in the same order.

For each class {\tt C} that contains an {\tt extern} method, the
\Xten{} compiler generates a text file {\tt C\_x10stub.c}.  This file
contains generated {\tt C} stub functions which are called from the
{\tt extern} routines.  The name of the stub function is derived from
the name of the {\tt extern} method. If the method is {\tt
C.process()}, the stub function will be {\tt
Java\_C\_C\_process()}. The name is suffixed with the signature of the
method if the method is overloaded.

The programmer must write {\tt C} code to implement the native method,
using the methods in the {\tt C} stub file to call the actual native
method.  The programmer must compile these files and link them into a
dynamically linked library (DLL).  Note that the {\tt jni.h} header file
must be in the include path.  The programmer must ensure this library
is loaded by the program before the method is called e.g.{} add a {\tt
System.loadlibrary} call (in a static initializer of the
\Xten{} class).

\paragraph{Example.}
The following class illustrates the use of {\tt unsafe} and native
linking. 
\begin{x10}
public class IntArrayExternUnsafe \{
  public static extern 
      void process(int [.] yy, int size);
  static {System.loadLibrary("IntArrayExternUnsafe");}
  public static void main(String args[]) \{
     boolean b= (new IntArrayExternUnsafe()).run();
     System.out.println("++++++ Test "
                         +(b?"succeeded.":"failed."));
     System.exit(b?0:1);
  \}
  public boolean run()\{
    int high = 10;
    boolean verified=false;
    distribution d= (0:high) -> here;
    int [.] y = new int unsafe[d]; 
    for( int j=0;j < 10;++j)
        y[j] = j;
    process(y,high);
    for(int j=0;j < 10;++j)\{
      int expected = j+100;
      if(y[j] != expected)\{
        System.out.println("y["+j+"]="
                           +y[j]+" != "+expected);
        return false;
       \}
    \}
    return true;
  \}
\}
\end{x10}

The programmer may then write the {\tt C} code thus:
\begin{x10}
void IntArrayExternUnsafe\_process(jlong yy, 
                                signed int size)\{
  int i;
  int* array = (int *)(long)yy;
  for(i = 0;i < size;++i)\{
    array[i] += 100;
  \}
\}
/* automatically generated in \_x10stub.c*/
void 
 Java\_IntArrayExternUnsafe\_IntArrayExternUnsafe\_process
 (JNIEnv *env,  jobject obj,jlong yy,jint size)\{
   IntArrayExternUnsafe\_process(yy,size);
\}
\end{x10}

This code may be linked with the stub file (or textually placed in
it). The programmer must then compile and link the {\tt C} code and
ensure that the DLL is on the appropriate classpath. 


\chapter{Lost Bits}

\bard{This chapter should not exist.  The material in it was misplaced.  It 
is stored here temporarily
so it doesn't get lost}

\section{Visibility of Local Variables and Formals}

In general, variables (\ie, local variables, parameters,
properties, fields) are visible at
a point in the c
if they are defined before \xcd"T" in the program. This rule applies to
types in property lists as well as parameter lists (for methods and
constructors).
A formal parameter is visible in the types of all other formal
parameters of the same method, constructor, or type definition,
as well as in the method or constructor body itself.
Properties are accessible via their containing object--\xcd"this"
within the body of their class declaration.  The special
variable \xcd"this" is in scope at each property
declaration, constructor signatures and bodies, instance method signatures
and bodies,
and instance field signatures and initializers, but not in scope
at \xcd"static" method or field declarations or \xcd"static"
initializers.  


%\chapter{Grammar}

The following chapter contains the grammar for the \Xten{} language.

\todo{Paste in the grammar}

\renewcommand{\bibname}{References}
\bibliographystyle{plain}
\bibliography{master}

%%\extrapart{Bibliography and references}

% My reference for proper reference format is:
%    Mary-Claire van Leunen.
%    {\em A Handbook for Scholars.}
%    Knopf, 1978.
% I think the references list would look better in ``open'' format,
% i.e. with the three blocks for each entry appearing on separate
% lines.  I used the compressed format for SIGPLAN in the interest of
% space.  In open format, when a block runs over one line,
% continuation lines should be indented; this could probably be done
% using some flavor of latex list environment.  Maybe the right thing
% to do in the long run would be to convert to Bibtex, which probably
% does the right thing, since it was implemented by one of van
% Leunen's colleagues at DEC SRC.
%  -- Jonathan

% This is just a personal remark on your question on the RRRS:
% The language CUCH (Curry-Church) was implemented by 1964 and 
% is a practical version of the lambda-calculus (call-by-name).
% One reference you may find in Formal Language Description Languages
% for Computer Programming T.~B.~Steele, 1965 (or so).
%  -- Matthias Felleisen


\begin{thebibliography}{99}

\bibitem{SICP}
Harold Abelson and Gerald Jay Sussman with Julie Sussman.
{\em Structure and Interpretation of Computer Programs.}
MIT Press, Cambridge, 1985.

\bibitem{readfloat}
William Clinger.
How to read floating point numbers accurately.
In {\em Proceedings of the 1990 ACM SIGPLAN Conference on Programming
  Language Design and Implementation}.  Forthcoming.

University of Oregon Technical Report CIS-TR-90-01.

\bibitem{RRRS}
William Clinger, editor.
The revised revised report on Scheme, or an uncommon Lisp.
MIT Artificial Intelligence Memo 848, August 1985.
Also published as Computer Science Department Technical Report 174,
  Indiana University, June 1985.

\bibitem{R4RS}
William Clinger and Jonathan Rees, editors.
The revised$^4$ report on the algorithmic language Scheme.
University of Oregon Technical Report CIS-TR-90-02.

\bibitem{Scheme311}
Carol Fessenden, William Clinger, Daniel P.~Friedman, and Christopher Haynes.
Scheme 311 version 4 reference manual.
Indiana University Computer Science Technical Report 137, February 1983.
Superceded by~\cite{Scheme84}.

\bibitem{Scheme84}
D.~Friedman, C.~Haynes, E.~Kohlbecker, and M.~Wand.
Scheme 84 interim reference manual.
Indiana University Computer Science Technical Report 153, January 1985.

\bibitem{CFractions}
G.~H.~Hardy and E.~M.~Wright.
{\em An Introduction to the Theory of Numbers.} 5th ed.
Oxford University Press, New York NY, 1979.

\bibitem{Haskell}
Paul Hudak and Philip Wadler, editors.
Report on the Functional Programming Language Haskell.
Yale University Research Report YALEU/DCS/RR-666, December 1988.

\bibitem{IEEE}
{\em IEEE Standard 754-1985.  IEEE Standard for Binary Floating-Point
Arithmetic.}  IEEE, New York, 1985.

\bibitem{Knuth}
Donald E. Knuth.
The Art of Computer Programming, volume 2: Seminumerical Algorithms.
Addison-Wesley, Reading MA, 1969.

\bibitem{Landin65}
Peter Landin.
A correspondence between Algol 60 and Church's lambda notation: Part I.
{\em Communications of the ACM} 8(2):89--101, February 1965.

\bibitem{Matula68}
David W. Matula.
In-and-Out Conversions.
{\em Communications of the ACM} 11(1):47--50, January 1968.

\bibitem{Matula70}
David W. Matula.
A Formalization of Floating-Point Numeric Base Conversion.
{\em IEEE Transactions on Computers} C-19, 8:681-692, August 1970.

\bibitem{MITScheme}
MIT Department of Electrical Engineering and Computer Science.
Scheme manual, seventh edition.
September 1984.

\bibitem{Penfield81}
Paul Penfield, Jr.
Principal values and branch cuts in complex APL.
In {\em APL '81 Conference Proceedings,} pages 248--256.
ACM SIGAPL, San Francisco, September 1981.
Proceedings published as {\em APL Quote Quad} 12(1), ACM, September 1981.

\bibitem{Pitman83}
Kent M.~Pitman.
The revised MacLisp manual (Saturday evening edition).
MIT Laboratory for Computer Science Technical Report 295, May 1983.

\bibitem{Rees82}
Jonathan A.~Rees and Norman I.~Adams IV.
T: A dialect of Lisp or, lambda: The ultimate software tool.
In {\em Conference Record of the 1982 ACM Symposium on Lisp and
  Functional Programming}, pages 114--122.

\bibitem{R3RS}
Jonathan Rees and William Clinger, editors.
The revised$^3$ report on the algorithmic language Scheme.
In {\em ACM SIGPLAN Notices} 21(12), ACM, December 1986.

\bibitem{Reynolds72}
John Reynolds.
Definitional interpreters for higher order programming languages.
In {\em ACM Conference Proceedings}, pages 717--740.
ACM, \todo{month?}~1972.

\bibitem{Rabbit}
Guy Lewis Steele Jr.
Rabbit: a compiler for Scheme.
MIT Artificial Intelligence Laboratory Technical Report 474, May 1978.

\bibitem{CLtL}
Guy Lewis Steele Jr.
{\em Common Lisp: The Language.}
Digital Press, Burlington MA, 1984.

\bibitem{CLtL2}
Guy Lewis Steele Jr.
{\em Common Lisp: The Language.} 2d ed.
Digital Press, Bedford MA, 1990.

\bibitem{Scheme78}
Guy Lewis Steele Jr.~and Gerald Jay Sussman.
The revised report on Scheme, a dialect of Lisp.
MIT Artificial Intelligence Memo 452, January 1978.

\bibitem{Heuristic}
Guy Lewis Steele Jr.~and Jon L White.
How to Print Floating-Point Numbers Accurately.
In {\em Proceedings of the 1990 ACM SIGPLAN Conference on Programming
  Language Design and Implementation}.  Forthcoming.

\bibitem{Stoy77}
Joseph E.~Stoy.
{\em Denotational Semantics: The Scott-Strachey Approach to
  Programming Language Theory.}
MIT Press, Cambridge, 1977.

\bibitem{Scheme75}
Gerald Jay Sussman and Guy Lewis Steele Jr.
Scheme: an interpreter for extended lambda calculus.
MIT Artificial Intelligence Memo 349, December 1975.

\bibitem{Vuillemin}
Jean Vuillemin.
Exact real computer arithmetic with continued fractions.
In {\em Proceedings of the 1988 ACM Conference on Lisp and
  Functional Programming}, pages 14--27.

\end{thebibliography}
	

% Adjustment to avoid having the last index entry on a page by itself.
%\addtolength{\baselineskip}{-0.1pt}

\clearpage
\documentclass[orivec,twoside,twocolumn]{algol60}
\usepackage{changebar}
%\documentclass[twoside]{algol60}
\def\Hat{{\tt \char`\^}}
\def\ccfont{\sf}
\usepackage{url}
\usepackage{times}
\pagestyle{headings}
\showboxdepth=0
\makeindex
% Macros for R^nRS.

\makeatletter

\newcommand{\topnewpage}{\@topnewpage}
\newcommand{\authorsc}[1]{{\scriptsize\scshape #1}}

% Chapters, sections, etc.

\newcommand{\extrapart}[1]{
 % \chapter{#1}
  \chapter*{#1}
  \markboth{#1}{#1}
  \vskip 1ex
  \addcontentsline{toc}{chapter}{#1}}

\newcommand{\clearchapterstar}[1]{
  \clearpage
  \topnewpage[
    \centerline{\large\bf\uppercase{#1}}
    \bigskip]}

\newcommand{\clearextrapart}[1]{
  \clearchapterstar{#1}
  \markboth{#1}{#1}
  \addcontentsline{toc}{chapter}{#1}}

\newcommand{\vest}{}
\newcommand{\dotsfoo}{$\ldots\,$}

\newcommand{\sharpfoo}[1]{{\tt\##1}}
\newcommand{\schfalse}{\sharpfoo{f}}
\newcommand{\schtrue}{\sharpfoo{t}}

\newcommand{\singlequote}{{\tt'}}  %\char19
\newcommand{\doublequote}{{\tt"}}
\newcommand{\backquote}{{\tt\char18}}
\newcommand{\backwhack}{{\tt\char`\\}}
\newcommand{\atsign}{{\tt\char`\@}}
\newcommand{\sharpsign}{{\tt\#}}
\newcommand{\verticalbar}{{\tt|}}

\newcommand{\coerce}{\discretionary{->}{}{->}}

% Knuth's \in sucks big boulders
\def\elem{\hbox{\raise.13ex\hbox{$\scriptstyle\in$}}}

\newcommand{\meta}[1]{{\noindent\hbox{\rm$\langle$#1$\rangle$}}}
\let\hyper=\meta
\newcommand{\hyperi}[1]{\hyper{#1$_1$}}
\newcommand{\hyperii}[1]{\hyper{#1$_2$}}
\newcommand{\hyperj}[1]{\hyper{#1$_i$}}
\newcommand{\hypern}[1]{\hyper{#1$_n$}}
\newcommand{\var}[1]{\noindent\hbox{\it{}#1\/}}  % Careful, is \/ always the right thing?
\newcommand{\vari}[1]{\var{#1$_1$}}
\newcommand{\varii}[1]{\var{#1$_2$}}
\newcommand{\variii}[1]{\var{#1$_3$}}
\newcommand{\variv}[1]{\var{#1$_4$}}
\newcommand{\varj}[1]{\var{#1$_j$}}
\newcommand{\varn}[1]{\var{#1$_n$}}

\newcommand{\vr}[1]{{\noindent\hbox{$#1$\/}}}  % Careful, is \/ always the right thing?
\newcommand{\vri}[1]{\vr{#1_1}}
\newcommand{\vrii}[1]{\vr{#1_2}}
\newcommand{\vriii}[1]{\vr{#1_3}}
\newcommand{\vriv}[1]{\vr{#1_4}}
\newcommand{\vrv}[1]{\vr{#1_5}}
\newcommand{\vrj}[1]{\vr{#1_j}}
\newcommand{\vrn}[1]{\vr{#1_n}}


\newcommand{\defining}[1]{\mainindex{#1}{\em #1}}
\newcommand{\ide}[1]{{\schindex{#1}\frenchspacing\tt{#1}}}

\newcommand{\lambdaexp}{{\cf lambda} expression}
\newcommand{\Lambdaexp}{{\cf Lambda} expression}
\newcommand{\callcc}{{\tt call-with-current-continuation}}

% \reallyindex{SORTKEY}{HEADCS}{TYPE}
% writes (index-entry "SORTKEY" "HEADCS" TYPE PAGENUMBER)
% which becomes  \item \HEADCS{SORTKEY} mainpagenumber ; auxpagenumber ...

\global\def\reallyindex#1#2#3{%
\write\@indexfile{"#1" "#2" #3 \thepage}}

\newcommand{\mainschindex}[1]{\label{#1}\reallyindex{#1}{tt}{main}}
\newcommand{\mainindex}[1]{\reallyindex{#1@{\rm #1}{main}}}
\newcommand{\schindex}[1]{\reallyindex{#1}{tt}{aux}}
\newcommand{\sharpindex}[1]{\reallyindex{#1}{sharpfoo}{aux}}
%vj%\renewcommand{\index}[1]{\reallyindex{#1}{rm}{aux}}

\newcommand{\domain}[1]{#1}
\newcommand{\nodomain}[1]{}
%\newcommand{\todo}[1]{{\rm$[\![$!!~#1$]\!]$}}
\newcommand{\todo}[1]{}

% \frobq will make quote and backquote look nicer.
\def\frobqcats{%\catcode`\'=13 %\catcode`\{=13{}\catcode`\}=13{}
\catcode`\`=13{}}
{\frobqcats
\gdef\frobqdefs{%\def'{\singlequote}
\def`{\backquote}}}%\def\{{\char`\{}\def\}{\char`\}}
\def\frobq{\frobqcats\frobqdefs}

% \cf = code font
% Unfortunately, \cf \cf won't work at all, so don't even attempt to
% next constructions which use them...
\newcommand{\cf}{\frenchspacing\tt}

% Same as \obeycr, but doesn't do a \@gobblecr.
{\catcode`\^^M=13 \gdef\myobeycr{\catcode`\^^M=13 \def^^M{\\}}%
\gdef\restorecr{\catcode`\^^M=5 }}

{\catcode`\^^I=13 \gdef\obeytabs{\catcode`\^^I=13 \def^^I{\hbox{\hskip 4em}}}}

{\obeyspaces\gdef {\hbox{\hskip0.5em}}}

\gdef\gobblecr{\@gobblecr}

\def\setupcode{\@makeother\^}

% Scheme example environment
% At 11 points, one column, these are about 56 characters wide.
% That's 32 characters to the left of the => and about 20 to the right.

\newenvironment{x10noindent}{
  % Commands for scheme examples
  \newcommand{\ev}{\>\>\evalsto}
  \newcommand{\lev}{\\\>\evalsto}
  \newcommand{\unspecified}{{\em{}unspecified}}
  \newcommand{\scherror}{{\em{}error}}
  \setupcode
  \small \cf \obeytabs \obeyspaces \myobeycr
  \begin{tabbing}%
\qquad\=\hspace*{5em}\=\hspace*{9em}\=\kill%   was 16em
\gobblecr}{\unskip\end{tabbing}}

%\newenvironment{scheme}{\begin{schemenoindent}\+\kill}{\end{schemenoindent}}
\newenvironment{x10}{
  % Commands for scheme examples
  \newcommand{\ev}{\>\>\evalsto}
  \newcommand{\lev}{\\\>\evalsto}
  \renewcommand{\em}{\rmfamily\itshape}
  \newcommand{\unspecified}{{\em{}unspecified}}
  \newcommand{\scherror}{{\em{}error}}
  \setupcode
  \small \cf \obeyspaces \myobeycr
  \footnotesize
  \begin{tabbing}%
\qquad\=\hspace*{5em}\=\hspace*{9em}\=\+\kill%   was 16em
\gobblecr}{\unskip\end{tabbing}\normalsize}

\newcommand{\evalsto}{$\Longrightarrow$}

% Rationale

\newenvironment{rationale}{%
\bgroup\small\noindent{\em Rationale:}\space}{%
\egroup}

% Notes

\newenvironment{note}{%
\bgroup\small\noindent{\em Note:}\space}{%
\egroup}

% Manual entries

\newenvironment{entry}[1]{
  \vspace{3.1ex plus .5ex minus .3ex}\noindent#1%
\unpenalty\nopagebreak}{\vspace{0ex plus 1ex minus 1ex}}

\newcommand{\exprtype}{syntax}

% Primitive prototype
\newcommand{\pproto}[2]{\unskip%
\hbox{\cf\spaceskip=0.5em#1}\hfill\penalty 0%
\hbox{ }\nobreak\hfill\hbox{\rm #2}\break}

% Parenthesized prototype
\newcommand{\proto}[3]{\pproto{(\mainschindex{#1}\hbox{#1}{\it#2\/})}{#3}}

% Variable prototype
\newcommand{\vproto}[2]{\mainschindex{#1}\pproto{#1}{#2}}

% Extending an existing definition (\proto without the index entry)
\newcommand{\rproto}[3]{\pproto{(\hbox{#1}{\it#2\/})}{#3}}

% Grammar environment

\newenvironment{grammar}{
  \def\:{\goesto{}}
  \def\|{$\vert$}
  \cf \myobeycr
  \begin{tabbing}
    %\qquad\quad \= 
    \qquad \= $\vert$ \= \kill
  }{\unskip\end{tabbing}}

%\newcommand{\unsection}{\unskip}
\newcommand{\unsection}{{\vskip -2ex}}

% Commands for grammars
\newcommand{\arbno}[1]{#1\hbox{\rm*}}  
\newcommand{\atleastone}[1]{#1\hbox{$^+$}}

\newcommand{\goesto}{$\longrightarrow$}

% mark modifications (for the grammar) From Igor Pechtchanski/Watson/IBM@IBMUS
\newlength{\modwidth}\setlength{\modwidth}{0.005in}
\newlength{\modskip}\setlength{\modskip}{.4em}
\newlength{\@modheight}
\newlength{\@modpos}
\providecommand{\markmod}[1]{%
  \setlength{\@modheight}{#1}%
  \addtolength{\@modheight}{-0.06in}%
  \setlength{\@modpos}{\linewidth}%
  \addtolength{\@modpos}{0.285in}%         Magic
  \addtolength{\@modpos}{\modwidth}%
  \addtolength{\@modpos}{\modskip}%
  \marginpar{\vspace{-\@modheight}%
             \hspace{-\@modpos}%
             \rule{\modwidth}{#1}}%
}

% The index

\def\theindex{%\@restonecoltrue\if@twocolumn\@restonecolfalse\fi
%\columnseprule \z@
%!! \columnsep 35pt
\clearpage
\@topnewpage[
    \centerline{\large\bf\uppercase{Alphabetic index of definitions of concepts,}}
    \centerline{\large\bf\uppercase{keywords, and procedures}}
    \vskip 1ex \bigskip]
    \markboth{Index}{Index}
    \addcontentsline{toc}{chapter}{Alphabetic index of 
 definitions of concepts,\hfil\penalty0 \hbox{\hspace*{2em} keywords, and procedures}}
    \bgroup %\small
    \parindent\z@
    \parskip\z@ plus .1pt\relax\let\item\@idxitem}

\def\@idxitem{\par\hangindent 40pt}

\def\subitem{\par\hangindent 40pt \hspace*{20pt}}

\def\subsubitem{\par\hangindent 40pt \hspace*{30pt}}

\def\endtheindex{%\if@restonecol\onecolumn\else\clearpage\fi
\egroup}

\def\indexspace{\par \vskip 10pt plus 5pt minus 3pt\relax}

\makeatother
\newcommand{\Xten}{{\sf X10}}
\newcommand{\XtenCurrVer}{{\sf X10 v1.1}}
\newcommand{\java}{{\sf Java}}
\newcommand{\Java}{{\sf Java}}
\newcommand{\notfouro}[1]{}
\newcommand{\notinfouro}[1]{}
\newcommand{\futureext}[1]{{\em \paragraph{Future Extensions.}#1}}
\newcommand{\tbd}{} % marker for things to be done later.
\newcommand{\limitation}[1]{{\em Limitation: #1}} % marker for things to be done later.


\def\headertitle{The \XtenCurrVer{} Report }
\def\integerversion{1.0}

% Sizes and dimensions

\topmargin -.375in       %    Nominal distance from top of page to top of
                         %    box containing running head.
\headsep 15pt            %    Space between running head and text.

\textheight 663pt        % Height of text (including footnotes and figures, 
                         % excluding running head and foot).

\textwidth 523pt         % Width of text line.
\columnsep 15pt          % Space between columns 
\columnseprule 0pt       % Width of rule between columns.

\parskip 5pt plus 2pt minus 2pt % Extra vertical space between paragraphs.
\parindent 0pt                  % Width of paragraph indentation.
\topsep 0pt plus 2pt            % Extra vertical space, in addition to 
                                % \parskip, added above and below list and
                                % paragraphing environments.

\oddsidemargin  -.5in    % Left margin on odd-numbered pages.
\evensidemargin -.5in    % Left margin on even-numbered pages.

%% End of sizes and dimensions
\makeatletter
\newsavebox{\eStop}
\savebox{\eStop}{\raisebox{0.6ex}{\framebox[0.5em]\relax}}

\def\newtenv#1{\@ifnextchar[{\@otxm{#1}}{\@ntxm{#1}}}

\def\@ntxm#1#2{\@ifnextchar[{\@xntxm{#1}{#2}}{\@yntxm{#1}{#2}}}

\def\@xntxm#1#2[#3]{\expandafter\@ifdefinable\csname #1\endcsname
{\@definecounter{#1}\@addtoreset{#1}{#3}%
\expandafter\xdef\csname the#1\endcsname{\expandafter\noexpand
  \csname the#3\endcsname \@thmcountersep \@thmcounter{#1}}%
\global\@namedef{#1}{\@txm{#1}{#2}}\global\@namedef{end#1}{\@endtenv}}}

\def\@yntxm#1#2{\expandafter\@ifdefinable\csname #1\endcsname
{\@definecounter{#1}%
\expandafter\xdef\csname the#1\endcsname{\@thmcounter{#1}}%
\global\@namedef{#1}{\@txm{#1}{#2}}\global\@namedef{end#1}{\@endtenv}}}

\def\@otxm#1[#2]#3{\expandafter\@ifdefinable\csname #1\endcsname
  {\global\@namedef{the#1}{\@nameuse{the#2}}%
\global\@namedef{#1}{\@txm{#2}{#3}}%
\global\@namedef{end#1}{\@endtenv}}}

\def\@txm#1#2{\refstepcounter
    {#1}\@ifnextchar[{\@ytxm{#1}{#2}}{\@xtxm{#1}{#2}}}

\def\@xtxm#1#2{\@begintenv{#2}{\csname the#1\endcsname}\ignorespaces}
\def\@ytxm#1#2[#3]{\@opargbegintenv{#2}{\csname
       the#1\endcsname}{#3}\ignorespaces}

%DEFAULT VALUES
\def\@begintenv#1#2{\trivlist \item[\hskip \labelsep{\bf #1\ #2}]}
\def\@opargbegintenv#1#2#3{\trivlist
      \item[\hskip \labelsep{\bf #1\ #2\ (#3)}]}
\def\@endtenv{\hfill\usebox{\eStop}\endtrivlist}
\makeatother

\newtenv{example}{Example}[section]

\begin{document}

\parindent 0pt %!! 15pt                    % Width of paragraph indentation.

%\hfil {\bf 7 Feb 2005}
%\hfil \today{}

% First page

\thispagestyle{empty}

% \todo{"another" report?}

\title{Report on the Experimental Language \Xten \\
\large Version \integerversion}
\author{Please send comments to \\
Vijay Saraswat at \texttt{vsaraswa@us.ibm.com}}
\date\today
\maketitle

\if 0
\topnewpage[{
\begin{center}   
{\huge\bf Report on the Experimental Language \Xten{}}
\vskip 1ex
$$
\begin{tabular}{l@{\extracolsep{.5in}}lll}
\multicolumn{4}{c}{\sc Version \integerversion}\\
\multicolumn{4}{c}{\sc Please send comments to 
Vijay Saraswat at 
{\tt vsaraswa@us.ibm.com}}\\
%\multicolumn{4}{c}{({\sc IBM Confidential})}

%\ldots
\end{tabular}
$$
\vskip 2ex
% {\it Dedicated to the Memory of APL} % vj
{\bf \today}
\vskip 2.6ex
\end{center}


}]
\fi

\newcommand\authorsc[1]{#1}
%\newcommand\authorsc[1]{\textsc{#1}}


\chapter*{Summary}
This draft report provides an initial description of the programming
language \Xten. \Xten{} is a single-inheritance class-based object-oriented
(OO) programming language designed for high-performance, high-productivity
computing on high-end computers supporting $\approx 10^5$ hardware threads
and $\approx 10^{15}$ operations per second. 

{}\Xten{} is based on state-of-the-art object-oriented programming
languages and deviates from them only as necessary to support its
design goals. The language is intended to have a simple and clear
semantics and be readily accessible to mainstream OO programmers. It
is intended to support a wide variety of concurrent programming
idioms.
%, incuding data parallelism, task parallelism, pipelining.
%producer/consumer and divide and conquer.

%We expect to revise this document in the light of experience gained in implementing
%and using this language.

The \Xten{} design team consists of
\authorsc{David Bacon}, 
\authorsc{Raj Barik}, 
\authorsc{Ganesh Bikshandi}, 
\authorsc{Bob Blainey}, 
\authorsc{Philippe Charles}, 
\authorsc{Perry Cheng}, 
\authorsc{Christopher Donawa}, 
\authorsc{Julian Dolby}, 
\authorsc{Kemal Ebcio\u{g}lu},
\authorsc{Robert Fuhrer},
\authorsc{Patrick Gallop}, 
\authorsc{Christian Grothoff}, 
\authorsc{Allan Kielstra}, 
\authorsc{Sreedhar Kodali}, 
\authorsc{Sriram Krishnamoorthy}, 
\authorsc{Nathaniel Nystrom}, 
\authorsc{Igor Peshansky}, 
\authorsc{Vijay Saraswat} (contact author), 
\authorsc{Vivek Sarkar},
\authorsc{Armando Solar-Lezama},  
\authorsc{S. Alexander Spoon}, 
\authorsc{Sayantan Sur}, 
\authorsc{Christoph von Praun},
\authorsc{Pradeep Varma},
\authorsc{Krishna Venkata},
\authorsc{Jan Vitek}, and
\authorsc{Tong Wen}.

For extended discussions and support we would like to thank: 
Robert Callahan, Calin
Cascaval, Norman Cohen, Elmootaz Elnozahy, John Field, Bob Fuhrer,
Orren Krieger, Doug Lea, John McCalpin, Paul McKenney, Ram Rajamony,
R.K.~Shyamasundar, Filip Pizlo, V.T.~Rajan, Frank Tip, and Mandana Vaziri.

We thank Jonathan Rhees and William Clinger with help in obtaining the
\LaTeX{} style file and macros used in producing the Scheme report,
on which this document is based. We acknowledge the influence of
the $\mbox{\Java}^{\mbox{\authorsc{\small tm}}}$ Language
Specification \cite{jls2}.
%document, as evidenced by the numerous citations in the text.

This document revises Version 1.1 of the Report, released in
June 2007. It documents the language corresponding to the second
revision of the first version of the implementation.  This
revision was done by
\authorsc{Raj Barik}, 
\authorsc{Philippe Charles}, 
\authorsc{Christopher Donawa}, 
\authorsc{Robert Fuhrer},
\authorsc{Nathaniel Nystrom},  
\authorsc{Vijay Saraswat},
\authorsc{Vivek Sarkar},
\authorsc{Pradeep Varma}, and
\authorsc{Krishna Venkata}.
(Earlier implementations benefited from significant contributions by
\authorsc{Christian Grothoff} and 
\authorsc{Christoph von Praun}.)
\authorsc{Tong Wen} has written many application programs
in \Xten{}. \authorsc{Guojing Cong} has helped in the
development of many applications.


%\vfill
%\begin{center}
%{\large \bf
%*** DRAFT*** \\
%%August 31, 1989
%\today
%}\end{center}

\vfill
\eject


\chapter*{Contents}
\addvspace{3.5pt}                  % don't shrink this gap
\renewcommand{\tocshrink}{-3.5pt}  % value determined experimentally
{\footnotesize
\tableofcontents
}

\vfill
\eject


   \par  % vj: first page
\clearextrapart{Introduction}

\subsection*{Background}

Bigger computational problems need bigger computers capable of
performing a larger number of operations per second. The era of
increasing performance by simply increasing clocking frequency now
seems to be behind us; faster chips run hotter and current cooling
technology does not scale as rapidly as the clock. Instead, computer
designers are starting to look at ``scale out'' systems in which the
system's computational capacity is increased by adding additional
nodes of comparable power to existing nodes, and connecting nodes with
a high-speed communication network.

A central problem with scale out systems is a definition of the {\em
memory model}, that is, a model of the interaction between shared
memory and  simultaneous (read, write) operations on that
memory by multiple processors. The traditional ``one operation at a
time, to completion'' model that underlies Lamport's notion of {\em
sequential consistency} (SC) proves too expensive to implement in
hardware, at scale. Various models of {\em relaxed consistency} have
proven too difficult for programmers to work with.  

One response to this problem has been to move to a {\em fragmented
memory model}. Multiple processors -- each sequentially consistent
internally -- are made to interact via a relatively language-neutral
message-passing format such as MPI \cite{mpi}. This model has enjoyed
some success: several high-performance applications have been written
in this style. Unfortunately, this model leads to a {\em loss of
programmer productivity}: the mesage-passing format is integrated into
the host language by means of an application-programming interface
(API), the programmer must explicitly represent and manage the
interaction between multiple processes and choreograph their data
exchange; large data-structures (such as distributed arrays, graphs,
hash-tables) that are conceptually unitary must be thought of as
fragmented across different nodes; all processors must generally
execute the same code (in an SPMD fashion) etc.

One response to this problem has been the advent of the {\em
partitioned global address space} (PGAS) model underlying languages such as
UPC, Titanium and Co-Array Fortran \cite{pgas}. These languages permit
the programmer to think of a single computation running across the
multiple processors, sharing a common address space. All data resides
at some processors, which is said to have {\em affinity} to the
data. Each processor may operate directly on the data it contains but
must use some indirect mechanism to access or update data at other
processors. Some kind of global {\em barriers} are used to ensure that
processors remain roughly in lock-step.

\Xten{} is a modern object-oriented programming language
in the PGAS family. The fundamental goal of \Xten{} is to enable
high-performance, high-productivity programming for high-end
(scale-out) computers -- for traditional numerical computation
workloads (such as weather simulation, molecular dynamics, particle
transport problems etc) as well as commercial server
workloads. \Xten{} is based on state-of-the-art object-oriented
programming ideas primarily to take advantage of their proven
flexibility and ease-of-use for a wide spectrum of programming
problems. \Xten{} takes advantage of several years of research (e.g.{}
in the context of the Java Grande forum,
\cite{moreira00java,kava}) on how to adapt such languages to the context of
high-performance numerical computing. Thus \Xten{} provides support
for user-defined {\em value types} (such as {\tt int}, {\tt float},
{\tt complex} etc), including operator overloading, supports a very
flexible form of multi-dimensional arrays (based on ideas in ZPL
\cite{zpl}) and supports IEEE-standard floating point arithmetic.

The major novel contribution of \Xten{} however is its flexible
treatment of concurrency, distribution and locality, within an
integrated type system. \Xten{} introduces {\em places} as an
abstraction for a {\em virtual shared-memory multi-processor}; a
computation runs over a large collection of places. Each place hosts
some data and runs one or more {\em activities}. Activities are
extremely lightweight threads of execution and may dynamically spawn
new activities locally or at remote places. {\em Clocks} are used to
ensure that a programmer-specified, data-dependent set of activities
has quiesced before another action is initiated. Arrays may be
distributed across multiple places. A static type system allows the
programmer to keep track of the location of objects and ensures
statically that an activity does not synchronously attempt to
read/write remote data.

%% say something about native.

\Xten{} is an experimental language. This document lays out an initial set 
of ideas which we expect to be the basis of an initial
implementation. Several representative concurrent idioms have found
pleasant expression in \Xten. We intend to develop several full-scale
applications to get better experience with the language, and revisit
the design in the light of this experience. Future versions of the
language are expected to support user-definable operators and permit
the specification of generic classes and methods. 


   \par  % 0.1
\chapter{Overview of \Xten}
\section{Semantics}
\Xten{} may be thought of as (generic) \java{} less concurrency, arrays and built-in types,  plus {\em places}, {\em activities}, {\em clocks}, (distributed,
multi-dimensional) {\em arrays} and {\em value} types. All these changes are
motivated by the desire to use the new language for high-end,
high-performance, high-productivity computing.

\subsection{Places and activities}
The central new concept in \Xten{} is that of a {\em place}
(\S~\ref{XtenPlaces}).  A place may be thought of conceptually as a
``virtual shared-memory multi-processor'': a computational unit with a
finite, though perhaps dynamically varying, number of hardware threads
and a bounded amount of shared memory uniformly accessible by all
threads.  An \Xten{} program is intended to run on a computer capable
of supporting millions of places.

An \Xten{} computation acts on {\em data
objects}(\S~\ref{XtenObjects}) through the execution of lightweight
threads called {\em activities}(\S~\ref{XtenActivities}).  Objects are
of two kinds. A {\em scalar} object has a small, statically fixed set
of fields, each of which has a distinct name. A scalar object is
located at a single place and stays at that place throughout its
lifetime.  An {\em aggregate} object has many fields (the number may
be known only when the object is created), uniformly accessed through
an index (e.g.{} an integer) and may be distributed across many
places. The distribution of an aggregate object remains unchanged
throughout the computation. \Xten{} assumes an underlying garbage
collector will dispose of (scalar and aggregate) objects and reclaim
the memory associated with them once it can be determined that these
objects are no longer accessible from the current state of the
computation. (There are no operations in the language to allow a
programmer to explicitly release memory.)

{}\Xten{} has a {\em unified} or {\em global address space}. This
means that an activity can reference objects at other places.
However, an activity may synchronously access data items only in the
current place (the place in which the activity is running). It may
atomically update one or more data items, but only in the current
place.  Indeed, all accesses to mutable shared data must occur from
within an {\em atomic section}. To read a remote location, an activity
must spawn another activitiy {\em asynchronously}
(\S~\ref{AsynchronousActivity}). This operation returns immediately,
leaving the spawning activity with a {\em future}
(\S~\ref{XtenFutures}) for the result. Similarly, remote location can
be written into only by asynchronously spawning an activity to run at
that location.

Throughout its lifetime an activity executes at the same place. An
activity may dynamically spawn activities in the current or remote
places.

\paragraph{Atomic sections}

\Xten{} introduces statements of the form {\cf atomic S} where {\cf S}
is a statement.  The type system ensures that such a statement will
dynamically access only local data. (The statement may throw
a {\cf BadPlaceException} -- but only because of a failed place cast.)
Such a statement is executed by the activity as if in a single step
during which all other activities are frozen.

\paragraph{Asynch activities}

An asynch activity is a statement of the form {\cf async (P) S} where
{\cf P} is a place expression and {\cf S} is a statement.  Such a
statement is executed by spawning an activity at the place designated
by {\cf P} to execute statement {\cf S}.

An async expression of type {\cf future T}} has the form {\cf future
(P) E} where {\tt E} is an expression of type {\tt T}. It executess
the expression {\tt E} at the place {\tt P} as an async activity,
immediately returning with a future. The future may later be forced
causing the activity to be blocked until the return value has been
computed by the async activity.

\subsection{Clocks}
The MPI style of coordinating the activity of multiple processes with
a single barrier is not suitable for the dynamic, asynchronous network
of activities in an \Xten{} computation. Instead, it becomes necessary
to allow a computation to use multiple barriers. \Xten{} {\em clocks}
(\S~\ref{XtenClocks}) are designed to offer the functionality of
multiple barriers in a dynamic context while still supporting
determinate, deadlock-free parallel computation.

Activities may use clocks to repeatedly detect quiescence of arbitrary
programmer-specified, data-dependent set of activities. Each activity
is spawned with a known set of clocks and may dynamically create new
clocks. At any given time an activity is {\em registered} with zero or
more clocks. It may register newly created activities with a clock,
un-register itself with a clock, suspend on a clock or require that a
statement (possibly involving execution of new async activities) be
executed to completion before the clock can advance.  At any given
step of the execution a clock is in a given phase. It advances to the
next phase only when all its registered activities have {\em quiesced}
(by executing a {\tt continue} operation on the clock), and all
statements scheduled for execution in this phase have terminated.
When a clock advances, all its activities may now resume execution.

Thus clocks act as {\em barriers} for a dynamically varying collection
of activities. They generalize the barriers found in MPI style program
in that an activity may use multiple clocks simultaneously. Yet
programs using clocks are guaranteed not to suffer from
deadlock. Clocks are also integrated into the \Xten{} type system,
permitting variables to be declared so that they are {\tt final} in each
phase of a clock.

\subsection{Interfaces and Classes}
Programmers write \Xten{} code by writing {\em generic interfaces}
(\S~\ref{XtenInterfaces}) and {\em generic classes}
(\S~\ref{XtenClasses}). Generic interfaces and classes may be
defined over a collection of {\em type parameters}. Instances can be
created only from {\em concrete} classes; such a class has all its
type parameters (if any) instantiated with concrete classes and
concrete interfaces.

\subsection{Scalar classes}
An \Xten{} scalar class (\S~\ref{XtenClasses}) has fields, methods and
inner types (interfaces, classes), subclasses another class, and
implements one or more interfaces. Thus \Xten{} classes live in a
single-inheritance code hierarchy.  \Xten{} allows the programmer to
define literals for classes, and overload infix/prefix/postfix
operators.

There are two kinds of scalar classes: {\em reference} classes
(\S~\ref{ReferenceClasses}) and {\em value} classes
(\S~\ref{ValueClasses}).

A reference class typically has updatable fields. Objects of such a
class may not be freely copied from place to place. Methods may be
invoked on such an object only by an activity in the same place.

A value class (\S~\ref{ValueClasses}) has no updatable fields (defined
directly or through inheritance), and allows no reference
subclasses. (Fields may be typed at reference classes, so may contain
references to objects with mutable state.) Objects of such a class may
be freely copied from place to place, and may be implemented very
efficiently. Methods may be invoked on such an object from any place.

\Xten{} has no primitive classes. However, the standard library {\cf x10.lang} supplies (final) value classes {\cf boolean}, {\cf byte}, {\cf short}, {\cf char}, {\cf int}, {\cf long}, {\cf float}, {\cf complex} and {\cf string}. The user may defined additional arithmetic value classes using the facilities of the language.

\subsection{Arrays, Regions and Distributions}
An \Xten{} array is a function from a {\em distribution}
(\S~\ref{XtenDistributions}) to a base type (which may itself be an
array type).

A distribution is a map from a {\em region} (\S~\ref{XtenRegions}) to a
subset of places.  A region is a collection of indices.

Operations are provided to construct regions from other regions, and
to iterate over regions. Standard set operations, such as union,
disjunction and set difference are available for regions.

A primitive set of distributions is provided, together with operations
on distributions. A {\em sub-distribution} of a distribution is one
which is defined on a smaller region and agrees with the distribution
at all points.  The standard operations on regions are extended to
distributions.

In future versions of the language, a programmer may specify new
distributions, and new operations on distributions.

A new array can be created by restricting an existing array to a
sub-distribution, by combining multiple arrays, and by performing
pointwise operations on arrays with the same distribution.

\Xten{} allows array constructors to iterate over the underlying
distribution and specify a value at each item in the underlying
region. Such a constructor may spawn activities at multiple places.


\subsection{Nullable type constructor}

\Xten{} has a {\cf nullable} type constructor which can be applied uniformly to
scalar (value or reference) and array types. This type constructor
returns a new type which adds a special value {\cf null} to the set of
values of its argument type, unless the argument type already has this
value.

\subsection{Statements and expressions}
\Xten{} supports the standard set of primitive operations (assignment, classcasts) and sequential control constructs (conditionals, looping, method
invocation, exception raising/catching) etc.

\paragraph{Place casts}
The programmer may use the standard classcast mechanism
(\S~\ref{ClassCast}) to cast a value to a located type. A {\cf
BadPlaceException} is thrown if the value is not of the given
type. This is the only language construct that throws a {\cf
BadPlaceException}.

\subsection{Translating MPI programs to \Xten{}}

While \Xten{} permits considerably greater flexibility in writing
distributed programs and data structures than MPI, it is instructive
to examine how to translate MPI programs to \Xten.

Each separate MPI process can be translated into an \Xten{}
place. Async activities may be used to read and write variables
located at different processes. A single clock may be used for barrier
synchronization between multiple MPI processes. \Xten{} collective
operations may be used to implement MPI collective operations.
\Xten{} is more general than MPI in (a)~not requiring synchronization
between two processes in order to enable one to read and write the
other's values, (b)~permitting the use of high-level atomic sections
within a process to obtain mutual exclusion between multiple
activities running in the same node (c)~permitting the use of multiple
clocks to combine the expression of different physics (e.g.{}
computations modeling blood coagulation together with computations
involving the flow of blood), (d)~not requiring an SPMD style of
computation.

%\note{Relaxed exception model}
\subsection{Summary and future work}

{}\Xten{} is considerably higher-level than thread-based languages in
that it supports dynamically spawning very lightweight activities, the
use of atomic operations for mutual exclusion, and the use of clocks
for repeated quiescence detection of a data-dependent set of
activities. Yet it is much more concrete than languages like HPF in
that it forces the programmer to explicitly deal with distribution of
data objects. In this the language reflects the designers belief that
issues of locality and distribution cannot be hidden from the
programmer of high-performance code in high-end computing.  A
performance model that distinguishes between computation and
communication must be made explicit and transparent.\footnote{In this
\Xten{} is similar to more modern languages such as ZPL \cite{zpl}.} At
the same time we believe that the place-based type system and support
for generic programming will allow the \Xten{} programmer to be highly
productive; many of the tedious details of distribution-specific code
can be handled in a generic fashion.

We expect the next version of the language to be significantly
informed by experience in implementing and using the language. We
expect it to have constructs to support continuous program
optimization, and allow the programmer to provide guidance on
clustering places to (hardware) nodes. For instance, we may introduce
a notion of hierarchical clustering of places.




  \par % Semantics section. What else?
\vskip 2ex
\clearchapterstar{Description of the language} %\unskip\vskip -2ex
\chapter{Lexical structure}

In general, \Xten{} follows \java{} rules \cite[Chapter 3]{jls2} for
lexical structure.

Lexically a program consists of a stream of white space, comments,
identifiers, keywords, literals, separators and operators.

\paragraph{Whitespace}
% Whitespace \index{whitespace} follows \java{} rules \cite[Chapter 3.6]{jls2}.
ASCII space, horizontal tab (HT), form feed (FF) and line
terminators constitute white space.

\paragraph{Comments}
% Comments \index{comments} follows \java{} rules
% \cite[Chapter 3.7]{jls2}. 
All text included within the ASCII characters ``\xcd"/*"'' and
``\xcd"*/"'' is
considered a comment and ignored; nested comments are not
allowed.  All text from the ASCII characters
``\xcd"//"'' to the end of line is considered a comment and is ignored.

\paragraph{Identifiers}

Identifiers \index{identifier} are defined as in \java.
Identifiers consist of a single letter followed by zero or more
letters or digits.
Letters are defined as the characters for which the \java{}
method \xcd"Character.isJavaIdentifierStart" returns true.
Digits are defined as the ASCII characters \xcd"0" through \xcd"9".

\paragraph{Keywords}
\Xten{} reserves the following keywords:
\begin{xten}
abstract        any             as              async
at              ateach          atomic          await
break           case            catch           class
clocked         const           continue        current
def             default         do              else
extends         extern          final           finally
finish          for             foreach         future
goto            has             here            if
implements      import          in              instanceof
interface       local           native          new
next            nonblocking     or              package
private         protected       property        public
return          safe            self            shared
static          super           switch          this
throw           throws          try             type
val             value           var             when
while
\end{xten}
Note that the primitive types are not considered keywords.
The keyword \xcd{goto} is reserved, but not used.

\paragraph{Literals}\label{Literals}\index{literals}

Literals are either integers, floating point numbers, booleans,
characters, strings, and \xcd"null".
\XtenCurrVer{} defines literal syntax in the same way as \java{} does.
Unsigned 32-bit integers are suffixed with
\xcd{U} or \xcd{u}.
Signed 64-bit integers are suffixed with
\xcd{L} or \xcd{l}.
Unsigned 64-bit integers are suffixed with
any of \xcd{LU}, \xcd{Lu}, \xcd{UL}, \xcd{Ul},
\xcd{lU}, \xcd{lu}, \xcd{uL}, or \xcd{ul}.

\paragraph{Separators}
\Xten{} has the following separators and delimiters:
\begin{xten}
( )  { }  [ ]  ;  ,  .
\end{xten}

\paragraph{Operators}
\Xten{} has the following operators:
\begin{xten}
==  !=  <   >   <=  >=
&&  ||  &   |   ^
<<  >>  >>>
+   -   *   /   %
++  --  !   ~
&=  |=  ^=
<<= >>= >>>
+=  -=  *=  /=  %=
=   ?   :   =>  ->
<:  :>  @   ..
\end{xten}




	\par % 0.1
\chapter{Types}
\label{XtenTypes}\index{types}

{}\Xten{} is a {\em strongly typed} object language: every variable
and expression has a type that is known at compile-time. Further,
\Xten{} has a {\em unified} type system: all data items created at
runtime are {\em objects} (\S~\ref{XtenObjects}. Types limit the
values that variables can hold, and specify the places at which these
values lie.

{}\Xten{} supports two kinds of objects, {\em reference objects} and
{\em value objects}.  Reference objects are instances of {\em
reference classes} (\S~\ref{ReferenceClasses}). They may contain
mutable fields and must stay resident in the place in which they were
created. Value objects are instances of {\em value classes}
(\S~\ref{ValueClasses}). They are immutable and may be freely copied
from place to place. Either reference or value objects may be 
{\em scalar} (instances of a non-array class) or {\em aggregate} (instances
of arrays).

An \Xten{} type is either a {\em reference type} or a {\em value
type}.  Each type consists of a {\em data type}, which is a set of
values, and a {\em place type} which specifies the place at which the
value resides.  Types are constructed through the application of {\em
type constructors} (\S~\ref{TypeConstructors}).

Types are used in variable declarations, casts, object creation, array
creation, class literals and {\cf instanceof} expressions.\footnote{In
order to allow this version of the language to focus on the core new
ideas, \XtenCurrVer{} does not have user-definable classloaders,
though there is no technical reason why they could not have been
added.}

A variable is a storage location (\S~\ref{XtenVariables}). All
variables are initialized with a value and cannot be observed without
a value. 

Variables whose value may not be changed after initialization are
called {\em final variables} (or sometimes {\em constants}).  The
programmer indicates that a variable is final by using the annotation
{\tt final} in the variable declaration.  

%% Final variables play an 
%% important role in \Xten{}, as we shall discuss below. For this reason,
%% \Xten{} enforces the lexical restriction that all variables whose name
%% starts with an upper case letter are implicitly declare final. (It is
%% not an error to also explicitly declare such variables as
%% final.)\index{Upper-case Convention}

\section{Type constructors}\index{type constructors}\label{TypeConstructors}

An \Xten{} type is a pair specifying a {\em datatype} and a {\em
placetype}. Semantically, a datatype specifies a set of values and a
placetype specifies the set of places at which these values may
live. Thus taken together, a type specifies both the kind of value
permitted and its location. 

\begin{x10}
509   Type ::=  DataType  PlaceTypeSpecifieropt
510     | nullable  Type
511     | future <  Type > 
512   DataType ::=  PrimitiveType
513   DataType ::=  ClassOrInterfaceType
514     |  ArrayType
\end{x10}

For simplicity, this version of \Xten{} does not permit the
specification of generic classes or interfaces. This is expected to be
remedied in future versions of the language.

Every class and interface definition in \Xten{} defines a type with
the same name. Additionally, {}\Xten{} specifies three {\em type
constructors}: {\tt nullable}, the {\tt future}, and array type
constructors. We discuss these constructors and place types in detail
in the secions that follow; here we briefly discss interface and class
declarations.

\paragraph{Interface declarations.}\label{InterfaceTypes}
An interface declaration specifies a name, a list of extended
interfaces, and constants ({\tt public static final} fields) and
method signatures associated with the interface. Each interface
declaration introduces a type with the same name as the declaration.
Semantically, the data type is the set of all objects which are
instances of (value or reference) classes that implement the
interface. A class implements an interface if it says it does and if
it implements all the methods defined in the interface.


The {\em interface declaration} (\S~\ref{XtenInterfaces}) takes as
argument one or more interfaces (the {\em extended} interfaces), one
or more type parameters and the definition of constants and method
signatures and the name of the defined interface.  Each such
declaration introduces a data type.

\begin{x10}
426   DataType ::= ClassOrInterfaceType
433   ClassOrInterfaceType ::= TypeName 
13    ClassType ::= TypeName
15    TypeName ::= identifier
16     | TypeName . identifier
\end{x10}

\paragraph{Reference class declarations.}\label{ReferenceTypes}
The {\em reference class declaration} (\S~\ref{ReferenceClasses}) takes
as argument a reference class (the {\em extended class}), one or more
interfaces (the {\em implemented interfaces}), the definition of
fields, methods and inner types, and returns a class of the named type
(\S~\ref{ReferenceClasses}). Each such declaration introduces a data
type. Semantically, the data type is the set of all objects which are
instances of (subclasses of) the class.

\paragraph{Value class declarations.}
The {\em value class declaration} (\S~\ref{ValueClasses}) is
similar to the reference class declaration except that it must extend
either a value class or a reference class that has no mutable fields.
It may be used to construct a value type in the same way as a
reference class declaration can be used to construct a reference type.

\section{The \Xcd{nullable} type constructor}
\label{NullableTypeConstructor}\index{nullable@{\tt nullable}}

\Xten{} supports the type constructor, \xcd"nullable[T]".  For any
type \xcd"T", the type \xcd"nullable[T]" contains all the values of
type \xcd"T", and a special \xcd"null" value, unless \xcd"T" already
contains \xcd"null". This value is designated by the literal
\xcd"null", which is special in that it has the type
\xcd"nullable T" for all types \xcd"T".\index{null@{\tt null}}

The visibility of the type \xcd"nullable[T]" is the same as the
visibility of \xcd"T". The members of the type \xcd"nullable[T]" are
the same as those of type \xcd"T". Note that because of this
\xcd"nullable" may not be regarded as a generic class; rather it is a
special type constructor.  In fact, \xcd"nullable[T]" can be
considered a mixin; it is a subtype of \xcd"T".

%% TODO: Visibility of nullable^T.

This type constructor can be used in any type expression used to
declare variables (e.g., local variable{s}, method parameter{s},
class field{s}, iterator parameter{s}, try/catch parameter{s} etc).
It may be applied to value types, reference types or aggregate types.
It may not be used in an \xcd"extends" clause or an \xcd"implements"
clause in a class or interface declaration. It may not be used 
in a \xcd"new" expression---a \xcd"new" expression is used only to construct 
non-null values.

If \xcd"T" is a value
(respectively, reference) type, then \xcd"nullable[T]" is defined to be
a value (respectively, reference) type.

An immediate consequence of the definition of \xcd"nullable" is that
for any type \xcd"T", the type \xcd"nullable[nullable[T]]" is equal to
the type \xcd"nullable[T]".

Any attempt to access a field or invoke a method on the value
\xcd"null" results in a \xcd"NullPointerException".

An expression \xcd"e" of type \xcd"nullable[T]" may be checked for nullity
using the expression \xcd"e==null". (It is a compile-time error for
the static type of \xcd"e" to not be \xcd"nullable[T]", for some \xcd"T".)

\paragraph{Conversions}
\xcd"null" can be passed as an argument to a method call whose
corresponding formal parameter is of type \xcd"nullable[T]" for some type
\xcd"T". (This is a widening reference conversion, per \cite[Sec
5.1.4]{jls2}.) Similarly it may be returned from a method call of
return type \xcd"nullable T" for some type \xcd"T".

For any value \xcd"v" of type \xcd"T", the class cast expression
\xcd"(nullable[T]) v" succeeds and specifies a value of type \xcd"nullable[T]".
This value may be seen as the ``boxed'' version of \xcd"v".

\Xten{} permits the widening reference conversion from any type \xcd"T"
to the type \xcd"nullable[T1]" if \xcd"T" can be widened to the
type \xcd"T1". Thus, the type \xcd"T" is a subtype of the type \xcd"nullable[T]".
%in accordance with the LiskovSubstitutionPrinciple.

Correspondingly, a value \xcd"e" of type \xcd"nullable[T]" can be cast to the
type \xcd"T", resulting in a \xcd"NullPointerException" if \xcd"e" is
\xcd"null" and \xcd"nullable[T]" is not equal to \xcd"T", and in the
corresponding value of type \xcd"T" otherwise.  If \xcd"T" is a value
type this may be seen as the ``unboxing'' operator.

The expression \xcd"(T) null" throws a \xcd"ClassCastException"
if \xcd"T" is not equal to \xcd"nullable[T]"; otherwise it
returns \xcd"null" at type \xcd"T". Thus it may be used to check
whether \xcd"T" = \xcd"nullable[T]".

\paragraph{Arrays of nullary type}
The nullary type constructor may also be used in (aggregate) instance
creation expressions (e.g., \xcd"new nullable[T](R)"). In such a
case \xcd"T" must designate a class. Each member of the array is
initialized to \xcd"null", unless an explicit array initializer is
specified.

\paragraph{Implementation notes}
A value of type \xcd"nullable[T]" may be implemented by boxing a value of
type \xcd"T" unless the value is already boxed. The literal \xcd"null"
may be represented as the unique null reference.

\paragraph{\Java{} compatibility}

\java{} provides a somewhat different treatment of \xcd"null".  A
class definition extends a nullable type to produce a nullable type,
whereas primitive types such as \xcd"int" are not nullable---the
programmer has to explicitly use a boxed version of \xcd"int",
\xcd"Integer", to get the effect of \xcd"nullable int". Wherever \Java{} uses a
variable at reference type \xcd"T", and at runtime the variable may
carry the value \xcd"null", the \Xten{} programmer should declare the
variable at type \xcd"nullable[T]". However, there are many situations
in \java{} in which a variable at reference type \xcd"T" can be
statically determined to not carry null as a value. Such variables
should be declared at type \xcd"T" in \Xten{}.

\paragraph{Design rationale}

The need for \xcd"nullable" arose because \Xten{} has value types and
reference types, and arguably the ability to add a \xcd"null" value to
a type is orthogonal to whether the type is a value type or a
reference type. This argues for the notion of nullability as a type
constructor.

The key question that remains is whether it should be possible to
define ``towers'', that is, define the type constructor in such a way
that \xcd"nullable[nullable[T]]" is distinct from \xcd"nullable[T]". Here
one would think of nullable as a disjoint sum type constructor that
adds a value \xcd"null" to the interpretation of its argument type
even if it already has that value. Thus \xcd"nullable[nullable[T]]" is
distinct from \xcd"nullable T" because it has one more \xcd"null"
value. Explicit injection and projection functions of signature
\xcd"T => nullable[T]" and \xcd"nullable[T] => T", respectively,
would need to be provided.

The designers of \Xten{} felt that while such a definition might be
mathematically tenable, and programmatically interesting, it was
likely to be too confusing for programmers. More importantly, it would
be a deviation from current practice that is not forced by the core
focus of \Xten{} (concurrency and distribution). Hence the decision to
collapse the tower.  As discussed below, this results in no loss of
expressiveness because towers can be obtained through explicit
programming.

\paragraph{Examples}

Consider the following class:

\begin{xten}
final value Box { 
  public def this(v: nullable[Object]) { this.datum = v; }
  public var datum: nullable[Object];
}
\end{xten}

Now one may use a variable \xcd"x" at type \xcd"nullable[Box]" to
distinguish between the \xcd"null" at type \xcd"nullable[Box]" and at type
\xcd"nullable[Object]". In the first case the value
of \xcd"x" will be \xcd"null", in the second case the value of \xcd"x.datum"
will be \xcd"null".

Such a type may be used to define efficient code for memoization:

\begin{xten}
abstract class Memo {
  var values: Array[nullable[Box]];
  def this(n: int) {
    // initialized to all nulls
    values = new nullable[Box](n); 
  }
  abstract def compute(key: int): nullable[Object];
  def lookup(key: int) nullable[Object] = { 
   if (values(key) != null) 
     return values(key).datum;
   val v = compute(key);
   values(key) = new Box(v);
   return v;
  }
}
\end{xten}


% C#: http://blogs.msdn.com/ericgu/archive/2004/05/27/143221.aspx
% Nice: http://nice.sourceforge.net/cgi-bin/twiki/view/Doc/OptionTypes

 



\subsection{Future types}

For any type \xcd"T", \xcd"future[T]" is a type:
\begin{verbatim}
424   Type ::=   future [ Type ] 
\end{verbatim}
The type  represents a value which when forced will return a value of type
\xcd"T". Thus the type makes available the following methods:

\begin{xten}
  public def force(): T;
\end{xten}

%% Nothing to say about future/nullable yet.

\subsection{Array types}
\label{ArrayTypeConstructors}\index{array types}
\index{\Xcd{Array}}
\index{\Xcd{ValArray}}
\index{\Xcd{x10.lang.Array}}
\index{\Xcd{x10.lang.ValArray}}

Arrays in \Xten{} are instances of the value classes
\xcd"x10.lang.Array" and \xcd"x10.lang.ValArray".
Because of the importance of arrays in \Xten{}, the language
supports more concise syntax for accessing array elements and
performing operations on arrays.

% The array type constructor takes as argument a type (the {\em base
% type}), an optional distribution (\Sref{XtenDistributions}), and
% optionally the keyword \xcd"value":

The array type \xcd"Array[T]" is the type of all
reference arrays of base type \xcd"T". Such an array can take on any
distribution, over any region. 

The array type \xcd"ValArray[T]" specifies the type of all
values arrays of base type \xcd"T".
The array elements of a \xcd"ValArray" are
all final.\footnote{Note that the base type of a
\xcd"ValArray" can be a value class or a reference class, just as the 
type of a final variable can be a value class or a reference class.}

Both array classes implement the function type
\xcd"(Point) => T"; the element of array \xcd"A" at point
\xcd"p" may be accessed using the 
syntax \xcd"A(p)".  The \xcd"Array" class 
also implements the \xcd"Settable[Point,T]" interface 
permitting assignment to an array element using the syntax
\xcd"A(p) = v".

\XtenCurrVer{} also allows a distribution to be specified 
as a property initializer on the array type.
The distribution must be an expression of type
\xcd"Dist" whose
value does not depend on the value of any mutable variable.

\Xten{} also supports dependent types for arrays,
e.g.,
\xcd"Array[Double]{rank==3}" is the type of all arrays of 
\xcd"Double" of rank \xcd"3".

\subsection{Rails}

A \emph{rail} is a 1-dimensional, zero-based, local array. 
The type \xcd"Rail[T](n)" is an alias of the mutable array type
\xcd"Array[T](Region(0..n-1)->here)".
The type \xcd"ValRail[T](n)" is an alias of the mutable array type
\xcd"ValArray[T](Region(0..n-1)->here)".

\begin{xten}
package x10.lang;
type Rail = Array{rail};
type Rail[T] = Array[T]{rail};
type Rail[T](n: Int) = Array[T]{rail, region==[0..n-1]);
type Rail[T](n: Int) = Array[T]{rail, region==[0..n-1]);
\end{xten}

\Xten{} supports shorthand syntax for rail construction
(\Sref{RailConstructors}).


\notfouro{\section{Place types and Type reconstruction}\label{PlaceTypes}\index{place types}

\Xten{} distinguishes two kinds of places: {\em shared places}\index{places!shared}\label{SharedPlaces}, 
which correspond to the memory of individual processors which is
shared across multiple activities, and {\em scoped
places}\index{places!scoped}\label{ScopedPlaces} such as threadlocal
memory and method memory which is available only in limited
scope. \XtenCurrVer{} supports only {\tt threadlocal} scoped memory,
i.e.{} memory accessible only to the current activity.  Future
versions may support {\tt methodlocal} and {\tt blocklocal} memory.

\subsection{Place expressions}
The following are place expressions:
\begin{itemize}

\item {\tt here}: the place of the current computation
(in a method) or the place of the current object (in field
declarations).  At the top level of a method, {\tt here} is the place
at which the method invocation's object lives. Inside an {\tt async},
{\tt here} is the place of the nearest enclosing {\tt async}.

{}\item {\tt threadlocal}: the scoped place for the current thread
(only in scope for method activations)

{}\item {\tt D[i]}: where {\tt D} is a distribution
(\S~\ref{XtenDistributions}) and {\tt i} is a point in the region of
the distribution.

{}\item {\tt e.place} where {\tt e} is a variable
(\S~\ref{XtenVariables}) at a reference type. \Xten{} allows the use
of the expression {\tt e} in contexts expecting a {\tt place}
parameter as shorthand for {\tt e.place}.
\end{itemize}
Note that \XtenCurrVer{} does not permit the dynamic creation of a
place. Each \Xten{} computation is initiated with a fixed number of
places, as determined by a configuration parameter. The number is available
from {\tt place.MAX\_PLACES}.  The set of places is available 

Place expressions are used in the following contexts: 
\begin{itemize}
\item As a  place type in a reference type (\S~\ref{ReferenceTypes}).
\item As a target for an {\tt async} activity (\S~\ref{AsyncActivity}).
\item In a class cast expression  (\S~\ref{ClassCast}).
\item In an {\tt instanceof} expression (\S~\ref{instanceOf}).
\item In stable equality comparisons, at type {\tt place}.
\item As a parameter to a method invocation (\S~\ref{MethodInvocation}).
{}\item As a type parameter to a generic class or interface
(\S~\ref{TypeParameter}), or an actual argument to a generic class or interface (\S~\ref{ClassCreation}).
\end{itemize}

An example:
\begin{x10}
<P> void foo(Foo@P foo, Bar@P bar) \{
   int x = async (P) \{ return foo.f + bar.b; \};
   ...
\}
\end{x10}


Places can be passed as parameters to classes. The place where the
object lives is an implicit parameter, accessible within constructors
and instance initializers as here:

\begin{x10}
    class Foo <place P, place Q> \{
        Bar@P bar;
        Baz@Q baz;
        Qux@here qux;
    \}
\end{x10}

A reference type {\tt Foo} (with no place annotation) is always taken
as shorthand for {\tt Foo@here}.

An object can be cast to a particular place. If the object is not at
the right place, a {\tt BadPlaceException} is thrown:

\begin{x10}
    <P> m(Object@P obj) \{
        Object@here h = (Object@here)obj;
        h.blah();
    \}
\end{x10}

Places can be checked for equality. One can view this as the analog of
the {\tt instanceof} operator for places.

\begin{x10}
 <P> m(Object@P obj) \{
     if (here == P) \{
        // will never throw an exception
       Object@here h = (Object@here)obj;
       h.m();
   \}
\}
\end{x10}

A special form of object reference type, {\tt anywhere} or {\tt ?}, 
constrains the object to be at a shared place but does not constrain
the place. This allows collections of objects at heterogenous places.

\begin{x10}
    Object@?[] objects;
\end{x10}

We allow {\tt async} and {\tt future} statements to use {\tt ?} to
stand for any place. For instance:

\begin{x10}
     Object o = !future(?)\{new Object();\}
\end{x10}

\noindent spawns an activity at some arbitrary place and return an object created at that place.

Similarly the object reference type {\tt current} can be used in
(i)~constructing a distribution for a reference array or (ii)
specifying the location of the base type of a reference array. In the
first case points mapped to {\tt current} by the distribution will
reside in the same place as the array object itself, and in the second
case the value of the array at a particular point is an object in the
same place as that array component. Example:
\begin{x10}
    Object@current[] objects;
\end{x10}

\subsubsection{Type reconstruction}\index{type!reconstruction}

{}\Xten{} permits the use of the generic method syntax in variable
declarations. Any variable declaration may be prefixed with a type
parameter list and may use these parameters in type expressions in the
declaration. Such a parameterized variable declaration succeeds at
compile time if it is possible for the compiler to assign unique types
to the parameters in such a way that the declaration type-checks.  The
scope of the parameter is the scope of the variable introduced by the
declaration. Throughout this scope the parameter has the value
inferred by the compiler.

For instance:
\begin{x10}
  // This introduces P as a constant place, the
  // location of objects[0]
 <P> Object@P obj = objects[0];
 async (P) { obj.blah(); };
\end{x10}

Often it is the case that a type parameter is not referenced after it
is introduced. In such cases \Xten{} permits the use of ``\_'' (the
{\em anonymous
parameter})\label{AnonymousParameter}\index{parameter!anonymous@{\tt\_}}
as a parameter. Multiple occurrences of ``\_'' are taken to stand for
``fresh names'' in each occurrence.  For instance:
\begin{x10}
 <P> \_@P obj = objects[0];
 async (P) { obj.blah(); };
\end{x10}
\noindent should be taken as shorthand for
\begin{x10}
 <P,Q> Q@P obj = objects[0];
 async (P) { obj.blah(); };
\end{x10}
{}\noindent where {\tt Q} does not occur anywhere else in the
program. If a declaration only uses the anoymous parameter the angle
brackets may be omitted. Thus for example the often used special case:
\begin{x10}
  \_ obj = new int[D];
\end{x10}
\noindent is shorthand for:
\begin{x10}
  <> \_@\_ obj = new int[D];
\end{x10}

{}\noindent By default, objects are created {\tt here}. An object can
be created in thread local storage by using an {\tt @threadlocal}
suffix to the data type name:
\begin{x10}
   new Foo@threadlocal<here, here>();
\end{x10}


Note that the effect of creating an object at a place {\tt P} and
returning a reference to it may be obtained by:
\begin{x10}
    Foo@P x = future (P) { new Foo();}
\end{x10}

{}\Xten{} imposes the rule that the lifetime of places passed as place
parameters to objects must be no shorter than the lifetime of the
object itself. It also maintains the invariant that all place variables
in scope are guaranteed to last longer than this method. This implies
that given an invocation:
\begin{x10}
    new Foo@P<..., Q, ...>
\end{x10}
\noindent the following constraints must be true
\begin{itemize}
\item {\cf P} is {\cf threadlocal},  or
\item {\cf P} is {\cf here} (and not in an {\cf async}) and {\cf Q} is a parameter to this class, or {\cf Q} is {\cf shared} or {\cf P} is {\cf Q}
\end{itemize}

In a generic object constructor invocation, a type parameter is always
replaced with a type which includes the place. For instance

\begin{x10}
    \_ s = new Stack<Point@P>();
\end{x10}

Constructor calls, method calls and field accesses have the following place constraints.
\begin{x10}
  Foo@P foo = new Foo@P();
  int[]@P data;
  foo.m();
  int i = foo.f;
  foo.f = i;
  int i = data[j];
  data[j] = i;
\end{x10}
Here {\tt P} is scoped or {\tt here} or {\tt accessible}. {\tt f}
must be a final field.

\Xten{} also provides place inference for {\tt asyncs} and {\tt futures} (\S~\ref{AsyncActivity}):

\begin{x10}
  async \{{\cf\em{}Statement;}\} 
  async (\_) \{{\cf\em{}Statement;}\}
  future \{{\cf\em{}Statement;}\} 
  future (\_) \{{\cf\em{}Statement;}\} 
\end{x10}

In the {\tt async} ({\tt future}) statement, the target place of the
{\tt async} ({\tt future}) is the unique place that will satisfy the
place constraints of the body; if there is more than one such place,
an error is thrown at compile-time. We permit the second form so as to
allow the programmer to document that s/he intends the compiler to
infer the location.

Place variables outside an async are accessible inside the async.

\paragraph{Examples}
Consider a {\tt Stack} containing elements of some type {\tt T} which must
all be located at a given place {\tt P}:
\begin{x10}
 class Stack<T@P> \{
     ?T@P[] elements;
     int size;
     Stack() \{
         // filled with nulls.
         elements = new ?T@P[1000]; 
         size = 0;
     \}
     void push(T@P t) \{ elements[size++] = t; \}
     T@P pop() \{ return elements[--size]; \}
     void fill(T@P v) \{ 
         for (int i : elements) \{ 
            elements[i] = v; 
         \} 
     \}
 \}
\end{x10}

Consider the following array initializers (\S~\ref{ArrayInitializer}):
\begin{x10}
  distribution D = block(1000);
  // region 1..1000 treated as 1..1000 here
  \_ data = new int@current[1000](i){ return i*i; }; 
    // gives in P the place to which data[2] was 
    // mapped, i.e. the first place.
  <P> int@P q = data[2]; 
    // initialize the array with 10 times 
    // the index value
  float[D] d = new float[D] (i){ return 10.0*i; };
  float[D] d2 = new float[D] (i){ return i*i; };
  float[D] result = 
    new float[D] (i){ return di] + d2[i]; };
\end{x10}

The code fragments type-check because the compiler may make the
inference that {\tt here} inside the {\tt ateach} is {\tt D[i]}, and
that the place of the elements {\tt d[i]} and {\tt d2[i]} is also 
{\tt D[i]}. The \Xten{} compiler uses only syntactic equality and simple
intraprocedural dataflow identities to determine which places are the
same.  

The place associated with a particular distribution element may be
accessed using array syntax.
\begin{x10}
  place P = D[i];
  Object@P obj = async { new Object@P(); };
  int x = async (D[77]) { return data[77]; };
\end{x10}
Distributions can be passed as parameters much as places can.

\begin{x10}
 <P> void m(Object@P f) \{
    \_ stack = new Stack@local <Object@P>();
    fill(stack);
 \}
\end{x10}

In the following, the type specifier for the argument expands to
{\tt Stack<Object@P>@here}:

\begin{x10}
 <P> void fill(Stack<Object@P> stack) \{
    for (int i : 1000 ) \{
        stack.push( new Object@P(););
    \}
 \}

 <P> void unordered\_fill(Stack<Object@P> stack) \{
    ^Object@P[] futures = new ^Object@P[1000] 
          \{return new Object@P();\}; 
    for (^Object@P f : futures) \{
        stack.push(f);
    \}
 \}
\end{x10}

\paragraph{Examples with type inference}
\begin{x10}
 <P> void m() \{
    \_ stack = new Stack@threadlocal<Object, P>();
    fill(stack);
\}

 <P> void fill(Stack@threadlocal<Object@P> stack) \{
    for (int i = 0; i < 1000; i++) \{
        stack.push(async \{return new Object@P();\});
    \}
\}

<P> void unordered\_fill(Stack@local<Object@P> stack) \{
    \_ values = new \_[1000] \{return new Object@P();\};
    for (\_ f : values) \{
        stack.push(f);
    \}
\}

 // Use of anywhere types and newplace 
 // to create heterogenous collections
 void anywhere\_test() \{
    // create 1000 objects at 
    // 1000 different shared places
   \_ objs = new Object@?[1000]@local 
          (i)\{ async(?) \{ return new Object();\}\};
    for (\_ o : objs) \{
        <P> Object@P v = o;
        async \{v.blah(); \};
    \}
 \}
\end{x10}

}

\section{Variables}\label{XtenVariables}\index{variables}

A variable of a reference data type {\tt reference R} where {\tt R} is
the name of an interface (possibly with type arguments) always holds a
reference to an instance of a class implementing the interface {\tt R}.

A variable of a reference data type {\tt R} where {\tt R} is the name
of a reference class (possibly with type arguments) always holds a
reference to an instance of the class {\tt R} or a class that is a
subclass of {\tt R}. 

A variable of a reference array data type {\tt R [D]} is always an
array which has as many variables as the size of the region underlying
the distribution {\cf D}. These variables are distributed across
places as specified by {\cf D} and have the type {\tt R}.

A variable of a nullary (reference or value) data type {\tt nullable
T} always holds either the value (named by) {\tt null} or a value of
type {\tt T} (these cases are not mutually exclusive).

A variable of a value data type {\tt value R} where {\tt R} is the
name of an interface (possibly with type arguments) always holds
either a reference to an instance of a class implementing {\tt R} or
an instance of a class implementing {\tt R}. No program can
distinguish between the two cases.

A variable of a value data type {\tt R} where {\tt R} is the name of a
value class always holds a reference to an instance of {\tt R} (or a
class that is a subclass of {\tt R}) or an instance of {\tt R} (or a
class that is a subclass of {\tt R}). No program can distinguish
between the two cases.

A variable of a value array data type {\tt V value [R]} is always an
array which has as many variables as the size of the region {\tt R}.
Each of these variables is immutable and has the type {\tt V}.

\Xten{} supports seven kinds of variables: final {\em class
variables} (static variables), {\em instance variables} (the instance
fields of a class), {\em array components}, {\em method parameters},
{\em constructor parameters}, {\em exception-handler parameters} and
{\em local variables}.

\subsection{Final variables}\label{FinalVariable}\index{variable!final}\index{final variable}
A final variable satisfies two conditions: 
\begin{itemize}
\item it can be assigned to at most once, 
\item it must be assigned to before use. 
\end{itemize}

\Xten{} follows \java{} language rules in this respect \cite[\S
4.5.4,8.3.1.2,16]{jls2}. Briefly, the compiler must undertake a
specific analysis to statically guarantee the two properties above.

\todo{Check if this analysis needs to be revisited.}

\subsection{Initial values of variables}
\label{NullaryConstructor}\index{nullary constructor}
\cbstart 
Every variable declared at a type must always contain a value of that type.

Every class variable must be initialized before it is read, through
the execution of an explicit initializer or a static block. Every
instance variable must be initialized before it is read, through the
execution of an explicit initializer or a constructor.  An instance
variable declared at a nullable type (and not declared to be {\tt
final}) is assumed to have an initializer which sets the value to {\tt
null}.

Each method and constructor parameter is initialized to the
corresponding argument value provided by the invoker of the method. An
exception-handling parameter is initialized to the object thrown by
the exception. A local variable must be explicitly given a value by
initialization or assignment, in a way that the compiler can verify
using the rules for definite assignment \cite[\S~16]{jls2}.

\cbend

\section{Objects}\label{XtenObjects}\index{Object}

An object is an instance of a scalar class or an array type.  It is
created by using a class instance creation expression
(\S~\ref{ClassCreation}) or an array creation
(\S~\ref{ArrayInitializer}) expression, such as an array
initializer. An object that is an instance of a reference (value) type
is called a {\em reference} ({\em value}) {\em object}.

All value and reference classes subclass from {\tt x10.lang.Object}.
This class has one {\tt const} field {\tt location} of type {\tt
x10.lang.place}. \index{place.location} Thus all objects in \Xten{}
are located (have a place). However, \Xten{} permits value objects to
be freely copied from place to place because they contain no mutable
state.  It is permissible for a read of the {\tt location} field of
such a value to always return {\tt here} (\S~\ref{Here});
therefore no space needs to be allocated in the object representation
for such a field.

In \XtenCurrVer{} a reference object stays resident at the place at
which it was created for its entire lifetime.

{}\Xten{} has no operation to dispose of a reference.  Instead the
collection of all objects across all places is globally garbage
collected.

{}\Xten{} objects do not have any synchronization information (e.g.{}
a lock) associated with them. Thus the methods on {\tt
java.lang.Object} for waiting/synchronizing/notification etc are not
available in \Xten. Instead the programmer should use atomic blocks
(\S~\ref{AtomicBlocks}) for mutual exclusion and clocks
(\S~\ref{XtenClocks}) for sequencing multiple parallel operations.

A reference object may have many references, stored in fields of
objects or components of arrays. A change to an object made through
one reference is visible through another reference. \Xten{} mandates
that all accesses to mutable objects shared between multiple
activities must occur in an atomic section (\S\ref{AtomicBlocks}).

\cbstart 
Note that the creation of a remote async activity
(\S~\ref{AsyncActivity}) {\cf A} at {\cf P} may cause the automatic creation of
references to remote objects at {\cf P}. (A reference to a remote
object is called a {\em remote object reference}, to a local object a
{\em local object reference}.)  For instance {\cf A} may be created
with a reference to an object at {\cf P} held in a variable referenced
by the statement in {\cf A}.  Similarly the return of a value by a
{\cf future} may cause the automatic creation of a remote object
reference, incurring some communication cost.  An {}\Xten{}
implementation should try to ensure that the creation of a second or
subsequent reference to the same remote object at a given place does
not incur any (additional) communication cost.

\cbend 

A reference to an object may carry with it the values of final fields
of the object. The implementation should try to ensure that the cost
of communicating the values of final fields of an object from the
place where it is hosted to any other place is not incurred more than
once for each target place.

{}\Xten{} does not have an operation (such as Pascal's ``dereference''
operation) which returns an object given a reference to the
object. Rather, most operations on object references are transparently
performed on the bound object, as indicated below. The operations on
objects and object references include:
\begin{itemize}

{}\item Field access (\S~\ref{FieldAccess}). An activity holding a
reference to a reference object may perform this operation only if the
object is local.  An activity holding a reference to a value object
may perform this operation regardless of the location of the object
(since value objects can be copied freely from place to place).  The
implementation should try to ensure that the cost of copying the field
from the place where the object was created to the referencing place
will be incurred at most once per referencing place, according to the
rule for final fields discussed above.

\item Method invocation (\S~\ref{MethodInvocation}).  An activity
holding a reference to a reference object may perform this operation
only if the object is local.  An activity holding a reference to a
value object may perform this operation regardless of the location of
the object (since value objects can be copied freely). The \Xten{}
implementation must attempt to ensure that the cost of copying enough
relevant state of the value object to enable this method invocation to
succeed is incurred at most once for each value object per place.

{}\item Casting (\S~\ref{ClassCast}).  An activity can perform this
operation on local or remote objects, and should not incur
communication costs (to bring over type information) more than once
per place.

{}\item {\cf instanceof} operator (\S~\ref{instanceOf}).  An activity
can perform this operation on local or remote objects, and should not
incur communication costs (to bring over type information) more than
once per place.

\item The stable equality operator {\cf ==} and {\cf !=}
(\S~\ref{StableEquality}). An activity can perform these operations on
local or remote objects, and should not incur communication costs
(to bring over relevant information) more than once per place.

% \item The ternary conditional operator {\cf ?:}
\end{itemize}

\section{Built-in types}
\cbstart 

The package {\tt x10.lang} provides a number of built-in class and
interface declarations that can be used to construct types.

\subsection{The class {\tt Object}}\label{Object}\index{Object}
The class {\cf x10.lang.Object} is a superclass of all other classes.
A variable of this type can hold a reference to an instance of any
scalar or array type.

\begin{x10}
package x10.lang;
public class Object \{
  public String toString() \{...\}
  public boolean equals(Object o) \{...\}
  public int hashCode() \{...\}
\}
\end{x10}

The method {\tt equals} and {\tt hashCode} are useful in hashtables,
and are defined as in \java. The default implementation of {\tt equals}
is stable equality, \S~\ref{StableEquality}. This method may be overridden
in a (value or reference) subclass.

\subsection{The class {\tt String}}
\Xten{} supports strings as in \java{}. A string object is immutable,
and has a concatenation operator ({\tt +})  available on it.

\subsection{Arithmetic classes}
Several value types are provided that encapsulate
abstractions (such as fixed point and floating point arithmetic)
commonly implemented in hardware by modern computers:

\begin{x10}
boolean byte short char 
int long 
double float 
\end{x10}

\XtenCurrVer{} defines these data types in the same way as the 
\java{} language. Specifically, a program may contain literals
that stand for specific instances of these classes. The syntax
for literals is the same as for \java{} (\S~ref{Literals}).

\futureext{
\Xten{} may provide mechansims in the future to permit the programmer
to specify how a specific value class is to be mapped to special
hardware operations (e.g.{} along the lines of
\cite{kava}). Similarly, mechanisms may be provided to permit the user
to specify new syntax for literals.
}
\subsection{Places, distributions, regions, clocks}
\Xten{} defines several other classes in the {\cf x10.lang}
package. Please consult the API documentation for more details.

\subsection{Java utility classes}
\XtenCurrVer{} programmers may import and use \java{} packages such as
{\tt java.util}, e.g.{} {\tt java.util.Set}, {\tt
java.lang.System}. \Xten{} programs should not invoke methods
that use the {\tt wait/notify/notifyAll} methods on such objects,
since this may interfere with \Xten{} synchronization. The
implementation does not make imported \java{} classes
automatically extend {\tt x10.lang.Object}. 

\futureext{
The above represents an {\em ad hoc} integration of \java{} libraries
into \Xten{}. It has the unfortunate consequence that not every run-time
value in an \Xten{} program execution is an instance of a subclass of
{\tt x10.lang.Object}. }

In the future a more principled and robust scheme will be worked
out. Such a scheme will need to attend to the integration of the
\java{} and \Xten{} type systems, and develop a notion of place for 
\java{} objects.

\cbend
\section{Conversions and Promotions}\label{XtenConversions}\label{XtenPromotions}\index{conversions}\index{promotions}

{}\XtenCurrVer{} supports \java's conversions and promotions
(identity, widening, narrowing, value set, assignment, method
invocation, string, casting conversions and numeric promotions)
appropriately modified to support \Xten's built-in numeric classes
rather than \java's primitive numeric types.

This decision may be revisited in future version of the language in
favor of a streamlined proposal for allowing user-defined
specification of conversions and promotions for value types, as part
of the syntax for user-defined operators.
	\par % empty
%% Fri Dec 08 06:15:28 2006
Mon Jul 03 16:00:11 2006

\chapter{Dependent types}\label{XtenDepTypes}\index{dependent types}
\def\withmath#1{\relax\ifmmode#1\else{$#1$}\fi}
\def\LL#1{\withmath{\lbrack\!\lbrack #1\rbrack\!\rbrack}}

Dependent types are a fundamental extension to the type system for
Java-like languages. A dependent type records constraints on
{\em properties} (final fields) of the type.  Since types are a fundamental
building block of Java-like languages, the introduction of dependent
types affects many facets of the language simultaneously --- the
definition of types, classes, interfaces, methods, constructors,
fields, inheritance, overriding, overloading, and type related
operators (cast, and instanceof).

Indeed dependent types bring substantial expressiveness to Java-like
languages. Just as generic classes permit a single definition for a
class $C$ to be treated as a template for an unbounded number of classes
obtained from $C$ by ``instantiating'' $C$ with type arguments, so also a
dependent type permits a single definition to produce a potentially
unbounded family of types. That is, just as generic types permit a
programmer to express and use {\em functions} from types to types, so also
dependent types permit a programmer to express and use functions from values
to types.  Indeed, the family of types generated from $C$ form a
lattice of subtypes of $C$, one for each constraint on the properties
of $C$ expressible in the underlying constraint system, but all sharing
the same "structure".

In \Xten{} dependent types are checked statically. However, as in
Java-like languages an instanceof relation is available to dynamically
check that an object belongs to a particular (dependent) type. Also a
cast operation is available to force the runtime system to treat an
object o as belonging to a particular dependent type (the operation
throws a ClassCast exception if it is not possible to do so).

Dependent types are also the basis for an implicit syntax for
\Xten{}. This is discussed in the last section.

\section{Properties}\label{DepType:Properties}\index{properties}
The dependent type system is built on the notion of {\em properties}, 
for types (classes and interfaces).

\begin{x10}
NormalClassDeclaration ::= 
   ClassModifiersopt class identifier 
   PropertyListopt Superopt Interfacesopt ClassBody

NormalInterfaceDeclaration ::= 
   InterfaceModifiersopt interface identifier 
   PropertyListopt ExtendsInterfacesopt InterfaceBody

PropertyList     ::= ( Properties WhereClauseopt )
Properties       ::= Property
                 | Properties , Property
Property         ::= Type identifier    
PropertyListopt  ::= \$Empty | PropertyList
\end{x10}


A property has a name and a type. The declaration of a type (class or
interface) introduces a sequence of defined properties for the
type. 

\begin{quotation}
   {\sc Static Semantics Rule:} It is a compile-time error for a class
  defining a property {\cf P p} to have an ancestor class that defines a property
  with the name {\cf p}.  
\end{quotation}


Each class {\cf C} defining a property {\cf P p} implicitly has a field

\begin{x10}
public final P p;  
\end{x10}

\noindent and a getter method 

\begin{x10}
public final P p() { return p;}  
\end{x10}

\noindent Each interface {\cf I} defining a property {\cf P p} implicitly has a getter method

\begin{x10}
public final P p() { return p;}
\end{x10}

\begin{quotation}
  {\sc Static Semantics Rule:} It is a compile-time error for a class or
  interface defining a property {\cf P p} to have an existing method with
  the signature {\cf P p()}.   
\end{quotation}


Properties are used to build dependent types from classes, as
described below (\S~\ref{DepType:DepType}).

The {\tt WhereClause} in a {\tt PropertyList} specifies an explicit
condition on the properties of the type, and is discussed further
below (\S~\ref{DepType:Class}, \ref{DepType:Interface}).

\begin{quotation}
    {\sc Static Semantics Rule:}  Every constructor for a class defining
   properties {\cf P1 p1, \ldots, Pn pn} must ensure that each of the fields
   corresponding to the properties is definitely initialized (cf
   requirement on initialization of final fields in Java) before the
   constructor returns.  
\end{quotation}


\begin{example}
 A class representing immutable 2d points, with two properties {\tt i} and 
{\tt j}.
  \begin{x10}
   value class Point(int i, int j) { ... }
   value class point(int rank) { ... }
  \end{x10}
  
\end{example}

\section{Dependent types}\label{DepType:DepType}\index{dependent type}

A dependent type (deptype) is of the form {\cf C(: c)} where {\cf C} is a class
or interface, and {\cf c} is a {\em constraint}. ({\cf C} is said to be 
{\em the base type} of the deptype, and {\cf c} the {\em constraint} of the deptype.)  A
constraint is a boolean expression that can only use a predefined set
of operators and methods. 

\begin{x10}
Type  ::=   PrimitiveType
         | ClassOrInterfaceType
         | ArrayType
         | nullable < Type > DepParamtersopt
         | future < Type > DepParametersopt
PrimitiveType ::= NumericType DepParametersopt
         | boolean DepParametersopt
ClassOrInterfaceType   ::= 
  TypeName DepParametersopt PlaceTypeSpecifieropt
ClassType              ::= 
  TypeName DepParametersopt PlaceTypeSpecifieropt
InterfaceType          ::= 
  TypeName DepParametersopt PlaceTypeSpecifieropt
PlaceTypeSpecifier     ::=  ! PlaceTypeopt
PlaceType              ::= current | Expression

DepParameters    ::= ( DepParameterExpr ) 
DepParameterExpr ::= ArgumentList WhereClauseopt
WhereClause      ::= : Constraint
Constraint       ::= Expression
ArgumentList     ::= Expression 
   | ArgumentList , Expression
DepParametersopt ::= null | DepParameters
WhereClauseopt   ::= null | WhereClause
PlaceTypeopt     ::= null | PlaceType
\end{x10}

\begin{quotation}
{\sc Static Semantics: Variable Occurrence}
  In a deptype T=C(:c), the only variables that may occur in c are (a)
  self, (b) properties visible at T, (c) final local variables, final
  method parameters or final constructor parameters visible at T, (d)
  final fields visible at T's lexical place in the source program.  
\end{quotation}

\begin{quotation}
{\sc Static Semantics: This restrictions}

  The special variable {\cf this} may be used in a depclause for a type {\cf T}
  only if (a)~it occurs in a property declaration for a class, (b)~it
  occurs in an instance method, (c)~it occurs in an instance field, (d)~it
  occurs in an instance initializer.
\end{quotation}

In particular it may not be used in types that occur in a static
context, or in the arguments, body or return type of a constructor or
in the extends or implements clauses of class and interface
definitions.  In these contexts the object that {\cf this} would
correspond to has either not been formed or is not well defined.

\begin{quotation}
{\sc Static Semantics: Variable visibility}
  If a type {\cf T} occurs in a field, method or constructor
  declaration, then all variables used in {\cf T} must have at least the
  same visibility as the declaration.  The relation ``at least the same
  visibility as'' is given by the transitive closure of:

  \begin{x10}
public > protected, protected > private
public > package, package > private
  \end{x10}

All inherited properties of a type {\cf T} are visible in the property
list of {\cf T}, and the body of {\cf T}.

\end{quotation}

In general local variables/parameters/properties/fields are visible at
{\cf T} if they are defined before {\cf T} in the program. This rule applies to
types in property lists as well as parameter lists (for methods and
constructors).  An exception is made for the return type of a method:
all the arguments to the method are considered to be visible, even
though they occur lexically after the return type (given the \Java{}
syntactic convention that the return type for a method precedes the
argument list for the method).

We permit variable declarations {\cf T v} where {\cf T} is obtained
from a dependent type {\cf C(:c)} by replacing one or more occurrences
of {\cf self} in {\cf c} by {\cf v}. (If such a declaration {\cf T v}
is type-correct, it must be the case that the variable {\cf v} is not
visible at the type {\cf T}. Hence we can always recover the
underlying deptype {\cf C(:c)} by replacing all occurrences of {\cf v}
in the constraint of {\cf T} by {\cf self}.)

For instance, {\cf  int(: v > 0) v} is shorthand for {\cf int(: self > 0) v}.
\begin{quotation}
{\sc Static Semantics: Constraint type}
  The type of c must be boolean.  
\end{quotation}

A variable occuring in the constraint {\cf c} of a deptype, other than
{\cf self} or a property of {\cf self}, is said to be a {\em
parameter} of{\cf c}.\label{DepType:Parameter} \index{parameter}

An instance {\cf o} of {\cf C} is said to be of type {\cf C (:c)} (or: {\em belong to}
{\cf C(:c)}) if the predicate {\cf c} evaluates to {\cf true} in the current lexical
environment, augmented with the binding {\cf [self |-> o]}. We shall
use the function \LL{{\cf C(:c)}} to denote the set of objects that belong
to {\cf C(:c)}. 

\section{Type definition}\label{DepType:Class}
A class definition 
\begin{x10}
ClassModifiersopt class identifier 
    PropertyListopt Superopt Interfacesopt ClassBody  
\end{x10}

\noindent and an interface definition
\begin{x10}
InterfaceModifiersopt interface identifier 
   PropertyListopt ExtendsInterfacesopt InterfaceBody  
\end{x10}

\noindent may reference several deptypes. The types of properties, the
specification of the super clause and the specification of interfaces
may each involve deptypes.

\begin{x10}
Super ::= extends ClassType
Superopt ::= null | Super
ClassType ::= 
  TypeName DepParametersopt PlaceTypeSpecifieropt

Interfaces ::= implements InterfaceTypeList
InterfaceTypeList ::= InterfaceType
       | InterfaceTypeList , InterfaceType
Interfacesopt ::= null | Interfaces
InterfaceType ::= 
 TypeName DepParametersopt PlaceTypeSpecifieropt
\end{x10}

\section{Where clauses}\label{DepType:WhereClauses}\index{where clauses}

There is a general recipe for constructing a list of parameters or
properties {\cf T1(:c1) x1, ... , Tk(:ck) xk} that must satisfy a given
(satisfiable) constraint {\cf c}. 

\begin{x10}
class Foo (T1 (: (T2 x2; ...; Tk xk;  c) x1, 
       T2 (: (T3 x3; ...; Tk xk;  c) x2, 
        ...
       Tk (:  c) xk) { 
 ...
}
\end{x10}

The first type {\cf T1 (:T2 x2;...;Tk xk; c) x1} is consistent iff
{\cf (exists T1 x1, T2 x2,..., Tk xk) c} is consistent. The second is
consistent iff
\begin{x10}
forall T1(: exists (T2 x2,..., Tk xk) c) x1
exists T2 x2. (exists T3 x2,..., Tk xk) c
\end{x10}
\noindent But this is always true. Similarly for the conditions for the other
properties.

Thus logically every satisfiable constraint {\cf c} on a list of parameters
{\cf x1,..., xk} can be expressed using the dependent types of xi, provided
that the constraint language is rich enough to permit existential
quantifiers.

Nevertheless we will find it convenient to permit the programmer to
explicitly specify a depclause after the list of properties, thus:
\begin{x10}
class Point(int i, int j) \{ ... \}
class Line(Point start, Point end :  end != start) 
  \{ ... \}
class Triangle (Line a, Line b, Line c 
       : a.end == b.start \&\& b.end == c.start \&\&
         c.end == a.start) \{ ... \}
class SolvableQuad(int a, int b, int c 
                   : a*x*x+b*x+c==0)  \{ ... \}
class Circle (int r, int x, int y 
              : r > 0 \&\& r*r==x*x+y*y)\{ ... \}
class NonEmptyList extends List(: n > 0) \{...\}
\end{x10}

Consider the definition of the class {\cf Line}. This may be thought of as
saying: the class {\cf Line} has two fields, {\cf Point start} and {\cf Point
end}. Further, every instance of Line must satisfy the constraint that
{\cf end !=start}. Similarly for the other class definitions. 

In the general case, the production for {\cf NormalClassDeclaration}
specifies that the list of properties may be followed by a {\cf
WhereClause}:

\begin{x10}
NormalClassDeclaration ::= 
    ClassModifiersopt class identifier 
    PropertyListopt Superopt Interfacesopt ClassBody

NormalInterfaceDeclaration ::= 
   InterfaceModifiersopt interface identifier 
   PropertyListopt ExtendsInterfacesopt InterfaceBody

PropertyList     ::= ( Properties WhereClauseopt )
\end{x10}

All the properties in the list, together with inherited properties,
may appear in the {\cf WhereClause}. A property list {\cf T1 x1, ...., Tn xn : c}
for a class {\cf Foo} is said to be consistent if each of the {\cf Ti} are
consistent and the constraint
\begin{x10}
      exists  T1 x1, ..., Tn xn, Foo self . c
\end{x10}
\noindent is valid (always true).

\section{Type invariants}\label{DepType:TypeInvariant}\index{Type Invariant}

With every defined class or interface {\cf T} we associate a {\em type
invariant} {\tt i(T)} as follows. The type invariant associated with
{\cf x10.lang.Object} is the proposition

\begin{x10}
nullable<place> self.loc  
\end{x10}

The type invariant associated with any interface {\cf I} that extends
interfaces {\cf I1,..., Ik} and defines properties {\cf P1 x1, ..., Pn xn} and
specifies a where clause {\cf c} is given by:

\begin{x10}
  ti(I1) \&\& ... \&\& tk(Ik) \&\& P1 self.x1 
  \&\& ... \&\& Pn self.xn \&\& c  
\end{x10}

Similarly the type invariant associated with any class {\cf C} that
implements interfaces {\cf I1,..., Ik}, extends class {\cf D} and defines
properties {\cf P1 x1,..., Pn xn} and specifies a where clause {\cf c} is given
by:
\begin{x10}
  i(D) \&\& P1 self.x1 \&\& ... \&\& Pn self.xn \&\& c  
\end{x10}

The {\sc Int Implements} Static Semantic rule below requires that the
type invariant associated with a class entail the type invariants of
each interface that it implements.

The static semantics rules below guarantee that for any variable {\cf v} of
type {\cf T(:c)} (where {\cf T} is an interface name or a class name) the only
objects {\cf o} that may be stored in {\cf v} are such that {\cf o} satisfies
{\cf i(T)[o/this] \&\& c[o/self]}.

\section{Consistency of deptypes}\label{DepType:Consistency}\index{deptype,consistency}

A dependent type {\cf C(:c)} may contain zero or more parameters. We require
that a type never be empty -- so that it is possible for a variable of
the type to contain a value. This is accomplished by requiring that
the constraint c must be satisfiable {\em regardless} of the value assumed
by parameters to the contraint (if any). Formally, consider a type
{\cf T=C(: c)}, with the variables {\cf F1 f1, ..., Fk fk} free in {\cf c}.  Let 
{\cf S={F1 f1, .., Fk fk, Fk+1 fk+1, ... Fn fn}} be the smallest set of
declarations containing {\cf F1 f1, ..., Fk fk} and closed under the rule: {\cf F
f} in {\cf S} if a reference to variable {\cf f} (which is declared as {\cf F f}) occurs
in a type in {\cf S}.

(NOTE: The syntax rules for the language ensure that {\cf S} is always
finite. The type for a variable {\cf v} cannot reference a variable whose
type depends on {\cf v}.)

We say that {\cf T=C(:c)} is {\em parametrically consistent} (in brief:
{\em consistent}) if

\begin{itemize}
  \item Each type {\cf F1, ..., Fn} is (recursively) parametrically consistent, and
\item It can be established that {\cf forall F1 f1, .., Fn fn. exists C self. c \&\& i(C)}.
\end{itemize}
\noindent where {\tt i(C)} is the invariant associated with the type {\cf C}
(\S~\ref{DepType:TypeInvariant}).  Note by definition of {\cf S} the formula on the
above has no free variables.

\begin{quotation}
   {\sc Static Semantics Rule:}
    For a declaration {\cf T v} to be type-correct, {\cf T} must be parametrically
    consistent. The compiler issues an error if it cannot determine
    the type is parametrically consistent.
\end{quotation}

\begin{example}

A class that represents a line has two distinct points:
\begin{x10}
class Array(int  rank, 
    region(:rank==this.rank) region) \{...\}  
\end{x10}
\end{example}

One can use deptypes to define other closed geometric figures as well.

\begin{example}
Here is an example:
\begin{x10}
 class Point(int x, int y) \{...\}
 class Line( Point start, 
        Point(: self != this.start) end) 
\{...\}      
\end{x10}
\end{example}


To see that the declaration {\cf Point(: self != start) end} is
parametrically consistent, note that the following formula is valid:
\begin{x10}
forall Line this. 
  exists Point self. self != this.start  
\end{x10}
\noindent since the set of all {\cf Points} has more than one element.

\begin{example}
A triangle has three lines sharing three vertices.
\begin{x10}
class Triangle 
 (Line a, 
  Line(: a.end == b.start) b, 
  Line(: b.end == c.start \&\& c.end == a.start) c) 
 \{ ...\}
\end{x10}
\end{example}


Given {\cf Line a}, the type {\cf Line(: a.end == b.start) b} is consistent, and
given the two, the type {\cf Line(: b.end == c.start \&\& c.end == a.start) c}
is consistent.

%%Similarly:
%%
%%   // A class with properties a, b,c,x satisfying the 
%%   // given constraints.
%%   class SolvableQuad(int a, int b, 
%%                      int(: b*b - 4*a*c >= 0) c, 
%%                      int(: a*x*x + b*x + c==0) x) { 
%%     ...
%%   }
%%
%%  // A class with properties r, x, and y satisfying
%%  // the conditions for (x,y) to lie on a circle with center (0,0)
%%  // and radius r.
%%   class Circle (int(: r> 0) r, 
%%                 int(: r*r - x*x >= 0) x,
%%                 int(: y*y == r*r -x*x) y) { 
%%   ...
%%   }

\section{Equivalence of deptypes}\label{DepType:Equivalence}\label{deptype,equivalence}

Two dependent types {\cf C(:c)} and {\cf C(:d)} are said to be {\em equivalent} if 
{\cf c} is true whenever {\cf d} is, and vice versa. Thus, 
$\LL{C(:c)} = \LL{C(:d)}$.

Note that two deptypes that are syntactically different may be
equivalent. For instance, {\cf int(:self >= 0)} and {\cf int(:self ==
0 || self > 0)} are equivalent though they are syntactically
distinct. The \Java{} type system is essentially a nominal system -- two
types are the same if and only if they have the same name. The \Xten{}
type system extends the nominal type system of \Java{} to permit
constraint-based equivalence.

A dependent type {\cf C(:c)} is said to refine a type {\cf C(:d)} if
{\cf c} implies {\cf d}.  In such a case we have $\LL{C(:c)}$ is a
subset of $\LL{C(:d)}$. All dependent types defined on {\cf C} refine
{\cf C} since {\cf C} is equivalent to {\cf C(:true)}.

\section{Type checking rules}
\subsection{Class definitions}

Consider a class definition
\begin{x10}
ClassModifiersopt 
 class C (P1 x1,..., Pn xn)  extends D(:d) 
   implements I1(:c1),..., Ik(:ck)
 ClassBody  
\end{x10}

Each of the following static semantics rules must be satisfied:

\begin{quotation}
{\sc Static Semantics: Int-implements}
    The type invariant {\cf i(C)} of {\cf C} must entail {\cf ci[this/self]} for each 
  {\cf i} in {\cf 1:k}.  

{\sc Static Semantics: Super-extends}
    The return type {\cf c} of each constructor in {\cf ClassBody} must entail {\cf d}.
\end{quotation}

\subsection{Constructor definitions}

A constructor for a class {\cf C} is guaranteed to return an object of the
class on successful termination. This object must satisfy i(C), the
class invariant associated with {\cf C} (\S~\ref{DepType:TypeInvariant}). However,
often the objects returned by a constructor may satisfy {\em stronger}
properties than the class invariant. \Xten{}'s dependent type system
permits these extra properties to be asserted with the constructor in
the form of a deptype (the ``return type'' of the constructor):

\begin{x10}
ConstructorDeclarator ::=  
  SimpleTypeName DepParametersopt 
 ( FormalParameterListopt WhereClauseopt )
\end{x10}

As with method declarations, the parameter list for the constructor
may specify a where clause that is to be satisfied by the parameters
to the list.

\begin{example}
Here is another example.
\begin{x10}
public class List(int(:n >=0) n) \{
 protected nullable<Object>   value;
 protected nullable<List(n-1)>  tail;
 public List(t.n+1)(Object o, final List t) \{
     n=t.n+1;
     tail=t;
     value=o;
 \}
 public List(0) () \{
     n=0;
     value=null;
     tail=null;
 \}
 ...
\}
\end{x10}
The second constructor returns a {\cf List} that is guaranteed to have {\cf n==0};
the first constructor is guaranteed to return a List with {\cf n>0}
(in fact, {\cf n==t.n+1}, where the argument to the constructor is {\cf t}). 
This is recorded by the programmer in the deptype associated with the
constructor.
\end{example}

\begin{quotation}
{\sc Static Semantics:  Super-invoke}
   Let {\cf C} be a class with properties {\cf P1 p1, ..., Pn pn}, invariant {\cf c}
   extending the deptype {\cf D(:d)} (where {\cf D} is the name of a class).

   For every constructor in {\cf C} the compiler checks that the call to
   super invokes a constructor for D whose return type is strong enough
   to entail d. Specifically, if the call to super is of the form 
     {\cf      super(e1, ..., ek)}
   and the static type of each expression ei is Si, and the invocation
   is statically resolved to a constructor
{\cf       D(:d1) (T1 x1,..., Tk xk : c)}
   then it must be the case that 
   \begin{x10}
S1 x1, ..., Si xi |- Ti xi  (for i in 1:k)
S1 x1, ..., Sk xk |- c  
d1[a/self] \&\& S1 x1 ... \&\& Sk xk |- d[a/self]      
   \end{x10}
\noindent   where {\cf a} is a constant that does not appear in 
{\cf S1 x1 \&\& ... \&\& Sk xk}.
  
\end{quotation}

\begin{quotation}
{\cf Static Semantics: Constructor return}
   The compiler checks that every constructor for {\cf C} ensures that
   the properties {\cf p1,..., pn} are initialized with values which satisfy
   {\cf t(C)}, and its own return type {\cf c'} as follows.  In each constructor, the
   compiler checks that the static types {\cf Ti} of the expressions {\cf ei}
   assigned to pi are such that the following is true
   \begin{x10}
    T1 p1, ...., Tn pn |- t(C) \&\& c'     
   \end{x10}
\end{quotation}
(Note that for the assignment of ei to pi to be type-correct it must be the
    case that Ti pi |- Pi pi.) 


\begin{quotation}
{\sc Static Semantics: constructor invocation}
The compiler must check that every invocation {\cf C(e1,..., en)} to a
constructor is type correct: each argument {\cf ei} must have a static type
that is a subtype of the declared type {\cf Ti} for the {\cf i}th argument of the
constructor, and the conjunction of static types of the argument must
entail the {\cf WhereClause} in the parameter-list of the constructor.
\end{quotation}

\section{Field definitions}

Not every instance of a class needs to have every field defined on the
class. In Java-like languages this is ensured by conditionally setting
fields to a default value, such as {\cf null}, in those instances where the
fields are not needed.  

Consider the class {\cf List} used earlier.  Here all instances of {\cf List}
returned by the second constructor do not need the fields {\cf value} and
{\cf tail}; their value is set to null.

\Xten{} permits a much cleaner solution that does not require default
values such as null to be stored in such fields. \Xten{} permits fields to
be {\em guarded}, that is defined only if a certain constraint on the
properties of the class, called the {\cf guard} of that field, is true.

\begin{x10}
FieldDeclaration  ::= 
   FieldModifiersopt ThisClauseopt 
   Type VariableDeclarators ;
ThisClause       ::= this DepParameters
ThisClauseopt    ::= null | ThisClause
\end{x10}

It is illegal for code to access a guarded field through a reference
whose static type does not satisfy the associated guard, even
implicitly (i.e.{} through an implicit {\cf this}). Rather the source
program should contain an explict cast, e.g.{} {\cf C(:c) me = (C(:c)) this}.

\begin{quotation}
{\sc Static Semantics Rule:} Let {\cf f} be a field defined in class
{\cf C} with guard {\cf this(:c)}.  The compiler declares an error if
field {\cf f} is accessed through a reference {\cf o} whose static
type is not a subtype of {\cf C(:c)}.
\end{quotation}

\begin{example}

We may now rewrite the List example:
\begin{x10}
public class List(int(:n >=0) n) \{
  protected this(:n>0) Object  value;
  protected this(:n>0) List(n-1)  tail;
  public List(t.n+1)(Object o, final List t)\{
     n=t.n+1;
     List(:n>0) me = (List(:n>0)) this;
     me.tail=t;
     me.value=o;
  \}
  public List(0) () \{
     n=0;
  \}
  ...
\}
\end{x10}

The fields {\cf value} and {\cf tail} do not exist for instances of the class
{\cf List(0)}.
\end{example}

It is a compile-time error for a class to have two fields of the same
name, even if their {\cf ThisClauses} are different. A class {\cf C} with a field
named {\cf f} is said to {\em hide} a field in a superclass named {\cf f}.

\begin{quotation}
 {\sc Static Semantics Rule:}
     A class may not declare two fields with the same name.
\end{quotation}

\subsection{Field hiding}

The definition of field hiding does not take {\cf ThisClauses} in
account. Suppose a class {\cf C} has a field

\begin{x10}
 this(:c) Foo f;  
\end{x10}
\noindent and a subclass {\cf D} of {\cf C} has a field
\begin{x10}
 this(:d) Fum f;  
\end{x10}

We will say that {\cf D.f} hides {\cf C.f}, {\em regardless} of the
constraints {\cf c} and {\cf d}. This is in keeping with \Java, and
permits a naive implementation which always allocates space for a
conditional field.

{\sc DESIGN RATIONALE} It might seem attractive to require that {\cf
D.f} hides {\cf C.f} only if {\cf d} entails {\cf c}. This would seem
to necessitate a rather complex implementation structure for classes,
requiring some kind of a heterogenous translation for deptypes of {\cf C}
and {\cf D}. This bears further investigation.

\section{Method definitions}

\Xten{} permits guarded method definitions, similar to guarded
field definitions. Additionally, the parameter list for a method may
contain a WhereClause.

\begin{x10}
MethodHeader ::= 
  MethodModifiersopt ResultType 
  MethodDeclarator Throwsopt
MethodDeclarator ::= 
  ThisClauseopt identifier 
  ( FormalParameterListopt WhereClauseopt )
 | MethodDeclarator [ ]

ResultType ::= Type | void
\end{x10}

The guard (specified by {\cf ThisClause}) speciifes a constraint {\cf c} on the
properties of the class {\cf C} on which the method is being defined. The
method exists only for those instances of {\cf C} which satisfy {\cf c}.  It is
illegal for code to invoke the method on objects whose static type is
not a subtype of {\cf C(:c)}.

We relax the rules of lexical visibility and finality for variable
references in deptypes for method parameters.  Method
parameters not necessarily declared to be final are permitted to occur
in the types of parameters that occur after them in textual
order. Method parameters may also occur in the ReturnType for the
method, as long as they are declared final. (Even though the ReturnType
occurs lexically before the parameter list, it is considered to lie in
the scope of the declarations in the parameter list.)

\begin{quotation}
 {\sc  Static semantics Rule: }
    The compiler checks that every method invocation {\cf o.m(e1,..., en)}
    for a method is type correct. Each each argument ei must have a
    static type Si that is a subtype of the declared type Ti for the ith
    argument of the method, and the conjunction of static types
    of the arguments must entail the WhereClause in the parameter-list
    of the method.

    The compiler checks that in every method invocation {\cf o.m(e1,...,
    en)} the static type of o, S, is a subtype of C(:c), where the method
    is defined in class C and the ThisClause for m is equivalent to
    {\cf this(:c)}.

    Finally, if the declared return type of the method is D(:d), the
    return type computed for the call is {\cf D(: final S a; S1 x1; ...; Sn
    xn; d[a/this])}, where a is a new variable that does not occur in
    {\cf d, S, S1, ... , Sn}, and {\cf x1,...,xn} are the formal parameters of the
    method.
\end{quotation}

\begin{example}
Consider the program:
  \begin{x10}
public class List(int(:n >=0) n) \{
  protected this(:n > 0) Object  value;
  protected this(:n > 0) List(n-1)  tail;
  public List(t.n+1)(Object o, List t) \{
      n=t.n+1;
      tail = t;
      value = o;
  \}
  public List(:self.n==0) () \{
      n=0;
  \}
  public List(:self.n==this.n+l.n) append(List l) \{
      return (n==0)? l 
         : new List( value, tail.append(l)); 
  \}
  public this(:n>0) 
    Object nth(final int(:k >= 1 \&\& k <= n) k) \{
      return k==1 ? value : tail.nth(k-1);
  \}
\}
  \end{x10}

The following code fragment
\begin{x10}
List(:self.n==3) u = ...
List(:self.n==x) t = ...;
List(:self.n==x+3) s = t.append(u);
\end{x10}
\noindent will typecheck. The type of the expression {\cf t.append(u)} is 
\begin{x10}
List(: final List(:self.n==x) a; 
       List(:self.n==3) l; self.n== a.n+l.n)  
\end{x10}
\noindent and this simplifies to
\begin{x10}
List(: final List(:self.n==x) a; 
       List(:self.n==3) l; self.n== x+3)  
\end{x10}
\noindent which, after dropping unused local variables, reduces to:
\begin{x10}
List(: self.n== x+3)  
\end{x10}
\end{example}

\subsection{Method overloading, overriding, hiding, shadowing and obscuring}

The definitions of method overloading, overriding, hiding, shadowing
and obscuring in \Xten{} are the same as in \Java, modulo the following
considerations motivated by dependent types.

The definition of a method declaration {\cf m1} ``having the same signature
as'' a method declaration {\cf m2} involves identity of types. Two \Xten{} types
are defined to be identical iff they are equivalent (\S~\ref{DepType:Equivalence}).
Two methods are said to have {\em the same signature} if (a)
they have the same number of formal parameters, (b) for each parameter
their types are equivalent, and (c) the constraints associated with
their ThisTypes are equivalent. It is a compile-time error for there
to be two methods with the same name and same signature in a class
(either defined in that class or in a superclass).

\begin{quotation}
   {\sc Static Semantics Rule:}
  A class {\cf C} may not have two declarations for a method named {\cf m} -- either
  defined at {\cf C} or inherited --
\begin{x10}
T this(:tc) m(T1(:t1) v1,..., Tn(:tn) vn) \{...\}
S this(:sc) m(S1(:s1) v1,..., Sn(:sn) vn) \{...\}
\end{x10}
\noindent   if it is the case that the types {\cf this(:tc), T1(:t1), ...., Tn(:tn)} are
  equivalent to the types {\cf this(:sc), S1(:t1), ...., Tn(:tn)}
  respectively.
\end{quotation}

A class {\cf C} inherits from its direct superclass and superinterfaces all
their methods visible according to the access restriction modifiers
public/private/protected/(package) of the superclass/superinterfaces
that are not hidden or overridden. A method {\cf M1} in a class {\cf C} overrides
a method {\cf M2} in a superclass {\cf D} if {\cf M1} and {\cf M2} have the same signature.
Methods are overriden on a signature-by-signature basis.

A method invocation {\cf o.m(e1,..., en)} is said to have the {\em static
signature} {\cf <T, T1,...,Tn>} where {\cf T} is the static type of {\cf o}, and
{\cf T1,..., Tn} are the static types of {\cf e1,..., en} respectively.  As in
\Java, it must be the case that the compiler can determine a single
method defined on {\cf T} with argument type {\cf T1,..., Tn}, otherwise a
compile-time error is declared. However, unlike \Java, the \Xten{} type {\cf T}
may be a dependent type {\cf C(:c)}. Therefore, given a class definition for
{\cf C} we must determine which methods of {\cf C} are available at a type
{\cf C(:c)}. But the answer to this question is clear: exactly those methods
defined on {\cf C} are available at the type {\cf C(:c)} whose guard {\cf d} is implied
by {\cf c}.


\begin{example}
  Consider the definitions:
  \begin{x10}
class Point(int i, int j) \{...\}
class Line(Point s, Point(:self != i) e) \{
//m1: Both points lie in the right half of the plane
  this(:s.i>= 0 \&\& e.i >= 0) void draw() \{...\}
// m2 -- Both points lie on the y-axis
  this(:s.i== 0 \&\& e.i == 0) void draw() \{...\}
// m3 -- Both points lie in the top half of the plane
  this(:s.j>= 0 \&\& e.j >= 0) void draw() \{...\}
  // m4  -- The general method
  void draw() \{...\}
\} 
  \end{x10} 
\noindent  Three different implementations are given for the draw method, one
  for the case in which the line lies in the right half of the plane,
  one for the case that the line lies on the y-axis and the third for
  the case that the line lies in the top half of the plane.


\noindent  Consider the invocation
Line(:s.i < 0) m = ...
m.draw();

\noindent  This generates a compile time error because there is no applicable
  method definition.

\noindent  Consider the invocation

\begin{x10}
Line(:s.i>=0 \&\& s.j>=0 \&\& e.i>=0 \&\& e.j>=0) 
  m = ...
m.draw();
\end{x10}

\noindent  This generates a compile time error because both m1 and m3 are applicable.

\noindent  Consider the invocation
\begin{x10}
Line(:s.i>=0 \&\& s.j>=0 \&\& e.i>=0) m = ...
m.draw();
\end{x10}
  This does not generate any compile-time error since only m1 is
  applicable. 
\end{example}


In the last example, notice that at runtime {\cf m1} will be invoked
(assuming {\cf m} contains an instance of the {\cf Line} class, and not some
subclass of {\cf Line} that overrides this method). This will be the case
even if {\cf m} satisfies at runtime the stronger conditions for {\cf m2} (i.e.,
{\cf s.i==0 \&\& e.i==0}). That is, dynamic method lookup will not take into
account the  ``strongest'' constraint that the receiver may satisfy, i.e.{}
its ``strongest deptype''. 

{\em {\sc DESIGN RATIONALE.}
  The design decision that dynamic method lookup should ignore
  dependent type information was made to keep the design and the
  implementation simple and to ensure that serious errors such as
  method invocation errors are captured at compile-time.
 
  Consider the above example and the invocation
  \begin{x10}
   Line m = ...
   m.draw();    
  \end{x10}


   Statically the compiler will not report an error because m4 is the
   only method that is applicable. However, if dynamic method lookup
   were to use deptypes then we would face the problem that if m is a
   line that lives in the upper right quadrant then *both* m2 and m3
   are applicable and one does not override the other. Hence we must
   report an error dynamically.

   As discussed above, the programmer can write code with {\cf instanceof}
   and classcasts that perform any application-appropriate
   discrimination.  
}

\section{Interfaces with properties}\label{DepType:Interface}

\Xten{} permits interfaces to have properties and specify an interface
invariant. This is necessary so that programmers can build dependent
types on top of interfaces and not just classes.

\begin{x10}
NormalInterfaceDeclaration ::= 
     InterfaceModifiersopt interface identifier 
        PropertyListopt ExtendsInterfacesopt InterfaceBody
PropertyList     ::= ( Properties WhereClauseopt )
Properties       ::= Property
                    | Properties , Property
Property         ::= Type identifier    
PropertyListopt  ::= null | PropertyList
\end{x10}
The invariant associated with an interface is the conjunction of the
invariants associated with its superinterfaces and the invariant
defined at the interface. 

\begin{quotation}
   {\sc Static Semantics Rule:} The compiler declares an error if this constraint
   is not consistent (\S~\ref{DepType:Consistency}).  
\end{quotation}

Each interface implicitly defines a nullary getter method {\cf T p()} for
each property {\cf T p}. 

\begin{quotation}
   {\sc Static Semantics Rule:} The compiler issues a warning if the programmer
   explicitly defines a method with this signature for an interface.
  
\end{quotation}

A class {\cf C} (with properties) is said to implement an interface {\cf I} if
\begin{itemize}
  \item its properties contains all the properties of I,
\item its class invariant, i(C), implies i(I)
\end{itemize}


\subsection{instanceof} 

\Xten{} permits Types to be used in an in instanceof expression to
determine whether an object is an instance of the given type:

\begin{x10}
RelationalExpression ::= 
  RelationalExpression instanceof Type  
\end{x10}

In the above expression, {\cf Type} is any type including deptypes and
``primitive'' types. The expression {\cf e instanceof T} evaluates to true
if and only if the evaluation of {\cf e} results in a value {\cf v} which belongs
to the type {\cf T}. This determination may involve checking that the
constraint associated with the type is true for the value {\cf v}.

\subsection{Class casts}

\Xten{} permits types to be used in a cast expression:

\begin{x10}
CastExpression ::= 
  ( Type ) UnaryExpressionNotPlusMinus  
\end{x10}

In the above expression, Type is any type including deptypes and
``primitive'' types. The expression {\cf ((T) e)} evaluates {\cf e} to
a value {\cf v}, and results in {\cf v} if {\cf v} is an instance of
the type {\cf T}. Otherwise a {\cf ClassCastException} is thrown. The
static type of the expression {\cf (T) e} is {\cf T}.

   \begin{quotation}
 {\sc  Static Semantics Rule:}
    The compiler checks that the static type of e is either a subtype
    of T (this situation is sometimes called a "stupid cast"), or a
    supertype of T. If neither is the case, it throws a compile-time
    error since the cast must necessarily fail at runtime.     
   \end{quotation}

\subsection{Local variables}

Dependent types may be used to specify the types of local variables,
including loop variables and parameters for catch clauses.

\section{Array types}

\begin{x10}
    ArrayType ::= Type [  ] 
     | Type value  [  ]
     | Type [ DepParameterExpr ]
     | Type  value [ DepParameterExpr ]  
\end{x10}

\Xten{} has the following built in types:

\begin{x10}
class point(int(:rank>=0) rank) { ...}
class region
  (int(:rank>=0) rank,
   // region is a product of rank-1 convex regions.
   boolean rect,  
   // on each dim, the low bound is 0. 
   boolean lowZero
  ) { ...}
class dist
  (int (:rank>=0) rank, 
   boolean rect,
   boolean lowZero,
   region(: self.rank==rank\&\&self.rect==rect
           \&\&self.lowZero==lowZero) region, 
   place onePlace 
   )  { ...}
class Array<T>
  (int (:rank>=0) rank,  
   boolean rect,
   boolean lowZero,
   place onePlace,
   region region,
   dist(:self.rank==rank \&\&self.rect=rect
         \&\&self.lowZero==lowZero
        \&\&self.region==region
        \&\&self.onePlace==onePlace) dist
    ) \{...\}
class ValueArray<T>
 (int (:rank>=0) rank,  
   boolean rect,
   boolean lowZero,
   place onePlace,
   region region,
   dist(:self.rank==rank \&\&self.rect=rect
        \&\&self.lowZero==lowZero
       \&\&self.region==region
       \&\&self.onePlace==onePlace) dist
) \{...\}
\end{x10}


The array types on the left are shorthand for the deptypes on the right:
\begin{x10}
Type []  => Array<Type>
Type value [] => ValueArray<Type>
Type  [ DepParameterExpr ] 
   => Array<Type>( DepParameterExpr )
Type Value [ DepParameterExpr ] 
   => ValueArray<Type>( DepParameterExpr )
\end{x10}

For {\cf R} is a reference type, the type {\cf R(:c)!current[]} is interpreted as the
type of all arrays which are such that the value at a point {\cf p} in its
region has the type {\cf R(:c)!self.dist[p]}. (Recall that in \Xten{} a
distribution is itself a value array that maps a point to a place.)

\begin{example}
  The type {\cf double[:rail]} is the Java type {\cf double[]}.
  The type {\cf double[:rail][:rail]} is the Java type {\cf double[][]}.

  The type {\cf double[:rank==N]} is the type of all N-dimensional arrays of
  doubles.

  The type {\cf double[:rank==N\&\&onePlace==here]} is the type of all $N$-dimensional
  arrays of doubles that are local (mapped to one place).  
\end{example}

\section{Constraint system}

The initial release of \Xten{} has a very simple constraint system,
permitting only conjunctions of equalities between variables and
constants, and existential quantification over typed variables.

Subsquent implementations are intended to support boolean algebra,
arithmetic, relational algebra etc to permit types over regions and
distributions. We envision this as a major step towards removing most
(if not all) dynamic array bounds and places check from \Xten{}.
  
%%\subsection{Normalization of constraints}
%%
%%The constraint system satisfies the following structural rule
%%
%%  Gamma |- c
%%---------------------------------
%%Gamma, S(: c') x |- S(:c' \&\& c) x
%%
%%An additional rule, SELF, relates self to the variable being defined:
%%
%%   T(: c) x |- c[x/self]
%%
%%The rule TI ensures that the type invariant is implicitly available:
%%
%%   T(: c) x |- T (: i(T)[x/this] \&\& c) x
%%
%%Examples
%%
%%Consider the definitions
%%  class region(int rank) { ...}
%%  class intArray(region r, int(:self==r.rank) rank) { ...}
%%
%%Now we have the following invariants:
%%
%%   i(region) =def=  int this.rank
%%   i(intArray) =def= region this.r \&\& int(:self==this.r.rank) this.rank
%%
%%Note that i(intArray) is equivalent to
%%   
%%   region this.r \&\& int(:self==this.r.rank) this.rank \&\& this.rank == this.r.rank
%%
%%Now we have the following derivations
%%
%%  intArray(:self.rank==2) a |- region(:self.rank==2) a.r
%%
%%through the following derivation. 
%%
%%  intArray(:self.rank==2) a |- intArray(:self.rank==2 \&\& i(intArray)) a  (TI)
%%
%%Now |intArray(:self.rank==2 \&\& i(intArray)) a|  is equivalent to
%%
%%  intArray(:self.rank==2 \&\& region a.r \&\& int(:self==a.r.rank) a.rank 
%%            \&\& a.rank == a.r.rank) a 
%%
%%and entails each one of 
%%
%%    a.rank==2
%%    a.rank == a.r.rank
%%
%%which together entail a.r.rank==2.

\subsection{Syntactic abbreviations}\label{DepType:SyntaxAbbrev}

\section{Place types}\label{DepType:PlaceType}\index{placetype}

\begin{x10}
PlaceTypeSpecifier ::= ! PlaceType
PlaceType ::=  any | current | Expression  
\end{x10}

All \Xten{} reference classes extend the class x10.lang.Ref:

\begin{x10}
package x10.lang;
ublic class Ref(place loc) { ... }  
\end{x10}

Because of the importance of places in the \Xten{} design, special
syntactic support is provided for deptypes involving places.

We now consider the expansions of deptypes with place information.

Unless a deptype {\cf T} (whose base is a reference type) has an {\cf !} suffix,
the constraint for {\cf T} is implicitly assumed to contain the clause
{\cf self.loc==here}.
\begin{x10}
C( : c) => C(: self.loc==here \&\& c )  
\end{x10}
\noindent The expansions for a deptype with an ``{\cf !}'' suffix are:
\begin{x10}
C( : c)!  => C(: c )  // no self.loc clause.
C( : c)!p => C(: self.loc==p \&\& c ) 
\end{x10}


\begin{quotation}
  {\cf Static Semantics Rule:} It is a compile time error for the {\tt !}-annotation to
  be used for deptypes whose base type does not extend {\tt x10.lang.Ref}.  
\end{quotation}

\section{Implicit Syntax}\label{ImplicitSyntax}\index{implicit syntax}

Recall that the explicit syntax for \Xten{} requires the programmer to use
asyncs/future to ensure the Locality Principle: An activity accesses
only those mutable locations that reside in the same place as the
activity. 

Explicit syntax has the advantage that the performance model for \Xten{}
is explicit from the syntax. It has the disadvantage that the
programmer has to manually reason about the placement of various
objects. If the programmer reasons incorrectly then computation may
abort at runtime with an exception.

The place-based type system enables the compiler to support the
Locality principle. The programmer uses the type system to establish
that the types of various objects are local. These assertions are
checked by the compiler (as a side-effect of checking dependent
types). Additionally, the programmer may now use normal variable
syntax to access (read/write) variables, and invoke methods on
objects. Suppose the type of the variable v is C(:c). If c establishes
loc==here then the compiler generates code for performing the relevant
operation on the local variable (read, write, method invocation)
synchronously. 

Otherwise the compiler generates code in explicit syntax as
follows. If the operation is a read, the compiler generates code to
perform a future/force on the variable

\begin{x10}
  future(v){v}.force();  
\end{x10}

If the operation is a write |v=e|, the compiler generates code to perform
the write synchronously:

\begin{x10}
  final T temp = e;
  finish async (v){ v = w;}
\end{x10}

If the operation is a read on an array variable a[p] the compiler
generates the code:

\begin{x10}
  future(a.dist[p]){ a[p]}.force();  
\end{x10}


If the operation is a write |a[p]=e|, the compiler generates code to perform
the write synchronously:

\begin{x10}
  final point tp = p;
  final T t = e;
  finish async (a.dist[tp]){ a[tp] = t;}
\end{x10}

If the operation is a method invocation |e.m(e1,..., en)| for a void
method, the compiler generates code to perform the method invocation
synchronously:

\begin{x10}
  final T! t = e;
  final P1! t1 = e1;
  ...
  final Pn! tn = en;
  finish async (t){ 
    t.m(t1,..., tn);
  }  
\end{x10}


If the operation is a method invocation |e.m(e1,...,en)| for a method
that returns a value of type E, then the compiler generates the
following code:

\begin{x10}
  new Runnable() {
  public E run() {
    final T! t = e;
    final P1! t1 = e1;
    ...
    final Pn! tn = en;
  
    return future(t){t.m(t1,...tn)}.force();
  }}.run()
\end{x10}

	\par % empty
\chapter{Names and packages}
\label{packages} \index{names}\index{packages}

\Xten{} supports \java's mechanisms for names and packages \cite[\S 6,\S 7]{jls2}, including {\tt public}, {\tt protected}, {\tt private} and package-specific access control. \Xten{} supports \java's naming conventions.

\Xten{} also supports \java{} 1.5 static imports \cite{jsr201}.	\par % \vfill\eject % empty
\chapter{Places}\label{XtenPlaces}\index{places}

An \Xten{} place is a repository for data and activities. Each place
is to be thought of as a locality boundary: the activities running in
a place may access data items located at that place with the
efficiency of on-chip access. Accesses to remote places may take
orders of magnitude longer.

{}\Xten{} provides a built-in value class, \xcd"x10.lang.place"; all
places are instances of this class.  This class is \xcd"final" in
{}\XtenCurrVer.

In \XtenCurrVer{}, the set of places available to a computation is
determined at the time that the program is run and remains fixed
through the run of the program. The number of places available 
may be determined by reading (\xcd"place.MAX_PLACES"). (This number
is specified from the command line/configuration information; 
see associated {\tt README} documentation.)

All scalar objects created during program execution are located in one
place, though they may be referenced from other places. Aggregate
objects (arrays) may be distributed across multiple places using
distributions.

The set of all places in a running instance of an \Xten{} program may
be obtained through the \xcd"const" field \xcd"place.places".  (This
set may be used to define distributions, for instance,
\S~\ref{XtenDistributions}.) 


The set of all places is totally ordered.  The first place may be
obtained by reading \xcd"place.FIRST_PLACE". The initial activity for
an \Xten{} computation starts in this place
(\S~\ref{initial-computation}). For any place, the operation \xcd"next()"
returns the next place in the total order (wrapping around at the
end). Further details on the methods and fields available on this
class may be obtained by consulting the API documentation.

{\em Note: Future versions of the language may permit user-definable
places, and the ability to dynamically create places. }

\paragraph{Static semantics.}
Variables of type \xcd"place" must be initialized and are implicitly
\xcd"final".  

\section{Place expressions}
Any expression of type \xcd"place" is called a place expression. 
Examples of place expressions are \xcd"this.location" (the place
at which the current object lives), \xcd"place.FIRST_PLACE"
(the first place in the system in canonical order). 

Place expressions are used in the following contexts: 
\begin{itemize}
%\item As a  place type in a type (\S~\ref{PlaceTypes}).
\item As a target for an \xcd"async" activity or a future
(\S~\ref{AsyncActivity}).
\item In a class cast expression  (\S~\ref{ClassCast}).
\item In an \xcd"instanceof" expression (\S~\ref{instanceOf}).
\item In stable equality comparisons, at type \xcd"place".
\end{itemize}

Like values of any other type, places may be passed as arguments
to methods, returned from methods, stored in fields etc.

\section{\Xcd{here}}\index{here}\label{Here}
\Xten{} supports a special indexical constant\footnote{
An indexical constant is one whose value depends on its context
of use.} \xcd"here":
\begin{verbatim}
22 ExpressionName ::= here
\end{verbatim}
The constant evaluates to the place at which the current activity is
running. Unlike other place expressions, this constant cannot be 
used as the placetype of fields, since the type of a field 
should be independent of the activity accessing it.

\paragraph{Example.}
The code:
\begin{xten}
public class F {
  public def m(a: F) {
    val OldHere: place = here;
    async (a) {
      System.out.println("OldHere == here:" 
                         + (OldHere == here));
    }
  }
  public static void main(s: array[String]) {
    new F().m(future(place.FIRST_PLACE.next())
              { new F() }.force());
  }
}  
\end{xten}
\noindent will print out \xcd"true" iff the computation was configured
to start with the number of places set to \xcd"1". 


\section{Implicit syntax}\label{ImplicitSyntax}\index{implicit syntax}

Recall that the explicit syntax for \Xten{} requires the programmer to use
asyncs/future to ensure the Locality Principle: An activity accesses
only those mutable locations that reside in the same place as the
activity. 

Explicit syntax has the advantage that the performance model for \Xten{}
is explicit from the syntax. It has the disadvantage that the
programmer has to manually reason about the placement of various
objects. If the programmer reasons incorrectly then computation may
abort at runtime with an exception.

The place-based type system enables the compiler to support the
Locality principle. The programmer uses the type system to establish
that the types of various objects are local. These assertions are
checked by the compiler (as a side-effect of checking dependent
types). Additionally, the programmer may now use normal variable
syntax to access (read/write) variables, and invoke methods on
objects. Suppose the type of the variable \xcd"v" is \xcd"C{c}".
If \xcd"c" establishes
\xcd"location==here" then the compiler generates code for
performing the relevant operation on the local variable (read,
write, method invocation) synchronously. 

Otherwise the compiler generates code in explicit syntax as
follows. If the operation is a read, the compiler generates code to
perform a future/force on the variable

\begin{xten}
future(v) { v }.force();  
\end{xten}

If the operation is a write \xcd"v=e", the compiler generates code to perform
the write synchronously:

\begin{xten}
val temp: T = e;
finish async (v) { v = w; }
\end{xten}

If the operation is a read on an array variable a[p] the compiler
generates the code:

\begin{xten}
future(a.dist(p)) { a(p) }.force();  
\end{xten}


If the operation is a write \xcd"a[p]=e", the compiler generates code to perform
the write synchronously:

\begin{xten}
val tp: point = p;
val t: T = e;
finish async (a.dist(tp)) { a(tp) = t; }
\end{xten}

If the operation is a method invocation \xcdmath"e.m(e$_1$,..., e$_n$)"
for a void method, the compiler generates code to
perform the method invocation synchronously:

\begin{xten}
val t: T! = e;
val t1: P1! = e1;
...
val tn: Pn! = en;
finish async (t) {
  t.m(t1,..., tn);
}
\end{xten}


If the operation is a method invocation
\xcdmath"e.m(e$_1$,...,e$_n$)" for a method that returns a value
of type E, then the compiler generates the following code:

XXX remove this -- do not define semantics by translation

\begin{xten}
( (t: T!, t1: P1!, ..., tn: Pn!) => { 
    return future(t) { t.m(t1,...tn) }.force();
  } )(e, e1, ..., en)
\end{xten}


	\par %0.1
\chapter{Activities}\label{XtenActivities}

An \Xten{} computation may have many concurrent {\em activities} ``in
flight'' at any give time. We use the term activity to denote a
dynamic execution instance of a piece of code (with references to
data). An activity is intended to execute in parallel with other
activities. An activity may be thought of as a very light-weight
thread.  In \XtenCurrVer{}, an activity may not be interrupted,
suspended or resumed as the result of actions taken by any other
activity.

An activity is spawned in a given place and stays in that place for
its lifetime.  An activity may be {\em running}, {\em blocked} on some
condition or {\em terminated}. When the statement associated with an
activity terminates normally, the activity terminates normally; when
it terminates abruptly with some reason $R$, the activity terminates
with the same reason (\Sref{ExceptionModel}).

An activity may be long-running and may invoke recursive methods (thus
may have a stack associated with it). On the other hand, an activity
may be short-running, involving a fine-grained operation such as a
single read or write.

% An activity may have an {\em activitylocal} heap accessible only
%to the activity. 

An activity may asynchronously and in parallel launch activities at
other places.

\Xten{} distinguishes between {\em local} termination and {\em global}
termination of a statement. The execution of a statement by an
activity is said to terminate locally when the activity has finished
all its computation related to that statement. (For instance the
creation of an asynchronous activity terminates locally when the
activity has been created.)  It is said to terminate globally when it
has terminated locally and all activities that it may have spawned at
any place (if any) have, recursively, terminated globally.

An \Xten{} computation is initiated as a single activity from the
command line. This activity is the {\em root activity}\index{root
activity} for the entire computation. The entire computation
terminates when (and only when) this activity globally
terminates. Thus \Xten{} does not permit the creation of so called
``daemon threads''---threads that outlive the lifetime of the root
activity. We say that an \Xten{} computation is {\em rooted}
(\Sref{initial-computation}).

\futureext{ We may permit the initial activity to be a daemon activity
to permit reactive computations, such as webservers, that may not
terminate.}

\section{The \Xten{} rooted exception model}
\label{ExceptionModel}

The rooted nature of \Xten{} computations permits the definition of a
{\em rooted} exception model. In multi-threaded programming languages
there is a natural parent-child relationship between a thread and a
thread that it spawns. Typically the parent thread continues execution
in parallel with the child thread. Therefore the parent thread cannot
serve to catch any exceptions thrown by the child thread. 

The presence of a root activity permits \Xten{} to adopt a different
model.  In any state of the computation, say that an activity $A$ is
{\em a root of} an activity $B$ if $A$ is an ancestor of $B$ and $A$
is suspended at a statement (such as the \xcd"finish" statement
\Sref{finish}) awaiting the termination of $B$ (and possibly other
activities). For every \Xten{} computation, the
\emph{root-of} relation
is guaranteed to be a tree. The root of the tree is the root activity
of the entire computation. If $A$ is the nearest root of $B$, the path
from $A$ to $B$ is called the {\em activation path} for the
activity.\footnote{Note that depending on the state of the computation
the activation path may traverse activities that are running,
suspended or terminated.}

We may now state the exception model for \Xten.  An uncaught exception
propagates up the activation path to its nearest root activity, where
it may be handled locally or propagated up the \emph{root-of} tree when
the activity terminates (based on the semantics of the statement being
executed by the activity).\footnote{In \XtenCurrVer{} the \xcd"finish"
statement is the only statement that marks its activity as a root
activity. Future versions of the language may introduce more such
statements.}  Thus, unlike concurrent languages such as \java{}, no
exception is ``thrown on the floor''.

\section{Spawning an activity}\label{AsynchronousActivity}\label{AsyncActivity}

Asynchronous activities serve as a single abstraction for supporting a
wide range of concurrency constructs such as message passing, threads,
DMA, streaming, data prefetching. (In general, asynchronous operations
are better suited for supporting scalability than synchronous
operations.)

An activity is created by executing the statement:

\begin{grammar}
Statement \: AsyncStatement \\
AsyncStatement \: \xcd"async" PlaceExpressionSingleList\opt Statement \\
PlaceExpressionSingleList \: \xcd"(" PlaceExpression \xcd")" \\
PlaceExpression \: Expression \\
\end{grammar} 

The place expression \xcd"e" is expected to be of type \xcd"Place",
e.g., \xcd"here" or \xcd"d(p)" for some
distribution \xcd"d" and point \xcd"p" (\Sref{XtenPlaces}).  
If not, the compiler replaces
\xcd"e" with \xcd"e.location" if
\xcd"e" is of type \xcd"x10.lang.Ref". Otherwise the compiler reports a type error. 

Note specifically that the expression \xcd"a(i)" when used as a place
expression may evaluate to \xcd"a(i).location", which may not be
the same place as \xcd"a.dist(i)". The programmer must be 
careful to choose the right expression, appropriate for the statement.
Accesses to \xcd"a(i)" within \grammarrule{Statement} should typically be guarded 
by the place expression \xcd"a.dist(i)".

In many cases the compiler may infer the unique place at which the
statement is to be executed by an analysis of the types of the
variables occuring in the statement. (The place must be such that the
statement can be executed safely, without generating a
\xcd"BadPlaceException".) In such cases the programmer may omit the
place designator; the compiler will throw an error if it cannot
determine the unique designated place.\footnote{\XtenCurrVer{} does
not specify a particular algorithm; this will be fixed in future
versions.}

An activity $A$ executes the statement \xcd"async (P) S" by launching
a new activity $B$ at the designated place, to execute the specified
statement. The statement terminates locally as soon as $B$ is
launched.  The activation path for $B$ is that of $A$, augmented with
information about the line number at which $B$ was spawned.  $B$
terminates normally when $S$ terminates normally.  It terminates
abruptly if $S$ throws an (uncaught) exception. The exception is
propagated to $A$ if $A$ is a root activity (see \Sref{finish}),
otherwise through $A$ to $A$'s root activity. Note that while an
activity is running, exceptions thrown by activities it has already
generated may propagate through it up to its root activity.

Multiple activities launched by a single activity at another place are
not ordered in any way. They are added to the pool of activities at
the target place and will be executed in sequence or in parallel based
on the local scheduler's decisions. If the programmer wishes to
sequence their execution s/he must use \Xten{} constructs, such as
clocks and \xcd"finish" to obtain the desired effect.  Further, the
\Xten{} implementations are not required to have fair schedulers,
though every implementation should make a best faith effort to ensure
that every activity eventually gets a chance to make forward progress.

\begin{staticrule*}
The statement in the body of an \xcd"async" is subject to the
restriction that it must be acceptable as the body of a \xcd"void"
method for an anoymous inner class declared at that point in the code,
which throws no checked exceptions. As such, it may reference
variables in lexically enclosing scopes (including \xcd"clock"
variables, \Sref{XtenClocks}) provided that such variables are
(implicitly or explicitly) \xcd"final".
\end{staticrule*}

\paragraph{Returning from within an \xcd"async".}
The body \xcd"S" of an  \xcd"async S" is not permitted to contain a 
\xcd"return" statement since there are two candidate scopes from which
the programmer might intend the return: the   
\xcd"async" itself, and the enclosing method.

Programmers wishing to use a \xcd"return" statement in \xcd"S" to
return from the \xcd"async S" should use the idiom 
\xcd"val x=()=>S;async x();" instead (for some new variable \xcd"x". 
(Note that it does not make sense for code executing in the body of an
\xcd"async" to attempt to return from the enclosing method -- the
method may already have returned asynchronously.)


\section{Place changes}\label{AtStatement}

An activity may change place using the \xcd"at" statement or
\xcd"at" expression:

\begin{grammar}
Statement \: AtStatement \\
AtStatement \: \xcd"at" PlaceExpressionSingleList Statement \\
Expression \: AtExpression \\
AtExpression \: \xcd"at" PlaceExpressionSingleList ClosureBody \\
\end{grammar}

The statement \xcd"at (p) S" executes the statement \xcd"S"
synchronously at place \xcd"p".
The expression \xcd"at (p) E" executes the statement \xcd"E"
synchronously at place \xcd"p", returning the result to the
originating place.

\section{Finish}\index{finish}\label{finish}
The statement \xcd"finish S" converts global termination to local
termination and introduces a root activity. 

\begin{grammar}
Statement \: FinishStatement \\
FinishStatement \: \xcd"finish" Statement \\
\end{grammar}

An activity $A$ executes \xcd"finish S" by executing \xcd"S".  The
execution of \xcd"S" may spawn other asynchronous activities (here or
at other places).  Uncaught exceptions thrown or propagated by any
activity spawned by \xcd"S" are accumulated at \xcd"finish S".
\xcd"finish S" terminates locally when all activities spawned by \xcd"S"
terminate globally (either abruptly or normally). If
\xcd"S" terminates normally, then \xcd"finish S" terminates normally
and $A$ continues execution with the next statement after \xcd"finish S".
If \xcd"S" terminates abruptly, then \xcd"finish S"
terminates abruptly and throws a single exception formed 
from the collection of exceptions accumulated at \xcd"finish S".

Thus a \xcd"finish S" statement serves as a collection point for
uncaught exceptions generated during the execution of \xcd"S".

Note that repeatedly \xcd"finish"ing a statement has no effect after
the first \xcd"finish": \xcd"finish finish S" is indistinguishable
from \xcd"finish S".

\paragraph{Interaction with clocks.}\label{sec:finish:clock-rule}
\xcd"finish S" interacts with clocks (\Sref{XtenClocks}). 

While executing \xcd"S", an activity must not spawn any \xcd"clocked"
asyncs. (Asyncs spawned during the execution of \xcd"S" may spawn
clocked asyncs.) A
\xcd"ClockUseException"\index{clock!ClockUseException} is thrown
if (and when) this condition is violated.

In \XtenCurrVer{} this condition is checked dynamically; future
versions of the language will introduce type qualifiers which permit
this condition to be checked statically.

\futureext{
The semantics of \xcd"finish S" is conjunctive; it terminates when all
the activities created during the execution of \xcd"S" (recursively)
terminate. In many situations (e.g., nondeterministic search) it is
natural to require a statement to terminate when any {\em one} of the
activities it has spawned succeeds. The other activities may then be
safely aborted. Future versions of the language may introduce a
\xcd"finishone S" construct to support such speculative or nondeterministic
computation.
}
%% Need an example here.

\section{Initial activity}\label{initial-computation}\index{initial activity}

An \Xten{} computation is initiated from the command line on the
presentation of a classname \xcd"C". The class must have a
\xcd"public static def main(a: array[String])" method, otherwise an
exception is thrown
and the computation terminates.  The single statement
\begin{xten}
finish async (place.FIRST_PLACE) {
  C.main(s);
}
\end{xten} 
\noindent is executed where \xcd"s" is an array of strings created
from command line arguments. This single activity is the root activity
for the entire computation. (See \Sref{XtenPlaces} for a discussion of
placs.)

%% Say something about configuration information? 

\section{Foreach statements}
\index{foreach@\xcd"foreach"}\label{ForLoop}


\begin{grammar}
Statement \: ForEachStatement \\
ForEachStatement \: 
      \xcd"foreach" \xcd"(" Formal \xcd"in" Expression \xcd")"
          Statement \\
\end{grammar}

The \xcd"foreach" statement is similar to the enhanced \xcd"for"
statement (\Sref{ForAllLoop}).

An activity executes a \xcd"foreach" statement in a similar fashion
except that separate \xcd"async" activities are launched in parallel
in the local place of each object returned by the iteration.
The statement
terminates locally when all the activities have been spawned. It never
throws an exception, though exceptions thrown by the spawned
activities are propagated through to the root activity.

In a common case, the
the collection is intended to be of type
\xcd"Region" and the formal parameter is of type \xcd"Point".  Expressions \xcd"e" of type \xcd"Dist" and
\xcd"Array" are also accepted, and treated as if they were \xcd"e.region".


\section{Ateach statements}

\begin{grammar}
Statement \: AtEachStatement \\
AtEachStatement \:
      \xcd"ateach" \xcd"(" Formal \xcd"in" Expression \xcd")"
         Statement \\
\end{grammar}

The \xcd"ateach" statement is similar to the \xcd"foreach"
statement.  The collection must be of type \xcd"Dist"
and the formal parameter of type \xcd"Point".
Expressions \xcd"e" of type \xcd"Array" are also accepted, and treated
as if they were \xcd"e.dist". The compiler reports a type error
in all other cases.

This statement differs from \xcd"foreach" only in
that each activity is spawned at the place specified by the
distribution for the point. That is, 
\xcd"ateach(p(i1,...,ik): point in D) S" may
be thought of as standing for:
\begin{xten}
foreach (p(i1,...,ik): point in D.region) 
  async (D(p)) S
\end{xten}

\section{Futures}\label{XtenFutures}

\Xten{} provides syntactic support for {\em asynchronous expressions}, also
known as futures:

\begin{grammar}
Primary \: FutureExpression \\
FutureExpression \:
  \xcd"future" PlaceExpressionSingleList\opt ClosureBody
\end{grammar} 

Intuitively such an expression evaluates its body asynchronously at
the given place. The resulting value may be obtained from the future
returned by this expression, by using the \xcd"force" operation.

In more detail, in an expression \xcd"future (Q) e", the place
expression \xcd"Q" is treated as in an \xcd"async" statement. \xcd"e"
is an expression of some type \xcd"T". \xcd"e" may reference only
those variables in the enclosing lexical environment which are
declared to be \xcd"final".

If the type of \xcd"e" is \xcd"T" then the type of
\xcd"future (Q) e" is \xcd"future[T]".  This 
type \xcd"Future[T]" is defined as if by:
\begin{xten}
package x10.lang;
public interface Future[T] implements () => T {
  def forced(): Boolean;
}
\end{xten}

Evaluation of \xcd"future (Q) e" terminates locally with the creation
of a value \xcd"f" of type \xcd"Future[T]".  This value may be
stored in objects, passed as arguments to methods, returned from
method invocation etc. 

At any point, the method \xcd"forced" may be invoked on \xcd"f". This
method returns without blocking, with the value \xcd"true" if the
asynchronous evaluation of \xcd"e" has terminated globally and with
the value \xcd"false" if it has not.

\xcd"Future[T]" is a subtype of the function type \xcd"() => T".
Invoking---\emph{forcing}---the future \xcd"f" blocks until the
asynchronous evaluation of \xcd"e" has terminated globally. If the
evaluation terminates successfully with value \xcd"v", then the method
invocation returns \xcd"v". If the evaluation terminates abruptly with
exception \xcd"z", then the method throws exception \xcd"z". Multiple
invocations of the function (by this or any other activity) do not
result in multiple evaluations of \xcd"e". The results of the first
evaluation are stored in the future \xcd"f" and used to respond to all
queries.

\begin{example}
\begin{xten}
promise: Future[T] = future (a.dist(3)) a(3);
value: T = promise();
\end{xten}
\end{example}

\eat{
\subsection{Implementation notes}
Futures are provided in \Xten{} for convenience; they may be
programmed using latches, \xcd"async" and \xcd"finish" as
described in \Sref{future-imp}.
}

\section{At expressions}

\begin{grammar}
Expression \: \xcd"at" \xcd"(" Expression \xcd")" Expression
\\
\end{grammar}

An at expression evaluates an expression synchronously at a given place.
The expression \xcd"at (p) e" is equivalent to \xcd"future (p) e).force()".

\section{Shared variables}
\label{Shared}

A shared local variable is declared with the annotation
\xcd"shared".  It may be thought of as being accessible by any spawned
activity in its lexical scope.  Final variables are implicitly
shared.  An implementation may consider boxing shared 
variables and making a reference to the boxed value available to
any closures that use the variable.

\section{Atomic blocks}\label{AtomicBlocks}\index{atomic blocks}
Languages such as \java{} use low-level synchronization locks to allow
multiple interacting threads to coordinate the mutation of shared
data. \Xten{} eschews locks in favor of a very simple high-level
construct, the {\em atomic block}.

A programmer may use atomic blocks to guarantee that invariants of
shared data-structures are maintained even as they are being accessed
simultaneously by multiple activities running in the same place.

\subsection{Unconditional atomic blocks}
The simplest form of an atomic block is the {\em unconditional
atomic block}:

\begin{grammar}
Statement \: AtomicStatement \\
AtomicStatement \: \xcd"atomic"  Statement \\
MethodModifier \: \xcd"atomic" \\
\end{grammar}

For the sake of efficient implementation \XtenCurrVer{} requires
that the atomic block be {\em analyzable}, that is, the set of
locations that are read and written by the \grammarrule{BlockStatement} are
bounded and determined statically.\footnote{A static bound is a constant
that depends only on the program text, and is independent 
of any runtime parameters. }
The exact algorithm to be used by
the compiler to perform this analysis will be specified in future
versions of the language.
\tbd{}

Such a statement is executed by an activity as if in a single step
during which all other concurrent activities in the same place are
suspended. If execution of the statement may throw an exception, it is
the programmer's responsibility to wrap the atomic block within a
\xcd"try"/{\xcd"finally" clause and include undo code in the finally
clause. Thus the \xcd"atomic" statement only guarantees atomicity on
successful execution, not on a faulty execution.

%% A compiler is allowed to reorder two atomic blocks that have no
%%data-dependency between them, just as it may reorder any two
%%statements which have no data-dependencies between them. For the
%%purposes of data dependency analysis, an atomic block is deemed to
%%have read and written all data at a single program point, the
%%beginning of the atomic block.
%%%% I dont believe we need to say at some point in the atomic block.
%%
We allow methods of an object to be annotated with \xcd"atomic". Such
a method is taken to stand for a method whose body is wrapped within an
\xcd"atomic" statement.

Atomic blocks are closely related to non-blocking synchronization
constructs \cite{herlihy91waitfree}, and can be used to implement 
non-blocking concurrent algorithms.

\begin{staticrule*}
In \xcd"atomic S", \xcd"S" may include method calls,
conditionals, etc.
It may {\em not} include an \xcd"async" activity.
It may {\em not} include any statement that may potentially block at
runtime (e.g., \xcd"when", \xcd"force" operations, \xcd"next"
operations on clocks, \xcd"finish").

\limitation{Not checked in the current implementation.}
\end{staticrule*}


All locations accessed in an atomic block must reside \xcd"here"
(\Sref{Here}). A
\xcd"BadPlaceException"\index{place!BadPlaceException} is thrown
if (and when) this condition is violated.

All locations accessed in an atomic block must statically satisfy the
{\em locality condition}: they must belong to the place of the current
activity.\index{locality condition}\label{LocalityCondition} The
compiler checks for this condition by checking whether the statement
could be the body of a \xcd"void" method annotated with \xcd"local" at
that point in the code (\Sref{LocalAnnotation}).

\paragraph{Consequences.}
Note an important property of an (unconditonal) atomic block:

\begin{eqnarray}
 \mbox{\xcd"atomic {s1; atomic s2}"} &=& \mbox{\xcd"atomic {s1; s2}"}
\end{eqnarray}

Further, an atomic block will eventually terminate successfully or
thrown an exception; it may not introduce a deadlock.


\subsubsection{Example}

The following class method implements a (generic) compare and swap (CAS) operation:

\begin{xten}
// target defined in lexically enclosing environment.
public atomic def CAS(old: Object, new: Object): Boolean {
   if (target.equals(old)) {
     target = new;
     return true;
   }
   return false;
}
\end{xten}

\subsection{Conditional atomic blocks}

Conditional atomic blocks are of the form:

\begin{grammar}
Statement \:  WhenStatement \\
WhenStatement \:  \xcd"when" \xcd"(" Expression \xcd")" Statement \\
            \| WhenStatement \xcd"or" \xcd"(" Expression \xcd")" Statement \\
\end{grammar}

In such a statement the one or more expressions are called {\em
guards} and must be \xcd"Boolean" expressions. The statements are the
corresponding {\em guarded statements}. The first pair of expression
and statement is called the {\em main clause} and the additional pairs
are called {\em auxiliary clauses}. A statement must have a main
clause and may have no auxiliary clauses.

An activity executing such a statement suspends until such time as any
one of the guards is true in the current state. In that state, the
statement corresponding to the first guard that is true is executed.
The checking of the guards and the execution of the corresponding
guarded statement is done atomically. 


\Xten{} does not guarantee that a conditional atomic block
will execute if its condition holds only intermmitently. For, based on
the vagaries of the scheduler, the precise instant at which a
condition holds may be missed. Therefore the programmer is advised to
ensure that conditions being tested by conditional atomic blocks are
eventually stable, i.e., they will continue to hold until the block
executes (the action in the body of the block may cause the condition
to not hold any more).

%%Fourth, \Xten{} does not guarantees only {\em weak fairness} when executing
%%conditional atomic blocks. Let $c$ be the guard of some conditional
%%atomic block $A$. $A$ is required to make forward progress only if
%%$c$ is {\em eventually stable}. That is, any execution $s_1, s_2,
%%\ldots$ of the program is considered illegal only if there is a $j$
%%such that $c$ holds in all states $s_k$ for $k > j$ and in which $A$
%%does not execute. Specifically, if the system executes in such a way
%%that $c$ holds only intermmitently (that is, for some state in which
%%$c$ holds there is always a later state in which $c$ does not hold),
%%$A$ is not required to be executed (though it may be executed).

\begin{rationale}
The guarantee provided by \xcd"wait"/\xcd"notify" in \java{} is no
stronger. Indeed conditional atomic blocks may be thought of as a
replacement for \java's wait/notify functionality.
\end{rationale} 

We note two common abbreviations. The statement \xcd"when (true) S" is
behaviorally identical to \xcd"atomic S": it never suspends. Second,
\xcd"when (c) {;}" may be abbreviated to \xcd"await(c);"---it
simply indicates that the thread must await the occurrence of a
certain condition before proceeding.  Finally note that a \xcd"when"
statement with multiple branches is behaviorally identical to a
\xcd"when" statement with a single branch that checks the disjunction of
the condition of each branch, and whose body contains an
\xcd"if"/\xcd"then"/\xcd"else" checking each of the branch conditions.

\begin{staticrule*}
For the sake of efficient implementation certain restrictions are
placed on the guards and statements in a conditional atomic
block. 
\end{staticrule*}

Guards are required not to have side-effects, not to spawn
asynchronous activities and to have a statically determinable upper
bound on their execution. These conditions are expected to be checked
statically by the compiler.

The body of a \xcd"when" statement must satisfy the conditions
for the body of an \xcd"atomic" block.
%Second, as for unconditional atomic blocks, the set of memory
%locations accessed by a guarded statements are required to be bounded
%and statically anlayzable.

Note that this implies that guarded statements are required to be {\em
flat}, that is, they may not contain conditional atomic blocks. (The
implementation of nested conditional atomic blocks may require
sophisticated operational techniques such as rollbacks.)

\paragraph{Sample usage.} 
There are many ways to ensure that a guard is eventually
stable. Typically the set of activities are divided into those that
may enable a condition and those that are blocked on the
condition. Then it is sufficient to require that the threads that may
enable a condition do not disable it once it is enabled. Instead the
condition may be disabled in a guarded statement guarded by the
condition. This will ensure forward progress, given the weak-fairness
guarantee.

\begin{example}
The following class shows how to implement a bounded buffer of size
$1$ in \Xten{} for repeated communication between a sender and a
receiver.

\begin{xten}
class OneBuffer {
  datum: Object = null;
  filled: Boolean = false;
  public def send(v: Object) {
    when (!filled) {
      this.datum = v;
      this.filled = true;
    }
  }
  public def receive(): Object {
    when (filled) {
      v: Object = datum;
      datum = null;
      filled = false;
      return v;
    }
  }
}
\end{xten}
\end{example}

\eat{
\paragraph{Implementing a future with a latch.}\label{future-imp}
The following class shows how to implement a {\em latch}. A latch is
an object that is initially created in a state called the {\em
unlatched} state. During its lifetime it may transition once to a {\em
forced} state. Once forced, it stays forced for its lifetime. The
latch may be queried to determine if it is forced, and if so, an
associated value may be retrieved. Below, we will consider a latch set
when some activity invokes a \xcd"setValue" method on it. This method
provides two values, a normal value and an exceptional value. The
method \xcd"force" blocks until the latch is set. If an exceptional
value was specified when the latch was set, that value is thrown on
any attempt to read the latch. Otherwise the normal value is returned.

\begin{xten}
public interface Future[T] {
   def forced(): Boolean;
   def apply(): T;
}
public class Latch implements Future {
  protected var forced: Boolean = false;
  protected var result: Box[T] = null;
  protected var z: Box[Exception] = null;

  public atomic def setValue(val: T): Boolean {
    return setValue(val, null);
  }
  public atomic def setValue(z: Exception): Boolean {
    return setValue(null, z);
  }
  public atomic def setValue(val: T,
                             z: Exception): Boolean {
    if (forced) return false;
    // these assignment happens only once.
    this.result = val;
    this.z = z;
    this.forced = true;
    return true;
  }
  public atomic def forced(): Boolean {
    return forced;
  }
  public def apply(): T {
    when (forced) {
      if (z != null) throw z;
      return result to T;
    }
  }
}
\end{xten}

Latches, \xcd"aync" operations and \xcd"finish" operations may be used
to implement futures as follows. The expression \xcd"future(P) e"
can be translated to:
\begin{xten}
(() => {
    L: Latch = new Latch();
    async (P) {
      X: Object;
      try {
        finish X = e;
        async (L) {
          L.setValue(X); 
        }
      }
      catch (Z: Exception) {
        async (L) {
          L.setValue(Z);
        }
      }
    }
    return L;
  })()
\end{xten}

Here we assume that \xcd"RunnableLatch" is an interface defined by:
\begin{xten} 
public interface RunnableLatch {
  def run(): Latch;
}
\end{xten}

We use the standard \java{} idiom of wrapping the core translation
inside an inner class definition/method invocation pair (i.e.,
\xcd"new RunnableLatch() {....}.run()") so as to keep the resulting
expression completely self-contained, while executing statements
inside the evaluation of an expression.

Execution of a \xcd"future(P) e" causes a new latch to be created,
and an \xcd"async" activity spawned at \xcd"P". The activity attempts
to \xcd"finish" the assigned \xcd"x = e", where \xcd"x" is a local
variable.  This may cause new activities to be spawned, based on
\xcd"e". If the assignment terminates successfully, another activity is
spawned to invoke the \xcd"setValue" method on the latch.  Exceptions
thrown by these activities (if any) are accumulated at the \xcd"finish"
statement and thrown after global termination of all
activities spawned by \xcd"x=e". The exception will be caught by the 
\xcd"catch" clause and stored with the latch. 


\oldtodo{Conditional atomic blocks should be powerful enough to implement clocks as well.}

\paragraph{A future to execute a statement.}
Consider an expression \xcd"onFinish {S}". This should return
a \xcd"Boolean" latch which should be forced when \xcd"S" has terminated
globally. Unlike \xcd"finish S", the evaluation of \xcd"onFinish {S}"
should locally terminate immediately, returning a latch. The
latch may be passed around in method invocations and stored in
objects. An activity may perform \xcd"force"/\xcd"forced" method
invocations on the latch whenever it desires to determine whether \xcd"S"
has terminated.

Such an expression can be written as:
\begin{xten}
(=> {
    L: Latch = new Latch();
    async (here) {
      try {
        finish S;
        L.setValue(true);
      }
      catch (Z: Exception) {
        L.setValue(Z);
      }
    }
    return L;
  }
)()
\end{xten}
}
	\par %0.1
\chapter{Clocks}\label{XtenClocks}\index{clocks}
\cbstart
The standard library for \Xten{}, {\cf x10.lang} defines a {\cf final
value class}, {\tt clock} intended for repeated quiescence detection
of arbitrary, data-dependent collection of activities. Clocks are a
generalization of {\em barriers}. They permit dynamically created
activities to register and deregister. An activity may be registered
with multiple clocks at the same time. In particular, nested clocks
are permitted: an activity may create a nested clock and within one
phase of the outer clock schedule activities to run to completion on
the nested clock.  Neverthless the design of clocks ensures that
deadlock cannot be introduced by using clock operations.

This chapter describes the syntax and semantics of clocks and
statements in the language that have parameters of type {\cf clock}. 

The key invariants associated with clocks are as follows.  At any
stage of the computation, a clock has zero or more {\em registered}
activities. An activity may perform operations only on those clocks it
is registered with (these clocks constitute its {\em clock set}).  An
activity is registered with one or more clocks when it is created.
During its lifetime the only additional clocks it is registered with
are exactly those that it creates. In particular it is not possible
for an activity to register itself with a clock it discovers by
reading a data-structure.

An activity may perform the following operations on a clock {\cf
c}. It may {\em unregister} with {\cf c} by executing {\cf
c.drop();}. After this, it may perform no further actions on {\cf c}
for its lifetime. It may {\em check} to see if it is unregistered on a
clock. It may {\em register} a newly forked activity with {\cf c}.  It
may {\em post} a statement {\cf S} for completion in the current phase
of {\cf c} by executing the statement {\cf now(c) S}. It may {\em
resume} the clock by executing {\cf c.resume();}. This indicates to
{\cf c} that it has finished posting all statements it wishes to
perform in the current phase. Finally, it may {\em block} (by
executing {\cf next;}) on all the clocks that it is registered
with. (This operation implicitly {\cf resume}'s all clocks for the
activity.) It will resume from this statement only when all these
clocks are ready to advance to the next phase.

A clock becomes ready to advance to the next phase when every activity
registered with the clock has executed at least one {\cf resume}
operation on that clock and all statements posted for completion in
the current phase have been completed.

Though clocks introduce a blocking statement ({\cf next}) an important
property of \Xten{} is that clocks cannot introduce deadlocks. That is,
the system cannot reach a quiescent state (in which no activity is
progressing) from which it is unable to progress. For, before blocking
each activity resumes all clocks it is registered with. Thus if a
configuration were to be stuck (that is, no activity can progress) all
clocks will have been resumed. But this implies that all activities
blocked on {\cf next} may continue and the configuration is not stuck.

\section{Clock operations}
The special statements introduced for clock operations are listed below.
\begin{x10}
462 Statement ::= ClockedStatement
472 StatementNoShortIf ::= 
      ClockedStatementNoShortIf
479 NowStatement ::= 
      now ( Clock ) Statement
480 ClockedStatement ::= 
      clocked ( ClockList ) Statement
490 ClockedStatementNoShortIf ::= 
      clocked ( ClockList ) 
         StatementNoShortIf
501 NextStatement ::= next ;
\end{x10}

Note that {\tt x10.lang.clock} provides several useful methods on
clocks (e.g. {\tt drop}).

\subsection{Creating new clocks}\index{clock!creation}
Clocks are created using the nullary constructor for {\cf
x10.lang.clock} via a factory method:

\begin{x10}
clock timeSynchronizer = clock.factory.clock();
\end{x10}

All clocked variables are implicitly final. The initializer for a
local variable declaration of type {\tt clock} must be a new clock
expression. Thus \Xten{} does not permit aliasing of clocks.
Clocks are created in the place global heap and hence outlive the
lifetime of the creating activity.  Clocks are instances of value
classes, hence may be freely copied from place to
place. (Clock instances typically contain references to mutable state
that maintains the current state of the clock.)

The current activity is automatically registered with the newly
created clock.  It may deregister using the {\tt drop} method on
clocks (see the documentation of {\tt x10.lang.clock}). All activities
are automatically deregistered from all clocks they are registered
with on termination (normal or abrupt).

\subsection{Registering new activities on clocks}\index{clock!clocked statements}

The programmer may specify which clocks a new activity is to be registered with using the {\tt clocked} clause:
\begin{x10}
\end{x10}


\paragraph{Static semantics.} An activity may 
transmit only those clocks that is registered with and has not
quiesced on.  (\S~\ref{resume}). The compiler checks this
statically, inserting code to throw a {\tt ClockUseException}
if a violation is detected at runtime.

An activity may check that it is registered on a clock {\tt c} by
executing:
\begin{x10}
c.registered()
\end{x10}
\noindent This call returns a {\cf boolean} value: {\cf true} iff the
activity is registered on {\cf c}.

\paragraph{Note.} 
\Xten{} does not contain a ``register'' statement that would allow an
activity to discover a clock in a datastructure and register itself on
it. Therefore, while clocks may be stored in a datastructure by one
activity and read from that by another, the new activity cannot
``use'' the clock unless it is already registered with it.

\todo{Add text on arrays of clocks.}

\subsection{Resuming clocks}\index{clock!resume}\label{resume}
\Xten{} permits {\em split phase} clocks. An activity may wish
to indicate that it has completed whatever work it wishes to perform
in the current phase of a  clock {\tt c} it is registered with, without
suspending all activity. It may do so  by executing the method
invocation:
\begin{x10}
  c.resume();
\end{x10}
\noindent on a clock {\tt c} it is registered with.  

Nothing happens if the activity invokes this method on a clock it is
not registered with, or if it has already invoked a {\tt resume} on
this clock in the current phase.  Otherwise execution of this
statement indicates that the activity will not transmit {\cf c} to an
async or invoke {\cf now} until it terminates, drops {\cf c} or
executes a {\tt next}. The runtime throws a {\tt ClockUseException} if
it detects a violation of this condition.

\paragraph{Static semantics.} 
The compiler should issue an error if any activity has a potentially
live execution path from a {\cf resume} statement on a clock {\tt c}
to a {\cf now} or async spawn statement (which registers the new
activity on {\cf c}) unless the path goes through a {\cf next}
statement. (A {\cf c.drop()} following a {\cf c.resume()} is legal,
as is {\cf c.resume()} following a {\cf c.resume()}.

\subsection{Advancing clocks}\index{clock!next}
An activity may execute the statement
\begin{x10}
  next;
\end{x10}

\noindent 
Execution of this statement blocks until all the clocks that the
activity is registered with (if any) have advanced. (The activity
implicitly issues a {\cf resume} on all clocks it is registered
with before suspending.)

An \Xten{} computation is said to be {\em quiescent} on a clock {\cf
c} if each activity registered with {\cf c} has continued {\cf c}.
Note that once a computation is quiescent on {\cf c}, it will remain
quiescent on {\cf c} forever (unless the system takes some action),
since no other activity can become registered with {\cf c}.  That is,
quiescence on a clock is a {\em stable property}.

Once the implementation has detected quiecence on {\cf c}, the system
marks all activities registered with {\cf c} as being able to progress
on {\cf c}. An activity blocked on {\cf next} resumes execution once
it is marked for progress by all the clocks it is registered with.

\subsection{Dropping clocks}\index{clock!drop}
An activity may drop a clock by executing:
\begin{x10}
c.drop();
\end{x10}

\noindent{} 
The method does nothing if the activity has already dropped {\cf c}. 

\paragraph{Static semantics.}
The compiler should issue an error if it discovers a potentially live
execution path from a {\tt c.drop()} to a statement using {\tt c}.

\subsection{Posting statements on a clock}\index{clock!now}
\Xten{} provides syntactic support for a common idiom. Often it may be
necessary for an activity $A$ to require that a certain set of
statements be executed to completion before a clock $c$ can move
forward, without $A$ actually waiting for the completion
of the statement. We introduce the syntax:
\begin{x10}
461 Statement ::= NowStatement
471 StatementNoShortIf ::= 
       NowStatementNoShortIf
479 NowStatement ::= 
       now ( Clock ) Statement
489 NowStatementNoShortIf ::= 
       now ( Clock ) StatementNoShortIf
\end{x10}
\noindent 

A statement {\tt now (c) s} may be considered as shorthand for
\begin{x10}
  async clocked(c) \{ 
     finish async s; 
  \}
\end{x10}

\paragraph{Note.} Because of the static semantics of {\tt finish}
it is not possible to nest {\cf now} statements. Instead if it proves
useful, we may introduce a multi-clocked {\tt now} statement,
which permits the statement to be posted on multiple clocks
simultaneously.
\begin{x10}
479' NowStatement ::= 
       now ( ClockList ) Statement
489' NowStatementNoShortIf ::= 
       now ( ClockList ) StatementNoShortIf  
\end{x10}

\subsection{Program equivalences}

From the discussion above it should be clear that the following
equivalences hold:

\begin{eqnarray}
 {\cf c.resume(); next;}       &=& {\cf next;}\\
 {\cf c.resume(); d.resume();} &=& {\cf d.resume(); c.resume();}\\
 {\cf c.resume(); c.resume();} &=&  {\cf c.resume();}
\end{eqnarray}

Note that {\cf next; next;} is not the same as {\cf next;}. The
first will wait for clocks to advance twice, and the second
once.  

\cbend
%%\subsection{Implementation Notes}
%%Clocks may be implemented efficiently with message passing, e.g.{} by
%%using short-circuit ideas in \cite{SaraswatPODC88}.  Recall that every
%%activity is spawned with references to a fixed number of clocks. Each
%%reference should be thought of as a global pointer to a location in
%%some place representing the clock. (We shall discuss a further
%%optimization below.) Each clock keeps two counters: the total number
%%of outstanding references to the clock, and the number of activities
%%that are currently suspended on the clock.
%%
%%When an activity $A$ spawns another activity $B$ that will reference a
%%clock $c$ referenced by $A$, $A$ {\em splits} the reference by sending
%%a message to the clock. Whenever an activity drops a reference to a
%%clock, or suspends on it, it sends a message to the clock. 
%%
%%The optimization is that the clock can be represented in a distributed
%%fashion. Each place keeps a local counter for each clock that is
%%referenced by an activity in that place. The global location for the
%%clock simply keeps track of the places that have references and that
%%are quiescent. This can reduce the inter-place message traffic
%%significantly.

\todo{Reintroduce clocked types}
%%\section{Clocked types}\index{types!clocked}
%%
%%%We allow types to specify clocks, via a {\cf clocked(c)} modifier,
%%%e.g.{}
%%
%%%\begin{x10}
%%%  clocked(c) int r;
%%%\end{x10}
%%
%%%This declaration asserts that {\cf r} is accessible
%%%(readable/writable) only by those statements that are clocked on {\cf
%%%c}. Thus propagation of the clock provides some control over the
%%%``visibility'' of {\cf r}.
%%
%%The declaration 
%%
%%\begin{x10}
%%  clocked(c) final int l = r;
%%\end{x10}
%%
%%\noindent asserts additionally that in each clock instant {\cf l} is final, 
%%i.e.{} the value of {\cf l} may be reset at the beginning of each phase
%%of {\tt c} but stays constant during the phase.
%%
%%This statement terminates when the computation of {\tt r} has
%%terminated and the assignment has been performed.
%%
%%\todo{Generalize the syntax so that aggregate variables can be clocked with an aggregate clock of the same shape.}
%%
%%\subsection{Clocked assignment}\index{assignment!clocked}
%%We expect that most arrays containing application data will be
%%declared to be {\cf clocked final}. We support this very powerful type
%%declaration by providing a new statement:
%%{\footnotesize
%%\begin{verbatim}
%%  next(c) l = r; 
%%\end{verbatim}}
%%
%%
%%\noindent 
%%for a variable $l$ declared to be clocked on $c$. The statement
%%assigns $r$ to the {\em next} value of $l$. There may be multiple such
%%assignments before the clock advances. The last such assignment
%%specifies the value of the variable that will be visible after the
%%clock has advanced.  If the variable is {\cf clocked final} it is
%%guaranteed that {\em all} readers of the variable throughout this
%%phase will see the value $r$.
%%
%%The expression {\tt r} is implicitly treated as {\tt now(c) r}. That
%%is, the clock {\tt c} will not advance until the computation of {\tt r} has
%%terminated.
%%
%%\section{Examples}
%%
%%Consider the core of the ASCI Benchmark Sweep3D program for computing
%%solutions to mass transport problems.
%%
%%In a nutshell the core computation is a triply nested sequential loop
%%in which the value of a variable in the current iteration is dependent
%%on the values of neighboring variables in a past iteration. Such a
%%problem can be parallelized through pipelining. One visualizes a
%%diagonal wavefront sweeping through the array. An MPI version of the
%%program may be described as follows. There is a two dimensional grid
%%of processors which performs the following computation
%%repeatedly. Each processor synchronously receives a value from the
%%processor to its west, then to its north, then computes some function
%%of these values and computes a new value to be sent to the processor
%%to its east and then to its south.  Ignoring the behavior of the
%%boundary processors for the moment such a computation may be described
%%by the following \Xten{} program:
%%
%%\begin{x10}
%%region R = [1..n0,1..m0];
%%clock[R] W,N;
%%clock(W) final double [cyclic(R)] A; 
%%for (int t : 1..TMax) \{
%%  ateach( i,j:A) 
%%    clock (W[i-1,j],N[i,j-1],W[i,j],N[i,j]) \{
%%      double west = now (W[i-1,j]) future\{A[i-1,j]\}; 
%%      W[i-1,j].continue();           
%%      double north = now (N[i,j-1]) future\{A[i,j-1]\}; 
%%      N[i,j-1].continue();
%%      next(W[i,j]) A[i,j] = compute(west, north);
%%      next W[i-1,j],N[i,j-1],W[i,j],N[i,j];
%%  \}
%%\}
%%\end{x10}

	\par  %\vfill\eject %0.1
\chapter{Interfaces}
\label{XtenInterfaces}\index{interfaces}

{}\XtenCurrVer{} interfaces are essentially the same \java{}
interfaces \cite[\S 9]{jls2}. An interface primarily specifies
signatures for public methods. It may extend multiple interfaces. 
%The
%need for magic constants in interfaces is lessened with the
%introduction of {\tt enum} (\S~\ref{XtenEnums}).


Future version of \Xten{} will introduce additional structure in
interface definitions that will allow the programmer to state
additional properties of classes that implement that interface. For
instance a method may be declared {\tt pure} to indicate that its
evaluation cannot have any side-effects. A method may be declared {\tt
local} to indicate that its execution is confined purely to the
current place (no communication with other places). Similarly,
behavioral properties of the method as they relate to the usage of
clocks of the current activity may be specified.

	\par  %\vfill\eject % empty
\chapter{Classes}
\label{XtenClasses}\index{class}

The {\em class declaration} has
a list of type \params,
value properties, 
a constraint (the {\em class invariant}, a single superclass,
one or more interfaces, and a class body containing the
the definition of
fields, methods, and member types.
Each such declaration introduces a class
type (\Sref{ReferenceTypes}).

\begin{grammar}
NormalClassDeclaration \:
      ClassModifiers\opt \xcd"class" Identifier  \\
   && TypeParameterList\opt PropertyList\opt Guard\opt \\
   && Super\opt Interfaces\opt ClassBody \\
\\
TypeParameterList     \:  \xcd"[" TypeParameters \xcd"]" \\
TypeParameters        \:  TypeParameter ( \xcd"," Typearameter )\star \\
TypeParameter         \:  Variance\opt Annotation\star Identifier     \\
Variance \: \xcd"+" \\
         && \xcd"-" \\
\\
PropertyList     \:  \xcd"(" Properties \xcd")" \\
Properties       \:  Property ( \xcd"," Property )\star \\
Property         \:  Annotation\star \xcd"val"\opt Identifier \xcd":" Type \\
\\
Super \: \xcd"extends" ClassType \\
Interfaces \: \xcd"implements" InterfaceType ( \xcd"," InterfaceType)\star \\
\\
ClassBody \: ClassMember\star \\
ClassMember \: ClassDeclaration \\
            \| InterfaceDeclaration \\
            \| FieldDeclaration \\
            \| MethodDeclaration \\
            \| ConstructorDeclaration \\
\end{grammar}

A type parameter declaration is given by an optional variance
tag and an identifier.
A type parameter must be
bound to a concrete type when an instance of the class is created.


A value property has a name and a type.   Value properties
are accessible in the same way as \xcd"public" \xcd"final"
fields.

\begin{staticrule*}
It is a compile-time error for a class
defining a value property \xcd"x: T" to have an ancestor class that defines
a value property with the name \xcd"x".  
\end{staticrule*}

Each class \xcd"C" defining a property \xcd"x: T" implicitly has a field

\begin{xten}
public val x : T;
\end{xten} 

\noindent and a getter method

\begin{xten}
public final def x(): T { return x; }
\end{xten}

\noindent Each interface \xcd"I" defining a property \xcd"x: T"
implicitly has a getter method

\begin{xten}
public def x(): T;
\end{xten}

\begin{staticrule*}
It is a compile-time error for a class or
interface defining a property \xcd"x :T" to have an existing method with
the signature \xcd"x(): T".
\end{staticrule*}

Properties are used to build dependent types from classes, as
described in \Sref{DepType:DepType}.

\label{ClassGuard}

The \grammarrule{Guard} in a class or interface declaration specifies an
explicit condition on the properties of the type, and is discussed further
in \Sref{DepType:Guard}.

\begin{staticrule*}
     Every constructor for a class defining
   properties \xcdmath"x$_1$: T$_1$, $\ldots$, x$_n$: T$_n$" must ensure that each of the fields
   corresponding to the properties is definitely initialized
   (cf. requirement on initialization of final fields in Java) before the
   constructor returns.
\end{staticrule*}

Type \params{}
are used to define generic classes and
interfaces, as described in \Sref{Generics}.

Classes are structured in a single-inheritance code
hierarchy, may implement multiple interfaces, may have static and
instance fields, may have static and instance methods, may have
constructors, may have static and instance initializers, may have
static and instance inner classes and interfaces. \Xten{} does not
permit mutable static state, so the role of static methods and
initializers is quite limited. Instead programmers should use
singleton classes to carry mutable static state.

Method signatures may specify checked exceptions. Method definitions
may be overridden by subclasses; the overriding definition may have a
declared return type that is a subclass of the return type of the
definition being overridden. Multiple methods with the same name but
different signatures may be provided on a class (ad hoc
polymorphism). The public/private/protected/package-protected access
modification framework may be used.

\oldtodo{Add the new rule for preventing leakage of this from a constructor.}

Because of its different concurrency model, \Xten{} does not support
\xcd"transient" and \xcd"volatile" field modifiers.

\oldtodo{Figure out class modifiers. Figure out which new ones need to be added to support IEEE FP.}

\section{Reference classes}\index{class!reference class}\label{ReferenceClasses}
A reference class is declared with the optional keyword \xcd"reference" preceding \xcd"class" in a class declaration. Reference
class declarations may be used to construct reference types
(\Sref{ReferenceTypes}). Reference classes may have mutable
fields. Instances of a reference class are always created in a fixed
place and in \XtenCurrVer{} stay there for the lifetime of the
object. (Future versions of \Xten{} may support object migration.)
Variables declared at a reference type always store a reference to the
object, regardless of whether the object is local or remote.

\section{Value classes}\index{class!value class}\label{ValueClasses}

{}\Xten{} singles out a certain set of classes for additional
support. A class is said to be {\em stateless} if all of its fields
are declared to be \xcd"final" (\Sref{FinalVariable}), otherwise it
is {\em stateful}. (\Xten{} has syntax for specifying an array class
with final fields, unlike \java{}.) A {\em stateless (stateful)
object} is an instance of a stateless (stateful) class.

{}\Xten{} allows the programmer to signify that a class (and all its
descendents) are stateless. Such a class is called a {\em value
class}.  The programmer specifies a value class by prefixing the
modifier \xcd"value" before the keyword \xcd"class" in a class
declaration.  (A class not declared to be a value class will be called
a {\em reference class}.)  Each instance field of a value class is
treated as \xcd"final". It is legal (but neither required nor recommended)
for fields in a value class to be declared final. For brevity, the \Xten{}
compiler allows the programmer to omit the keyword \xcd"class" after
\xcd" value" in a value class declaration.


\begin{grammar}
ValueClassDeclaration \:
      ClassModifiers\opt \xcd"value" \xcd"class"\opt Identifier  \\
   && TypePropertyList\opt PropertyList\opt Guard\opt \\
   && Super\opt Interfaces\opt ValueClassBody \\
\end{grammar}


The \xcd"Box" type constructor (\Sref{BoxType}) can
be used to declare variables whose value may be \xcd"null" or a value
type.

Stable equality for value types is defined through a deep walk,
bottoming out in fields of reference types (\Sref{StableEquality}).

\begin{staticrule*}
It is a compile-time error for a value class to inherit from a
stateful class or for a reference class to inherit from a value
class. All fields of a value class are implicitly declared \xcd"final".
\end{staticrule*}

\subsection{Representation}

Since value objects do not contain any updatable locations, they can
be freely copied from place to place. An implementation may use
copying techniques even within a place to implement value types,
rather than references. This is transparent to the programmer.

More explicitly, \Xten{} guarantees that an implementation must always
behave as if a variable of a reference type takes up as much space as
needed to store a reference that is either null or is bound to an
object allocated on the (appropriate) heap. However, \Xten{} makes no
such guarantees about the representation of a variable of value
type. The implementation is free to behave as if the value is stored
``inline'', allocated on the heap (and a reference stored in the
variable) or use any other scheme (such as structure-sharing) it may
deem appropriate. Indeed, an implementation may even dynamically
change the representation of an object of a value type, or dynamically
use different representations for different instances (that is,
implement automatic box/unboxing of values).

Implementations are strongly encouraged to implement value types as
space-efficiently as possible (e.g., inlining them or passing them in
registers, as appropriate).  Implementations are expected to cache
values of remote final value variables by default. If a value is
large, the programmer may wish to consider spawning a remote activity
(at the place the value was created) rather than referencing the
containing variable (thus forcing it to be cached).

\oldtodo{Need to figure out whether we should let the programmer be
aware of lazy pull vs full-value push of value objects. This is the
idea of introducing a *-annotation. Need to make a decision on
this. Could leave this for 0.7.}

\begin{example}
A functional \xcd"LinkedList" program may be written as follows:


\begin{xten}
value LinkedList { 
  val first: Object;
  val rest: LinkedList;
  public def this(first: Object) {
     this(first, null);
  }
  public def this(first: Object, rest: LinkedList) {
    this.first = first;
    this.rest = rest;
  }
  public def first(): Object {
    return first;
  }
  public def rest(): LinkedList {
    return rest;
  } 
  public def append(l: LinkedList): LinkedList {
    return (this.rest == null) 
        ? new LinkedList(this.first, l) 
        : this.rest.append(l);
  }
}
\end{xten}

Similarly, a \xcd"Complex" class may be implemented as follows:
\begin{xten}
value Complex { 
  re: Double;
  im: Double;
  public def this(re: Double, im: Double) {
     this.re=re;
     this.im=im;
  }
  public def add(other: Complex): Complex {
    return new Complex(this.re+other.re,
                       this.im+other.im);
  }
  public def mult(other: Complex): Complex {
    return new Complex(this.re^2-other.re^2,
                       2*this.im*other.im);
  }
  ...
}
\end{xten}
\end{example}

\section{Type invariants}
\index{type invariants}
\index{guards}

There is a general recipe for constructing a list of parameters or
properties \xcdmath"x$_1$: T$_1${c$_1$}, $\dots$, x$_k$: T$_k${c$_k$}" that must satisfy a given
(satisfiable) constraint \xcd"c". 

\begin{xtenmath}
class Foo(x$_1$: T1{x$_2$: T$_2$; $\dots$; x$_k$: T$_k$; c},
          x$_2$: T2{x$_3$: T$_3$; $\dots$; x$_k$: T$_k$; c},
          $\dots$
          x$_k$: T$_k${c}) {
  $\dots$
}
\end{xtenmath}

The first type \xcdmath"x$_1$: T$_1${x$_2$: T$_2$; $\dots$; x$_k$: T$_k$; c}" is consistent iff
\xcdmath"$\exists$x$_1$: T$_1$, x$_2$: T$_2$, $\dots$, x$_k$: T$_k$. c" is consistent. The second is
consistent iff
\begin{xtenmath}
$\forall$x$_1$: T$_1${x$_2$: T$_2$; $\dots$; x$_k$: T$_k$; c}
$\exists$x$_2$: T$_2$. $\exists$x$_3$: T$_3$, $\dots$, x$_k$: T$_k$. c
\end{xtenmath}
\noindent But this is always true. Similarly for the conditions for the other
properties.

Thus logically every satisfiable constraint \xcd"c" on a list of parameters
\xcdmath"x$_1$", \dots, \xcdmath"x$_k$"
can be expressed using the dependent types of 
\xcdmath"x$_i$", provided
that the constraint language is rich enough to permit existential
quantifiers.

Nevertheless we will find it convenient to permit the programmer to
explicitly specify a depclause after the list of properties, thus:
\begin{xten}
class Point(i: Int, j: Int) { ... }
class Line(start: Point, end: Point){end != start}
  = { ... }
class Triangle (a: Line, b: Line, c: Line)
        {a.end == b.start && b.end == c.start &&
         c.end == a.start} = { ... }
class SolvableQuad(a: Int, b: Int, c: Int)
                   {a*x*x+b*x+c==0} = { ... }
class Circle (r: Int, x: Int, y: Int)
              {r > 0 && r*r==x*x+y*y} = { ... }
class NonEmptyList extends List{n > 0} {...}
\end{xten}

Consider the definition of the class \xcd"Line". This may be thought of as
saying: the class \xcd"Line" has two fields, \xcd"start: Point" and
\xcd"end: Point".
Further, every instance of \xcd"Line" must satisfy the constraint that
\xcd"end != start". Similarly for the other class definitions. 

In the general case, the production for \grammarrule{NormalClassDeclaration}
specifies that the list of properties may be followed by a
\grammarrule{Guard}.

\begin{grammar}
NormalClassDeclaration \:
      ClassModifiers\opt \xcd"class" Identifier  \\
   && TypeParameterList\opt PropertyList\opt Guard\opt \\
   && Extends\opt Interfaces\opt ClassBody \\
\\
NormalInterfaceDeclaration \:
      InterfaceModifiers\opt \xcd"interface" Identifier  \\
   && TypeParameterList\opt PropertyList\opt Guard\opt \\
   && ExtendsInterfaces\opt InterfaceBody \\
\end{grammar}

All the properties in the list, together with inherited properties,
may appear in the \grammarrule{Guard}. A guard \xcd"c" with
value property list \xcdmath"x$_1$: T$_1$, $\dots$, x$_n$: T$_n$"
for a class \xcd"C" is said to be consistent if each of the \xcdmath"T$_i$" are
consistent and the constraint
\begin{xtenmath}
$\exists$x$_1$: T$_1$, $\dots$, x$_n$: T$_n$, self: C. c
\end{xtenmath}
\noindent is valid (always true).

\section{Class definitions}

Consider a class definition
\begin{xtenmath}
$\mbox{\emph{ClassModifiers}}^{\mbox{?}}$
class C(x$_1$: P$_1$, $\dots$, x$_n$: P$_n$) extends D{d}
   implements I$_1${c$_1$}, $\dots$, I$_k${c$_k$}
$\mbox{\emph{ClassBody}}$
\end{xtenmath}

Each of the following static semantics rules must be satisfied:

\begin{staticrule}{Int-implements}
The type invariant \xcdmath"$\mathit{inv}$(C)" of \xcd"C" must entail
\xcdmath"c$_i$[this/self]" for each $i$ in $\{1, \dots, k\}$
\end{staticrule}

\begin{staticrule}{Super-extends}
The return type \xcd"c" of each constructor in \grammarrule{ClassBody}
must entail \xcd"d".
\end{staticrule}

\section{Constructor definitions}

A constructor for a class \xcd"C" is guaranteed to return an object of the
class on successful termination. This object must satisfy i(C), the
class invariant associated with \xcd"C" (\Sref{DepType:TypeInvariant}).
However,
often the objects returned by a constructor may satisfy {\em stronger}
properties than the class invariant. \Xten{}'s dependent type system
permits these extra properties to be asserted with the constructor in
the form of a constrained type (the ``return type'' of the constructor):

\begin{grammar}
ConstructorDeclarator \:
  \xcd"def" \xcd"this" TypeParameterList\opt \xcd"(" FormalParameterList\opt \xcd")" \\
  && ReturnType\opt Guard\opt Throws\opt \\
ReturnType    \: \xcd":" Type \\
Guard   \: "{" DepExpression "}" \\
Throws    \: \xcd"throws" ExceptionType  ( \xcd"," ExceptionType )\star \\
ExceptionType \: ClassBaseType Annotation\star \\
\end{grammar}

\label{ConstructorGuard}

The parameter list for the constructor
may specify a \emph{guard} that is to be satisfied by the parameters
to the list.

\begin{example}
Here is another example.
\begin{xten}
public class List[T](n: Int{n >= 0}) {
    protected head: Box[T];
    protected tail: List[T](n-1);
    public def this(o: T, t: List[T]) : List[T](t.n+1) = {
        n = t.n+1;
        tail = t;
        head = o;
    }
    public def this() : List[T](0) = {
        n = 0;
        head = null;
        tail = null;
    }
    ...
}
\end{xten}
The second constructor returns a \xcd"List" that is guaranteed to have
\xcd"n==0";
the first constructor is guaranteed to return a List with \xcd"n>0"
(in fact, \xcd"n==t.n+1", where the argument to the constructor is \xcd"t"). 
This is recorded by the programmer in the constrained type associated with the
constructor.
\end{example}

\begin{staticrule}{Super-invoke}
   Let \xcd"C" be a class with properties
   \xcdmath"p$_1$: P$_1$, $\dots$, p$_n$: P$_n$", invariant \xcd"c"
   extending the constrained type \xcd"D{d}" (where \xcd"D" is the name of a class).

   For every constructor in \xcd"C" the compiler checks that the call to
   super invokes a constructor for \xcd"D" whose return type is strong enough
   to entail \xcd"d". Specifically, if the call to super is of the form 
     \xcdmath"super(e$_1$, $\dots$, e$_k$)"
   and the static type of each expression \xcdmath"e$_i$" is
   \xcdmath"S$_i$", and the invocation
   is statically resolved to a constructor
\xcdmath"def this(x$_1$: T$_1$, $\dots$, x$_k$: T$_k$){c}: D{d$_1$}"
   then it must be the case that 
\begin{xtenmath}
x$_1$: S$_1$, $\dots$, x$_i$: S$_i$ $\vdash$ x$_i$: T$_i$  (for $i \in \{1, \dots, k\}$)
x$_1$: S$_1$, $\dots$, x$_k$: S$_k$ $\vdash$ c  
d$_1$[a/self] && x$_1$: S$_1$ ... && x$_k$: S$_k$ $\vdash$ d[a/self]      
\end{xtenmath}
\noindent where \xcd"a" is a constant that does not appear in 
\xcdmath"x$_1$: S$_1$ $\wedge$ ... $\wedge$ x$_k$: S$_k$".
  
\end{staticrule}

\begin{staticrule}{Constructor return}
   The compiler checks that every constructor for \xcd"C" ensures that
   the properties \xcdmath"p$_1$,..., p$_n$" are initialized with values which satisfy
   \xcdmath"t(C)", and its own return type \xcd"c'" as follows.  In each constructor, the
   compiler checks that the static types \xcdmath"T$_i$" of the expressions \xcdmath"e$_i$"
   assigned to \xcdmath"p$_i$" are such that the following is
   true:
\begin{xtenmath}
p$_1$: T$_1$, $\dots$, p$_n$: T$_n$ $\vdash$ t(C) $\wedge$ c'     
\end{xtenmath}
\end{staticrule}
(Note that for the assignment of \xcdmath"e$_i$" to \xcdmath"p$_i$"
to be type-correct it must be the
    case that \xcdmath"p$_i$: T$_i$ $\wedge$ p$_i$: P$_i$".) 


\begin{staticrule}{Constructor invocation}
The compiler must check that every invocation \xcdmath"C(e$_1$, $\dots$, e$_n$)" to a
constructor is type correct: each argument \xcdmath"e$_i$" must have a static type
that is a subtype of the declared type \xcdmath"T$_i$" for the $i$th
argument of the
constructor, and the conjunction of static types of the argument must
entail the \grammarrule{Guard} in the parameter list of the constructor.
\end{staticrule}

\section{Field definitions}

Not every instance of a class needs to have every field defined on the
class. In Java-like languages this is ensured by conditionally setting
fields to a default value, such as \xcd"null", in those instances where the
fields are not needed.  

Consider the class \xcd"List" used earlier.  Here all instances of \xcd"List"
returned by the second constructor do not need the fields \xcd"value" and
\xcd"tail"; their value is set to null.

\label{FieldGuard}

\Xten{} permits a much cleaner solution that does not require default
values such as null to be stored in such fields. \Xten{} permits fields to
be {\em guarded} with a constraint.  The field is accessible
only if the \emph{guard} constraint is satisified.

\begin{grammar}
FieldDeclaration  \:
   FieldModifiers\opt \xcd"val" VariableDeclarators \xcd";" \\
   \|
   FieldModifiers\opt \xcd"var" VariableDeclarators \xcd";" \\
VariableDeclarators \:
        VariableDeclarator ( \xcd"," VariableDeclarator )\star \\
VariableDeclarator \:
   Identifier ( Constraint )\opt ( \xcd":" Type )\opt ( \xcd"=" Expression )\opt \\
\end{grammar}

It is illegal for code to access a guarded field through a reference
whose static type does not satisfy the associated guard, even
implicitly (i.e., through an implicit \xcd"this"). Rather the source
program should contain an explict cast, e.g., \xcd"me: C{c} = this as C{c}".

\begin{staticrule*}
Let \xcd"f" be a field defined in class
\xcd"C" with guard \xcd"c".  The compiler declares an error if
field \xcd"f" is accessed through a reference \xcd"o" whose static
type is not a subtype of \xcd"C{c}".
\end{staticrule*}

\begin{example}

We may now rewrite the List example:
\begin{xten}
public class List(n: Int{n>=0}) {
  protected val head{n>0}: Object;
  protected val tail{n>0}: List(n-1);
  public def this(o: Object, t: List): List(t.n+1) {
     property(t.n+1);
     head=o;
     tail=t;
  }
  public def this(): List(0) {
     property(0);
  }
  ...
}
\end{xten}

The fields \xcd"value" and \xcd"tail" do not exist for instances of the class
\xcd"List(0)".
\end{example}

It is a compile-time error for a class to have two fields of the same
name, even if their guards are different. A class \xcd"C" with a field
named \xcd"f" is said to {\em hide} a field in a superclass named \xcd"f".

\begin{staticrule*}
     A class may not declare two fields with the same name.
\end{staticrule*}

To avoid an ambiguity, it is a static error for a class to
declare a field with a function type (\Sref{FunctionTypes}) with
the same name and signature  as a method of the same class.

\subsection{Field hiding}

A subclass that defines a field \xcd"f" hides any field \xcd"f"
declared in a superclass, regardless of their types.  The
superclass field \xcd"f" may be accessed within the body of
the subclass via the reference \xcd"super.f".

\eat{
The definition of field hiding does not take
\grammar{Guard} into
account. Suppose a class \xcd"C" has a field

\begin{xten}
var f{c}: Foo;
\end{xten}
\noindent and a subclass \xcd"D" of \xcd"C" has a field
\begin{xten}
var f{d}: Fum;
\end{xten}

We will say that \xcd"D.f" hides \xcd"C.f", {\em regardless} of the
constraints \xcd"c" and \xcd"d". This is in keeping with \Java, and
permits a naive implementation which always allocates space for a
conditional field.

\begin{rationale}
It might seem attractive to require that \xcd"D.f"
hides \xcd"C.f" only if \xcd"d" implies \xcd"c". This would seem
to necessitate a rather complex implementation structure for classes,
requiring some kind of a heterogenous translation for
constrained types of \xcd"C"
and \xcd"D". This bears further investigation.
\end{rationale}
}

\section{Method definitions}

\Xten{} permits guarded method definitions, similar to guarded
field definitions. Additionally, the parameter list for a method may
contain a \grammarrule{Guard}.

\begin{grammar}
MethodDeclaration \: MethodHeader \xcd";" \\
                  \| MethodHeader \xcd"=" ClosureBody \\
MethodHeader \:  
  MethodModifiers\opt \xcd"def" Identifier TypeParameters\opt \\
&& \xcd"(" 
  FormalParameterList\opt \xcd")" Guard\opt \\
  && ReturnType\opt Throws\opt \\
\end{grammar}

In the formal parameter list, variables may be declared with
\xcd"val" or \xcd"var".  If neither is specified, the variable
is \xcd"val".

\label{MethodGuard}

The guard (specified by \grammarrule{Guard})
specifies a constraint \xcd"c" on the
properties of the class \xcd"C" on which the method is being defined. The
method exists only for those instances of \xcd"C" which satisfy \xcd"c".  It is
illegal for code to invoke the method on objects whose static type is
not a subtype of \xcd"C{c}".

\begin{staticrule*}
    The compiler checks that every method invocation
    \xcdmath"o.m(e$_1$, $\dots$, e$_n$)"
    for a method is type correct. Each each argument
    \xcdmath"e$_i$" must have a
    static type \xcdmath"S$_i$" that is a subtype of the declared type
    \xcdmath"T$_i$" for the $i$th
    argument of the method, and the conjunction of static types
    of the arguments must entail the guard in the parameter list
    of the method.

    The compiler checks that in every method invocation
    \xcdmath"o.m(e$_1$, $\dots$, e$_n$)"
    the static type of \xcd"o", \xcd"S", is a subtype of \xcd"C{c}", where the method
    is defined in class \xcd"C" and the guard for \xcd"m" is equivalent to
    \xcd"c".

    Finally, if the declared return type of the method is
    \xcd"D{d}", the
    return type computed for the call is
    \xcdmath"D{a: S; x$_1$: S$_1$; $\dots$; x$_n$: S$_n$; d[a/this]}",
    where \xcd"a" is a new
    variable that does not occur in
    \xcdmath"d, S, S$_1$, $\dots$, S$_n$", and
    \xcdmath"x$_1$, $\dots$, x$_n$" are the formal
    parameters of the method.
\end{staticrule*}

\begin{example}
Consider the program:
\begin{xten}
public class List(n: Int{n>=0}) {
  protected val head{n>0}: Object;
  protected val tail{n>0}: List(n-1);
  public def this(o: Object, t: List): List(t.n+1) = {
     property(t.n+1);
     head=o;
     tail=t;
  }
  public def this(): List(0) = {
     property(0);
  }
  public def append(l: List): List{self.n==this.n+l.n} = {
      return (n==0)? l
         : new List(head, tail.append(l)); 
  }
  public def nth(k: Int{1 <= k && k <= n}){n > 0}: Object = {
      return k==1 ? head : tail.nth(k-1);
  }
}
\end{xten}

The following code fragment
\begin{xten}
u: List{self.n==3} = ...
t: List{self.n==x} = ...;
s: List{self.n==x+3} = t.append(u);
\end{xten}
\noindent will typecheck. The type of the expression \xcd"t.append(u)" is 
\begin{xten}
List{a: List{self.n==x}; 
     l: List{self.n==3}; self.n==a.n+l.n}  
\end{xten}
\noindent which is equivalent to:
\begin{xten}
List{self.n==x+3}
\end{xten}
\end{example}

The method body is either an expression, a block of statements,
or a block ending with an expression.

\subsection{Property methods}

A method declared with the modifier \xcd"property" may be used
in constraints.  A property method declared in a class must have
a body and must not be \xcd"void".  The body of the method must
consist of only a single \xcd"return" statement or a single
expression.  It is a static error of the expression cannot be
represented in the constraint system.

Property methods in classes are implicitly \xcd"final"; they cannot be
overridden.

A property method definition may omit the formal parameters and
the \xcd"def" keyword.  That is, the following are equivalent:

\begin{xten}
property def rail(): boolean = rect && onePlace == here && zeroBased;
property rail: boolean = rect && onePlace == here && zeroBased;
\end{xten}

\subsection{Method overloading, overriding, hiding, shadowing and obscuring}
\label{MethodOverload}

The definitions of method overloading, overriding, hiding, shadowing
and obscuring in \Xten{} are the same as in \Java, modulo the following
considerations motivated by type parameters and dependent types.

Two or more methods of a class or interface may have the same
name if they have a different number of type parameters, or
they have value parameters of different types.

The definition of a method declaration \xcdmath"m$_1$" ``having the same signature
as'' a method declaration \xcdmath"m$_2$" involves identity of types. Two \Xten{} types
are defined to be identical iff they are equivalent (\Sref{DepType:Equivalence}).
Two methods are said to have {\em the same signature} if (a)
they have the same number of type parameters, (b) they have the
same number of formal (value) parameters, and (c) for each formal parameter
their types are equivalent. It is a compile-time error for there
to be two methods with the same name and same signature in a class
(either defined in that class or in a superclass).

\begin{staticrule*}
  A class \xcd"C" may not have two declarations for a method named \xcd"m"---either
  defined at \xcd"C" or inherited:
\begin{xtenmath}
def m[X$_1$, $\dots$, X$_m$](v$_1$: T$_1${t$_1$}, $\dots$, v$_n$: T$_n${t$_n$}){tc}: T {...}
def m[X$_1$, $\dots$, X$_m$](v$_1$: S$_1${s$_1$}, $\dots$, v$_n$: S$_n${s$_n$}){sc}: S {...}
\end{xtenmath}
\noindent
if it is the case that the types \xcd"C{tc}", \xcdmath"T$_1${t$_1$}",
\dots, \xcdmath"T$_n${t$_n$}" are
equivalent to the types \xcdmath"C{sc}, S$_1${t$_1$}, $\dots$, T$_n${t$_n$}"
respectively.
\end{staticrule*}

In addition, the guard of a overriding method must be 
no stronger than the guard of the overridden method.   This
ensures that any virtual call to the method
satisfies the guard of the callee.

\begin{staticrule*}
  If a class \xcd"C" overrides a method of a class or interface
  \xcd"B", the guard of the method in \xcd"B" must entail
  the guard of the method in \xcd"C".
\end{staticrule*}

A class \xcd"C" inherits from its direct superclass and superinterfaces all
their methods visible according to the access restriction modifiers
public/private/protected/(package) of the superclass/superinterfaces
that are not hidden or overridden. A method \xcdmath"M$_1$" in a class
\xcd"C" overrides
a method \xcdmath"M$_2$" in a superclass \xcd"D" if
\xcdmath"M$_1$" and \xcdmath"M$_2$" have the same signature.
Methods are overriden on a signature-by-signature basis.

A method invocation \xcdmath"o.m(e$_1$, $\dots$, e$_n$)"
is said to have the {\em static signature}
\xcdmath"<T, T$_1$, $\dots$, T$_n$>" where \xcd"T" is the static type of
\xcd"o", and
\xcdmath"T$_1$",
\dots,
\xcdmath"T$_n$"
are the static types of \xcdmath"e$_1$", \dots, \xcdmath"e$_n$",
respectively.  As in
\Java, it must be the case that the compiler can determine a single
method defined on \xcd"T" with argument type
\xcdmath"T$_1$", \dots \xcdmath"T$_n$"; otherwise, a
compile-time error is declared. However, unlike \Java, the \Xten{} type \xcd"T"
may be a dependent type \xcd"C{c}". Therefore, given a class definition for
\xcd"C" we must determine which methods of \xcd"C" are available at a type
\xcd"C{c}". But the answer to this question is clear: exactly those methods
defined on \xcd"C" are available at the type \xcd"C{c}"
whose guard \xcd"d" is implied by \xcd"c".


\begin{example}
  Consider the definitions:
\begin{xten}
class Point(i: Int, j: Int) {...}
class Line(s: Point, e: Point{self != i}) {
  // m1: Both points lie in the right half of the plane
  def draw(){s.i>= 0 && e.i >= 0} = {...}
  // m2: Both points lie on the y-axis
  def draw(){s.i== 0 && e.i == 0} = {...}
  // m3: Both points lie in the top half of the plane
  def draw(){s.j>= 0 && e.j >= 0} = {...}
  // m4: The general method
  def draw() = {...}
} 
\end{xten} 
\noindent  Three different implementations are given for the
\xcd"draw" method, one
  for the case in which the line lies in the right half of the plane,
  one for the case that the line lies on the y-axis and the third for
  the case that the line lies in the top half of the plane.


\noindent  Consider the invocation
\begin{xten}
m: Line{s.i < 0} = ...
m.draw();
\end{xten}

\noindent  This generates a compile time error because there is no applicable
  method definition.

\noindent  Consider the invocation

\begin{xten}
m: Line{s.i>=0 && s.j>=0 && e.i>=0 && e.j>=0} = ...
m.draw();
\end{xten}

\noindent  This generates a compile time error because both
\xcd"m1" and \xcd"m3" are applicable.

\noindent  Consider the invocation
\begin{xten}
m: Line{s.i>=0 && s.j>=0 && e.i>=0} = ...
m.draw();
\end{xten}
  This does not generate any compile-time error since only m1 is
  applicable. 
\end{example}


In the last example, notice that at runtime \xcd"m1" will be invoked
(assuming \xcd"m" contains an instance of the \xcd"Line" class, and not some
subclass of \xcd"Line" that overrides this method). This will be the case
even if \xcd"m" satisfies at runtime the stronger conditions for \xcd"m2" (i.e.,
\xcd"s.i==0 && e.i==0"). That is, dynamic method lookup will not take into
account the  ``strongest'' constraint that the receiver may
satisfy, i.e.,
its ``strongest constrained type''. 

\begin{rationale}
  The design decision that dynamic method lookup should ignore
  dependent type information was made to keep the design and the
  implementation simple and to ensure that serious errors such as
  method invocation errors are captured at compile-time.
 
  Consider the above example and the invocation
\begin{xten}
m: Line = ...
m.draw();    
\end{xten}


   Statically the compiler will not report an error because m4 is the
   only method that is applicable. However, if dynamic method lookup
   were to use constrained types then we would face the problem that if m is a
   line that lives in the upper right quadrant then both \xcd"m2"
   and \xcd"m3"
   are applicable and one does not override the other. Hence we must
   report an error dynamically.

   As discussed above, the programmer can write code with \xcd"instanceof"
   and class casts that perform any application-appropriate
   discrimination.  
\end{rationale}

\subsection{Method annotations}

\subsubsection{\Xcd{atomic} annotation}

A method may be declared \xcd"atomic".

\begin{grammar}
  MethodModifier \: \xcd"atomic"  
\end{grammar}

Such a method is treated as if the statement in its body is wrapped 
implicitly in an \xcd"atomic" statement.

\subsubsection{\Xcd{local} annotation}\label{LocalAnnotation}\index{local!\xcd"local"}

A method may be declared \xcd"local".

\begin{grammar}
  MethodModifier \: \xcd"local"  
\end{grammar}

By declaring a method \xcd"local" the programmer asserts that while
executing this method an activity will only access local memory.

The compiler implements the following rules to guarantee this property.

Let \xcd"o" be any expression occurring in the body of the
method. Assume its static datatype is \xcd"F". 

\begin{itemize}
\item Local methods can only be overridden by local methods. 

\item If the body of the method contains any field access \xcd"o.e", then
the static placetype of \xcd"o" must be \xcd"here". 

The programmer can always ensure that this condition is satisfied
(albeit at the risk of introducing a runtime exception) by replacing
each field access \xcd"o.e" with \xcd"(o as F!here).e".

\item If the body of the method contains any assignments to fields
(e.g. \xcd"o.e Op= t", or \xcd"Op o.e" or \xcd"o.e Op") then the
static placetype of \xcd"o" must be \xcd"here".

The programmer can always ensure that this condition is satisfied by
replacing \xcd"o.e Op= t" by \xcd"o1.e Op=t" and preceding it (in the
same basic block) with the local variable declaration \xcd"o1: F!here = o as F!here" (for some new local variable \xcd"o1"). Similarly for
\xcd"Op o.e" and \xcd"o.e Op".

\item Recall that the static placetype of an array access \xcd"o(e)"
is \xcd"o.dist(e)". Therefore, any read/write array access
\xcd"o(e)" must be guarded by the condition \xcd"o.dist(e) == here".  (Since  \xcd"e" may have side-effects, the compiler must
ensure that the place check uses the value returned by the same
expression evaluation that is used to access the array element.)

\item If the body of the method contains any method invocation
\xcdmath"o.m(t$_1$,$\dots$,t$_k$)" then the method invoked must be local. Additionally,
the static place type of \xcd"o" must be \xcd"here". 
As above, the programmer can always ensure the second
condition is satisfied by writing such a method invocation
as \xcdmath"(o as F!here).m(t$_1$,$\dots$,t$_k$)".
\end{itemize}

Note that reads/writes to local variables or method parameters are
always local, hence the compiler does not have to check any extra
conditions.

A method declared \xcd"atomic" is automatically declared
to be \xcd"local".
	\par % 0.1
\chapter{Arrays}\label{XtenArrays}\index{arrays}

An array is a mapping from a region (set of points) to a range data
type distributed over one or more places.
Multiple arrays may be declared with the same underlying
distribution.
The distribution underlying an array \xcd"a" may be obtained through
the field \xcd"a.dist".
\index{arrays!distribution@{\tt distribution}}

\section{Points}\label{point-syntax}\index{point syntax}

Arrays are indexed by points--$n$-dimensional tuples of
integers, implemented by the class \xcd"x10.lang.Point".
\Xten{} specifies a simple syntax for the construction of points.
A rail constructor (\Sref{RailConstructors}) of type \xcd"ValRail[Int]"
%or
%\xcd"ValRail[Long]" array
of length $n$
can be implicitly coerced to a \xcd"Point" of rank $n$.  For
example, the following code initializes \xcd"p" to a point of
rank two using a rail constructor:

\begin{xten}
p: Point = [1,2];
\end{xten}

The \xcd"Point" constructor can take a rail constructor as
argument.  The assignment above can be written, without
implicit coercion, as:

\begin{xten}
p: Point = new Point([1,2]);
\end{xten}

Points implement the function type \xcd"(Int) => Int"; thus, the
\xcd"i"the element of a point \xcd"p" may be accessed as \xcd"p(i)".
If \xcd"i" is out of range, an
\xcd"ArrayIndexOutOfBoundsException" is thrown.

\section{Regions}\label{XtenRegions}\index{region}

A region is a set of indices (called {\em points}).  {}\Xten{}
provides a built-in value class, {\tt x10.lang.region}, to allow the
creation of new regions and to perform operations on regions. This
class is {\tt final} in {}\XtenCurrVer; future versions of the
language may permit user-definable regions. Since regions play a dual
role (values as well as types), variables of type \xcd"region" must be
initialized and are implicitly {\tt final}. Regions are first-class
objects---they may be stored in fields of objects, passed as
arguments to methods, returned from methods etc.

Each region \xcd"R" has a constant rank, \xcd"R.rank", which is a
non-negative integer. The literal \xcd"[]" represents the {\em empty
region} and has rank \xcd"0".

Here are several examples of region declarations:
\begin{xten}
Null: region = [];  // Empty 0-dimensional region          
R1: region = 1:100; // 1-dim region with extent 1..100.
R1: region = [1:100]; // Same as above.
R2: region = [0:99, -1:MAX_HEIGHT];   
R3: region = region.factory.upperTriangular(N);
R4: region = region.factory.banded(N, K);
   // A square region.
R5: region = [E, E];           
   // Same as: region above.
R6: region = [100, 100];       
   // Represents the intersection of two regions
AandB: region = A && B;       
  // represents the union of two regions
AOrB: region = A || B;        
\end{xten}

A region may be constructed using a comma-separated list of regions
(\S~\ref{point-syntax}) within square brackets, as above and represents
the Cartesian product of each of the arguments.  The bound of a
dimension may be any final variable of a fixed-point numeric
type. \XtenCurrVer{} does not support hierarchical regions.

Various built-in regions are provided through  factory
methods on {\tt region}.  For instance:\index{region!upperTriangular}
\index{region!lowerTriangular}\index{region!banded}
\begin{itemize}
{}\item \xcd"region.factory.upperTriangular(N)" returns a region corresponding
to the non-zero indices in an upper-triangular \xcd"N x N" matrix.
{}\item \xcd"region.factory.lowerTriangular(N)" returns a region corresponding
to the non-zero indices in a lower-triangular \xcd"N x N" matrix.
{}\item \xcd"region.banded(N, K)" returns a region corresponding to
the non-zero indices in a banded \xcd"N x N" matrix where the width of
the band is \xcd"K"
\end{itemize}

All the points in a region are ordered canonically by the lexicographic total order. Thus the points of a region \xcd"R=[1:2,1:2]" are ordered as 
\begin{xten}
(1,1), (1,2), (2,1), (2,2)
\end{xten}
Sequential iteration statements such as \xcd"for" (\S~\ref{ForAllLoop})
iterate over the points in a region in the canonical order.

A region is said to be {\em convex}\index{region!convex} if it is of
the form \xcd"[T1,..., Tk]" for some set of enumerations \xcd"Ti". Such a
region satisfies the property that if two points $p_1$ and $p_3$ are
in the region, then so is every point $p_2$ between them. (Note that
\xcd"||" may produce non-convex regions from convex regions, e.g.{}
\xcd"[1,1] || [3,3]" is a non-convex region.)

For each region \xcd"R", the {\em convex closure} of \xcd"R" is the
smallest convex region enclosing \xcd"R".  For each integer \xcd"i"
less than \xcd"R.rank", the term \xcd"R[i]" represents the enumeration
in the \xcd"i"th dimension of the convex closure of \xcd"R". It may be
used in a type expression wherever an enumeration may be used.

\subsection{Operations on regions}
Various non side-effecting operators (i.e.{} pure functions) are
provided on regions. These allow the programmer to express sparse as
well as dense regions.

Let \xcd"R" be a region. A subset of \xcd"R" is also called a {\em
sub-region}.\index{region!sub-region}

Let \xcd"R1" and \xcd"R2" be two regions.

\xcd"R1 && R2" is the intersection of \xcd"R1" and \xcd"R2".\index{region!intersection}

\xcd"R1 || R2" is the union of the \xcd"R1" and \xcd"R2".\index{region!union}

\xcd"R1 - R2" is the set difference of \xcd"R1" and \xcd"R2".\index{region!set difference}

Two regions are equal ({\tt ==}) if they represent the same set of
points.\index{region!==}


\todo{ Need to determine if regions can be passed to and returned from
methods.}

\todo{Can objects have region fields?}

\todo{ Need to determine if Xten control constructs already provide the nesting of regions of ZPL.}

\todo{ Need to determine if directions (and "of", wrap, reflect) make sense and should be included in Xten.}

\todo{ Need to add translations (ZPL @). Check whether they are widely useful.}

\todo{ Need to determine if {\tt index<d>} arrays are useful enough to include them.}




\section{Distributions}\label{XtenDistributions}
\index{distribution}

A {\em distribution} is a mapping from a region to a set of places.
{}\Xten{} provides a built-in value class, \xcd"x10.lang.Dist", to allow the creation of new distributions and
to perform operations on distributions. This class is \xcd"final" in
{}\XtenCurrVer; future versions of the language may permit
user-definable distributions. Since distributions play a dual role
(values as well as types), variables of type \xcd"Dist" must
be initialized and are implicitly \xcd"final".

The {\em rank} of a distribution is the rank of the underlying region.

%Recall that each program runs in a fixed number of places, determined
%by runtime parameters. The static constant place.MAX_PLACES specifies
%the maximum number of places. The collection of places is assumed to
%be totally ordered.


\begin{xten}
R: region = 1..100;
D: distribution = distribtion.factory.block(R);
D: distribution = distribution.factory.cyclic(R);
D: distribution = R -> here;
D: distribution = distribution.factory.random(R);
\end{xten}

Let \xcd"D" be a distribution. \xcd"D.region" denotes the underlying
region. \xcd"D.places" is the set of places constituting the range of
\xcd"D" (viewed as a function). Given a point \xcd"p", the expression
\xcd"D(p)" represents the application of \xcd"D" to \xcd"p", that is,
the place that \xcd"p" is mapped to by \xcd"D". The evaluation of the
expression \xcd"D(p)" throws an \xcd"ArrayIndexOutofBoundsException"
if \xcd"p" does not lie in the underlying region.

When operated on as a distribution, a region \xcd"R" implicitly
behaves as the distribution mapping each item in \xcd"R" to \xcd"here"
(i.e., \xcd"R->here", see below). Conversely, when used in a context
expecting a region, a distribution \xcd"D" should be thought of as
standing for \xcd"D.region".

{}\oldtodo{Allan: We do not specify how the values of an array at a place
are stored, e.g. in row-major or column major order. Need to work this
out.}

\subsection{Operations returning distributions}

Let \xcd"R" be a region, \xcd"Q" a set of places \{\xcd"p1", \dots, \xcd"pk"\}
(enumerated in canonical order), and \xcd"P" a place. All the operations
described below may be performed on \xcd"Dist.factory".

\paragraph{Unique distribution} \index{distribution!unique}
The distribution \xcd"unique(Q)" is the unique distribution from the
region \xcd"1:k" to \xcd"Q" mapping each point \xcd"i" to \xcd"pi".

\paragraph{Constant distributions.} \index{distribution!constant}
The distribution \xcd"R->P" maps every point in \xcd"R" to \xcd"P".

\paragraph{Block distributions.}\index{distribution!block}
The distribution \xcd"block(R, Q)" distributes the elements of \xcd"R"
(in order) over the set of places \xcd"Q" in blocks  as
follows. Let $p$ equal \xcd"|R| div N" and $q$ equal \xcd"|R| mod N",
where \xcd"N" is the size of \xcd"Q", and 
\xcd"|R|" is the size of \xcd"R".  The first $q$ places get
successive blocks of size $(p+1)$ and the remaining places get blocks of
size $p$.

The distribution \xcd"block(R)" is the same distribution as {\cf
block(R, place.places)}.

\oldtodo{Check into block distributions per dimension.}
\paragraph{Cyclic distributions.} \index{distribution!cyclic}
The distribution \xcd"cyclic(R, Q)" distributes the points in \xcd"R"
cyclically across places in \xcd"Q" in order.

The distribution \xcd"cyclic(R)" is the same distribution as \xcd"cyclic(R, place.places)".

Thus the distribution \xcd"cyclic(place.MAX_PLACES)" provides a 1--1
mapping from the region \xcd"place.MAX_PLACES" to the set of all
places and is the same as the distribution \xcd"unique(place.places)".

\paragraph{Block cyclic distributions.}\index{distribution!block cyclic}
The distribution \xcd"blockCyclic(R, N, Q)" distributes the elements
of \xcd"R" cyclically over the set of places \xcd"Q" in blocks of size
\xcd"N".

\paragraph{Arbitrary distributions.} \index{distribution!arbitrary}
The distribution \xcd"arbitrary(R,Q)" arbitrarily allocates points in {\cf
R} to \xcd"Q". As above, \xcd"arbitrary(R)" is the same distribution as
\xcd"arbitrary(R, place.places)".

\oldtodo{Determine which other built-in distributions to provide.}

\paragraph{Domain Restriction.} \index{distribution!restriction!domain}

If \xcd"D" is a distribution and \xcd"R" is a sub-region of {\cf
D.domain}, then \xcd"D | R" represents the restriction of \xcd"D" to
\xcd"R".  The compiler throws an error if it cannot determine that
\xcd"R" is a sub-region of \xcd"D.domain".

\paragraph{Range Restriction.}\index{distribution!restriction!range}

If \xcd"D" is a distribution and \xcd"P" a place expression, the term
\xcd"D | P" denotes the sub-distribution of \xcd"D" defined over all the
points in the domain of \xcd"D" mapped to \xcd"P".

Note that \xcd"D | here" does not necessarily contain adjacent points
in \xcd"D.region". For instance, if \xcd"D" is a cyclic distribution,
\xcd"D | here" will typically contain points that are \xcd"P" apart,
where \xcd"P" is the number of places. An implementation may find a
way to still represent them in contiguous memory, e.g., using a
complex arithmetic function to map from the region index to an index
into the array.

\subsection{User-defined distributions}\index{distribution!user-defined}

Future versions of \Xten{} may provide user-defined distributions, in
a way that supports static reasoning.

\oldtodo{TBD. Make distribution a value type. What is the API? Return a
set of indices. For each index point, return the place. A method to
return a subdistribution given a subregion. A method to check if a
given distribution is a subdistribution. But may need to provide
methods that the compiler can use to reason about the
distribution. \\
Postpone to 0.7.}

\subsection{Operations on distributions}

A {\em sub-distribution}\index{sub-distribution} of \xcd"D" is any
distribution \xcd"E" defined on some subset of the domain of \xcd"D",
which agrees with \xcd"D" on all points in its domain. We also say
that \xcd"D" is a {\em super-distribution} of \xcd"E". A distribution
\xcd"D1" {\em is larger than} \xcd"D2" if \xcd"D1" is a
super-distribution of \xcd"D2".

Let \xcd"D1" and \xcd"D2" be two distributions.  


\paragraph{Intersection of distributions.}\index{distribution!intersection}
\xcd"D1 && D2", the intersection of \xcd"D1" and \xcd"D2", is the
largest common sub-distribution of \xcd"D1" and \xcd"D2".

\paragraph{Asymmetric union of distributions.}\index{distribution!union!asymmetric}
\xcd"D1.overlay(D2", the asymmetric union of \xcd"D1" and \xcd"D2", is the
distribution whose domain is the union of the regions of \xcd"D1" and
\xcd"D2", and whose value at each point \xcd"p" in its domain is \xcd"D2[p]"
if \xcd"p" lies in \xcd"D.domain" otherwise it is \xcd"D1[p]". (\xcd"D1" provides the defaults.)

\paragraph{Disjoint union of distributions.}\index{distribution!union!disjoint}
\xcd"D1 || D2", the disjoint union of \xcd"D1" and \xcd"D2", is
defined only if the domains of \xcd"D1" and \xcd"D2" are disjoint. Its
value is \xcd"D1.overlay(D2)" (or equivalently \xcd"D2.overlay(D1)".
(It is the least super-distribution of \xcd"D1" and \xcd"D2".)

\paragraph{Difference of distributions.}\index{distribution!difference}
\xcd"D1 - D2" is the largest sub-distribution of \xcd"D1" whose domain is
disjoint from that of \xcd"D2".


\subsection{Example}
\begin{xten}
def dotProduct(a: array[T](D), b: array[T](D)): array[double](D) {
  return (new array[T]([1:D.places]) (j: point) => (
      (new array[T](D | here) (i: point) => a[i]*b[i]).sum();
  )).sum();
}
\end{xten}

This code returns the inner product of two \xcd"T" vectors defined
over the same (otherwise unknown) distribution. The result is the sum
reduction of an array of \xcd"T" with one element at each place in the
range of \xcd"D". The value of this array at each point is the sum
reduction of the array formed by multiplying the corresponding
elements of \xcd"a" and \xcd"b" in the local sub-array at the current
place.




\section{Array initializer}\label{ArrayInitializer}\label{array!creation}

Arrays are instantiated by invoking a factory method for the
class \xcd"Array".

\eat{
An array instantiation may be annotated
\xcd"unsafe"
if it is intended to be
allocated in an unmanaged region (e.g., for communication with native
code). A value array is an immutable array. An array creation
must take either an \xcd"Int" as an argument or a \xcd"Dist". In the first
case an array is created over the distribution \xcd"[0:N-1]->here";
in the second over the given distribution. 
}

An array creation operation may also specify an initializer
function.
The function is applied in parallel
at all points in the domain of the distribution. The array
construction operation terminates locally only when the array has been
fully created and initialized (at all places in the range of the
distribution).

For instance:
\begin{xten}
val data : Array[Int]
    = Array.make[Int](1000->here, Point(i) => i);
val data2 : Array[Int]
    = Array.make[Int]([1:1000,1:1000]->here, Point(i,j) => i*j);
\end{xten}

{}\noindent 
The first declaration stores in \xcd"data" a reference to a
array with \xcd"1000" elements each of which is located in the
same place as the array. Each array component is initialized to \xcd"i".

The second declaration stores in \xcd"data2" a
2-d array over \xcd"[1:1000, 1:1000]" initialized with \xcd"i*j"
at point \xcd"[i,j]". It uses a more abbreviated form to specify 
the array initializer function.

Other examples:
\begin{xten}
val data : Array[Int]
    = Array.make[Int](1000, ((i): Point) => i*i);
val d : Array[Float](D)
    = Array.make[Float](D, ((i): Point) => 10.0*i);
val result : Array[Float](D)
    = Array.make[Float](D, ((i,j): Point) => i+j);
\end{xten}

\section{Operations on arrays}
In the following let \xcd"a" be an array with distribution \xcd"D" and
base type \xcd"T". \xcd"a" may be mutable or immutable, unless
indicated otherwise.

\subsection{Element operations}\index{array!access}
The value of \xcd"a" at a point \xcd"p" in its region of definition is
obtained by using the indexing operation \xcd"a(p)". This operation
may be used on the left hand side of an assignment operation to update
the value. The operator assignments \xcd"a(i) op= e" are also available
in \Xten{}.

For array variables, the right-hand-side of an assignment must
have the same distribution \xcd"D" as an array being assigned. This
assignment involves
control communication between the sites hosting \xcd"D". Each
site performs the assignment(s) of array components locally. The
assignment terminates when assignment has terminated at all
sites hosting \xcd"D".

\subsection{Constant promotion}\label{ConstantArray}\index{arrays!constant promotion}

For a distribution \xcd"D" and a constant or final variable \xcd"v" of
type \xcd"T" the expression \xcd"Array.make[T](D, (p: Point) => v)"
denotes the mutable array with
distribution \xcd"D" and base type \xcd"T" initialized with \xcd"v"
at every point.

\subsection{Restriction of an array}\index{array!restriction}

Let \xcd"D1" be a sub-distribution of \xcd"D". Then \xcd"a | D1"
represents the sub-array of \xcd"a" with the distribution \xcd"D1".

Recall that a rich set of operators are available on distributions
(\Sref{XtenDistributions}) to obtain sub-distributions
(e.g. restricting to a sub-region, to a specific place etc).

\subsection{Assembling an array}
Let \xcd"a1,a2" be arrays of the same base type \xcd"T" defined over
distributions \xcd"D1" and \xcd"D2" respectively. Assume that both
arrays are value or reference arrays. 
\paragraph{Assembling arrays over disjoint regions}\index{array!union!disjoint}

If \xcd"D1" and \xcd"D2" are disjoint then the expression \xcd"a1 || a2" denotes the unique array of base type \xcd"T" defined over the
distribution \xcd"D1 || D2" such that its value at point \xcd"p" is
\xcd"a1(p)" if \xcd"p" lies in \xcd"D1" and \xcd"a2(p)"
otherwise. This array is a reference (value) array if \xcd"a1" is.

\paragraph{Overlaying an array on another}\index{array!union!asymmetric}
The expression
\xcd"a1.overlay(a2)" (read: the array \xcd"a1" {\em overlaid with} \xcd"a2")
represents an array whose underlying region is the union of that of
\xcd"a1" and \xcd"a2" and whose distribution maps each point \xcd"p"
in this region to \xcd"D2(p)" if that is defined and to \xcd"D1(p)"
otherwise. The value \xcd"a1.overlay(a2)(p)" is \xcd"a2(p)" if it is defined and \xcd"a1(p)" otherwise.

This array is a reference (value) array if \xcd"a1" is.

The expression \xcd"a1.update(a2)" updates the array \xcd"a1" in place
with the result of \xcd"a1.overlay(a2)".

\oldtodo{Define Flooding of arrays}

\oldtodo{Wrapping an array}

\oldtodo{Extending an array in a given direction.}

\subsection{Global operations }

\paragraph{Pointwise operations}\label{ArrayPointwise}\index{array!pointwise operations}
The unary \xcd"lift" operation applies a function to each element of
an array, returning a new array with the same distribution.
The \xcd"lift" operation is implemented by the following method
in \xcd"Array[T]":
\begin{xten}
def lift[S](f: (T) => S): Array[S](dist);
\end{xten}

The binary \xcd"lift" operation takes a binary function and
another
array over the same distribution and applies the function
pointwise to corresponding elements of the two arrays, returning
a new array with the same distribution.
The \xcd"lift" operation is implemented by the following method
in \xcd"Array[T]":
\begin{xten}
def lift[S,R](f: (T,S) => R, Array[S](dist)): Array[R](dist);
\end{xten}

\paragraph{Reductions}\label{ArrayReductions}\index{array!reductions}

Let \xcd"f" be a function of type \xcd"(T,T)=>T".  Let
\xcd"a" be a value or reference array over base type \xcd"T".
Let \xcd"unit" be a value of type \xcd"T".
Then the
operation \xcd"a.reduce(f, unit)" returns a value of type \xcd"T" obtained
by performing \xcd"f" on all points in \xcd"a" in some order, and in
parallel.  The function \xcd"f" must be associative and
commutative.  The value \xcd"unit" should satisfy
\xcd"f(unit,x)" \xcd"==" \xcd"x" \xcd"==" \xcd"f(x,unit)".

This operation involves communication between the places over which
the array is distributed. The \Xten{} implementation guarantees that
only one value of type \xcd"T" is communicated from a place as part of
this reduction process.

\paragraph{Scans}\label{ArrayScans}\index{array!scans}

Let \xcd"f" be a reduction operator defined on type \xcd"T". Let
\xcd"a" be a value or reference array over base type \xcd"T" and
distribution \xcd"D". Then the operation \xcd"a||f()" returns an array
of base type \xcd"T" and distribution \xcd"D" whose $i$th element
(in canonical order) is obtained by performing the reduction \xcd"f"
on the first $i$ elements of \xcd"a" (in canonical order).

This operation involves communication between the places over which
the array is distributed. The \Xten{} implementation will endeavour to
minimize the communication between places to implement this operation.

Other operations on arrays may be found in \xcd"x10.lang.Array" and
other related classes.
	\par % 0.1
\chapter{Statements and Expressions}\label{XtenStatements}\index{statements}

\Xten{} inherits all the standard statements of \Java{}, with the expected semantics:

\begin{x10}
\em\tt EmptyStatement      LabeledStatement  
\em\tt ExpressionStatement IfStatement
\em\tt SwitchStatement     WhileDo
\em\tt DoWhile             ForLoop           
\em\tt BreakStatement      ContinueStatement  
\em\tt ReturnStatement   ThrowStatement
\em\tt TryStatement
\end{x10}

We focus on the new statements in \Xten. 

\section{Assignment}\index{assignment}\label{AssignmentStatement}

%It is often the case that an \Xten{} variable is assigned to only
%once. The user may declare such variables as {\cf final}. However,
%this is sometimes syntactically cumbersome.
%
%{}\Xten{} supports the syntax {\cf l := r} for assignment to mutable
%variables.  The user is strongly enouraged to use this syntax to
%assign variables that are intended to be assigned to more than
%once. The \Xten{} compiler may issue a warning if it detects code 
%that uses {\cf =} assignment statements on {\cf mutable} variables.

{}\Xten{} supports assignment {\tt l = r} to array variables. In this
case {\tt r} must have the same distribution {\tt D} as {\tt l}. This
statement involves control communication between the sites hosting
{\tt D}. Each site performs the assignment(s) of array components
locally. The assignment terminates when assignment has terminated at
all sites hosting {\tt D}.

%% TODO: Sectional assignment??

\section{Point and region construction}\label{point-syntax}\index{[] syntax}
\Xten{} specifies a simple syntax for the construction of points and regions.
\begin{x10}
281   ArgumentList ::= Expression
282      | ArgumentList , Expression
512   Primary ::= [ ArgumentList ]
\end{x10}
Each element in the argument list must be either of type {\tt int} or 
of type {\tt region}. In the former case the expression 
{\tt [ a1,..., ak ] } is treated as syntactic shorthand for
\begin{x10}
  point.factory.point(a1,..., ak)
\end{x10}
\noindent and in the latter case as shorthand for
\begin{x10}
  region.factory.region(a1,..., ak)
\end{x10}

\section{Exploded variable declarations}\label{exploded-syntax}\index{variable declarator!exploded}

\Xten{} permits a richer form of specification for variable
declarators in method arguments, local variables and loop variables
(the ``exploded'' or {\em destructuring} syntax).
\begin{x10}
81    VariableDeclaratorId ::= 
           identifier [ IdentifierList ]
82       | [ IdentifierList ]
\end{x10}
In \XtenCurrVer{} the {\tt VariableDeclaratorId} must be declared at
type {\tt x10.lang.point}. Intuitively, this syntax allows a
point to be ``destructured'' into its corresponding {\tt int} 
indices in a pattern-matching style.
The $k$th identifier in the {\tt
IdentifierList} is treated as a {\tt final} variable of type {\tt int}
that is initialized with the value of the $k$th index of the point. 
The second form of the syntax (Rule 82) permits the specification of only
the index variables.

Future versions of the language may allow destructuring syntax for all
value classes.

\paragraph{Example.}
The following example succeeds when executed.
\begin{x10}
public class Array1Exploded \{
  public int select(point p[i,j], point [k,l]) \{
      return i+k;
  \}
  public boolean run() \{
    distribution d =  [1:10, 1:10] -> here;
    int[.] ia = new int[d];
    for(point p[i,j]: [1:10,1:10]) \{
        if(ia[p]!=0) return false;
        ia[p] = i+j;
    \}
    for(point p[i,j]: d) \{
      point q1 = [i,j];
      if (i != q1[0]) return false;
      if ( j != q1[1]) return false;
      if(ia[i,j]!= i+j) return false;
      if(ia[i,j]!=ia[p]) return false;
      if(ia[q1]!=ia[p]) return false;
   \}
    if (! (4 == select([1,2],[3,4]))) return false;
     return true;
   \}
        
  public static void main(String args[]) \{
     boolean b= (new Array1Exploded()).run();
     System.out.println("++++++ "
                        + (b? "Test succeeded."
                           :"Test failed."));
     System.exit(b?0:1);
 \}
\}
\end{x10}

\chapter{Expressions}\label{XtenExpressions}\index{expressions}

\Xten{} supports a rich expression language similar to
\Java{}'s.
Evaluating an expression produces a value, which may be either
an instance of a value class or an instance of a reference
class. Expressions may also be \xcd"void"; that is, they produce no
value.
Expression evaluation may have side effects: assignment to a
variable, allocation, method calls, or exceptional control-flow.
Evaluation is strict and is performed left to right.

\section{Literals}

\Xten{} supports the following literal expressions: 
\begin{itemize}
\item An 32-bit integer literal is a value of type \xcd"x10.lang.Int".
\item An 64-bit long literal is a value of type \xcd"x10.lang.Long".
\item A 32-bit floating-point literal is a value of type \xcd"x10.lang.Float".
\item A 64-bit floating-point literal is a value of type \xcd"x10.lang.Double".
\item A character literal is a value of type \xcd"x10.lang.Char".
\item A string literal is a value of type \xcd"x10.lang.String".
\item The boolean literals \xcd"true" and \xcd"false" are of type
\xcd"x10.lang.Boolean".
\item The \xcd"null" literal is of the null type,
a subtype of all reference types.
\end{itemize}

\section{\Xcd{this}}

\begin{grammar}
ThisExpression \: \xcd"this" \\
\| ClassName \xcd"." \xcd"this" \\
\end{grammar}

The expression \xcd"this" is a final local variable containing a reference
to an instance of the lexically enclosing class.
It may be used only within the body of an instance method, a
constructor, or in the initializer of a instance field.

Within an inner class, \xcd"this" may be qualified with the
name of a lexically enclosing class.  In this case, it
represents an instance of that enclosing class.

The type of a \xcd"this" expression is the
innermost enclosing class, or the qualifying class,
constrained by the class invariant and the
method where clause, if any.

\section{Local variables}



\section{Field access}
\label{FieldAccess}


\begin{grammar}
FieldExpression \: Expression \xcd"." Identifier \\
                \| \xcd"super" \xcd"." Identifier \\
                \| ClassName \xcd"." Identifier \\
                \| ClassName \xcd"." \xcd"super" \xcd"." Identifier \\
\end{grammar}

A field of an object instance may be  accessed
with a field access expression.

The type of the access is the declared type of the field with the
actual target substituted for \xcd"this" in the type.  If the actual
target is not a final access path, an anonymous path is
substituted for \xcd"this".

The field accessed is selected from the fields and value properties
of the static type of the target and its superclasses.

If the field target is given by the keyword \xcd"super",
the target's type is
the superclass of the enclosing class, as
constrained by the superclass's class invariant, if any.

If the field target is \xcd"null", a \xcd"NullPointerException"
is thrown.

If the field target is a class name, a static field is selected.

It is illegal to access  a field that is not visible from
the current context.
It is illegal to access a non-static field
through a static field access expression.

\section{Calls}
\label{MethodInvocation}
\label{MethodInvocationSubstitution}

\section{Closures}
\label{Closures}
\input{Closures}

\section{Increment and decrement}

The operators \xcd"++" and \xcd"--" increment and decrement
a variable, respectfully.  The variable must be non-final
and of numeric type.

When the operator is prefix, the variable is
incremented or decremented by \xcd"1" and the result of the expression is
the new value of the variable.
When the operator is postfix, the variable is incremented or
decremented by \xcd"1" and the result of the expression is the old value of
the variable.

The new value of the variable $v$ is identical to the result of
the expressions
\xcdmath"$v$+1" or \xcdmath"$v$-1", as appropriate.

\section{Numeric promotion}

The unary and binary operators promote their operands as
follows.
Values are sign extended and converted to instances of the
promoted type.

\begin{itemize}
\item  The unary promotion of \xcd"Byte", \xcd"Short",
\xcd"Char", and \xcd"Int" is \xcd"Int".
\item  The unary promotion of \xcd"Long" is \xcd"Long".
\item  The unary promotion of \xcd"Float" is \xcd"Float".
\item  The unary promotion of \xcd"Double" is \xcd"Double".
\item The binary promotion of two types is the  
greater of the unary promotion of each type
according to the following order:
\xcd"Int", \xcd"Long", \xcd"Float", \xcd"Double".
\end{itemize}

\section{Unary plus and unary minus}

The unary \xcd"+" operator applies unary numeric promotion to
its operand.   The operand must be of numeric type.

The unary \xcd"-" operator
applies unary numeric promotion to its operand
and then
subtracts the promoted operand from \xcd"0".
The operand must be of numeric type.
The type of the result is promoted type.

\section{Bitwise complement}

 The unary \xcd"~" operator
applies unary numeric promotion to its operand
and then
 evaluates to the bitwise complement of
 the promoted operand.
  The operand must be of integral type.
The type of the result is promoted type.

\section{Binary arithmetic operations} 

The binary arithmetic operations apply binary numeric promotion
to their operands. The operands must be of numeric type.
The type of the result is the promoted type.
The
\xcd"+" operator adds the promoted operands. The \xcd"-" operator
subtracts the second operand from the first. The \xcd"*" operator
multiplies the  promoted  operands. The \xcd"/" operator
divides the
first  operand  by the second.
The \xcd"%" operator evaluates to
the remainder of the division of the first operand by the
second.

Floating point operations are determined by the IEEE 754
standard. 



\section{Binary shift operations}

Unary promotion is performed on each operand separately. 
The operands must be of integral type.
The type of the result is the promoted type of the left operand.

If the promoted type of the left operand is \xcd"Int",
the right operand is masked with \xcd"0x1f" using the bitwise
AND (\xcd"&") operator.
If the promoted type of the left operand is \xcd"Long",
the right operand is masked with \xcd"0x3f" using the bitwise
AND (\xcd"&") operator.

The \xcd"<<" operator left-shifts the left operand by the number of
bits given by the right operand.

The \xcd">>" operator right-shifts the left operand by the number of
bits given by the right operand.  The result is sign extended;
that is, if the right operand is $k$,
the most significant $k$ bits of the result are set to the most
significant bit of the operand.

The \xcd">>>" operator right-shifts the left operand by the number of
bits given by the right operand.  The result is not sign extended;
that is, if the right operand is $k$,
the most significant $k$ bits of the result are set to \xcd"0".

\section{Binary bitwise operations}

The binary bitwise operations apply binary numeric promotion
to their operands. The operands must be of integral type.
The type of the result is the promoted type.
The \xcd"&" operator  performs the bitwise AND of the promoted operands.
The \xcd"|" operator  performs the bitwise inclusive OR of the promoted operands.
The \xcd"^" operator  performs the bitwise exclusive OR of the promoted operands.

\section{String concatenation}

The \xcd"+"  operator is used for string concatenation 
 as well as addition.
If either operand is of static type \xcd"x10.lang.String",
 the other operand is converted to a \xcd"String", if needed,
  and  the two strings  are concatenated.

 String conversion of a non-\xcd"null" value is  performed by invoking the
 \xcd"toString()" method of the value.
  If the value is \xcd"null", the value is converted to 
  \xcd'"null"'.

The type of the result is \xcd"String".

\section{Logical negation}

The operand of the  unary \xcd"!" operator 
must be of type \xcd"x10.lang.Boolean".
The type of the result is \xcd"Boolean".
If the value of the operand is \xcd"true", the result is \xcd"false"; if
if the value of the operand  is \xcd"false", the result is \xcd"true".

\section{Boolean logical operations}

Operands of the binary boolean logical operators must be of type \xcd"Boolean".
The type of the result is \xcd"Boolean"

The \xcd"&" operator  evaluates to \xcd"true" if both of its
operands evaluate to \xcd"true"; otherwise, the operator
evaluates to \xcd"false".

The \xcd"|" operator  evaluates to \xcd"false" if both of its
operands evaluate to \xcd"false"; otherwise, the operator
evaluates to \xcd"true".

The \xcd"^" operator  evaluates to \xcd"false" if both of its
operands are equal; otherwise, the operator
evaluates to \xcd"true".

\section{Relational operations} 

The relational operations apply binary numeric promotion
to their operands. The operands must be of numeric type.
The type of the result is \xcd"Boolean".

The \xcd"<" operator evalutates to \xcd"true" if the left operand is
less than the right.
The \xcd"<=" operator evalutates to \xcd"true" if the left operand is
less than or equal to the right.
The \xcd">" operator evalutates to \xcd"true" if the left operand is
greater than the right.
The \xcd">=" operator evalutates to \xcd"true" if the left operand is
greater than or equal to the right.

Floating point comparison is determined by the IEEE 754
standard.  Thus,
if either operand is NaN, the result is \xcd"false".
Negative zero and positive zero are considered to be equal.
All finite values are less than positive infinity and greater
than negative infinity.

\section{Stable equality}\label{StableEquality}\index{==}\index{!=}

The \xcd"==" and \xcd"!=" operators provide \emph{stable equality}.

Two operands may be compared with the infix predicate \xcd"==".
The operation
evaluates to \xcd"true" if and only if no action taken by any
user program can distinguish between the two operands.  In more detail,
the rules are as follows.

If the operands both have reference type, then the operation
evaluates to \xcd"true" if both are references to
the same object (even if the object has no mutable fields). 

If one operand evaluates to \xcd"null" then the predicate
evaluates to \xcd"true" if and only if the
other operand is also \xcd"null".

If the operands both have value type, then they must be structurally equal;
that is, they must be instances of the same value class or value array
data type and all their fields or components must be \xcd"==". 

If one operand is of reference type and the other is of value type,
the result is \xcd"false".

The predicate \xcd"!=" returns \xcd"true" (\xcd"false") on two
arguments if and only if the operand \xcd"==" returns \xcd"false"
(\xcd"true") on the same operands.

The predicates \xcd"==" and \xcd"!=" may not be overridden by the
programmer.

\section{Allocation}
\label{ClassCreation}

\begin{grammar}
NewExpression \: \xcd"new" ClassName TypeArguments\opt ValueArguments
        ClassBody\opt \\
  \| \xcd"new" InterfaceName TypeArguments\opt ValueArguments
        ClassBody
\end{grammar}

An allocation expression creates a new instance of a class and
invokes a constructor of the class.
The expression designates the class name and passes
type and value arguments to be constructor.

The allocation expression may have an optional class body.
In this case, an anonymous subclass of the given class is
allocated.   An anonymous class allocation may also specify a
single super-interface rather than a superclass; the superclass
of the anonymous class is \xcd"x10.lang.Object".

If the class is anonymous---that is, if a class body is
provided---in the constructor is selected from the superclass.
The constructor to invoke is selected using the same rules as
for method invocation (\Sref{MethodInvocation}).

The type of an allocation expression
is the return type of the constructor invoked, with appropriate
substitutions  of actual arguments for formal parameters, as
specified in \Sref{MethodInvocationSubstitution}.

It is illegal to allocate an instance of an \xcd"abstract" class.
It is illegal to allocate an instance of a class or to invoke a
constructor that is not visible at
the allocation expression.


\section{Casts}\label{ClassCast}\index{classcast}

The cast operation may be used to cast an expression to a given type:

\begin{grammar}
UnaryExpressionNotPlusMinus \: CastExpression \\
CastExpression \: UnaryExpressionNotPlusMinus \xcd"as" Type \\
\| UnaryExpressionNotPlusMinus \xcd"to" Type \\
\end{grammar}

The result of this operation is a value of the given type if the cast
is permissible at runtime. Both the data type and place type of the
value are checked. Data type conversion is checked according to the
rules of the \java{} language (e.g., \cite[\S 5.5]{jls2}). If the
value cannot be cast to the appropriate data type, a
\xcd"ClassCastException"
is thrown. Otherwise, if the value cannot be cast to the
appropriate place type a \xcd"BadPlaceException" is thrown. 

Any attempt to cast an expression of a reference type to a value type
(or vice versa) results in a compile-time error. Some casts---such as
those that seek to cast a value of a subtype to a supertype---are
known to succeed at compile-time. Such casts should not cause extra
computational overhead at runtime.

\section{\Xcd{instanceof}}\label{instanceOf}\index{instanceof@\xcd"instanceof"}

\Xten{} permits types to be used in an in instanceof expression
to determine whether an object is an instance of the given type:

\begin{grammar}
RelationalExpression \: RelationalExpression \xcd"instanceof" Type
\end{grammar}

In the above expression, \grammarrule{Type} is any type including
constrained types and value types. 
At run time, the result of this operator is
\xcd"true" if the \grammarrule{RelationalExpression} can be cast
to \grammarrule{Type} without a \xcd" ClassCastException" being
thrown.  Otherwise the result is \xcd"false".
This determination may involve checking
that the
constraint, if any, associated with the type is true for the
given expression.

\section{Rail constructors}

The rail constructor \xcdmath"{ a$_1$, $\dots$, a$_k$ }"
is creates an instance of \xcd"valrail" with distribution
\xcdmath"(0..$k$-1)->here" where the $i$th element is
\xcdmath"a$_{i+1}$".  The element type of the array (\xcd"T") is
bound to the least common ancestor of the types of the
\xcdmath"a$_i$".  Since arrays are subtypes of \xcd"point => T",
rail constructors can be passed into the \xcd"Array" and
\xcd"valarray" constructors as initializer functions.

\section{Point construction}\label{point-syntax}\index{point syntax}

\Xten{} specifies a simple syntax for the construction of points.

\begin{grammar}
ArgumentList \: Expression ( \xcd"," Expression )\star \\
Primary \: \xcd"Point" \xcd"(" ArgumentList \xcd")"
\end{grammar}

Each element in the argument list must be of type
\xcd"Int" or \xcd"Long".  The expression
\xcdmath"Point(a$_1$, $\dots$, a$_k$)" is treated as syntactic shorthand for
\begin{xtenmath}
new Point({ a$_1$, $\dots$, a$_k$ })
\end{xtenmath}

\section{Region construction}\label{region-syntax}\index{region syntax}

\Xten{} specifies a simple syntax for the construction of regions.

\begin{grammar}
Primary \: Expression \xcd".." Expression \\
Primary \: Expression \xcd"*" Expression \\
\end{grammar}

The expression \xcdmath"a$_1$..a$_2$"
is shorthand for the rectangular, rank-1 region
\begin{xten}
new Region(a$_1$, a$_2$)
\end{xten}
Each subexpression of \xcdmath"a$_i$" must be of type \xcd"Int".

The expression \xcdmath"r$_1$ * r$_2$" is shorthand for the
cross-product region formed by pairing each point in \xcdmath"r$_1$"
with every the point in \xcdmath"r$_2$".  Thus, \xcd"(1..2) * (3..4)"
is the region \xcd"(1,3), (1,4), (2,3), (2,4)".

\section{Exploded variable declarations}\label{exploded-syntax}\index{variable declarator!exploded}

XXX

\Xten{} permits a richer form of specification for variable
declarators in method arguments, local variables and loop variables
(the ``exploded'' or {\em destructuring} syntax).

\begin{grammar}
VariableDeclaratorId \:
           Identifier \xcd"(" IdentifierList \xcd")" \\
           \| \xcd"(" IdentifierList \xcd")" \\
\end{grammar}

In \XtenCurrVer{} the \grammarrule{VariableDeclaratorId} must be declared at
type \xcd{x10.lang.Point}. Intuitively, this syntax allows a
point to be ``destructured'' into its corresponding \xcd{int} 
indices in a pattern-matching style.
The $k$th identifier in the \grammarrule{
IdentifierList} is treated as a \xcd{final} variable of type \xcd{int}
that is initialized with the value of the $k$th index of the point. 
The second form of the syntax (Rule 82) permits the specification of only
the index variables.

Future versions of the language may allow destructuring syntax for all
value classes.

\paragraph{Example.}
The following example succeeds when executed.
\begin{xten}
public class Array1Exploded {
  public def select(p(i,j): Point, (k,l): Point): int {
      return i+k;
  }
  public def run(): boolean {
    d: Dist = new Region(1..10, 1..10) -> here;
    ia: Array[int] = new array[int](d);
    for (p(i,j): Point in new Region(1..10,1..10)) {
        if (ia(p) != 0) return false;
        ia(p) = i+j;
    }
    for (p(i,j): Point in d) {
      q1: Point = (i,j);
      if (i != q1(0)) return false;
      if (j != q1(1)) return false;
      if(ia(i,j) != i+j) return false;
      if(ia(i,j) != ia(p)) return false;
      if(ia(q1)  != ia(p)) return false;
    }
    if (! (4 == select([1,2],[3,4]))) return false;
    return true;
  }
        
  public static def main(args: array[String]) {
     b: boolean = (new Array1Exploded()).run();
     System.out.println("++++++ "
                        + (b ? "Test succeeded."
                             : "Test failed."));
     System.exit(b ? 0 : 1);
  }
}
\end{xten}



 \par  % empty



	\par  % 0.05
\chapter{Annotations and Compiler
Plugins}\label{XtenAnnotations}\index{annotations}


X10 provides an 
an annotation system and compiler plugin system for to allow the
compiler to be extended with new static analyses and new
transformations.

Annotations are interface types that decorate the abstract syntax tree
of an X10 program.  The X10 type-checker ensures that an annotation
is a legal interface type.
In X10, interfaces may declare
both methods and properties.  Therefore, like any interface type, an
annotation may instantiate
one or more of its interface's properties.
Unlike with Java
annotations,
property initializers need not be
compile-time constants;
however, a given compiler plugin
may do additional checks to constrain the allowable
initializer expressions.
The X10 type-checker does not check that
all properties of an annotation are initialized,
although this could be enforced by
a compiler plugin.

\section{Annotation syntax}

The annotation syntax consists of an ``\texttt{@}'' followed by an interface type.
\begin{x10}
532   Annotation ::= @ InterfaceType
533   Annotations ::= Annotation
534     | Annotations Annotation
535   Annotationsopt ::=
536     | Annotations
\end{x10}
Annotations can be applied to most syntactic constructs in the language
including class declarations, constructors, methods, field declarations,
local variable declarations and formal parameters, statements,
expressions, and types.
Multiple occurrences of the same annotation (i.e., multiple
annotations with the same interface type) on the same entity are permitted.

\begin{x10}
537   ClassModifier ::= Annotation
538   InterfaceModifier ::= Annotation
539   FieldModifier ::= Annotation
540   MethodModifier ::= Annotation
541   VariableModifier ::= Annotation
542   ConstructorModifier ::= Annotation
543   AbstractMethodModifier ::= Annotation
544   ConstantModifier ::= Annotation
545   Type ::= AnnotatedType
546   AnnotatedType ::= Type Annotations
547   Statement ::= AnnotatedStatement
548   AnnotatedStatement ::= Annotation Statement
549   Expression ::= AnnotatedExpression
550   AnnotatedExpression ::= ( Annotations ) Expression
\end{x10}
\noindent
Recall that interface types may have dependent parameters.

\noindent
The following examples illustrate the syntax:

\begin{itemize}
\item Declaration annotations:
\begin{x10}
  // class annotation
  @Value
  class Cons \{ ... \}

  // method annotation
  @PreCondition(0 <= i \&\& i < this.size)
  public Object get(int i) \{ ... \}

  // constructor annotation
  @Where(x != null)
  C(T x) \{ ... \}

  // constructor return type annotation
  C@Initialized(T x) \{ ... \}

  // variable annotation
  @Unique A x;
\end{x10}
\item Type annotations:
\begin{x10}
  List@Nonempty

  int@Range(1,4)

  double[][]@Size(n * n)
\end{x10}
\item Expression annotations:
\begin{x10}
  (@RemoteCall) m()

  3 == (@Bits(2)) 15
\end{x10}
\item Statement annotations:
\begin{x10}
  @Atomic \{ ... \}

  @MinIterations(1)
  @MaxIterations(n)
  for (int i = 0; i < n; i++) \{ ... \}

  // An annotated empty statement ;
  @Assert(x < y);
\end{x10}
\end{itemize}

\section{Annotation declarations}

Annotations are declared as interfaces.  They must be
subtypes of \texttt{x10.lang.annotation.Annotation}.
Annotations on types, expressions, statements, classes, fields,
methods, constructors, and local variable declarations (or
formal parameters)
must extend
\texttt{ExpressionAnnotation},
\texttt{StatementAnnotation},
\texttt{ClassAnnotation},
\texttt{FieldAnnotation},
\texttt{MethodAnnotation},
\texttt{ConstructorAnnotation}, and
\texttt{VariableAnnotation}, respectively.

\section{Compiler plugins}
\index{plugins}

After the base X10 semantic checking is completed, 
compiler plugins are loaded and run.  Plugins may perform
any number of compiler passes to implement
additional semantic checking and code transformations, including
transformations using the abstract syntax of the annotations
themselves.  Plugins should output valid X10 abstract
syntax trees.

Plugins are implemented in java as
Polyglot~\cite{ncm03} passes applied to the AST
after normal base X10 type checking.
Plugins to run are specified on the command-line.  The order of
execution is determined by the Polyglot pass scheduler.
\index{Polyglot}

To run compiler plugins, add the command-line option:
\begin{x10}
  -PLUGINS=P1,P2,...,Pn
\end{x10}
where \texttt{P1}, \texttt{P2}, \dots, \texttt{Pn} are classes that implement the
\texttt{CompilerPlugin} interface:
\index{CompilerPlugin}

\begin{x10}
  package polyglot.ext.x10.plugin;

  import polyglot.ext.x10.ExtensionInfo;
  import polyglot.frontend.Job;
  import polyglot.frontend.goals.Goal;

  public interface CompilerPlugin \{
      public Goal register(ExtensionInfo extInfo, Job job);
  \}
\end{x10}

\index{Goal}
The \texttt{Goal} object returned by the \texttt{register} method specifies dependencies on other passes.
Documentation for Polyglot can be found at:
\begin{x10}
  http://www.cs.cornell.edu/Projects/polyglot
\end{x10}
Most plugins should implement either \texttt{SimpleOnePassPlugin} or
\texttt{SimpleVisitorPlugin}.

The compiler loads plugin classes from the x10c classpath.

Plugins are given access to a Polyglot AST and type system.  Annotations are
represented in the AST as \texttt{Node}s with the following interface:
\index{Node}

\begin{x10}
  package polyglot.ext.x10.ast;

  public interface AnnotationNode extends Node \{
    X10ClassType annotation();
  \}
\end{x10}

Annotations for a \texttt{Node} object \texttt{n} can be accessed through the
node's extension object as follows:
\index{AnnotationNode}

\begin{x10}
  List<AnnotationNode> annotations =
    ((X10Ext) n.ext()).annotations();
  List<X10ClassType> annotationTypes =
    ((X10Ext) n.ext()).annotationInterfaces();
\end{x10}
In the type system, \texttt{X10TypeObject} has the following
method for accessing annotations:
\begin{x10}
  List<X10ClassType> annotations();
\end{x10}


%\balance
\bibliography{pm.bib,db.bib}

% \clearpage

\end{document}


	\par  % 0.05
\section{Linking with native code}\label{extern}\index{extern}
\XtenCurrVer{} supports a simple facility to permit the efficient
intra-thread communication of an array of primitive type to code
written in the language {\tt C}.  The array must be a ``local''
array. The primary intent of this design is to permit the reuse of
native code that efficiently implements some numeric array/matrix
calculation.

Future language releases are expected to support similar bindings to
{\sc Fortran}, and to support parallel native processing of
distributed \Xten{} arrays. 

The interface consists of two parts. First, an array intended to be
communicated to native code must be created as an {\tt unsafe} array:
\begin{x10}
450 ArrayCreationExpression ::= 
      new ArrayBaseType Unsafeopt [ ] 
        ArrayInitializer
451   | new ArrayBaseType Unsafeopt [ Expression ]
452   | new ArrayBaseType Unsafeopt 
          [ Expression ] Expression
453   | new ArrayBaseType Unsafeopt [ Expression ] 
          ( FormalParameter ) MethodBody
454   | new ArrayBaseType value 
           Unsafeopt [ Expression ]
455   | new ArrayBaseType value 
           Unsafeopt [ Expression ] Expression
456   | new ArrayBaseType value 
        Unsafeopt [ Expression ] 
          ( FormalParameter ) MethodBody
530   Unsafeopt ::=
531     | unsafe
\end{x10}
Unsafe arrays can be of any dimension. However, \XtenCurrVer{}
requires that unsafe arrays be of a primitive type, and local (i.e.{}
with an underlying distribution that maps all elements in its region
to {\tt here}).

Unsafe arrays are allocated in a special array of memory that permits
their efficient transmission to natively linked code.
%% Comment about when this memory is freed.

Second, the \Xten{} programmer may specify that certain methods are to
be implemented natively by using the keyword {\tt extern}:
\begin{x10}
446   MethodModifier ::= extern
\end{x10}
Such a method must have the statement ``{\tt ;}'' as its body.
\XtenCurrVer{} requires that the method be {\tt static}; this
restriction is likely to be lifted in the future.  Primitive types in
the method argument are translated to their corresponding JNI type
(e.g.{} {\tt float} is translated to {\tt jfloat}, {\tt double} to
{\tt jdouble} etc).  The only non-primitive type permitted in an {\tt
extern} method is an (unsafe) array. This is passed at type {\tt
jlong} as an eight byte address into the unsafe region which contains
the data for the array. ({\tt jlong} is not the same as {\tt long} on
32-bit machines.)


Since only the starting address of an array is passed, if the array is
multidimensional, the user must explicitly communicate (or have a
guarantee of) the rank of the passed array, and must either typecast
or explicitly code the address calculation.  Note that all \Xten{}
arrays are created in row-major order, and so any native routine must
also access them in the same order.

For each class {\tt C} that contains an {\tt extern} method, the
\Xten{} compiler generates a text file {\tt C\_x10stub.c}.  This file
contains generated {\tt C} stub functions which are called from the
{\tt extern} routines.  The name of the stub function is derived from
the name of the {\tt extern} method. If the method is {\tt
C.process()}, the stub function will be {\tt
Java\_C\_C\_process()}. The name is suffixed with the signature of the
method if the method is overloaded.

The programmer must write {\tt C} code to implement the native method,
using the methods in the {\tt C} stub file to call the actual native
method.  The programmer must compile these files and link them into a
dynamically linked library (DLL).  Note that the {\tt jni.h} header file
must be in the include path.  The programmer must ensure this library
is loaded by the program before the method is called e.g.{} add a {\tt
System.loadlibrary} call (in a static initializer of the
\Xten{} class).

\paragraph{Example.}
The following class illustrates the use of {\tt unsafe} and native
linking. 
\begin{x10}
public class IntArrayExternUnsafe \{
  public static extern 
      void process(int [.] yy, int size);
  static {System.loadLibrary("IntArrayExternUnsafe");}
  public static void main(String args[]) \{
     boolean b= (new IntArrayExternUnsafe()).run();
     System.out.println("++++++ Test "
                         +(b?"succeeded.":"failed."));
     System.exit(b?0:1);
  \}
  public boolean run()\{
    int high = 10;
    boolean verified=false;
    distribution d= (0:high) -> here;
    int [.] y = new int unsafe[d]; 
    for( int j=0;j < 10;++j)
        y[j] = j;
    process(y,high);
    for(int j=0;j < 10;++j)\{
      int expected = j+100;
      if(y[j] != expected)\{
        System.out.println("y["+j+"]="
                           +y[j]+" != "+expected);
        return false;
       \}
    \}
    return true;
  \}
\}
\end{x10}

The programmer may then write the {\tt C} code thus:
\begin{x10}
void IntArrayExternUnsafe\_process(jlong yy, 
                                signed int size)\{
  int i;
  int* array = (int *)(long)yy;
  for(i = 0;i < size;++i)\{
    array[i] += 100;
  \}
\}
/* automatically generated in \_x10stub.c*/
void 
 Java\_IntArrayExternUnsafe\_IntArrayExternUnsafe\_process
 (JNIEnv *env,  jobject obj,jlong yy,jint size)\{
   IntArrayExternUnsafe\_process(yy,size);
\}
\end{x10}

This code may be linked with the stub file (or textually placed in
it). The programmer must then compile and link the {\tt C} code and
ensure that the DLL is on the appropriate classpath. 


%\chapter{Performance Model}\label{PerformanceModel}
 \par \vfill\eject % empty
\extrapart{Example}

This example illustrates 2-d Jacobi iteration.

\begin{xten}
public class Jacobi {
   const N: int = 6;
   const epsilon: double = 0.002;
   const epsilon2: double = 0.000000001;
   const R: region = [0:N+1, 0:N+1];
   const RInner: region = [1:N, 1:N];
   const D: distribution = distribution.factory.block(R);
   const DInner: distribution = D | RInner;
   const DBoundary: distribution = D - RInner;
   const EXPECTED_ITERS: int  = 97;
   const double EXPECTED_ERR: double = 0.0018673382039402497;
     
   val B: Array[double](D) = Array.make[double](D,
        (p(i,j): point) => DBoundary.contains(p) ? (N-1)/2 : N*(i-1)+(j-1));
    
   public def run(): boolean {
      var iters: int = 0;
      var err: double;
      while (true) {
        val Temp: Array[double] = 
           new array[double](DInner, ((i,j): point) =>
             (read(i+1,j)+read(i-1,j) +read(i,j+1)+read(i,j-1))/4.0);
        if((err=((B | DInner) - Temp).abs().sum()) < epsilon)
           break; 
        B.update(Temp);
        iters++; 
      }
      Console.OUT.println("Error="+err);
      Console.OUT.println("Iterations="+iters);
      return Math.abs(err-EXPECTED_ERR)<epsilon2 
          && iters==EXPECTED_ITERS;
   }
   public def read(i: int, j: int): double {
      return (future(D(i,j)) => B(i,j)).force();
   }
   public static def main(args: Array[String]) {
      val b = new Jacobi().run();
      Console.OUT.println("++++++ "
                          + (b? "Test succeeded."
                             :"Test failed."));
      System.exit(b?0:1);
   }
}
\end{xten}
	\par  \vfill\eject % have an example
\notinfouro{\onecolumn
\extrapart{\Xten{} syntax}\label{X10 syntax}\index{X10 productions}

This section contains the complete grammar for \Xten{}. This includes
all the new constructs in \Xten{} discussed in the main body of this
reference manual, as well as constructs obtained from \java{} which
behave essentially identically to the corresponding {\tt java} constructs.

Note that in this version of the grammar productions for the same
non-terminal may occur non-contiguously. For instance 
{\tt MethodModifier} is defined on lines {\tt 111--119} and
{\tt 445-446}. This will be corrected in future versions of the grammar.

{\footnotesize
\begin{verbatim}
0     $accept ::= CompilationUnit
1     identifier ::= IDENTIFIER
2     PrimitiveType ::= NumericType
3      | boolean
4     NumericType ::= IntegralType
5      | FloatingPointType
6     IntegralType ::= byte
7      | char
8      | short
9      | int
10     | long
11    FloatingPointType ::= float
12     | double
13    ClassType ::= TypeName
14    InterfaceType ::= TypeName
15    TypeName ::= identifier
16     | TypeName . identifier
17    ClassName ::= TypeName
18    ArrayType ::= Type [ ]
19    PackageName ::= identifier
20      | PackageName . identifier
21    ExpressionName ::= identifier
22      | here
23      | AmbiguousName . identifier
24    MethodName ::= identifier
25      | AmbiguousName . identifier
26    PackageOrTypeName ::= identifier
27      | PackageOrTypeName . identifier
28    AmbiguousName ::= identifier
29      | AmbiguousName . identifier
30    CompilationUnit ::= PackageDeclarationopt ImportDeclarationsopt TypeDeclarationsopt
31    ImportDeclarations ::= ImportDeclaration
32      | ImportDeclarations ImportDeclaration
33    TypeDeclarations ::= TypeDeclaration
34      | TypeDeclarations TypeDeclaration
35    PackageDeclaration ::= package PackageName ;
36    ImportDeclaration ::= SingleTypeImportDeclaration
37      | TypeImportOnDemandDeclaration
38      | SingleStaticImportDeclaration
39      | StaticImportOnDemandDeclaration
40    SingleTypeImportDeclaration ::= import TypeName ;
41    TypeImportOnDemandDeclaration ::= import PackageOrTypeName . * ;
42    SingleStaticImportDeclaration ::= import static TypeName . identifier ;
43    StaticImportOnDemandDeclaration ::= import static TypeName . * ;
44    TypeDeclaration ::= ClassDeclaration
45      | InterfaceDeclaration
46      | ;
47    ClassDeclaration ::= NormalClassDeclaration
48    NormalClassDeclaration ::= ClassModifiersopt class identifier Superopt Interfacesopt ClassBody
49    ClassModifiers ::= ClassModifier
50      | ClassModifiers ClassModifier
51    ClassModifier ::= public
52      | protected
53      | private
54      | abstract
55      | static
56      | final
57      | strictfp
58    Super ::= extends ClassType
59    Interfaces ::= implements InterfaceTypeList
60    InterfaceTypeList ::= InterfaceType
61      | InterfaceTypeList , InterfaceType
62    ClassBody ::= { ClassBodyDeclarationsopt }
63    ClassBodyDeclarations ::= ClassBodyDeclaration
64      | ClassBodyDeclarations ClassBodyDeclaration
65    ClassBodyDeclaration ::= ClassMemberDeclaration
66      | InstanceInitializer
67      | StaticInitializer
68      | ConstructorDeclaration
69    ClassMemberDeclaration ::= FieldDeclaration
70      | MethodDeclaration
71      | ClassDeclaration
72      | InterfaceDeclaration
73      | ;
74    FieldDeclaration ::= FieldModifiersopt Type VariableDeclarators ;
75    VariableDeclarators ::= VariableDeclarator
76      | VariableDeclarators , VariableDeclarator
77    VariableDeclarator ::= VariableDeclaratorId
78      | VariableDeclaratorId = VariableInitializer
79    VariableDeclaratorId ::= identifier
80      | VariableDeclaratorId [ ]
81      | identifier [ IdentifierList ]
82      | [ IdentifierList ]
83    VariableInitializer ::= Expression
84      | ArrayInitializer
85    FieldModifiers ::= FieldModifier
86      | FieldModifiers FieldModifier
87    FieldModifier ::= public
88      | protected
89      | private
90      | static
91      | final
92      | transient
93      | volatile
94    MethodDeclaration ::= MethodHeader MethodBody
95    MethodHeader ::= MethodModifiersopt ResultType MethodDeclarator Throwsopt
96    ResultType ::= Type
97      | void
98    MethodDeclarator ::= identifier ( FormalParameterListopt )
99      | MethodDeclarator [ ]
100   FormalParameterList ::= LastFormalParameter
101     | FormalParameters , LastFormalParameter
102   FormalParameters ::= FormalParameter
103     | FormalParameters , FormalParameter
104   FormalParameter ::= VariableModifiersopt Type VariableDeclaratorId
105   VariableModifiers ::= VariableModifier
106     | VariableModifiers VariableModifier
107   VariableModifier ::= final
108   LastFormalParameter ::= VariableModifiersopt Type ...opt VariableDeclaratorId
109   MethodModifiers ::= MethodModifier
110     | MethodModifiers MethodModifier
111   MethodModifier ::= public
112     | protected
113     | private
114     | abstract
115     | static
116     | final
117     | synchronized
118     | native
119     | strictfp
120   Throws ::= throws ExceptionTypeList
121   ExceptionTypeList ::= ExceptionType
122     | ExceptionTypeList , ExceptionType
123   ExceptionType ::= ClassType
124   MethodBody ::= Block
125     | ;
126   InstanceInitializer ::= Block
127   StaticInitializer ::= static Block
128   ConstructorDeclaration ::= ConstructorModifiersopt ConstructorDeclarator Throwsopt ConstructorBody
129   ConstructorDeclarator ::= SimpleTypeName ( FormalParameterListopt )
130   SimpleTypeName ::= identifier
131   ConstructorModifiers ::= ConstructorModifier
132     | ConstructorModifiers ConstructorModifier
133   ConstructorModifier ::= public
134     | protected
135     | private
136   ConstructorBody ::= { ExplicitConstructorInvocationopt BlockStatementsopt }
137   ExplicitConstructorInvocation ::= this ( ArgumentListopt ) ;
138     | super ( ArgumentListopt ) ;
139     | Primary . this ( ArgumentListopt ) ;
140     | Primary . super ( ArgumentListopt ) ;
141   Arguments ::= ( ArgumentListopt )
142   InterfaceDeclaration ::= NormalInterfaceDeclaration
143   NormalInterfaceDeclaration ::= InterfaceModifiersopt interface identifier ExtendsInterfacesopt InterfaceBody
144   InterfaceModifiers ::= InterfaceModifier
145     | InterfaceModifiers InterfaceModifier
146   InterfaceModifier ::= public
147     | protected
148     | private
149     | abstract
150     | static
151     | strictfp
152   ExtendsInterfaces ::= extends InterfaceType
153     | ExtendsInterfaces , InterfaceType
154   InterfaceBody ::= { InterfaceMemberDeclarationsopt }
155   InterfaceMemberDeclarations ::= InterfaceMemberDeclaration
156     | InterfaceMemberDeclarations InterfaceMemberDeclaration
157   InterfaceMemberDeclaration ::= ConstantDeclaration
158     | AbstractMethodDeclaration
159     | ClassDeclaration
160     | InterfaceDeclaration
161     | ;
162   ConstantDeclaration ::= ConstantModifiersopt Type VariableDeclarators
163   ConstantModifiers ::= ConstantModifier
164     | ConstantModifiers ConstantModifier
165   ConstantModifier ::= public
166     | static
167     | final
168   AbstractMethodDeclaration ::= AbstractMethodModifiersopt ResultType MethodDeclarator Throwsopt ;
169   AbstractMethodModifiers ::= AbstractMethodModifier
170     | AbstractMethodModifiers AbstractMethodModifier
171   AbstractMethodModifier ::= public
172     | abstract
173   ArrayInitializer ::= { VariableInitializersopt ,opt }
174   VariableInitializers ::= VariableInitializer
175     | VariableInitializers , VariableInitializer
176   Block ::= { BlockStatementsopt }
177   BlockStatements ::= BlockStatement
178     | BlockStatements BlockStatement
179   BlockStatement ::= LocalVariableDeclarationStatement
180     | ClassDeclaration
181     | Statement
182   LocalVariableDeclarationStatement ::= LocalVariableDeclaration ;
183   LocalVariableDeclaration ::= VariableModifiersopt Type VariableDeclarators
184   Statement ::= StatementWithoutTrailingSubstatement
185     | LabeledStatement
186     | IfThenStatement
187     | IfThenElseStatement
188     | WhileStatement
189     | ForStatement
190   StatementWithoutTrailingSubstatement ::= Block
191     | EmptyStatement
192     | ExpressionStatement
193     | AssertStatement
194     | SwitchStatement
195     | DoStatement
196     | BreakStatement
197     | ContinueStatement
198     | ReturnStatement
199     | SynchronizedStatement
200     | ThrowStatement
201     | TryStatement
202   StatementNoShortIf ::= StatementWithoutTrailingSubstatement
203     | LabeledStatementNoShortIf
204     | IfThenElseStatementNoShortIf
205     | WhileStatementNoShortIf
206     | ForStatementNoShortIf
207   IfThenStatement ::= if ( Expression ) Statement
208   IfThenElseStatement ::= if ( Expression ) StatementNoShortIf else Statement
209   IfThenElseStatementNoShortIf ::= if ( Expression ) StatementNoShortIf else StatementNoShortIf
210   EmptyStatement ::= ;
211   LabeledStatement ::= identifier : Statement
212   LabeledStatementNoShortIf ::= identifier : StatementNoShortIf
213   ExpressionStatement ::= StatementExpression ;
214   StatementExpression ::= Assignment
215     | PreIncrementExpression
216     | PreDecrementExpression
217     | PostIncrementExpression
218     | PostDecrementExpression
219     | MethodInvocation
220     | ClassInstanceCreationExpression
221   AssertStatement ::= assert Expression ;
222     | assert Expression : Expression ;
223   SwitchStatement ::= switch ( Expression ) SwitchBlock
224   SwitchBlock ::= { SwitchBlockStatementGroupsopt SwitchLabelsopt }
225   SwitchBlockStatementGroups ::= SwitchBlockStatementGroup
226     | SwitchBlockStatementGroups SwitchBlockStatementGroup
227   SwitchBlockStatementGroup ::= SwitchLabels BlockStatements
228   SwitchLabels ::= SwitchLabel
229     | SwitchLabels SwitchLabel
230   SwitchLabel ::= case ConstantExpression :
231     | default :
232   WhileStatement ::= while ( Expression ) Statement
233   WhileStatementNoShortIf ::= while ( Expression ) StatementNoShortIf
234   DoStatement ::= do Statement while ( Expression ) ;
235   ForStatement ::= BasicForStatement
236     | EnhancedForStatement
237   BasicForStatement ::= for ( ForInitopt ; Expressionopt ; ForUpdateopt ) Statement
238   ForStatementNoShortIf ::= for ( ForInitopt ; Expressionopt ; ForUpdateopt ) StatementNoShortIf
239     | EnhancedForStatementNoShortIf
240   ForInit ::= StatementExpressionList
241     | LocalVariableDeclaration
242   ForUpdate ::= StatementExpressionList
243   StatementExpressionList ::= StatementExpression
244     | StatementExpressionList , StatementExpression
245   BreakStatement ::= break identifieropt ;
246   ContinueStatement ::= continue identifieropt ;
247   ReturnStatement ::= return Expressionopt ;
248   ThrowStatement ::= throw Expression ;
249   SynchronizedStatement ::= synchronized ( Expression ) Block
250   TryStatement ::= try Block Catches
251     | try Block Catchesopt Finally
252   Catches ::= CatchClause
253     | Catches CatchClause
254   CatchClause ::= catch ( FormalParameter ) Block
255   Finally ::= finally Block
256   Primary ::= PrimaryNoNewArray
257     | ArrayCreationExpression
258   PrimaryNoNewArray ::= Literal
259     | Type . class
260     | void . class
261     | this
262     | ClassName . this
263     | ( Expression )
264     | ClassInstanceCreationExpression
265     | FieldAccess
266     | MethodInvocation
267     | ArrayAccess
268   Literal ::= IntegerLiteral
269     | LongLiteral
270     | FloatingPointLiteral
271     | DoubleLiteral
272     | BooleanLiteral
273     | CharacterLiteral
274     | StringLiteral
275     | null
276   BooleanLiteral ::= true
277     | false
278   ClassInstanceCreationExpression ::= new ClassOrInterfaceType ( ArgumentListopt ) ClassBodyopt
279     | Primary . new identifier ( ArgumentListopt ) ClassBodyopt
280     | AmbiguousName . new identifier ( ArgumentListopt ) ClassBodyopt
281   ArgumentList ::= Expression
282     | ArgumentList , Expression
283   FieldAccess ::= Primary . identifier
284     | super . identifier
285     | ClassName . super . identifier
286   MethodInvocation ::= MethodName ( ArgumentListopt )
287     | Primary . identifier ( ArgumentListopt )
288     | super . identifier ( ArgumentListopt )
289     | ClassName . super . identifier ( ArgumentListopt )
290   PostfixExpression ::= Primary
291     | ExpressionName
292     | PostIncrementExpression
293     | PostDecrementExpression
294   PostIncrementExpression ::= PostfixExpression ++
295   PostDecrementExpression ::= PostfixExpression --
296   UnaryExpression ::= PreIncrementExpression
297     | PreDecrementExpression
298     | + UnaryExpression
299     | - UnaryExpression
300     | UnaryExpressionNotPlusMinus
301   PreIncrementExpression ::= ++ UnaryExpression
302   PreDecrementExpression ::= -- UnaryExpression
303   UnaryExpressionNotPlusMinus ::= PostfixExpression
304     | ~ UnaryExpression
305     | ! UnaryExpression
306     | CastExpression
307   MultiplicativeExpression ::= UnaryExpression
308     | MultiplicativeExpression * UnaryExpression
309     | MultiplicativeExpression / UnaryExpression
310     | MultiplicativeExpression % UnaryExpression
311   AdditiveExpression ::= MultiplicativeExpression
312     | AdditiveExpression + MultiplicativeExpression
313     | AdditiveExpression - MultiplicativeExpression
314   ShiftExpression ::= AdditiveExpression
315     | ShiftExpression << AdditiveExpression
316     | ShiftExpression >> AdditiveExpression
317     | ShiftExpression >>> AdditiveExpression
318   RelationalExpression ::= ShiftExpression
319     | RelationalExpression < ShiftExpression
320     | RelationalExpression GREATER ShiftExpression
321     | RelationalExpression <_= ShiftExpression
322     | RelationalExpression GREATER = ShiftExpression
323   EqualityExpression ::= RelationalExpression
324     | EqualityExpression == RelationalExpression
325     | EqualityExpression != RelationalExpression
326   AndExpression ::= EqualityExpression
327     | AndExpression AND EqualityExpression
328   ExclusiveOrExpression ::= AndExpression
329     | ExclusiveOrExpression XOR AndExpression
330   InclusiveOrExpression ::= ExclusiveOrExpression
331     | InclusiveOrExpression OR ExclusiveOrExpression
332   ConditionalAndExpression ::= InclusiveOrExpression
333     | ConditionalAndExpression AND_AND InclusiveOrExpression
334   ConditionalOrExpression ::= ConditionalAndExpression
335     | ConditionalOrExpression OR_OR ConditionalAndExpression
336   ConditionalExpression ::= ConditionalOrExpression
337     | ConditionalOrExpression QUESTION Expression : ConditionalExpression
338   AssignmentExpression ::= ConditionalExpression
339     | Assignment
340   Assignment ::= LeftHandSide AssignmentOperator AssignmentExpression
341   LeftHandSide ::= ExpressionName
342     | FieldAccess
343     | ArrayAccess
344   AssignmentOperator ::= =
345     | *=
346     | /=
347     | %=
348     | +=
349     | -=
350     | <<=
351     | >>=
352     | >>>=
353     | &=
354     | ^=
355     | |=
356   Expression ::= AssignmentExpression
357   ConstantExpression ::= Expression
358   Catchesopt ::=
359     | Catches
360   identifieropt ::=
361     | identifier
362   ForUpdateopt ::=
363     | ForUpdate
364   Expressionopt ::=
365     | Expression
366   ForInitopt ::=
367     | ForInit
368   SwitchLabelsopt ::=
369     | SwitchLabels
370   SwitchBlockStatementGroupsopt ::=
371     | SwitchBlockStatementGroups
372   VariableModifiersopt ::=
373     | VariableModifiers
374   VariableInitializersopt ::=
375     | VariableInitializers
376   AbstractMethodModifiersopt ::=
377     | AbstractMethodModifiers
378   ConstantModifiersopt ::=
379     | ConstantModifiers
380   InterfaceMemberDeclarationsopt ::=
381     | InterfaceMemberDeclarations
382   ExtendsInterfacesopt ::=
383     | ExtendsInterfaces
384   InterfaceModifiersopt ::=
385     | InterfaceModifiers
386   ClassBodyopt ::=
387     | ClassBody
388   ,opt ::=
389     | ,
390   ArgumentListopt ::=
391     | ArgumentList
392   BlockStatementsopt ::=
393     | BlockStatements
394   ExplicitConstructorInvocationopt ::=
395     | ExplicitConstructorInvocation
396   ConstructorModifiersopt ::=
397     | ConstructorModifiers
398   ...opt ::=
399     | ...
400   FormalParameterListopt ::=
401     | FormalParameterList
402   Throwsopt ::=
403     | Throws
404   MethodModifiersopt ::=
405     | MethodModifiers
406   FieldModifiersopt ::=
407     | FieldModifiers
408   ClassBodyDeclarationsopt ::=
409     | ClassBodyDeclarations
410   Interfacesopt ::=
411     | Interfaces
412   Superopt ::=
413     | Super
414   ClassModifiersopt ::=
415     | ClassModifiers
416   TypeDeclarationsopt ::=
417     | TypeDeclarations
418   ImportDeclarationsopt ::=
419     | ImportDeclarations
420   PackageDeclarationopt ::=
421     | PackageDeclaration
422   Type ::= DataType PlaceTypeSpecifieropt
423     | nullable Type
424     | future < Type GREATER
425   DataType ::= PrimitiveType
426   DataType ::= ClassOrInterfaceType
427     | ArrayType
428   PlaceTypeSpecifier ::= AT PlaceType
429   PlaceType ::= placelocal
430     | activitylocal
431     | current
432     | PlaceExpression
433   ClassOrInterfaceType ::= TypeName DepParametersopt
434   DepParameters ::= ( DepParameterExpr )
435   DepParameterExpr ::= ArgumentList WhereClauseopt
436     | WhereClause
437   WhereClause ::= : Expression
438   ArrayType ::= X10ArrayType
439   X10ArrayType ::= Type [ . ]
440     | Type reference [ . ]
441     | Type value [ . ]
442     | Type [ DepParameterExpr ]
443     | Type reference [ DepParameterExpr ]
444     | Type value [ DepParameterExpr ]
445   MethodModifier ::= atomic
446     | extern
447   ClassDeclaration ::= ValueClassDeclaration
448   ValueClassDeclaration ::= ClassModifiersopt value identifier Superopt Interfacesopt ClassBody
449     | ClassModifiersopt value class identifier Superopt Interfacesopt ClassBody
450   ArrayCreationExpression ::= new ArrayBaseType Unsafeopt [ ] ArrayInitializer
451     | new ArrayBaseType Unsafeopt [ Expression ]
452     | new ArrayBaseType Unsafeopt [ Expression ] Expression
453     | new ArrayBaseType Unsafeopt [ Expression ] ( FormalParameter ) MethodBody
454     | new ArrayBaseType value Unsafeopt [ Expression ]
455     | new ArrayBaseType value Unsafeopt [ Expression ] Expression
456     | new ArrayBaseType value Unsafeopt [ Expression ] ( FormalParameter ) MethodBody
457   ArrayBaseType ::= PrimitiveType
458     | ClassOrInterfaceType
459   ArrayAccess ::= ExpressionName [ ArgumentList ]
460     | PrimaryNoNewArray [ ArgumentList ]
461   Statement ::= NowStatement
462     | ClockedStatement
463     | AsyncStatement
464     | AtomicStatement
465     | WhenStatement
466     | ForEachStatement
467     | AtEachStatement
468     | FinishStatement
469   StatementWithoutTrailingSubstatement ::= NextStatement
470     | AwaitStatement
471   StatementNoShortIf ::= NowStatementNoShortIf
472     | ClockedStatementNoShortIf
473     | AsyncStatementNoShortIf
474     | AtomicStatementNoShortIf
475     | WhenStatementNoShortIf
476     | ForEachStatementNoShortIf
477     | AtEachStatementNoShortIf
478     | FinishStatementNoShortIf
479   NowStatement ::= now ( Clock ) Statement
480   ClockedStatement ::= clocked ( ClockList ) Statement
481   AsyncStatement ::= async PlaceExpressionSingleListopt Statement
482   AtomicStatement ::= atomic PlaceExpressionSingleListopt Statement
483   WhenStatement ::= when ( Expression ) Statement
484     | WhenStatement or ( Expression ) Statement
485   ForEachStatement ::= foreach ( FormalParameter : Expression ) Statement
486   AtEachStatement ::= ateach ( FormalParameter : Expression ) Statement
487   EnhancedForStatement ::= for ( FormalParameter : Expression ) Statement
488   FinishStatement ::= finish Statement
489   NowStatementNoShortIf ::= now ( Clock ) StatementNoShortIf
490   ClockedStatementNoShortIf ::= clocked ( ClockList ) StatementNoShortIf
491   AsyncStatementNoShortIf ::= async PlaceExpressionSingleListopt StatementNoShortIf
492   AtomicStatementNoShortIf ::= atomic StatementNoShortIf
493   WhenStatementNoShortIf ::= when ( Expression ) StatementNoShortIf
494     | WhenStatement or ( Expression ) StatementNoShortIf
495   ForEachStatementNoShortIf ::= foreach ( FormalParameter : Expression ) StatementNoShortIf
496   AtEachStatementNoShortIf ::= ateach ( FormalParameter : Expression ) StatementNoShortIf
497   EnhancedForStatementNoShortIf ::= for ( FormalParameter : Expression ) StatementNoShortIf
498   FinishStatementNoShortIf ::= finish StatementNoShortIf
499   PlaceExpressionSingleList ::= ( PlaceExpression )
500   PlaceExpression ::= Expression
501   NextStatement ::= next ;
502   AwaitStatement ::= await Expression ;
503   ClockList ::= Clock
504     | ClockList , Clock
505   Clock ::= identifier
506   CastExpression ::= ( Type ) UnaryExpressionNotPlusMinus
507   MethodInvocation ::= Primary ARROW identifier ( ArgumentListopt )
508   RelationalExpression ::= RelationalExpression instanceof Type
509   IdentifierList ::= IdentifierList , identifier
510     | identifier
511   Primary ::= FutureExpression
512   Primary ::= [ ArgumentList ]
513   AssignmentExpression ::= Expression ARROW Expression
514   Primary ::= Expression : Expression
515   FutureExpression ::= future PlaceExpressionSingleListopt { Expression }
516   FieldModifier ::= mutable
517     | const
518   PlaceTypeSpecifieropt ::=
519     | PlaceTypeSpecifier
520   DepParametersopt ::=
521     | DepParameters
522   WhereClauseopt ::=
523     | WhereClause
524   PlaceExpressionSingleListopt ::=
525     | PlaceExpressionSingleList
526   ArgumentListopt ::=
527     | ArgumentList
528   DepParametersopt ::=
529     | DepParameters
530   Unsafeopt ::=
531     | unsafe
\end{verbatim}
}
\twocolumn\par  \vfill\eject} % syntax
\extrapart{Changes from v0.32}

This is the first reference manual that corresponds to a working
implementation. As such a number of details missing from v0.32 have
been spelt out. A number of mistakes have been corrected, and
clarifications added.

The semantics of exception handling across asynchronous activities has
been clarified.

Exploded syntax has been introduced to make it convenient to
destructure points. 

\subsection{Limitations}

Exception propagation from an activity to its invoking activity is not
yet implemented.

All the type checking rules are not implemented. Thus if your program
is already correct, it will exeute correctly. If it is not correct, it
may still execute and give a result.

The predicate {\tt ==} for value types is not yet implemented.

\todo{Update this list from Mantisa.}\par  \vfill\eject % changes
%\newpage                   %  Put bib on it's own page (it's just one)
%\twocolumn[\vspace{-.18in}]%  Last bib item was on a page by itself.
\renewcommand{\bibname}{References}
\newpage\bibliographystyle{plain}
\bibliography{master}

%%\extrapart{Bibliography and references}

% My reference for proper reference format is:
%    Mary-Claire van Leunen.
%    {\em A Handbook for Scholars.}
%    Knopf, 1978.
% I think the references list would look better in ``open'' format,
% i.e. with the three blocks for each entry appearing on separate
% lines.  I used the compressed format for SIGPLAN in the interest of
% space.  In open format, when a block runs over one line,
% continuation lines should be indented; this could probably be done
% using some flavor of latex list environment.  Maybe the right thing
% to do in the long run would be to convert to Bibtex, which probably
% does the right thing, since it was implemented by one of van
% Leunen's colleagues at DEC SRC.
%  -- Jonathan

% This is just a personal remark on your question on the RRRS:
% The language CUCH (Curry-Church) was implemented by 1964 and 
% is a practical version of the lambda-calculus (call-by-name).
% One reference you may find in Formal Language Description Languages
% for Computer Programming T.~B.~Steele, 1965 (or so).
%  -- Matthias Felleisen


\begin{thebibliography}{99}

\bibitem{SICP}
Harold Abelson and Gerald Jay Sussman with Julie Sussman.
{\em Structure and Interpretation of Computer Programs.}
MIT Press, Cambridge, 1985.

\bibitem{readfloat}
William Clinger.
How to read floating point numbers accurately.
In {\em Proceedings of the 1990 ACM SIGPLAN Conference on Programming
  Language Design and Implementation}.  Forthcoming.

University of Oregon Technical Report CIS-TR-90-01.

\bibitem{RRRS}
William Clinger, editor.
The revised revised report on Scheme, or an uncommon Lisp.
MIT Artificial Intelligence Memo 848, August 1985.
Also published as Computer Science Department Technical Report 174,
  Indiana University, June 1985.

\bibitem{R4RS}
William Clinger and Jonathan Rees, editors.
The revised$^4$ report on the algorithmic language Scheme.
University of Oregon Technical Report CIS-TR-90-02.

\bibitem{Scheme311}
Carol Fessenden, William Clinger, Daniel P.~Friedman, and Christopher Haynes.
Scheme 311 version 4 reference manual.
Indiana University Computer Science Technical Report 137, February 1983.
Superceded by~\cite{Scheme84}.

\bibitem{Scheme84}
D.~Friedman, C.~Haynes, E.~Kohlbecker, and M.~Wand.
Scheme 84 interim reference manual.
Indiana University Computer Science Technical Report 153, January 1985.

\bibitem{CFractions}
G.~H.~Hardy and E.~M.~Wright.
{\em An Introduction to the Theory of Numbers.} 5th ed.
Oxford University Press, New York NY, 1979.

\bibitem{Haskell}
Paul Hudak and Philip Wadler, editors.
Report on the Functional Programming Language Haskell.
Yale University Research Report YALEU/DCS/RR-666, December 1988.

\bibitem{IEEE}
{\em IEEE Standard 754-1985.  IEEE Standard for Binary Floating-Point
Arithmetic.}  IEEE, New York, 1985.

\bibitem{Knuth}
Donald E. Knuth.
The Art of Computer Programming, volume 2: Seminumerical Algorithms.
Addison-Wesley, Reading MA, 1969.

\bibitem{Landin65}
Peter Landin.
A correspondence between Algol 60 and Church's lambda notation: Part I.
{\em Communications of the ACM} 8(2):89--101, February 1965.

\bibitem{Matula68}
David W. Matula.
In-and-Out Conversions.
{\em Communications of the ACM} 11(1):47--50, January 1968.

\bibitem{Matula70}
David W. Matula.
A Formalization of Floating-Point Numeric Base Conversion.
{\em IEEE Transactions on Computers} C-19, 8:681-692, August 1970.

\bibitem{MITScheme}
MIT Department of Electrical Engineering and Computer Science.
Scheme manual, seventh edition.
September 1984.

\bibitem{Penfield81}
Paul Penfield, Jr.
Principal values and branch cuts in complex APL.
In {\em APL '81 Conference Proceedings,} pages 248--256.
ACM SIGAPL, San Francisco, September 1981.
Proceedings published as {\em APL Quote Quad} 12(1), ACM, September 1981.

\bibitem{Pitman83}
Kent M.~Pitman.
The revised MacLisp manual (Saturday evening edition).
MIT Laboratory for Computer Science Technical Report 295, May 1983.

\bibitem{Rees82}
Jonathan A.~Rees and Norman I.~Adams IV.
T: A dialect of Lisp or, lambda: The ultimate software tool.
In {\em Conference Record of the 1982 ACM Symposium on Lisp and
  Functional Programming}, pages 114--122.

\bibitem{R3RS}
Jonathan Rees and William Clinger, editors.
The revised$^3$ report on the algorithmic language Scheme.
In {\em ACM SIGPLAN Notices} 21(12), ACM, December 1986.

\bibitem{Reynolds72}
John Reynolds.
Definitional interpreters for higher order programming languages.
In {\em ACM Conference Proceedings}, pages 717--740.
ACM, \todo{month?}~1972.

\bibitem{Rabbit}
Guy Lewis Steele Jr.
Rabbit: a compiler for Scheme.
MIT Artificial Intelligence Laboratory Technical Report 474, May 1978.

\bibitem{CLtL}
Guy Lewis Steele Jr.
{\em Common Lisp: The Language.}
Digital Press, Burlington MA, 1984.

\bibitem{CLtL2}
Guy Lewis Steele Jr.
{\em Common Lisp: The Language.} 2d ed.
Digital Press, Bedford MA, 1990.

\bibitem{Scheme78}
Guy Lewis Steele Jr.~and Gerald Jay Sussman.
The revised report on Scheme, a dialect of Lisp.
MIT Artificial Intelligence Memo 452, January 1978.

\bibitem{Heuristic}
Guy Lewis Steele Jr.~and Jon L White.
How to Print Floating-Point Numbers Accurately.
In {\em Proceedings of the 1990 ACM SIGPLAN Conference on Programming
  Language Design and Implementation}.  Forthcoming.

\bibitem{Stoy77}
Joseph E.~Stoy.
{\em Denotational Semantics: The Scott-Strachey Approach to
  Programming Language Theory.}
MIT Press, Cambridge, 1977.

\bibitem{Scheme75}
Gerald Jay Sussman and Guy Lewis Steele Jr.
Scheme: an interpreter for extended lambda calculus.
MIT Artificial Intelligence Memo 349, December 1975.

\bibitem{Vuillemin}
Jean Vuillemin.
Exact real computer arithmetic with continued fractions.
In {\em Proceedings of the 1988 ACM Conference on Lisp and
  Functional Programming}, pages 14--27.

\end{thebibliography}
	\par

\vfill\eject

% Adjustment to avoid having the last index entry on a page by itself.
%\addtolength{\baselineskip}{-0.1pt}

\bigskip

\documentclass[orivec,twoside,twocolumn]{algol60}
\usepackage{changebar}
%\documentclass[twoside]{algol60}
\def\Hat{{\tt \char`\^}}
\def\ccfont{\sf}
\usepackage{url}
\usepackage{times}
\pagestyle{headings}
\showboxdepth=0
\makeindex
% Macros for R^nRS.

\makeatletter

\newcommand{\topnewpage}{\@topnewpage}
\newcommand{\authorsc}[1]{{\scriptsize\scshape #1}}

% Chapters, sections, etc.

\newcommand{\extrapart}[1]{
 % \chapter{#1}
  \chapter*{#1}
  \markboth{#1}{#1}
  \vskip 1ex
  \addcontentsline{toc}{chapter}{#1}}

\newcommand{\clearchapterstar}[1]{
  \clearpage
  \topnewpage[
    \centerline{\large\bf\uppercase{#1}}
    \bigskip]}

\newcommand{\clearextrapart}[1]{
  \clearchapterstar{#1}
  \markboth{#1}{#1}
  \addcontentsline{toc}{chapter}{#1}}

\newcommand{\vest}{}
\newcommand{\dotsfoo}{$\ldots\,$}

\newcommand{\sharpfoo}[1]{{\tt\##1}}
\newcommand{\schfalse}{\sharpfoo{f}}
\newcommand{\schtrue}{\sharpfoo{t}}

\newcommand{\singlequote}{{\tt'}}  %\char19
\newcommand{\doublequote}{{\tt"}}
\newcommand{\backquote}{{\tt\char18}}
\newcommand{\backwhack}{{\tt\char`\\}}
\newcommand{\atsign}{{\tt\char`\@}}
\newcommand{\sharpsign}{{\tt\#}}
\newcommand{\verticalbar}{{\tt|}}

\newcommand{\coerce}{\discretionary{->}{}{->}}

% Knuth's \in sucks big boulders
\def\elem{\hbox{\raise.13ex\hbox{$\scriptstyle\in$}}}

\newcommand{\meta}[1]{{\noindent\hbox{\rm$\langle$#1$\rangle$}}}
\let\hyper=\meta
\newcommand{\hyperi}[1]{\hyper{#1$_1$}}
\newcommand{\hyperii}[1]{\hyper{#1$_2$}}
\newcommand{\hyperj}[1]{\hyper{#1$_i$}}
\newcommand{\hypern}[1]{\hyper{#1$_n$}}
\newcommand{\var}[1]{\noindent\hbox{\it{}#1\/}}  % Careful, is \/ always the right thing?
\newcommand{\vari}[1]{\var{#1$_1$}}
\newcommand{\varii}[1]{\var{#1$_2$}}
\newcommand{\variii}[1]{\var{#1$_3$}}
\newcommand{\variv}[1]{\var{#1$_4$}}
\newcommand{\varj}[1]{\var{#1$_j$}}
\newcommand{\varn}[1]{\var{#1$_n$}}

\newcommand{\vr}[1]{{\noindent\hbox{$#1$\/}}}  % Careful, is \/ always the right thing?
\newcommand{\vri}[1]{\vr{#1_1}}
\newcommand{\vrii}[1]{\vr{#1_2}}
\newcommand{\vriii}[1]{\vr{#1_3}}
\newcommand{\vriv}[1]{\vr{#1_4}}
\newcommand{\vrv}[1]{\vr{#1_5}}
\newcommand{\vrj}[1]{\vr{#1_j}}
\newcommand{\vrn}[1]{\vr{#1_n}}


\newcommand{\defining}[1]{\mainindex{#1}{\em #1}}
\newcommand{\ide}[1]{{\schindex{#1}\frenchspacing\tt{#1}}}

\newcommand{\lambdaexp}{{\cf lambda} expression}
\newcommand{\Lambdaexp}{{\cf Lambda} expression}
\newcommand{\callcc}{{\tt call-with-current-continuation}}

% \reallyindex{SORTKEY}{HEADCS}{TYPE}
% writes (index-entry "SORTKEY" "HEADCS" TYPE PAGENUMBER)
% which becomes  \item \HEADCS{SORTKEY} mainpagenumber ; auxpagenumber ...

\global\def\reallyindex#1#2#3{%
\write\@indexfile{"#1" "#2" #3 \thepage}}

\newcommand{\mainschindex}[1]{\label{#1}\reallyindex{#1}{tt}{main}}
\newcommand{\mainindex}[1]{\reallyindex{#1@{\rm #1}{main}}}
\newcommand{\schindex}[1]{\reallyindex{#1}{tt}{aux}}
\newcommand{\sharpindex}[1]{\reallyindex{#1}{sharpfoo}{aux}}
%vj%\renewcommand{\index}[1]{\reallyindex{#1}{rm}{aux}}

\newcommand{\domain}[1]{#1}
\newcommand{\nodomain}[1]{}
%\newcommand{\todo}[1]{{\rm$[\![$!!~#1$]\!]$}}
\newcommand{\todo}[1]{}

% \frobq will make quote and backquote look nicer.
\def\frobqcats{%\catcode`\'=13 %\catcode`\{=13{}\catcode`\}=13{}
\catcode`\`=13{}}
{\frobqcats
\gdef\frobqdefs{%\def'{\singlequote}
\def`{\backquote}}}%\def\{{\char`\{}\def\}{\char`\}}
\def\frobq{\frobqcats\frobqdefs}

% \cf = code font
% Unfortunately, \cf \cf won't work at all, so don't even attempt to
% next constructions which use them...
\newcommand{\cf}{\frenchspacing\tt}

% Same as \obeycr, but doesn't do a \@gobblecr.
{\catcode`\^^M=13 \gdef\myobeycr{\catcode`\^^M=13 \def^^M{\\}}%
\gdef\restorecr{\catcode`\^^M=5 }}

{\catcode`\^^I=13 \gdef\obeytabs{\catcode`\^^I=13 \def^^I{\hbox{\hskip 4em}}}}

{\obeyspaces\gdef {\hbox{\hskip0.5em}}}

\gdef\gobblecr{\@gobblecr}

\def\setupcode{\@makeother\^}

% Scheme example environment
% At 11 points, one column, these are about 56 characters wide.
% That's 32 characters to the left of the => and about 20 to the right.

\newenvironment{x10noindent}{
  % Commands for scheme examples
  \newcommand{\ev}{\>\>\evalsto}
  \newcommand{\lev}{\\\>\evalsto}
  \newcommand{\unspecified}{{\em{}unspecified}}
  \newcommand{\scherror}{{\em{}error}}
  \setupcode
  \small \cf \obeytabs \obeyspaces \myobeycr
  \begin{tabbing}%
\qquad\=\hspace*{5em}\=\hspace*{9em}\=\kill%   was 16em
\gobblecr}{\unskip\end{tabbing}}

%\newenvironment{scheme}{\begin{schemenoindent}\+\kill}{\end{schemenoindent}}
\newenvironment{x10}{
  % Commands for scheme examples
  \newcommand{\ev}{\>\>\evalsto}
  \newcommand{\lev}{\\\>\evalsto}
  \renewcommand{\em}{\rmfamily\itshape}
  \newcommand{\unspecified}{{\em{}unspecified}}
  \newcommand{\scherror}{{\em{}error}}
  \setupcode
  \small \cf \obeyspaces \myobeycr
  \footnotesize
  \begin{tabbing}%
\qquad\=\hspace*{5em}\=\hspace*{9em}\=\+\kill%   was 16em
\gobblecr}{\unskip\end{tabbing}\normalsize}

\newcommand{\evalsto}{$\Longrightarrow$}

% Rationale

\newenvironment{rationale}{%
\bgroup\small\noindent{\em Rationale:}\space}{%
\egroup}

% Notes

\newenvironment{note}{%
\bgroup\small\noindent{\em Note:}\space}{%
\egroup}

% Manual entries

\newenvironment{entry}[1]{
  \vspace{3.1ex plus .5ex minus .3ex}\noindent#1%
\unpenalty\nopagebreak}{\vspace{0ex plus 1ex minus 1ex}}

\newcommand{\exprtype}{syntax}

% Primitive prototype
\newcommand{\pproto}[2]{\unskip%
\hbox{\cf\spaceskip=0.5em#1}\hfill\penalty 0%
\hbox{ }\nobreak\hfill\hbox{\rm #2}\break}

% Parenthesized prototype
\newcommand{\proto}[3]{\pproto{(\mainschindex{#1}\hbox{#1}{\it#2\/})}{#3}}

% Variable prototype
\newcommand{\vproto}[2]{\mainschindex{#1}\pproto{#1}{#2}}

% Extending an existing definition (\proto without the index entry)
\newcommand{\rproto}[3]{\pproto{(\hbox{#1}{\it#2\/})}{#3}}

% Grammar environment

\newenvironment{grammar}{
  \def\:{\goesto{}}
  \def\|{$\vert$}
  \cf \myobeycr
  \begin{tabbing}
    %\qquad\quad \= 
    \qquad \= $\vert$ \= \kill
  }{\unskip\end{tabbing}}

%\newcommand{\unsection}{\unskip}
\newcommand{\unsection}{{\vskip -2ex}}

% Commands for grammars
\newcommand{\arbno}[1]{#1\hbox{\rm*}}  
\newcommand{\atleastone}[1]{#1\hbox{$^+$}}

\newcommand{\goesto}{$\longrightarrow$}

% mark modifications (for the grammar) From Igor Pechtchanski/Watson/IBM@IBMUS
\newlength{\modwidth}\setlength{\modwidth}{0.005in}
\newlength{\modskip}\setlength{\modskip}{.4em}
\newlength{\@modheight}
\newlength{\@modpos}
\providecommand{\markmod}[1]{%
  \setlength{\@modheight}{#1}%
  \addtolength{\@modheight}{-0.06in}%
  \setlength{\@modpos}{\linewidth}%
  \addtolength{\@modpos}{0.285in}%         Magic
  \addtolength{\@modpos}{\modwidth}%
  \addtolength{\@modpos}{\modskip}%
  \marginpar{\vspace{-\@modheight}%
             \hspace{-\@modpos}%
             \rule{\modwidth}{#1}}%
}

% The index

\def\theindex{%\@restonecoltrue\if@twocolumn\@restonecolfalse\fi
%\columnseprule \z@
%!! \columnsep 35pt
\clearpage
\@topnewpage[
    \centerline{\large\bf\uppercase{Alphabetic index of definitions of concepts,}}
    \centerline{\large\bf\uppercase{keywords, and procedures}}
    \vskip 1ex \bigskip]
    \markboth{Index}{Index}
    \addcontentsline{toc}{chapter}{Alphabetic index of 
 definitions of concepts,\hfil\penalty0 \hbox{\hspace*{2em} keywords, and procedures}}
    \bgroup %\small
    \parindent\z@
    \parskip\z@ plus .1pt\relax\let\item\@idxitem}

\def\@idxitem{\par\hangindent 40pt}

\def\subitem{\par\hangindent 40pt \hspace*{20pt}}

\def\subsubitem{\par\hangindent 40pt \hspace*{30pt}}

\def\endtheindex{%\if@restonecol\onecolumn\else\clearpage\fi
\egroup}

\def\indexspace{\par \vskip 10pt plus 5pt minus 3pt\relax}

\makeatother
\newcommand{\Xten}{{\sf X10}}
\newcommand{\XtenCurrVer}{{\sf X10 v1.1}}
\newcommand{\java}{{\sf Java}}
\newcommand{\Java}{{\sf Java}}
\newcommand{\notfouro}[1]{}
\newcommand{\notinfouro}[1]{}
\newcommand{\futureext}[1]{{\em \paragraph{Future Extensions.}#1}}
\newcommand{\tbd}{} % marker for things to be done later.
\newcommand{\limitation}[1]{{\em Limitation: #1}} % marker for things to be done later.


\def\headertitle{The \XtenCurrVer{} Report }
\def\integerversion{1.0}

% Sizes and dimensions

\topmargin -.375in       %    Nominal distance from top of page to top of
                         %    box containing running head.
\headsep 15pt            %    Space between running head and text.

\textheight 663pt        % Height of text (including footnotes and figures, 
                         % excluding running head and foot).

\textwidth 523pt         % Width of text line.
\columnsep 15pt          % Space between columns 
\columnseprule 0pt       % Width of rule between columns.

\parskip 5pt plus 2pt minus 2pt % Extra vertical space between paragraphs.
\parindent 0pt                  % Width of paragraph indentation.
\topsep 0pt plus 2pt            % Extra vertical space, in addition to 
                                % \parskip, added above and below list and
                                % paragraphing environments.

\oddsidemargin  -.5in    % Left margin on odd-numbered pages.
\evensidemargin -.5in    % Left margin on even-numbered pages.

%% End of sizes and dimensions
\makeatletter
\newsavebox{\eStop}
\savebox{\eStop}{\raisebox{0.6ex}{\framebox[0.5em]\relax}}

\def\newtenv#1{\@ifnextchar[{\@otxm{#1}}{\@ntxm{#1}}}

\def\@ntxm#1#2{\@ifnextchar[{\@xntxm{#1}{#2}}{\@yntxm{#1}{#2}}}

\def\@xntxm#1#2[#3]{\expandafter\@ifdefinable\csname #1\endcsname
{\@definecounter{#1}\@addtoreset{#1}{#3}%
\expandafter\xdef\csname the#1\endcsname{\expandafter\noexpand
  \csname the#3\endcsname \@thmcountersep \@thmcounter{#1}}%
\global\@namedef{#1}{\@txm{#1}{#2}}\global\@namedef{end#1}{\@endtenv}}}

\def\@yntxm#1#2{\expandafter\@ifdefinable\csname #1\endcsname
{\@definecounter{#1}%
\expandafter\xdef\csname the#1\endcsname{\@thmcounter{#1}}%
\global\@namedef{#1}{\@txm{#1}{#2}}\global\@namedef{end#1}{\@endtenv}}}

\def\@otxm#1[#2]#3{\expandafter\@ifdefinable\csname #1\endcsname
  {\global\@namedef{the#1}{\@nameuse{the#2}}%
\global\@namedef{#1}{\@txm{#2}{#3}}%
\global\@namedef{end#1}{\@endtenv}}}

\def\@txm#1#2{\refstepcounter
    {#1}\@ifnextchar[{\@ytxm{#1}{#2}}{\@xtxm{#1}{#2}}}

\def\@xtxm#1#2{\@begintenv{#2}{\csname the#1\endcsname}\ignorespaces}
\def\@ytxm#1#2[#3]{\@opargbegintenv{#2}{\csname
       the#1\endcsname}{#3}\ignorespaces}

%DEFAULT VALUES
\def\@begintenv#1#2{\trivlist \item[\hskip \labelsep{\bf #1\ #2}]}
\def\@opargbegintenv#1#2#3{\trivlist
      \item[\hskip \labelsep{\bf #1\ #2\ (#3)}]}
\def\@endtenv{\hfill\usebox{\eStop}\endtrivlist}
\makeatother

\newtenv{example}{Example}[section]

\begin{document}

\parindent 0pt %!! 15pt                    % Width of paragraph indentation.

%\hfil {\bf 7 Feb 2005}
%\hfil \today{}

% First page

\thispagestyle{empty}

% \todo{"another" report?}

\title{Report on the Experimental Language \Xten \\
\large Version \integerversion}
\author{Please send comments to \\
Vijay Saraswat at \texttt{vsaraswa@us.ibm.com}}
\date\today
\maketitle

\if 0
\topnewpage[{
\begin{center}   
{\huge\bf Report on the Experimental Language \Xten{}}
\vskip 1ex
$$
\begin{tabular}{l@{\extracolsep{.5in}}lll}
\multicolumn{4}{c}{\sc Version \integerversion}\\
\multicolumn{4}{c}{\sc Please send comments to 
Vijay Saraswat at 
{\tt vsaraswa@us.ibm.com}}\\
%\multicolumn{4}{c}{({\sc IBM Confidential})}

%\ldots
\end{tabular}
$$
\vskip 2ex
% {\it Dedicated to the Memory of APL} % vj
{\bf \today}
\vskip 2.6ex
\end{center}


}]
\fi

\newcommand\authorsc[1]{#1}
%\newcommand\authorsc[1]{\textsc{#1}}


\chapter*{Summary}
This draft report provides an initial description of the programming
language \Xten. \Xten{} is a single-inheritance class-based object-oriented
(OO) programming language designed for high-performance, high-productivity
computing on high-end computers supporting $\approx 10^5$ hardware threads
and $\approx 10^{15}$ operations per second. 

{}\Xten{} is based on state-of-the-art object-oriented programming
languages and deviates from them only as necessary to support its
design goals. The language is intended to have a simple and clear
semantics and be readily accessible to mainstream OO programmers. It
is intended to support a wide variety of concurrent programming
idioms.
%, incuding data parallelism, task parallelism, pipelining.
%producer/consumer and divide and conquer.

%We expect to revise this document in the light of experience gained in implementing
%and using this language.

The \Xten{} design team consists of
\authorsc{David Bacon}, 
\authorsc{Raj Barik}, 
\authorsc{Ganesh Bikshandi}, 
\authorsc{Bob Blainey}, 
\authorsc{Philippe Charles}, 
\authorsc{Perry Cheng}, 
\authorsc{Christopher Donawa}, 
\authorsc{Julian Dolby}, 
\authorsc{Kemal Ebcio\u{g}lu},
\authorsc{Robert Fuhrer},
\authorsc{Patrick Gallop}, 
\authorsc{Christian Grothoff}, 
\authorsc{Allan Kielstra}, 
\authorsc{Sreedhar Kodali}, 
\authorsc{Sriram Krishnamoorthy}, 
\authorsc{Nathaniel Nystrom}, 
\authorsc{Igor Peshansky}, 
\authorsc{Vijay Saraswat} (contact author), 
\authorsc{Vivek Sarkar},
\authorsc{Armando Solar-Lezama},  
\authorsc{S. Alexander Spoon}, 
\authorsc{Sayantan Sur}, 
\authorsc{Christoph von Praun},
\authorsc{Pradeep Varma},
\authorsc{Krishna Venkata},
\authorsc{Jan Vitek}, and
\authorsc{Tong Wen}.

For extended discussions and support we would like to thank: 
Robert Callahan, Calin
Cascaval, Norman Cohen, Elmootaz Elnozahy, John Field, Bob Fuhrer,
Orren Krieger, Doug Lea, John McCalpin, Paul McKenney, Ram Rajamony,
R.K.~Shyamasundar, Filip Pizlo, V.T.~Rajan, Frank Tip, and Mandana Vaziri.

We thank Jonathan Rhees and William Clinger with help in obtaining the
\LaTeX{} style file and macros used in producing the Scheme report,
on which this document is based. We acknowledge the influence of
the $\mbox{\Java}^{\mbox{\authorsc{\small tm}}}$ Language
Specification \cite{jls2}.
%document, as evidenced by the numerous citations in the text.

This document revises Version 1.1 of the Report, released in
June 2007. It documents the language corresponding to the second
revision of the first version of the implementation.  This
revision was done by
\authorsc{Raj Barik}, 
\authorsc{Philippe Charles}, 
\authorsc{Christopher Donawa}, 
\authorsc{Robert Fuhrer},
\authorsc{Nathaniel Nystrom},  
\authorsc{Vijay Saraswat},
\authorsc{Vivek Sarkar},
\authorsc{Pradeep Varma}, and
\authorsc{Krishna Venkata}.
(Earlier implementations benefited from significant contributions by
\authorsc{Christian Grothoff} and 
\authorsc{Christoph von Praun}.)
\authorsc{Tong Wen} has written many application programs
in \Xten{}. \authorsc{Guojing Cong} has helped in the
development of many applications.


%\vfill
%\begin{center}
%{\large \bf
%*** DRAFT*** \\
%%August 31, 1989
%\today
%}\end{center}

\vfill
\eject


\chapter*{Contents}
\addvspace{3.5pt}                  % don't shrink this gap
\renewcommand{\tocshrink}{-3.5pt}  % value determined experimentally
{\footnotesize
\tableofcontents
}

\vfill
\eject


   \par  % vj: first page
\clearextrapart{Introduction}

\subsection*{Background}

Bigger computational problems need bigger computers capable of
performing a larger number of operations per second. The era of
increasing performance by simply increasing clocking frequency now
seems to be behind us; faster chips run hotter and current cooling
technology does not scale as rapidly as the clock. Instead, computer
designers are starting to look at ``scale out'' systems in which the
system's computational capacity is increased by adding additional
nodes of comparable power to existing nodes, and connecting nodes with
a high-speed communication network.

A central problem with scale out systems is a definition of the {\em
memory model}, that is, a model of the interaction between shared
memory and  simultaneous (read, write) operations on that
memory by multiple processors. The traditional ``one operation at a
time, to completion'' model that underlies Lamport's notion of {\em
sequential consistency} (SC) proves too expensive to implement in
hardware, at scale. Various models of {\em relaxed consistency} have
proven too difficult for programmers to work with.  

One response to this problem has been to move to a {\em fragmented
memory model}. Multiple processors -- each sequentially consistent
internally -- are made to interact via a relatively language-neutral
message-passing format such as MPI \cite{mpi}. This model has enjoyed
some success: several high-performance applications have been written
in this style. Unfortunately, this model leads to a {\em loss of
programmer productivity}: the mesage-passing format is integrated into
the host language by means of an application-programming interface
(API), the programmer must explicitly represent and manage the
interaction between multiple processes and choreograph their data
exchange; large data-structures (such as distributed arrays, graphs,
hash-tables) that are conceptually unitary must be thought of as
fragmented across different nodes; all processors must generally
execute the same code (in an SPMD fashion) etc.

One response to this problem has been the advent of the {\em
partitioned global address space} (PGAS) model underlying languages such as
UPC, Titanium and Co-Array Fortran \cite{pgas}. These languages permit
the programmer to think of a single computation running across the
multiple processors, sharing a common address space. All data resides
at some processors, which is said to have {\em affinity} to the
data. Each processor may operate directly on the data it contains but
must use some indirect mechanism to access or update data at other
processors. Some kind of global {\em barriers} are used to ensure that
processors remain roughly in lock-step.

\Xten{} is a modern object-oriented programming language
in the PGAS family. The fundamental goal of \Xten{} is to enable
high-performance, high-productivity programming for high-end
(scale-out) computers -- for traditional numerical computation
workloads (such as weather simulation, molecular dynamics, particle
transport problems etc) as well as commercial server
workloads. \Xten{} is based on state-of-the-art object-oriented
programming ideas primarily to take advantage of their proven
flexibility and ease-of-use for a wide spectrum of programming
problems. \Xten{} takes advantage of several years of research (e.g.{}
in the context of the Java Grande forum,
\cite{moreira00java,kava}) on how to adapt such languages to the context of
high-performance numerical computing. Thus \Xten{} provides support
for user-defined {\em value types} (such as {\tt int}, {\tt float},
{\tt complex} etc), including operator overloading, supports a very
flexible form of multi-dimensional arrays (based on ideas in ZPL
\cite{zpl}) and supports IEEE-standard floating point arithmetic.

The major novel contribution of \Xten{} however is its flexible
treatment of concurrency, distribution and locality, within an
integrated type system. \Xten{} introduces {\em places} as an
abstraction for a {\em virtual shared-memory multi-processor}; a
computation runs over a large collection of places. Each place hosts
some data and runs one or more {\em activities}. Activities are
extremely lightweight threads of execution and may dynamically spawn
new activities locally or at remote places. {\em Clocks} are used to
ensure that a programmer-specified, data-dependent set of activities
has quiesced before another action is initiated. Arrays may be
distributed across multiple places. A static type system allows the
programmer to keep track of the location of objects and ensures
statically that an activity does not synchronously attempt to
read/write remote data.

%% say something about native.

\Xten{} is an experimental language. This document lays out an initial set 
of ideas which we expect to be the basis of an initial
implementation. Several representative concurrent idioms have found
pleasant expression in \Xten. We intend to develop several full-scale
applications to get better experience with the language, and revisit
the design in the light of this experience. Future versions of the
language are expected to support user-definable operators and permit
the specification of generic classes and methods. 


   \par  % 0.1
\chapter{Overview of \Xten}
\section{Semantics}
\Xten{} may be thought of as (generic) \java{} less concurrency, arrays and built-in types,  plus {\em places}, {\em activities}, {\em clocks}, (distributed,
multi-dimensional) {\em arrays} and {\em value} types. All these changes are
motivated by the desire to use the new language for high-end,
high-performance, high-productivity computing.

\subsection{Places and activities}
The central new concept in \Xten{} is that of a {\em place}
(\S~\ref{XtenPlaces}).  A place may be thought of conceptually as a
``virtual shared-memory multi-processor'': a computational unit with a
finite, though perhaps dynamically varying, number of hardware threads
and a bounded amount of shared memory uniformly accessible by all
threads.  An \Xten{} program is intended to run on a computer capable
of supporting millions of places.

An \Xten{} computation acts on {\em data
objects}(\S~\ref{XtenObjects}) through the execution of lightweight
threads called {\em activities}(\S~\ref{XtenActivities}).  Objects are
of two kinds. A {\em scalar} object has a small, statically fixed set
of fields, each of which has a distinct name. A scalar object is
located at a single place and stays at that place throughout its
lifetime.  An {\em aggregate} object has many fields (the number may
be known only when the object is created), uniformly accessed through
an index (e.g.{} an integer) and may be distributed across many
places. The distribution of an aggregate object remains unchanged
throughout the computation. \Xten{} assumes an underlying garbage
collector will dispose of (scalar and aggregate) objects and reclaim
the memory associated with them once it can be determined that these
objects are no longer accessible from the current state of the
computation. (There are no operations in the language to allow a
programmer to explicitly release memory.)

{}\Xten{} has a {\em unified} or {\em global address space}. This
means that an activity can reference objects at other places.
However, an activity may synchronously access data items only in the
current place (the place in which the activity is running). It may
atomically update one or more data items, but only in the current
place.  Indeed, all accesses to mutable shared data must occur from
within an {\em atomic section}. To read a remote location, an activity
must spawn another activitiy {\em asynchronously}
(\S~\ref{AsynchronousActivity}). This operation returns immediately,
leaving the spawning activity with a {\em future}
(\S~\ref{XtenFutures}) for the result. Similarly, remote location can
be written into only by asynchronously spawning an activity to run at
that location.

Throughout its lifetime an activity executes at the same place. An
activity may dynamically spawn activities in the current or remote
places.

\paragraph{Atomic sections}

\Xten{} introduces statements of the form {\cf atomic S} where {\cf S}
is a statement.  The type system ensures that such a statement will
dynamically access only local data. (The statement may throw
a {\cf BadPlaceException} -- but only because of a failed place cast.)
Such a statement is executed by the activity as if in a single step
during which all other activities are frozen.

\paragraph{Asynch activities}

An asynch activity is a statement of the form {\cf async (P) S} where
{\cf P} is a place expression and {\cf S} is a statement.  Such a
statement is executed by spawning an activity at the place designated
by {\cf P} to execute statement {\cf S}.

An async expression of type {\cf future T}} has the form {\cf future
(P) E} where {\tt E} is an expression of type {\tt T}. It executess
the expression {\tt E} at the place {\tt P} as an async activity,
immediately returning with a future. The future may later be forced
causing the activity to be blocked until the return value has been
computed by the async activity.

\subsection{Clocks}
The MPI style of coordinating the activity of multiple processes with
a single barrier is not suitable for the dynamic, asynchronous network
of activities in an \Xten{} computation. Instead, it becomes necessary
to allow a computation to use multiple barriers. \Xten{} {\em clocks}
(\S~\ref{XtenClocks}) are designed to offer the functionality of
multiple barriers in a dynamic context while still supporting
determinate, deadlock-free parallel computation.

Activities may use clocks to repeatedly detect quiescence of arbitrary
programmer-specified, data-dependent set of activities. Each activity
is spawned with a known set of clocks and may dynamically create new
clocks. At any given time an activity is {\em registered} with zero or
more clocks. It may register newly created activities with a clock,
un-register itself with a clock, suspend on a clock or require that a
statement (possibly involving execution of new async activities) be
executed to completion before the clock can advance.  At any given
step of the execution a clock is in a given phase. It advances to the
next phase only when all its registered activities have {\em quiesced}
(by executing a {\tt continue} operation on the clock), and all
statements scheduled for execution in this phase have terminated.
When a clock advances, all its activities may now resume execution.

Thus clocks act as {\em barriers} for a dynamically varying collection
of activities. They generalize the barriers found in MPI style program
in that an activity may use multiple clocks simultaneously. Yet
programs using clocks are guaranteed not to suffer from
deadlock. Clocks are also integrated into the \Xten{} type system,
permitting variables to be declared so that they are {\tt final} in each
phase of a clock.

\subsection{Interfaces and Classes}
Programmers write \Xten{} code by writing {\em generic interfaces}
(\S~\ref{XtenInterfaces}) and {\em generic classes}
(\S~\ref{XtenClasses}). Generic interfaces and classes may be
defined over a collection of {\em type parameters}. Instances can be
created only from {\em concrete} classes; such a class has all its
type parameters (if any) instantiated with concrete classes and
concrete interfaces.

\subsection{Scalar classes}
An \Xten{} scalar class (\S~\ref{XtenClasses}) has fields, methods and
inner types (interfaces, classes), subclasses another class, and
implements one or more interfaces. Thus \Xten{} classes live in a
single-inheritance code hierarchy.  \Xten{} allows the programmer to
define literals for classes, and overload infix/prefix/postfix
operators.

There are two kinds of scalar classes: {\em reference} classes
(\S~\ref{ReferenceClasses}) and {\em value} classes
(\S~\ref{ValueClasses}).

A reference class typically has updatable fields. Objects of such a
class may not be freely copied from place to place. Methods may be
invoked on such an object only by an activity in the same place.

A value class (\S~\ref{ValueClasses}) has no updatable fields (defined
directly or through inheritance), and allows no reference
subclasses. (Fields may be typed at reference classes, so may contain
references to objects with mutable state.) Objects of such a class may
be freely copied from place to place, and may be implemented very
efficiently. Methods may be invoked on such an object from any place.

\Xten{} has no primitive classes. However, the standard library {\cf x10.lang} supplies (final) value classes {\cf boolean}, {\cf byte}, {\cf short}, {\cf char}, {\cf int}, {\cf long}, {\cf float}, {\cf complex} and {\cf string}. The user may defined additional arithmetic value classes using the facilities of the language.

\subsection{Arrays, Regions and Distributions}
An \Xten{} array is a function from a {\em distribution}
(\S~\ref{XtenDistributions}) to a base type (which may itself be an
array type).

A distribution is a map from a {\em region} (\S~\ref{XtenRegions}) to a
subset of places.  A region is a collection of indices.

Operations are provided to construct regions from other regions, and
to iterate over regions. Standard set operations, such as union,
disjunction and set difference are available for regions.

A primitive set of distributions is provided, together with operations
on distributions. A {\em sub-distribution} of a distribution is one
which is defined on a smaller region and agrees with the distribution
at all points.  The standard operations on regions are extended to
distributions.

In future versions of the language, a programmer may specify new
distributions, and new operations on distributions.

A new array can be created by restricting an existing array to a
sub-distribution, by combining multiple arrays, and by performing
pointwise operations on arrays with the same distribution.

\Xten{} allows array constructors to iterate over the underlying
distribution and specify a value at each item in the underlying
region. Such a constructor may spawn activities at multiple places.


\subsection{Nullable type constructor}

\Xten{} has a {\cf nullable} type constructor which can be applied uniformly to
scalar (value or reference) and array types. This type constructor
returns a new type which adds a special value {\cf null} to the set of
values of its argument type, unless the argument type already has this
value.

\subsection{Statements and expressions}
\Xten{} supports the standard set of primitive operations (assignment, classcasts) and sequential control constructs (conditionals, looping, method
invocation, exception raising/catching) etc.

\paragraph{Place casts}
The programmer may use the standard classcast mechanism
(\S~\ref{ClassCast}) to cast a value to a located type. A {\cf
BadPlaceException} is thrown if the value is not of the given
type. This is the only language construct that throws a {\cf
BadPlaceException}.

\subsection{Translating MPI programs to \Xten{}}

While \Xten{} permits considerably greater flexibility in writing
distributed programs and data structures than MPI, it is instructive
to examine how to translate MPI programs to \Xten.

Each separate MPI process can be translated into an \Xten{}
place. Async activities may be used to read and write variables
located at different processes. A single clock may be used for barrier
synchronization between multiple MPI processes. \Xten{} collective
operations may be used to implement MPI collective operations.
\Xten{} is more general than MPI in (a)~not requiring synchronization
between two processes in order to enable one to read and write the
other's values, (b)~permitting the use of high-level atomic sections
within a process to obtain mutual exclusion between multiple
activities running in the same node (c)~permitting the use of multiple
clocks to combine the expression of different physics (e.g.{}
computations modeling blood coagulation together with computations
involving the flow of blood), (d)~not requiring an SPMD style of
computation.

%\note{Relaxed exception model}
\subsection{Summary and future work}

{}\Xten{} is considerably higher-level than thread-based languages in
that it supports dynamically spawning very lightweight activities, the
use of atomic operations for mutual exclusion, and the use of clocks
for repeated quiescence detection of a data-dependent set of
activities. Yet it is much more concrete than languages like HPF in
that it forces the programmer to explicitly deal with distribution of
data objects. In this the language reflects the designers belief that
issues of locality and distribution cannot be hidden from the
programmer of high-performance code in high-end computing.  A
performance model that distinguishes between computation and
communication must be made explicit and transparent.\footnote{In this
\Xten{} is similar to more modern languages such as ZPL \cite{zpl}.} At
the same time we believe that the place-based type system and support
for generic programming will allow the \Xten{} programmer to be highly
productive; many of the tedious details of distribution-specific code
can be handled in a generic fashion.

We expect the next version of the language to be significantly
informed by experience in implementing and using the language. We
expect it to have constructs to support continuous program
optimization, and allow the programmer to provide guidance on
clustering places to (hardware) nodes. For instance, we may introduce
a notion of hierarchical clustering of places.




  \par % Semantics section. What else?
\vskip 2ex
\clearchapterstar{Description of the language} %\unskip\vskip -2ex
\chapter{Lexical structure}

In general, \Xten{} follows \java{} rules \cite[Chapter 3]{jls2} for
lexical structure.

Lexically a program consists of a stream of white space, comments,
identifiers, keywords, literals, separators and operators.

\paragraph{Whitespace}
% Whitespace \index{whitespace} follows \java{} rules \cite[Chapter 3.6]{jls2}.
ASCII space, horizontal tab (HT), form feed (FF) and line
terminators constitute white space.

\paragraph{Comments}
% Comments \index{comments} follows \java{} rules
% \cite[Chapter 3.7]{jls2}. 
All text included within the ASCII characters ``\xcd"/*"'' and
``\xcd"*/"'' is
considered a comment and ignored; nested comments are not
allowed.  All text from the ASCII characters
``\xcd"//"'' to the end of line is considered a comment and is ignored.

\paragraph{Identifiers}

Identifiers \index{identifier} are defined as in \java.
Identifiers consist of a single letter followed by zero or more
letters or digits.
Letters are defined as the characters for which the \java{}
method \xcd"Character.isJavaIdentifierStart" returns true.
Digits are defined as the ASCII characters \xcd"0" through \xcd"9".

\paragraph{Keywords}
\Xten{} reserves the following keywords:
\begin{xten}
abstract        any             as              async
at              ateach          atomic          await
break           case            catch           class
clocked         const           continue        current
def             default         do              else
extends         extern          final           finally
finish          for             foreach         future
goto            has             here            if
implements      import          in              instanceof
interface       local           native          new
next            nonblocking     or              package
private         protected       property        public
return          safe            self            shared
static          super           switch          this
throw           throws          try             type
val             value           var             when
while
\end{xten}
Note that the primitive types are not considered keywords.
The keyword \xcd{goto} is reserved, but not used.

\paragraph{Literals}\label{Literals}\index{literals}

Literals are either integers, floating point numbers, booleans,
characters, strings, and \xcd"null".
\XtenCurrVer{} defines literal syntax in the same way as \java{} does.
Unsigned 32-bit integers are suffixed with
\xcd{U} or \xcd{u}.
Signed 64-bit integers are suffixed with
\xcd{L} or \xcd{l}.
Unsigned 64-bit integers are suffixed with
any of \xcd{LU}, \xcd{Lu}, \xcd{UL}, \xcd{Ul},
\xcd{lU}, \xcd{lu}, \xcd{uL}, or \xcd{ul}.

\paragraph{Separators}
\Xten{} has the following separators and delimiters:
\begin{xten}
( )  { }  [ ]  ;  ,  .
\end{xten}

\paragraph{Operators}
\Xten{} has the following operators:
\begin{xten}
==  !=  <   >   <=  >=
&&  ||  &   |   ^
<<  >>  >>>
+   -   *   /   %
++  --  !   ~
&=  |=  ^=
<<= >>= >>>
+=  -=  *=  /=  %=
=   ?   :   =>  ->
<:  :>  @   ..
\end{xten}




	\par % 0.1
\chapter{Types}
\label{XtenTypes}\index{types}

{}\Xten{} is a {\em strongly typed} object language: every variable
and expression has a type that is known at compile-time. Further,
\Xten{} has a {\em unified} type system: all data items created at
runtime are {\em objects} (\S~\ref{XtenObjects}. Types limit the
values that variables can hold, and specify the places at which these
values lie.

{}\Xten{} supports two kinds of objects, {\em reference objects} and
{\em value objects}.  Reference objects are instances of {\em
reference classes} (\S~\ref{ReferenceClasses}). They may contain
mutable fields and must stay resident in the place in which they were
created. Value objects are instances of {\em value classes}
(\S~\ref{ValueClasses}). They are immutable and may be freely copied
from place to place. Either reference or value objects may be 
{\em scalar} (instances of a non-array class) or {\em aggregate} (instances
of arrays).

An \Xten{} type is either a {\em reference type} or a {\em value
type}.  Each type consists of a {\em data type}, which is a set of
values, and a {\em place type} which specifies the place at which the
value resides.  Types are constructed through the application of {\em
type constructors} (\S~\ref{TypeConstructors}).

Types are used in variable declarations, casts, object creation, array
creation, class literals and {\cf instanceof} expressions.\footnote{In
order to allow this version of the language to focus on the core new
ideas, \XtenCurrVer{} does not have user-definable classloaders,
though there is no technical reason why they could not have been
added.}

A variable is a storage location (\S~\ref{XtenVariables}). All
variables are initialized with a value and cannot be observed without
a value. 

Variables whose value may not be changed after initialization are
called {\em final variables} (or sometimes {\em constants}).  The
programmer indicates that a variable is final by using the annotation
{\tt final} in the variable declaration.  

%% Final variables play an 
%% important role in \Xten{}, as we shall discuss below. For this reason,
%% \Xten{} enforces the lexical restriction that all variables whose name
%% starts with an upper case letter are implicitly declare final. (It is
%% not an error to also explicitly declare such variables as
%% final.)\index{Upper-case Convention}

\section{Type constructors}\index{type constructors}\label{TypeConstructors}

An \Xten{} type is a pair specifying a {\em datatype} and a {\em
placetype}. Semantically, a datatype specifies a set of values and a
placetype specifies the set of places at which these values may
live. Thus taken together, a type specifies both the kind of value
permitted and its location. 

\begin{x10}
509   Type ::=  DataType  PlaceTypeSpecifieropt
510     | nullable  Type
511     | future <  Type > 
512   DataType ::=  PrimitiveType
513   DataType ::=  ClassOrInterfaceType
514     |  ArrayType
\end{x10}

For simplicity, this version of \Xten{} does not permit the
specification of generic classes or interfaces. This is expected to be
remedied in future versions of the language.

Every class and interface definition in \Xten{} defines a type with
the same name. Additionally, {}\Xten{} specifies three {\em type
constructors}: {\tt nullable}, the {\tt future}, and array type
constructors. We discuss these constructors and place types in detail
in the secions that follow; here we briefly discss interface and class
declarations.

\paragraph{Interface declarations.}\label{InterfaceTypes}
An interface declaration specifies a name, a list of extended
interfaces, and constants ({\tt public static final} fields) and
method signatures associated with the interface. Each interface
declaration introduces a type with the same name as the declaration.
Semantically, the data type is the set of all objects which are
instances of (value or reference) classes that implement the
interface. A class implements an interface if it says it does and if
it implements all the methods defined in the interface.


The {\em interface declaration} (\S~\ref{XtenInterfaces}) takes as
argument one or more interfaces (the {\em extended} interfaces), one
or more type parameters and the definition of constants and method
signatures and the name of the defined interface.  Each such
declaration introduces a data type.

\begin{x10}
426   DataType ::= ClassOrInterfaceType
433   ClassOrInterfaceType ::= TypeName 
13    ClassType ::= TypeName
15    TypeName ::= identifier
16     | TypeName . identifier
\end{x10}

\paragraph{Reference class declarations.}\label{ReferenceTypes}
The {\em reference class declaration} (\S~\ref{ReferenceClasses}) takes
as argument a reference class (the {\em extended class}), one or more
interfaces (the {\em implemented interfaces}), the definition of
fields, methods and inner types, and returns a class of the named type
(\S~\ref{ReferenceClasses}). Each such declaration introduces a data
type. Semantically, the data type is the set of all objects which are
instances of (subclasses of) the class.

\paragraph{Value class declarations.}
The {\em value class declaration} (\S~\ref{ValueClasses}) is
similar to the reference class declaration except that it must extend
either a value class or a reference class that has no mutable fields.
It may be used to construct a value type in the same way as a
reference class declaration can be used to construct a reference type.

\input{NullableTypeConstructor.tex}
\input{FutureTypes}
\input{ArrayTypes}
\notfouro{\input{PlaceTypes}}

\section{Variables}\label{XtenVariables}\index{variables}

A variable of a reference data type {\tt reference R} where {\tt R} is
the name of an interface (possibly with type arguments) always holds a
reference to an instance of a class implementing the interface {\tt R}.

A variable of a reference data type {\tt R} where {\tt R} is the name
of a reference class (possibly with type arguments) always holds a
reference to an instance of the class {\tt R} or a class that is a
subclass of {\tt R}. 

A variable of a reference array data type {\tt R [D]} is always an
array which has as many variables as the size of the region underlying
the distribution {\cf D}. These variables are distributed across
places as specified by {\cf D} and have the type {\tt R}.

A variable of a nullary (reference or value) data type {\tt nullable
T} always holds either the value (named by) {\tt null} or a value of
type {\tt T} (these cases are not mutually exclusive).

A variable of a value data type {\tt value R} where {\tt R} is the
name of an interface (possibly with type arguments) always holds
either a reference to an instance of a class implementing {\tt R} or
an instance of a class implementing {\tt R}. No program can
distinguish between the two cases.

A variable of a value data type {\tt R} where {\tt R} is the name of a
value class always holds a reference to an instance of {\tt R} (or a
class that is a subclass of {\tt R}) or an instance of {\tt R} (or a
class that is a subclass of {\tt R}). No program can distinguish
between the two cases.

A variable of a value array data type {\tt V value [R]} is always an
array which has as many variables as the size of the region {\tt R}.
Each of these variables is immutable and has the type {\tt V}.

\Xten{} supports seven kinds of variables: final {\em class
variables} (static variables), {\em instance variables} (the instance
fields of a class), {\em array components}, {\em method parameters},
{\em constructor parameters}, {\em exception-handler parameters} and
{\em local variables}.

\subsection{Final variables}\label{FinalVariable}\index{variable!final}\index{final variable}
A final variable satisfies two conditions: 
\begin{itemize}
\item it can be assigned to at most once, 
\item it must be assigned to before use. 
\end{itemize}

\Xten{} follows \java{} language rules in this respect \cite[\S
4.5.4,8.3.1.2,16]{jls2}. Briefly, the compiler must undertake a
specific analysis to statically guarantee the two properties above.

\todo{Check if this analysis needs to be revisited.}

\subsection{Initial values of variables}
\label{NullaryConstructor}\index{nullary constructor}
\cbstart 
Every variable declared at a type must always contain a value of that type.

Every class variable must be initialized before it is read, through
the execution of an explicit initializer or a static block. Every
instance variable must be initialized before it is read, through the
execution of an explicit initializer or a constructor.  An instance
variable declared at a nullable type (and not declared to be {\tt
final}) is assumed to have an initializer which sets the value to {\tt
null}.

Each method and constructor parameter is initialized to the
corresponding argument value provided by the invoker of the method. An
exception-handling parameter is initialized to the object thrown by
the exception. A local variable must be explicitly given a value by
initialization or assignment, in a way that the compiler can verify
using the rules for definite assignment \cite[\S~16]{jls2}.

\cbend

\section{Objects}\label{XtenObjects}\index{Object}

An object is an instance of a scalar class or an array type.  It is
created by using a class instance creation expression
(\S~\ref{ClassCreation}) or an array creation
(\S~\ref{ArrayInitializer}) expression, such as an array
initializer. An object that is an instance of a reference (value) type
is called a {\em reference} ({\em value}) {\em object}.

All value and reference classes subclass from {\tt x10.lang.Object}.
This class has one {\tt const} field {\tt location} of type {\tt
x10.lang.place}. \index{place.location} Thus all objects in \Xten{}
are located (have a place). However, \Xten{} permits value objects to
be freely copied from place to place because they contain no mutable
state.  It is permissible for a read of the {\tt location} field of
such a value to always return {\tt here} (\S~\ref{Here});
therefore no space needs to be allocated in the object representation
for such a field.

In \XtenCurrVer{} a reference object stays resident at the place at
which it was created for its entire lifetime.

{}\Xten{} has no operation to dispose of a reference.  Instead the
collection of all objects across all places is globally garbage
collected.

{}\Xten{} objects do not have any synchronization information (e.g.{}
a lock) associated with them. Thus the methods on {\tt
java.lang.Object} for waiting/synchronizing/notification etc are not
available in \Xten. Instead the programmer should use atomic blocks
(\S~\ref{AtomicBlocks}) for mutual exclusion and clocks
(\S~\ref{XtenClocks}) for sequencing multiple parallel operations.

A reference object may have many references, stored in fields of
objects or components of arrays. A change to an object made through
one reference is visible through another reference. \Xten{} mandates
that all accesses to mutable objects shared between multiple
activities must occur in an atomic section (\S\ref{AtomicBlocks}).

\cbstart 
Note that the creation of a remote async activity
(\S~\ref{AsyncActivity}) {\cf A} at {\cf P} may cause the automatic creation of
references to remote objects at {\cf P}. (A reference to a remote
object is called a {\em remote object reference}, to a local object a
{\em local object reference}.)  For instance {\cf A} may be created
with a reference to an object at {\cf P} held in a variable referenced
by the statement in {\cf A}.  Similarly the return of a value by a
{\cf future} may cause the automatic creation of a remote object
reference, incurring some communication cost.  An {}\Xten{}
implementation should try to ensure that the creation of a second or
subsequent reference to the same remote object at a given place does
not incur any (additional) communication cost.

\cbend 

A reference to an object may carry with it the values of final fields
of the object. The implementation should try to ensure that the cost
of communicating the values of final fields of an object from the
place where it is hosted to any other place is not incurred more than
once for each target place.

{}\Xten{} does not have an operation (such as Pascal's ``dereference''
operation) which returns an object given a reference to the
object. Rather, most operations on object references are transparently
performed on the bound object, as indicated below. The operations on
objects and object references include:
\begin{itemize}

{}\item Field access (\S~\ref{FieldAccess}). An activity holding a
reference to a reference object may perform this operation only if the
object is local.  An activity holding a reference to a value object
may perform this operation regardless of the location of the object
(since value objects can be copied freely from place to place).  The
implementation should try to ensure that the cost of copying the field
from the place where the object was created to the referencing place
will be incurred at most once per referencing place, according to the
rule for final fields discussed above.

\item Method invocation (\S~\ref{MethodInvocation}).  An activity
holding a reference to a reference object may perform this operation
only if the object is local.  An activity holding a reference to a
value object may perform this operation regardless of the location of
the object (since value objects can be copied freely). The \Xten{}
implementation must attempt to ensure that the cost of copying enough
relevant state of the value object to enable this method invocation to
succeed is incurred at most once for each value object per place.

{}\item Casting (\S~\ref{ClassCast}).  An activity can perform this
operation on local or remote objects, and should not incur
communication costs (to bring over type information) more than once
per place.

{}\item {\cf instanceof} operator (\S~\ref{instanceOf}).  An activity
can perform this operation on local or remote objects, and should not
incur communication costs (to bring over type information) more than
once per place.

\item The stable equality operator {\cf ==} and {\cf !=}
(\S~\ref{StableEquality}). An activity can perform these operations on
local or remote objects, and should not incur communication costs
(to bring over relevant information) more than once per place.

% \item The ternary conditional operator {\cf ?:}
\end{itemize}

\section{Built-in types}
\cbstart 

The package {\tt x10.lang} provides a number of built-in class and
interface declarations that can be used to construct types.

\subsection{The class {\tt Object}}\label{Object}\index{Object}
The class {\cf x10.lang.Object} is a superclass of all other classes.
A variable of this type can hold a reference to an instance of any
scalar or array type.

\begin{x10}
package x10.lang;
public class Object \{
  public String toString() \{...\}
  public boolean equals(Object o) \{...\}
  public int hashCode() \{...\}
\}
\end{x10}

The method {\tt equals} and {\tt hashCode} are useful in hashtables,
and are defined as in \java. The default implementation of {\tt equals}
is stable equality, \S~\ref{StableEquality}. This method may be overridden
in a (value or reference) subclass.

\subsection{The class {\tt String}}
\Xten{} supports strings as in \java{}. A string object is immutable,
and has a concatenation operator ({\tt +})  available on it.

\subsection{Arithmetic classes}
Several value types are provided that encapsulate
abstractions (such as fixed point and floating point arithmetic)
commonly implemented in hardware by modern computers:

\begin{x10}
boolean byte short char 
int long 
double float 
\end{x10}

\XtenCurrVer{} defines these data types in the same way as the 
\java{} language. Specifically, a program may contain literals
that stand for specific instances of these classes. The syntax
for literals is the same as for \java{} (\S~ref{Literals}).

\futureext{
\Xten{} may provide mechansims in the future to permit the programmer
to specify how a specific value class is to be mapped to special
hardware operations (e.g.{} along the lines of
\cite{kava}). Similarly, mechanisms may be provided to permit the user
to specify new syntax for literals.
}
\subsection{Places, distributions, regions, clocks}
\Xten{} defines several other classes in the {\cf x10.lang}
package. Please consult the API documentation for more details.

\subsection{Java utility classes}
\XtenCurrVer{} programmers may import and use \java{} packages such as
{\tt java.util}, e.g.{} {\tt java.util.Set}, {\tt
java.lang.System}. \Xten{} programs should not invoke methods
that use the {\tt wait/notify/notifyAll} methods on such objects,
since this may interfere with \Xten{} synchronization. The
implementation does not make imported \java{} classes
automatically extend {\tt x10.lang.Object}. 

\futureext{
The above represents an {\em ad hoc} integration of \java{} libraries
into \Xten{}. It has the unfortunate consequence that not every run-time
value in an \Xten{} program execution is an instance of a subclass of
{\tt x10.lang.Object}. }

In the future a more principled and robust scheme will be worked
out. Such a scheme will need to attend to the integration of the
\java{} and \Xten{} type systems, and develop a notion of place for 
\java{} objects.

\cbend
\section{Conversions and Promotions}\label{XtenConversions}\label{XtenPromotions}\index{conversions}\index{promotions}

{}\XtenCurrVer{} supports \java's conversions and promotions
(identity, widening, narrowing, value set, assignment, method
invocation, string, casting conversions and numeric promotions)
appropriately modified to support \Xten's built-in numeric classes
rather than \java's primitive numeric types.

This decision may be revisited in future version of the language in
favor of a streamlined proposal for allowing user-defined
specification of conversions and promotions for value types, as part
of the syntax for user-defined operators.
	\par % empty
%% Fri Dec 08 06:15:28 2006
Mon Jul 03 16:00:11 2006

\chapter{Dependent types}\label{XtenDepTypes}\index{dependent types}
\def\withmath#1{\relax\ifmmode#1\else{$#1$}\fi}
\def\LL#1{\withmath{\lbrack\!\lbrack #1\rbrack\!\rbrack}}

Dependent types are a fundamental extension to the type system for
Java-like languages. A dependent type records constraints on
{\em properties} (final fields) of the type.  Since types are a fundamental
building block of Java-like languages, the introduction of dependent
types affects many facets of the language simultaneously --- the
definition of types, classes, interfaces, methods, constructors,
fields, inheritance, overriding, overloading, and type related
operators (cast, and instanceof).

Indeed dependent types bring substantial expressiveness to Java-like
languages. Just as generic classes permit a single definition for a
class $C$ to be treated as a template for an unbounded number of classes
obtained from $C$ by ``instantiating'' $C$ with type arguments, so also a
dependent type permits a single definition to produce a potentially
unbounded family of types. That is, just as generic types permit a
programmer to express and use {\em functions} from types to types, so also
dependent types permit a programmer to express and use functions from values
to types.  Indeed, the family of types generated from $C$ form a
lattice of subtypes of $C$, one for each constraint on the properties
of $C$ expressible in the underlying constraint system, but all sharing
the same "structure".

In \Xten{} dependent types are checked statically. However, as in
Java-like languages an instanceof relation is available to dynamically
check that an object belongs to a particular (dependent) type. Also a
cast operation is available to force the runtime system to treat an
object o as belonging to a particular dependent type (the operation
throws a ClassCast exception if it is not possible to do so).

Dependent types are also the basis for an implicit syntax for
\Xten{}. This is discussed in the last section.

\section{Properties}\label{DepType:Properties}\index{properties}
The dependent type system is built on the notion of {\em properties}, 
for types (classes and interfaces).

\begin{x10}
NormalClassDeclaration ::= 
   ClassModifiersopt class identifier 
   PropertyListopt Superopt Interfacesopt ClassBody

NormalInterfaceDeclaration ::= 
   InterfaceModifiersopt interface identifier 
   PropertyListopt ExtendsInterfacesopt InterfaceBody

PropertyList     ::= ( Properties WhereClauseopt )
Properties       ::= Property
                 | Properties , Property
Property         ::= Type identifier    
PropertyListopt  ::= \$Empty | PropertyList
\end{x10}


A property has a name and a type. The declaration of a type (class or
interface) introduces a sequence of defined properties for the
type. 

\begin{quotation}
   {\sc Static Semantics Rule:} It is a compile-time error for a class
  defining a property {\cf P p} to have an ancestor class that defines a property
  with the name {\cf p}.  
\end{quotation}


Each class {\cf C} defining a property {\cf P p} implicitly has a field

\begin{x10}
public final P p;  
\end{x10}

\noindent and a getter method 

\begin{x10}
public final P p() { return p;}  
\end{x10}

\noindent Each interface {\cf I} defining a property {\cf P p} implicitly has a getter method

\begin{x10}
public final P p() { return p;}
\end{x10}

\begin{quotation}
  {\sc Static Semantics Rule:} It is a compile-time error for a class or
  interface defining a property {\cf P p} to have an existing method with
  the signature {\cf P p()}.   
\end{quotation}


Properties are used to build dependent types from classes, as
described below (\S~\ref{DepType:DepType}).

The {\tt WhereClause} in a {\tt PropertyList} specifies an explicit
condition on the properties of the type, and is discussed further
below (\S~\ref{DepType:Class}, \ref{DepType:Interface}).

\begin{quotation}
    {\sc Static Semantics Rule:}  Every constructor for a class defining
   properties {\cf P1 p1, \ldots, Pn pn} must ensure that each of the fields
   corresponding to the properties is definitely initialized (cf
   requirement on initialization of final fields in Java) before the
   constructor returns.  
\end{quotation}


\begin{example}
 A class representing immutable 2d points, with two properties {\tt i} and 
{\tt j}.
  \begin{x10}
   value class Point(int i, int j) { ... }
   value class point(int rank) { ... }
  \end{x10}
  
\end{example}

\section{Dependent types}\label{DepType:DepType}\index{dependent type}

A dependent type (deptype) is of the form {\cf C(: c)} where {\cf C} is a class
or interface, and {\cf c} is a {\em constraint}. ({\cf C} is said to be 
{\em the base type} of the deptype, and {\cf c} the {\em constraint} of the deptype.)  A
constraint is a boolean expression that can only use a predefined set
of operators and methods. 

\begin{x10}
Type  ::=   PrimitiveType
         | ClassOrInterfaceType
         | ArrayType
         | nullable < Type > DepParamtersopt
         | future < Type > DepParametersopt
PrimitiveType ::= NumericType DepParametersopt
         | boolean DepParametersopt
ClassOrInterfaceType   ::= 
  TypeName DepParametersopt PlaceTypeSpecifieropt
ClassType              ::= 
  TypeName DepParametersopt PlaceTypeSpecifieropt
InterfaceType          ::= 
  TypeName DepParametersopt PlaceTypeSpecifieropt
PlaceTypeSpecifier     ::=  ! PlaceTypeopt
PlaceType              ::= current | Expression

DepParameters    ::= ( DepParameterExpr ) 
DepParameterExpr ::= ArgumentList WhereClauseopt
WhereClause      ::= : Constraint
Constraint       ::= Expression
ArgumentList     ::= Expression 
   | ArgumentList , Expression
DepParametersopt ::= null | DepParameters
WhereClauseopt   ::= null | WhereClause
PlaceTypeopt     ::= null | PlaceType
\end{x10}

\begin{quotation}
{\sc Static Semantics: Variable Occurrence}
  In a deptype T=C(:c), the only variables that may occur in c are (a)
  self, (b) properties visible at T, (c) final local variables, final
  method parameters or final constructor parameters visible at T, (d)
  final fields visible at T's lexical place in the source program.  
\end{quotation}

\begin{quotation}
{\sc Static Semantics: This restrictions}

  The special variable {\cf this} may be used in a depclause for a type {\cf T}
  only if (a)~it occurs in a property declaration for a class, (b)~it
  occurs in an instance method, (c)~it occurs in an instance field, (d)~it
  occurs in an instance initializer.
\end{quotation}

In particular it may not be used in types that occur in a static
context, or in the arguments, body or return type of a constructor or
in the extends or implements clauses of class and interface
definitions.  In these contexts the object that {\cf this} would
correspond to has either not been formed or is not well defined.

\begin{quotation}
{\sc Static Semantics: Variable visibility}
  If a type {\cf T} occurs in a field, method or constructor
  declaration, then all variables used in {\cf T} must have at least the
  same visibility as the declaration.  The relation ``at least the same
  visibility as'' is given by the transitive closure of:

  \begin{x10}
public > protected, protected > private
public > package, package > private
  \end{x10}

All inherited properties of a type {\cf T} are visible in the property
list of {\cf T}, and the body of {\cf T}.

\end{quotation}

In general local variables/parameters/properties/fields are visible at
{\cf T} if they are defined before {\cf T} in the program. This rule applies to
types in property lists as well as parameter lists (for methods and
constructors).  An exception is made for the return type of a method:
all the arguments to the method are considered to be visible, even
though they occur lexically after the return type (given the \Java{}
syntactic convention that the return type for a method precedes the
argument list for the method).

We permit variable declarations {\cf T v} where {\cf T} is obtained
from a dependent type {\cf C(:c)} by replacing one or more occurrences
of {\cf self} in {\cf c} by {\cf v}. (If such a declaration {\cf T v}
is type-correct, it must be the case that the variable {\cf v} is not
visible at the type {\cf T}. Hence we can always recover the
underlying deptype {\cf C(:c)} by replacing all occurrences of {\cf v}
in the constraint of {\cf T} by {\cf self}.)

For instance, {\cf  int(: v > 0) v} is shorthand for {\cf int(: self > 0) v}.
\begin{quotation}
{\sc Static Semantics: Constraint type}
  The type of c must be boolean.  
\end{quotation}

A variable occuring in the constraint {\cf c} of a deptype, other than
{\cf self} or a property of {\cf self}, is said to be a {\em
parameter} of{\cf c}.\label{DepType:Parameter} \index{parameter}

An instance {\cf o} of {\cf C} is said to be of type {\cf C (:c)} (or: {\em belong to}
{\cf C(:c)}) if the predicate {\cf c} evaluates to {\cf true} in the current lexical
environment, augmented with the binding {\cf [self |-> o]}. We shall
use the function \LL{{\cf C(:c)}} to denote the set of objects that belong
to {\cf C(:c)}. 

\section{Type definition}\label{DepType:Class}
A class definition 
\begin{x10}
ClassModifiersopt class identifier 
    PropertyListopt Superopt Interfacesopt ClassBody  
\end{x10}

\noindent and an interface definition
\begin{x10}
InterfaceModifiersopt interface identifier 
   PropertyListopt ExtendsInterfacesopt InterfaceBody  
\end{x10}

\noindent may reference several deptypes. The types of properties, the
specification of the super clause and the specification of interfaces
may each involve deptypes.

\begin{x10}
Super ::= extends ClassType
Superopt ::= null | Super
ClassType ::= 
  TypeName DepParametersopt PlaceTypeSpecifieropt

Interfaces ::= implements InterfaceTypeList
InterfaceTypeList ::= InterfaceType
       | InterfaceTypeList , InterfaceType
Interfacesopt ::= null | Interfaces
InterfaceType ::= 
 TypeName DepParametersopt PlaceTypeSpecifieropt
\end{x10}

\section{Where clauses}\label{DepType:WhereClauses}\index{where clauses}

There is a general recipe for constructing a list of parameters or
properties {\cf T1(:c1) x1, ... , Tk(:ck) xk} that must satisfy a given
(satisfiable) constraint {\cf c}. 

\begin{x10}
class Foo (T1 (: (T2 x2; ...; Tk xk;  c) x1, 
       T2 (: (T3 x3; ...; Tk xk;  c) x2, 
        ...
       Tk (:  c) xk) { 
 ...
}
\end{x10}

The first type {\cf T1 (:T2 x2;...;Tk xk; c) x1} is consistent iff
{\cf (exists T1 x1, T2 x2,..., Tk xk) c} is consistent. The second is
consistent iff
\begin{x10}
forall T1(: exists (T2 x2,..., Tk xk) c) x1
exists T2 x2. (exists T3 x2,..., Tk xk) c
\end{x10}
\noindent But this is always true. Similarly for the conditions for the other
properties.

Thus logically every satisfiable constraint {\cf c} on a list of parameters
{\cf x1,..., xk} can be expressed using the dependent types of xi, provided
that the constraint language is rich enough to permit existential
quantifiers.

Nevertheless we will find it convenient to permit the programmer to
explicitly specify a depclause after the list of properties, thus:
\begin{x10}
class Point(int i, int j) \{ ... \}
class Line(Point start, Point end :  end != start) 
  \{ ... \}
class Triangle (Line a, Line b, Line c 
       : a.end == b.start \&\& b.end == c.start \&\&
         c.end == a.start) \{ ... \}
class SolvableQuad(int a, int b, int c 
                   : a*x*x+b*x+c==0)  \{ ... \}
class Circle (int r, int x, int y 
              : r > 0 \&\& r*r==x*x+y*y)\{ ... \}
class NonEmptyList extends List(: n > 0) \{...\}
\end{x10}

Consider the definition of the class {\cf Line}. This may be thought of as
saying: the class {\cf Line} has two fields, {\cf Point start} and {\cf Point
end}. Further, every instance of Line must satisfy the constraint that
{\cf end !=start}. Similarly for the other class definitions. 

In the general case, the production for {\cf NormalClassDeclaration}
specifies that the list of properties may be followed by a {\cf
WhereClause}:

\begin{x10}
NormalClassDeclaration ::= 
    ClassModifiersopt class identifier 
    PropertyListopt Superopt Interfacesopt ClassBody

NormalInterfaceDeclaration ::= 
   InterfaceModifiersopt interface identifier 
   PropertyListopt ExtendsInterfacesopt InterfaceBody

PropertyList     ::= ( Properties WhereClauseopt )
\end{x10}

All the properties in the list, together with inherited properties,
may appear in the {\cf WhereClause}. A property list {\cf T1 x1, ...., Tn xn : c}
for a class {\cf Foo} is said to be consistent if each of the {\cf Ti} are
consistent and the constraint
\begin{x10}
      exists  T1 x1, ..., Tn xn, Foo self . c
\end{x10}
\noindent is valid (always true).

\section{Type invariants}\label{DepType:TypeInvariant}\index{Type Invariant}

With every defined class or interface {\cf T} we associate a {\em type
invariant} {\tt i(T)} as follows. The type invariant associated with
{\cf x10.lang.Object} is the proposition

\begin{x10}
nullable<place> self.loc  
\end{x10}

The type invariant associated with any interface {\cf I} that extends
interfaces {\cf I1,..., Ik} and defines properties {\cf P1 x1, ..., Pn xn} and
specifies a where clause {\cf c} is given by:

\begin{x10}
  ti(I1) \&\& ... \&\& tk(Ik) \&\& P1 self.x1 
  \&\& ... \&\& Pn self.xn \&\& c  
\end{x10}

Similarly the type invariant associated with any class {\cf C} that
implements interfaces {\cf I1,..., Ik}, extends class {\cf D} and defines
properties {\cf P1 x1,..., Pn xn} and specifies a where clause {\cf c} is given
by:
\begin{x10}
  i(D) \&\& P1 self.x1 \&\& ... \&\& Pn self.xn \&\& c  
\end{x10}

The {\sc Int Implements} Static Semantic rule below requires that the
type invariant associated with a class entail the type invariants of
each interface that it implements.

The static semantics rules below guarantee that for any variable {\cf v} of
type {\cf T(:c)} (where {\cf T} is an interface name or a class name) the only
objects {\cf o} that may be stored in {\cf v} are such that {\cf o} satisfies
{\cf i(T)[o/this] \&\& c[o/self]}.

\section{Consistency of deptypes}\label{DepType:Consistency}\index{deptype,consistency}

A dependent type {\cf C(:c)} may contain zero or more parameters. We require
that a type never be empty -- so that it is possible for a variable of
the type to contain a value. This is accomplished by requiring that
the constraint c must be satisfiable {\em regardless} of the value assumed
by parameters to the contraint (if any). Formally, consider a type
{\cf T=C(: c)}, with the variables {\cf F1 f1, ..., Fk fk} free in {\cf c}.  Let 
{\cf S={F1 f1, .., Fk fk, Fk+1 fk+1, ... Fn fn}} be the smallest set of
declarations containing {\cf F1 f1, ..., Fk fk} and closed under the rule: {\cf F
f} in {\cf S} if a reference to variable {\cf f} (which is declared as {\cf F f}) occurs
in a type in {\cf S}.

(NOTE: The syntax rules for the language ensure that {\cf S} is always
finite. The type for a variable {\cf v} cannot reference a variable whose
type depends on {\cf v}.)

We say that {\cf T=C(:c)} is {\em parametrically consistent} (in brief:
{\em consistent}) if

\begin{itemize}
  \item Each type {\cf F1, ..., Fn} is (recursively) parametrically consistent, and
\item It can be established that {\cf forall F1 f1, .., Fn fn. exists C self. c \&\& i(C)}.
\end{itemize}
\noindent where {\tt i(C)} is the invariant associated with the type {\cf C}
(\S~\ref{DepType:TypeInvariant}).  Note by definition of {\cf S} the formula on the
above has no free variables.

\begin{quotation}
   {\sc Static Semantics Rule:}
    For a declaration {\cf T v} to be type-correct, {\cf T} must be parametrically
    consistent. The compiler issues an error if it cannot determine
    the type is parametrically consistent.
\end{quotation}

\begin{example}

A class that represents a line has two distinct points:
\begin{x10}
class Array(int  rank, 
    region(:rank==this.rank) region) \{...\}  
\end{x10}
\end{example}

One can use deptypes to define other closed geometric figures as well.

\begin{example}
Here is an example:
\begin{x10}
 class Point(int x, int y) \{...\}
 class Line( Point start, 
        Point(: self != this.start) end) 
\{...\}      
\end{x10}
\end{example}


To see that the declaration {\cf Point(: self != start) end} is
parametrically consistent, note that the following formula is valid:
\begin{x10}
forall Line this. 
  exists Point self. self != this.start  
\end{x10}
\noindent since the set of all {\cf Points} has more than one element.

\begin{example}
A triangle has three lines sharing three vertices.
\begin{x10}
class Triangle 
 (Line a, 
  Line(: a.end == b.start) b, 
  Line(: b.end == c.start \&\& c.end == a.start) c) 
 \{ ...\}
\end{x10}
\end{example}


Given {\cf Line a}, the type {\cf Line(: a.end == b.start) b} is consistent, and
given the two, the type {\cf Line(: b.end == c.start \&\& c.end == a.start) c}
is consistent.

%%Similarly:
%%
%%   // A class with properties a, b,c,x satisfying the 
%%   // given constraints.
%%   class SolvableQuad(int a, int b, 
%%                      int(: b*b - 4*a*c >= 0) c, 
%%                      int(: a*x*x + b*x + c==0) x) { 
%%     ...
%%   }
%%
%%  // A class with properties r, x, and y satisfying
%%  // the conditions for (x,y) to lie on a circle with center (0,0)
%%  // and radius r.
%%   class Circle (int(: r> 0) r, 
%%                 int(: r*r - x*x >= 0) x,
%%                 int(: y*y == r*r -x*x) y) { 
%%   ...
%%   }

\section{Equivalence of deptypes}\label{DepType:Equivalence}\label{deptype,equivalence}

Two dependent types {\cf C(:c)} and {\cf C(:d)} are said to be {\em equivalent} if 
{\cf c} is true whenever {\cf d} is, and vice versa. Thus, 
$\LL{C(:c)} = \LL{C(:d)}$.

Note that two deptypes that are syntactically different may be
equivalent. For instance, {\cf int(:self >= 0)} and {\cf int(:self ==
0 || self > 0)} are equivalent though they are syntactically
distinct. The \Java{} type system is essentially a nominal system -- two
types are the same if and only if they have the same name. The \Xten{}
type system extends the nominal type system of \Java{} to permit
constraint-based equivalence.

A dependent type {\cf C(:c)} is said to refine a type {\cf C(:d)} if
{\cf c} implies {\cf d}.  In such a case we have $\LL{C(:c)}$ is a
subset of $\LL{C(:d)}$. All dependent types defined on {\cf C} refine
{\cf C} since {\cf C} is equivalent to {\cf C(:true)}.

\section{Type checking rules}
\subsection{Class definitions}

Consider a class definition
\begin{x10}
ClassModifiersopt 
 class C (P1 x1,..., Pn xn)  extends D(:d) 
   implements I1(:c1),..., Ik(:ck)
 ClassBody  
\end{x10}

Each of the following static semantics rules must be satisfied:

\begin{quotation}
{\sc Static Semantics: Int-implements}
    The type invariant {\cf i(C)} of {\cf C} must entail {\cf ci[this/self]} for each 
  {\cf i} in {\cf 1:k}.  

{\sc Static Semantics: Super-extends}
    The return type {\cf c} of each constructor in {\cf ClassBody} must entail {\cf d}.
\end{quotation}

\subsection{Constructor definitions}

A constructor for a class {\cf C} is guaranteed to return an object of the
class on successful termination. This object must satisfy i(C), the
class invariant associated with {\cf C} (\S~\ref{DepType:TypeInvariant}). However,
often the objects returned by a constructor may satisfy {\em stronger}
properties than the class invariant. \Xten{}'s dependent type system
permits these extra properties to be asserted with the constructor in
the form of a deptype (the ``return type'' of the constructor):

\begin{x10}
ConstructorDeclarator ::=  
  SimpleTypeName DepParametersopt 
 ( FormalParameterListopt WhereClauseopt )
\end{x10}

As with method declarations, the parameter list for the constructor
may specify a where clause that is to be satisfied by the parameters
to the list.

\begin{example}
Here is another example.
\begin{x10}
public class List(int(:n >=0) n) \{
 protected nullable<Object>   value;
 protected nullable<List(n-1)>  tail;
 public List(t.n+1)(Object o, final List t) \{
     n=t.n+1;
     tail=t;
     value=o;
 \}
 public List(0) () \{
     n=0;
     value=null;
     tail=null;
 \}
 ...
\}
\end{x10}
The second constructor returns a {\cf List} that is guaranteed to have {\cf n==0};
the first constructor is guaranteed to return a List with {\cf n>0}
(in fact, {\cf n==t.n+1}, where the argument to the constructor is {\cf t}). 
This is recorded by the programmer in the deptype associated with the
constructor.
\end{example}

\begin{quotation}
{\sc Static Semantics:  Super-invoke}
   Let {\cf C} be a class with properties {\cf P1 p1, ..., Pn pn}, invariant {\cf c}
   extending the deptype {\cf D(:d)} (where {\cf D} is the name of a class).

   For every constructor in {\cf C} the compiler checks that the call to
   super invokes a constructor for D whose return type is strong enough
   to entail d. Specifically, if the call to super is of the form 
     {\cf      super(e1, ..., ek)}
   and the static type of each expression ei is Si, and the invocation
   is statically resolved to a constructor
{\cf       D(:d1) (T1 x1,..., Tk xk : c)}
   then it must be the case that 
   \begin{x10}
S1 x1, ..., Si xi |- Ti xi  (for i in 1:k)
S1 x1, ..., Sk xk |- c  
d1[a/self] \&\& S1 x1 ... \&\& Sk xk |- d[a/self]      
   \end{x10}
\noindent   where {\cf a} is a constant that does not appear in 
{\cf S1 x1 \&\& ... \&\& Sk xk}.
  
\end{quotation}

\begin{quotation}
{\cf Static Semantics: Constructor return}
   The compiler checks that every constructor for {\cf C} ensures that
   the properties {\cf p1,..., pn} are initialized with values which satisfy
   {\cf t(C)}, and its own return type {\cf c'} as follows.  In each constructor, the
   compiler checks that the static types {\cf Ti} of the expressions {\cf ei}
   assigned to pi are such that the following is true
   \begin{x10}
    T1 p1, ...., Tn pn |- t(C) \&\& c'     
   \end{x10}
\end{quotation}
(Note that for the assignment of ei to pi to be type-correct it must be the
    case that Ti pi |- Pi pi.) 


\begin{quotation}
{\sc Static Semantics: constructor invocation}
The compiler must check that every invocation {\cf C(e1,..., en)} to a
constructor is type correct: each argument {\cf ei} must have a static type
that is a subtype of the declared type {\cf Ti} for the {\cf i}th argument of the
constructor, and the conjunction of static types of the argument must
entail the {\cf WhereClause} in the parameter-list of the constructor.
\end{quotation}

\section{Field definitions}

Not every instance of a class needs to have every field defined on the
class. In Java-like languages this is ensured by conditionally setting
fields to a default value, such as {\cf null}, in those instances where the
fields are not needed.  

Consider the class {\cf List} used earlier.  Here all instances of {\cf List}
returned by the second constructor do not need the fields {\cf value} and
{\cf tail}; their value is set to null.

\Xten{} permits a much cleaner solution that does not require default
values such as null to be stored in such fields. \Xten{} permits fields to
be {\em guarded}, that is defined only if a certain constraint on the
properties of the class, called the {\cf guard} of that field, is true.

\begin{x10}
FieldDeclaration  ::= 
   FieldModifiersopt ThisClauseopt 
   Type VariableDeclarators ;
ThisClause       ::= this DepParameters
ThisClauseopt    ::= null | ThisClause
\end{x10}

It is illegal for code to access a guarded field through a reference
whose static type does not satisfy the associated guard, even
implicitly (i.e.{} through an implicit {\cf this}). Rather the source
program should contain an explict cast, e.g.{} {\cf C(:c) me = (C(:c)) this}.

\begin{quotation}
{\sc Static Semantics Rule:} Let {\cf f} be a field defined in class
{\cf C} with guard {\cf this(:c)}.  The compiler declares an error if
field {\cf f} is accessed through a reference {\cf o} whose static
type is not a subtype of {\cf C(:c)}.
\end{quotation}

\begin{example}

We may now rewrite the List example:
\begin{x10}
public class List(int(:n >=0) n) \{
  protected this(:n>0) Object  value;
  protected this(:n>0) List(n-1)  tail;
  public List(t.n+1)(Object o, final List t)\{
     n=t.n+1;
     List(:n>0) me = (List(:n>0)) this;
     me.tail=t;
     me.value=o;
  \}
  public List(0) () \{
     n=0;
  \}
  ...
\}
\end{x10}

The fields {\cf value} and {\cf tail} do not exist for instances of the class
{\cf List(0)}.
\end{example}

It is a compile-time error for a class to have two fields of the same
name, even if their {\cf ThisClauses} are different. A class {\cf C} with a field
named {\cf f} is said to {\em hide} a field in a superclass named {\cf f}.

\begin{quotation}
 {\sc Static Semantics Rule:}
     A class may not declare two fields with the same name.
\end{quotation}

\subsection{Field hiding}

The definition of field hiding does not take {\cf ThisClauses} in
account. Suppose a class {\cf C} has a field

\begin{x10}
 this(:c) Foo f;  
\end{x10}
\noindent and a subclass {\cf D} of {\cf C} has a field
\begin{x10}
 this(:d) Fum f;  
\end{x10}

We will say that {\cf D.f} hides {\cf C.f}, {\em regardless} of the
constraints {\cf c} and {\cf d}. This is in keeping with \Java, and
permits a naive implementation which always allocates space for a
conditional field.

{\sc DESIGN RATIONALE} It might seem attractive to require that {\cf
D.f} hides {\cf C.f} only if {\cf d} entails {\cf c}. This would seem
to necessitate a rather complex implementation structure for classes,
requiring some kind of a heterogenous translation for deptypes of {\cf C}
and {\cf D}. This bears further investigation.

\section{Method definitions}

\Xten{} permits guarded method definitions, similar to guarded
field definitions. Additionally, the parameter list for a method may
contain a WhereClause.

\begin{x10}
MethodHeader ::= 
  MethodModifiersopt ResultType 
  MethodDeclarator Throwsopt
MethodDeclarator ::= 
  ThisClauseopt identifier 
  ( FormalParameterListopt WhereClauseopt )
 | MethodDeclarator [ ]

ResultType ::= Type | void
\end{x10}

The guard (specified by {\cf ThisClause}) speciifes a constraint {\cf c} on the
properties of the class {\cf C} on which the method is being defined. The
method exists only for those instances of {\cf C} which satisfy {\cf c}.  It is
illegal for code to invoke the method on objects whose static type is
not a subtype of {\cf C(:c)}.

We relax the rules of lexical visibility and finality for variable
references in deptypes for method parameters.  Method
parameters not necessarily declared to be final are permitted to occur
in the types of parameters that occur after them in textual
order. Method parameters may also occur in the ReturnType for the
method, as long as they are declared final. (Even though the ReturnType
occurs lexically before the parameter list, it is considered to lie in
the scope of the declarations in the parameter list.)

\begin{quotation}
 {\sc  Static semantics Rule: }
    The compiler checks that every method invocation {\cf o.m(e1,..., en)}
    for a method is type correct. Each each argument ei must have a
    static type Si that is a subtype of the declared type Ti for the ith
    argument of the method, and the conjunction of static types
    of the arguments must entail the WhereClause in the parameter-list
    of the method.

    The compiler checks that in every method invocation {\cf o.m(e1,...,
    en)} the static type of o, S, is a subtype of C(:c), where the method
    is defined in class C and the ThisClause for m is equivalent to
    {\cf this(:c)}.

    Finally, if the declared return type of the method is D(:d), the
    return type computed for the call is {\cf D(: final S a; S1 x1; ...; Sn
    xn; d[a/this])}, where a is a new variable that does not occur in
    {\cf d, S, S1, ... , Sn}, and {\cf x1,...,xn} are the formal parameters of the
    method.
\end{quotation}

\begin{example}
Consider the program:
  \begin{x10}
public class List(int(:n >=0) n) \{
  protected this(:n > 0) Object  value;
  protected this(:n > 0) List(n-1)  tail;
  public List(t.n+1)(Object o, List t) \{
      n=t.n+1;
      tail = t;
      value = o;
  \}
  public List(:self.n==0) () \{
      n=0;
  \}
  public List(:self.n==this.n+l.n) append(List l) \{
      return (n==0)? l 
         : new List( value, tail.append(l)); 
  \}
  public this(:n>0) 
    Object nth(final int(:k >= 1 \&\& k <= n) k) \{
      return k==1 ? value : tail.nth(k-1);
  \}
\}
  \end{x10}

The following code fragment
\begin{x10}
List(:self.n==3) u = ...
List(:self.n==x) t = ...;
List(:self.n==x+3) s = t.append(u);
\end{x10}
\noindent will typecheck. The type of the expression {\cf t.append(u)} is 
\begin{x10}
List(: final List(:self.n==x) a; 
       List(:self.n==3) l; self.n== a.n+l.n)  
\end{x10}
\noindent and this simplifies to
\begin{x10}
List(: final List(:self.n==x) a; 
       List(:self.n==3) l; self.n== x+3)  
\end{x10}
\noindent which, after dropping unused local variables, reduces to:
\begin{x10}
List(: self.n== x+3)  
\end{x10}
\end{example}

\subsection{Method overloading, overriding, hiding, shadowing and obscuring}

The definitions of method overloading, overriding, hiding, shadowing
and obscuring in \Xten{} are the same as in \Java, modulo the following
considerations motivated by dependent types.

The definition of a method declaration {\cf m1} ``having the same signature
as'' a method declaration {\cf m2} involves identity of types. Two \Xten{} types
are defined to be identical iff they are equivalent (\S~\ref{DepType:Equivalence}).
Two methods are said to have {\em the same signature} if (a)
they have the same number of formal parameters, (b) for each parameter
their types are equivalent, and (c) the constraints associated with
their ThisTypes are equivalent. It is a compile-time error for there
to be two methods with the same name and same signature in a class
(either defined in that class or in a superclass).

\begin{quotation}
   {\sc Static Semantics Rule:}
  A class {\cf C} may not have two declarations for a method named {\cf m} -- either
  defined at {\cf C} or inherited --
\begin{x10}
T this(:tc) m(T1(:t1) v1,..., Tn(:tn) vn) \{...\}
S this(:sc) m(S1(:s1) v1,..., Sn(:sn) vn) \{...\}
\end{x10}
\noindent   if it is the case that the types {\cf this(:tc), T1(:t1), ...., Tn(:tn)} are
  equivalent to the types {\cf this(:sc), S1(:t1), ...., Tn(:tn)}
  respectively.
\end{quotation}

A class {\cf C} inherits from its direct superclass and superinterfaces all
their methods visible according to the access restriction modifiers
public/private/protected/(package) of the superclass/superinterfaces
that are not hidden or overridden. A method {\cf M1} in a class {\cf C} overrides
a method {\cf M2} in a superclass {\cf D} if {\cf M1} and {\cf M2} have the same signature.
Methods are overriden on a signature-by-signature basis.

A method invocation {\cf o.m(e1,..., en)} is said to have the {\em static
signature} {\cf <T, T1,...,Tn>} where {\cf T} is the static type of {\cf o}, and
{\cf T1,..., Tn} are the static types of {\cf e1,..., en} respectively.  As in
\Java, it must be the case that the compiler can determine a single
method defined on {\cf T} with argument type {\cf T1,..., Tn}, otherwise a
compile-time error is declared. However, unlike \Java, the \Xten{} type {\cf T}
may be a dependent type {\cf C(:c)}. Therefore, given a class definition for
{\cf C} we must determine which methods of {\cf C} are available at a type
{\cf C(:c)}. But the answer to this question is clear: exactly those methods
defined on {\cf C} are available at the type {\cf C(:c)} whose guard {\cf d} is implied
by {\cf c}.


\begin{example}
  Consider the definitions:
  \begin{x10}
class Point(int i, int j) \{...\}
class Line(Point s, Point(:self != i) e) \{
//m1: Both points lie in the right half of the plane
  this(:s.i>= 0 \&\& e.i >= 0) void draw() \{...\}
// m2 -- Both points lie on the y-axis
  this(:s.i== 0 \&\& e.i == 0) void draw() \{...\}
// m3 -- Both points lie in the top half of the plane
  this(:s.j>= 0 \&\& e.j >= 0) void draw() \{...\}
  // m4  -- The general method
  void draw() \{...\}
\} 
  \end{x10} 
\noindent  Three different implementations are given for the draw method, one
  for the case in which the line lies in the right half of the plane,
  one for the case that the line lies on the y-axis and the third for
  the case that the line lies in the top half of the plane.


\noindent  Consider the invocation
Line(:s.i < 0) m = ...
m.draw();

\noindent  This generates a compile time error because there is no applicable
  method definition.

\noindent  Consider the invocation

\begin{x10}
Line(:s.i>=0 \&\& s.j>=0 \&\& e.i>=0 \&\& e.j>=0) 
  m = ...
m.draw();
\end{x10}

\noindent  This generates a compile time error because both m1 and m3 are applicable.

\noindent  Consider the invocation
\begin{x10}
Line(:s.i>=0 \&\& s.j>=0 \&\& e.i>=0) m = ...
m.draw();
\end{x10}
  This does not generate any compile-time error since only m1 is
  applicable. 
\end{example}


In the last example, notice that at runtime {\cf m1} will be invoked
(assuming {\cf m} contains an instance of the {\cf Line} class, and not some
subclass of {\cf Line} that overrides this method). This will be the case
even if {\cf m} satisfies at runtime the stronger conditions for {\cf m2} (i.e.,
{\cf s.i==0 \&\& e.i==0}). That is, dynamic method lookup will not take into
account the  ``strongest'' constraint that the receiver may satisfy, i.e.{}
its ``strongest deptype''. 

{\em {\sc DESIGN RATIONALE.}
  The design decision that dynamic method lookup should ignore
  dependent type information was made to keep the design and the
  implementation simple and to ensure that serious errors such as
  method invocation errors are captured at compile-time.
 
  Consider the above example and the invocation
  \begin{x10}
   Line m = ...
   m.draw();    
  \end{x10}


   Statically the compiler will not report an error because m4 is the
   only method that is applicable. However, if dynamic method lookup
   were to use deptypes then we would face the problem that if m is a
   line that lives in the upper right quadrant then *both* m2 and m3
   are applicable and one does not override the other. Hence we must
   report an error dynamically.

   As discussed above, the programmer can write code with {\cf instanceof}
   and classcasts that perform any application-appropriate
   discrimination.  
}

\section{Interfaces with properties}\label{DepType:Interface}

\Xten{} permits interfaces to have properties and specify an interface
invariant. This is necessary so that programmers can build dependent
types on top of interfaces and not just classes.

\begin{x10}
NormalInterfaceDeclaration ::= 
     InterfaceModifiersopt interface identifier 
        PropertyListopt ExtendsInterfacesopt InterfaceBody
PropertyList     ::= ( Properties WhereClauseopt )
Properties       ::= Property
                    | Properties , Property
Property         ::= Type identifier    
PropertyListopt  ::= null | PropertyList
\end{x10}
The invariant associated with an interface is the conjunction of the
invariants associated with its superinterfaces and the invariant
defined at the interface. 

\begin{quotation}
   {\sc Static Semantics Rule:} The compiler declares an error if this constraint
   is not consistent (\S~\ref{DepType:Consistency}).  
\end{quotation}

Each interface implicitly defines a nullary getter method {\cf T p()} for
each property {\cf T p}. 

\begin{quotation}
   {\sc Static Semantics Rule:} The compiler issues a warning if the programmer
   explicitly defines a method with this signature for an interface.
  
\end{quotation}

A class {\cf C} (with properties) is said to implement an interface {\cf I} if
\begin{itemize}
  \item its properties contains all the properties of I,
\item its class invariant, i(C), implies i(I)
\end{itemize}


\subsection{instanceof} 

\Xten{} permits Types to be used in an in instanceof expression to
determine whether an object is an instance of the given type:

\begin{x10}
RelationalExpression ::= 
  RelationalExpression instanceof Type  
\end{x10}

In the above expression, {\cf Type} is any type including deptypes and
``primitive'' types. The expression {\cf e instanceof T} evaluates to true
if and only if the evaluation of {\cf e} results in a value {\cf v} which belongs
to the type {\cf T}. This determination may involve checking that the
constraint associated with the type is true for the value {\cf v}.

\subsection{Class casts}

\Xten{} permits types to be used in a cast expression:

\begin{x10}
CastExpression ::= 
  ( Type ) UnaryExpressionNotPlusMinus  
\end{x10}

In the above expression, Type is any type including deptypes and
``primitive'' types. The expression {\cf ((T) e)} evaluates {\cf e} to
a value {\cf v}, and results in {\cf v} if {\cf v} is an instance of
the type {\cf T}. Otherwise a {\cf ClassCastException} is thrown. The
static type of the expression {\cf (T) e} is {\cf T}.

   \begin{quotation}
 {\sc  Static Semantics Rule:}
    The compiler checks that the static type of e is either a subtype
    of T (this situation is sometimes called a "stupid cast"), or a
    supertype of T. If neither is the case, it throws a compile-time
    error since the cast must necessarily fail at runtime.     
   \end{quotation}

\subsection{Local variables}

Dependent types may be used to specify the types of local variables,
including loop variables and parameters for catch clauses.

\section{Array types}

\begin{x10}
    ArrayType ::= Type [  ] 
     | Type value  [  ]
     | Type [ DepParameterExpr ]
     | Type  value [ DepParameterExpr ]  
\end{x10}

\Xten{} has the following built in types:

\begin{x10}
class point(int(:rank>=0) rank) { ...}
class region
  (int(:rank>=0) rank,
   // region is a product of rank-1 convex regions.
   boolean rect,  
   // on each dim, the low bound is 0. 
   boolean lowZero
  ) { ...}
class dist
  (int (:rank>=0) rank, 
   boolean rect,
   boolean lowZero,
   region(: self.rank==rank\&\&self.rect==rect
           \&\&self.lowZero==lowZero) region, 
   place onePlace 
   )  { ...}
class Array<T>
  (int (:rank>=0) rank,  
   boolean rect,
   boolean lowZero,
   place onePlace,
   region region,
   dist(:self.rank==rank \&\&self.rect=rect
         \&\&self.lowZero==lowZero
        \&\&self.region==region
        \&\&self.onePlace==onePlace) dist
    ) \{...\}
class ValueArray<T>
 (int (:rank>=0) rank,  
   boolean rect,
   boolean lowZero,
   place onePlace,
   region region,
   dist(:self.rank==rank \&\&self.rect=rect
        \&\&self.lowZero==lowZero
       \&\&self.region==region
       \&\&self.onePlace==onePlace) dist
) \{...\}
\end{x10}


The array types on the left are shorthand for the deptypes on the right:
\begin{x10}
Type []  => Array<Type>
Type value [] => ValueArray<Type>
Type  [ DepParameterExpr ] 
   => Array<Type>( DepParameterExpr )
Type Value [ DepParameterExpr ] 
   => ValueArray<Type>( DepParameterExpr )
\end{x10}

For {\cf R} is a reference type, the type {\cf R(:c)!current[]} is interpreted as the
type of all arrays which are such that the value at a point {\cf p} in its
region has the type {\cf R(:c)!self.dist[p]}. (Recall that in \Xten{} a
distribution is itself a value array that maps a point to a place.)

\begin{example}
  The type {\cf double[:rail]} is the Java type {\cf double[]}.
  The type {\cf double[:rail][:rail]} is the Java type {\cf double[][]}.

  The type {\cf double[:rank==N]} is the type of all N-dimensional arrays of
  doubles.

  The type {\cf double[:rank==N\&\&onePlace==here]} is the type of all $N$-dimensional
  arrays of doubles that are local (mapped to one place).  
\end{example}

\section{Constraint system}

The initial release of \Xten{} has a very simple constraint system,
permitting only conjunctions of equalities between variables and
constants, and existential quantification over typed variables.

Subsquent implementations are intended to support boolean algebra,
arithmetic, relational algebra etc to permit types over regions and
distributions. We envision this as a major step towards removing most
(if not all) dynamic array bounds and places check from \Xten{}.
  
%%\subsection{Normalization of constraints}
%%
%%The constraint system satisfies the following structural rule
%%
%%  Gamma |- c
%%---------------------------------
%%Gamma, S(: c') x |- S(:c' \&\& c) x
%%
%%An additional rule, SELF, relates self to the variable being defined:
%%
%%   T(: c) x |- c[x/self]
%%
%%The rule TI ensures that the type invariant is implicitly available:
%%
%%   T(: c) x |- T (: i(T)[x/this] \&\& c) x
%%
%%Examples
%%
%%Consider the definitions
%%  class region(int rank) { ...}
%%  class intArray(region r, int(:self==r.rank) rank) { ...}
%%
%%Now we have the following invariants:
%%
%%   i(region) =def=  int this.rank
%%   i(intArray) =def= region this.r \&\& int(:self==this.r.rank) this.rank
%%
%%Note that i(intArray) is equivalent to
%%   
%%   region this.r \&\& int(:self==this.r.rank) this.rank \&\& this.rank == this.r.rank
%%
%%Now we have the following derivations
%%
%%  intArray(:self.rank==2) a |- region(:self.rank==2) a.r
%%
%%through the following derivation. 
%%
%%  intArray(:self.rank==2) a |- intArray(:self.rank==2 \&\& i(intArray)) a  (TI)
%%
%%Now |intArray(:self.rank==2 \&\& i(intArray)) a|  is equivalent to
%%
%%  intArray(:self.rank==2 \&\& region a.r \&\& int(:self==a.r.rank) a.rank 
%%            \&\& a.rank == a.r.rank) a 
%%
%%and entails each one of 
%%
%%    a.rank==2
%%    a.rank == a.r.rank
%%
%%which together entail a.r.rank==2.

\subsection{Syntactic abbreviations}\label{DepType:SyntaxAbbrev}

\section{Place types}\label{DepType:PlaceType}\index{placetype}

\begin{x10}
PlaceTypeSpecifier ::= ! PlaceType
PlaceType ::=  any | current | Expression  
\end{x10}

All \Xten{} reference classes extend the class x10.lang.Ref:

\begin{x10}
package x10.lang;
ublic class Ref(place loc) { ... }  
\end{x10}

Because of the importance of places in the \Xten{} design, special
syntactic support is provided for deptypes involving places.

We now consider the expansions of deptypes with place information.

Unless a deptype {\cf T} (whose base is a reference type) has an {\cf !} suffix,
the constraint for {\cf T} is implicitly assumed to contain the clause
{\cf self.loc==here}.
\begin{x10}
C( : c) => C(: self.loc==here \&\& c )  
\end{x10}
\noindent The expansions for a deptype with an ``{\cf !}'' suffix are:
\begin{x10}
C( : c)!  => C(: c )  // no self.loc clause.
C( : c)!p => C(: self.loc==p \&\& c ) 
\end{x10}


\begin{quotation}
  {\cf Static Semantics Rule:} It is a compile time error for the {\tt !}-annotation to
  be used for deptypes whose base type does not extend {\tt x10.lang.Ref}.  
\end{quotation}

\section{Implicit Syntax}\label{ImplicitSyntax}\index{implicit syntax}

Recall that the explicit syntax for \Xten{} requires the programmer to use
asyncs/future to ensure the Locality Principle: An activity accesses
only those mutable locations that reside in the same place as the
activity. 

Explicit syntax has the advantage that the performance model for \Xten{}
is explicit from the syntax. It has the disadvantage that the
programmer has to manually reason about the placement of various
objects. If the programmer reasons incorrectly then computation may
abort at runtime with an exception.

The place-based type system enables the compiler to support the
Locality principle. The programmer uses the type system to establish
that the types of various objects are local. These assertions are
checked by the compiler (as a side-effect of checking dependent
types). Additionally, the programmer may now use normal variable
syntax to access (read/write) variables, and invoke methods on
objects. Suppose the type of the variable v is C(:c). If c establishes
loc==here then the compiler generates code for performing the relevant
operation on the local variable (read, write, method invocation)
synchronously. 

Otherwise the compiler generates code in explicit syntax as
follows. If the operation is a read, the compiler generates code to
perform a future/force on the variable

\begin{x10}
  future(v){v}.force();  
\end{x10}

If the operation is a write |v=e|, the compiler generates code to perform
the write synchronously:

\begin{x10}
  final T temp = e;
  finish async (v){ v = w;}
\end{x10}

If the operation is a read on an array variable a[p] the compiler
generates the code:

\begin{x10}
  future(a.dist[p]){ a[p]}.force();  
\end{x10}


If the operation is a write |a[p]=e|, the compiler generates code to perform
the write synchronously:

\begin{x10}
  final point tp = p;
  final T t = e;
  finish async (a.dist[tp]){ a[tp] = t;}
\end{x10}

If the operation is a method invocation |e.m(e1,..., en)| for a void
method, the compiler generates code to perform the method invocation
synchronously:

\begin{x10}
  final T! t = e;
  final P1! t1 = e1;
  ...
  final Pn! tn = en;
  finish async (t){ 
    t.m(t1,..., tn);
  }  
\end{x10}


If the operation is a method invocation |e.m(e1,...,en)| for a method
that returns a value of type E, then the compiler generates the
following code:

\begin{x10}
  new Runnable() {
  public E run() {
    final T! t = e;
    final P1! t1 = e1;
    ...
    final Pn! tn = en;
  
    return future(t){t.m(t1,...tn)}.force();
  }}.run()
\end{x10}

	\par % empty
\chapter{Names and packages}
\label{packages} \index{names}\index{packages}

\Xten{} supports \java's mechanisms for names and packages \cite[\S 6,\S 7]{jls2}, including {\tt public}, {\tt protected}, {\tt private} and package-specific access control. \Xten{} supports \java's naming conventions.

\Xten{} also supports \java{} 1.5 static imports \cite{jsr201}.	\par % \vfill\eject % empty
\chapter{Places}\label{XtenPlaces}\index{places}

An \Xten{} place is a repository for data and activities. Each place
is to be thought of as a locality boundary: the activities running in
a place may access data items located at that place with the
efficiency of on-chip access. Accesses to remote places may take
orders of magnitude longer.

{}\Xten{} provides a built-in value class, \xcd"x10.lang.place"; all
places are instances of this class.  This class is \xcd"final" in
{}\XtenCurrVer.

In \XtenCurrVer{}, the set of places available to a computation is
determined at the time that the program is run and remains fixed
through the run of the program. The number of places available 
may be determined by reading (\xcd"place.MAX_PLACES"). (This number
is specified from the command line/configuration information; 
see associated {\tt README} documentation.)

All scalar objects created during program execution are located in one
place, though they may be referenced from other places. Aggregate
objects (arrays) may be distributed across multiple places using
distributions.

The set of all places in a running instance of an \Xten{} program may
be obtained through the \xcd"const" field \xcd"place.places".  (This
set may be used to define distributions, for instance,
\S~\ref{XtenDistributions}.) 


The set of all places is totally ordered.  The first place may be
obtained by reading \xcd"place.FIRST_PLACE". The initial activity for
an \Xten{} computation starts in this place
(\S~\ref{initial-computation}). For any place, the operation \xcd"next()"
returns the next place in the total order (wrapping around at the
end). Further details on the methods and fields available on this
class may be obtained by consulting the API documentation.

{\em Note: Future versions of the language may permit user-definable
places, and the ability to dynamically create places. }

\paragraph{Static semantics.}
Variables of type \xcd"place" must be initialized and are implicitly
\xcd"final".  

\section{Place expressions}
Any expression of type \xcd"place" is called a place expression. 
Examples of place expressions are \xcd"this.location" (the place
at which the current object lives), \xcd"place.FIRST_PLACE"
(the first place in the system in canonical order). 

Place expressions are used in the following contexts: 
\begin{itemize}
%\item As a  place type in a type (\S~\ref{PlaceTypes}).
\item As a target for an \xcd"async" activity or a future
(\S~\ref{AsyncActivity}).
\item In a class cast expression  (\S~\ref{ClassCast}).
\item In an \xcd"instanceof" expression (\S~\ref{instanceOf}).
\item In stable equality comparisons, at type \xcd"place".
\end{itemize}

Like values of any other type, places may be passed as arguments
to methods, returned from methods, stored in fields etc.

\section{\Xcd{here}}\index{here}\label{Here}
\Xten{} supports a special indexical constant\footnote{
An indexical constant is one whose value depends on its context
of use.} \xcd"here":
\begin{verbatim}
22 ExpressionName ::= here
\end{verbatim}
The constant evaluates to the place at which the current activity is
running. Unlike other place expressions, this constant cannot be 
used as the placetype of fields, since the type of a field 
should be independent of the activity accessing it.

\paragraph{Example.}
The code:
\begin{xten}
public class F {
  public def m(a: F) {
    val OldHere: place = here;
    async (a) {
      System.out.println("OldHere == here:" 
                         + (OldHere == here));
    }
  }
  public static void main(s: array[String]) {
    new F().m(future(place.FIRST_PLACE.next())
              { new F() }.force());
  }
}  
\end{xten}
\noindent will print out \xcd"true" iff the computation was configured
to start with the number of places set to \xcd"1". 


\section{Implicit syntax}\label{ImplicitSyntax}\index{implicit syntax}

Recall that the explicit syntax for \Xten{} requires the programmer to use
asyncs/future to ensure the Locality Principle: An activity accesses
only those mutable locations that reside in the same place as the
activity. 

Explicit syntax has the advantage that the performance model for \Xten{}
is explicit from the syntax. It has the disadvantage that the
programmer has to manually reason about the placement of various
objects. If the programmer reasons incorrectly then computation may
abort at runtime with an exception.

The place-based type system enables the compiler to support the
Locality principle. The programmer uses the type system to establish
that the types of various objects are local. These assertions are
checked by the compiler (as a side-effect of checking dependent
types). Additionally, the programmer may now use normal variable
syntax to access (read/write) variables, and invoke methods on
objects. Suppose the type of the variable \xcd"v" is \xcd"C{c}".
If \xcd"c" establishes
\xcd"location==here" then the compiler generates code for
performing the relevant operation on the local variable (read,
write, method invocation) synchronously. 

Otherwise the compiler generates code in explicit syntax as
follows. If the operation is a read, the compiler generates code to
perform a future/force on the variable

\begin{xten}
future(v) { v }.force();  
\end{xten}

If the operation is a write \xcd"v=e", the compiler generates code to perform
the write synchronously:

\begin{xten}
val temp: T = e;
finish async (v) { v = w; }
\end{xten}

If the operation is a read on an array variable a[p] the compiler
generates the code:

\begin{xten}
future(a.dist(p)) { a(p) }.force();  
\end{xten}


If the operation is a write \xcd"a[p]=e", the compiler generates code to perform
the write synchronously:

\begin{xten}
val tp: point = p;
val t: T = e;
finish async (a.dist(tp)) { a(tp) = t; }
\end{xten}

If the operation is a method invocation \xcdmath"e.m(e$_1$,..., e$_n$)"
for a void method, the compiler generates code to
perform the method invocation synchronously:

\begin{xten}
val t: T! = e;
val t1: P1! = e1;
...
val tn: Pn! = en;
finish async (t) {
  t.m(t1,..., tn);
}
\end{xten}


If the operation is a method invocation
\xcdmath"e.m(e$_1$,...,e$_n$)" for a method that returns a value
of type E, then the compiler generates the following code:

XXX remove this -- do not define semantics by translation

\begin{xten}
( (t: T!, t1: P1!, ..., tn: Pn!) => { 
    return future(t) { t.m(t1,...tn) }.force();
  } )(e, e1, ..., en)
\end{xten}


	\par %0.1
\chapter{Activities}\label{XtenActivities}

An \Xten{} computation may have many concurrent {\em activities} ``in
flight'' at any give time. We use the term activity to denote a
dynamic execution instance of a piece of code (with references to
data). An activity is intended to execute in parallel with other
activities. An activity may be thought of as a very light-weight
thread.  In \XtenCurrVer{}, an activity may not be interrupted,
suspended or resumed as the result of actions taken by any other
activity.

An activity is spawned in a given place and stays in that place for
its lifetime.  An activity may be {\em running}, {\em blocked} on some
condition or {\em terminated}. When the statement associated with an
activity terminates normally, the activity terminates normally; when
it terminates abruptly with some reason $R$, the activity terminates
with the same reason (\Sref{ExceptionModel}).

An activity may be long-running and may invoke recursive methods (thus
may have a stack associated with it). On the other hand, an activity
may be short-running, involving a fine-grained operation such as a
single read or write.

% An activity may have an {\em activitylocal} heap accessible only
%to the activity. 

An activity may asynchronously and in parallel launch activities at
other places.

\Xten{} distinguishes between {\em local} termination and {\em global}
termination of a statement. The execution of a statement by an
activity is said to terminate locally when the activity has finished
all its computation related to that statement. (For instance the
creation of an asynchronous activity terminates locally when the
activity has been created.)  It is said to terminate globally when it
has terminated locally and all activities that it may have spawned at
any place (if any) have, recursively, terminated globally.

An \Xten{} computation is initiated as a single activity from the
command line. This activity is the {\em root activity}\index{root
activity} for the entire computation. The entire computation
terminates when (and only when) this activity globally
terminates. Thus \Xten{} does not permit the creation of so called
``daemon threads''---threads that outlive the lifetime of the root
activity. We say that an \Xten{} computation is {\em rooted}
(\Sref{initial-computation}).

\futureext{ We may permit the initial activity to be a daemon activity
to permit reactive computations, such as webservers, that may not
terminate.}

\section{The \Xten{} rooted exception model}
\label{ExceptionModel}

The rooted nature of \Xten{} computations permits the definition of a
{\em rooted} exception model. In multi-threaded programming languages
there is a natural parent-child relationship between a thread and a
thread that it spawns. Typically the parent thread continues execution
in parallel with the child thread. Therefore the parent thread cannot
serve to catch any exceptions thrown by the child thread. 

The presence of a root activity permits \Xten{} to adopt a different
model.  In any state of the computation, say that an activity $A$ is
{\em a root of} an activity $B$ if $A$ is an ancestor of $B$ and $A$
is suspended at a statement (such as the \xcd"finish" statement
\Sref{finish}) awaiting the termination of $B$ (and possibly other
activities). For every \Xten{} computation, the
\emph{root-of} relation
is guaranteed to be a tree. The root of the tree is the root activity
of the entire computation. If $A$ is the nearest root of $B$, the path
from $A$ to $B$ is called the {\em activation path} for the
activity.\footnote{Note that depending on the state of the computation
the activation path may traverse activities that are running,
suspended or terminated.}

We may now state the exception model for \Xten.  An uncaught exception
propagates up the activation path to its nearest root activity, where
it may be handled locally or propagated up the \emph{root-of} tree when
the activity terminates (based on the semantics of the statement being
executed by the activity).\footnote{In \XtenCurrVer{} the \xcd"finish"
statement is the only statement that marks its activity as a root
activity. Future versions of the language may introduce more such
statements.}  Thus, unlike concurrent languages such as \java{}, no
exception is ``thrown on the floor''.

\section{Spawning an activity}\label{AsynchronousActivity}\label{AsyncActivity}

Asynchronous activities serve as a single abstraction for supporting a
wide range of concurrency constructs such as message passing, threads,
DMA, streaming, data prefetching. (In general, asynchronous operations
are better suited for supporting scalability than synchronous
operations.)

An activity is created by executing the statement:

\begin{grammar}
Statement \: AsyncStatement \\
AsyncStatement \: \xcd"async" PlaceExpressionSingleList\opt Statement \\
PlaceExpressionSingleList \: \xcd"(" PlaceExpression \xcd")" \\
PlaceExpression \: Expression \\
\end{grammar} 

The place expression \xcd"e" is expected to be of type \xcd"Place",
e.g., \xcd"here" or \xcd"d(p)" for some
distribution \xcd"d" and point \xcd"p" (\Sref{XtenPlaces}).  
If not, the compiler replaces
\xcd"e" with \xcd"e.location" if
\xcd"e" is of type \xcd"x10.lang.Ref". Otherwise the compiler reports a type error. 

Note specifically that the expression \xcd"a(i)" when used as a place
expression may evaluate to \xcd"a(i).location", which may not be
the same place as \xcd"a.dist(i)". The programmer must be 
careful to choose the right expression, appropriate for the statement.
Accesses to \xcd"a(i)" within \grammarrule{Statement} should typically be guarded 
by the place expression \xcd"a.dist(i)".

In many cases the compiler may infer the unique place at which the
statement is to be executed by an analysis of the types of the
variables occuring in the statement. (The place must be such that the
statement can be executed safely, without generating a
\xcd"BadPlaceException".) In such cases the programmer may omit the
place designator; the compiler will throw an error if it cannot
determine the unique designated place.\footnote{\XtenCurrVer{} does
not specify a particular algorithm; this will be fixed in future
versions.}

An activity $A$ executes the statement \xcd"async (P) S" by launching
a new activity $B$ at the designated place, to execute the specified
statement. The statement terminates locally as soon as $B$ is
launched.  The activation path for $B$ is that of $A$, augmented with
information about the line number at which $B$ was spawned.  $B$
terminates normally when $S$ terminates normally.  It terminates
abruptly if $S$ throws an (uncaught) exception. The exception is
propagated to $A$ if $A$ is a root activity (see \Sref{finish}),
otherwise through $A$ to $A$'s root activity. Note that while an
activity is running, exceptions thrown by activities it has already
generated may propagate through it up to its root activity.

Multiple activities launched by a single activity at another place are
not ordered in any way. They are added to the pool of activities at
the target place and will be executed in sequence or in parallel based
on the local scheduler's decisions. If the programmer wishes to
sequence their execution s/he must use \Xten{} constructs, such as
clocks and \xcd"finish" to obtain the desired effect.  Further, the
\Xten{} implementations are not required to have fair schedulers,
though every implementation should make a best faith effort to ensure
that every activity eventually gets a chance to make forward progress.

\begin{staticrule*}
The statement in the body of an \xcd"async" is subject to the
restriction that it must be acceptable as the body of a \xcd"void"
method for an anoymous inner class declared at that point in the code,
which throws no checked exceptions. As such, it may reference
variables in lexically enclosing scopes (including \xcd"clock"
variables, \Sref{XtenClocks}) provided that such variables are
(implicitly or explicitly) \xcd"final".
\end{staticrule*}

\paragraph{Returning from within an \xcd"async".}
The body \xcd"S" of an  \xcd"async S" is not permitted to contain a 
\xcd"return" statement since there are two candidate scopes from which
the programmer might intend the return: the   
\xcd"async" itself, and the enclosing method.

Programmers wishing to use a \xcd"return" statement in \xcd"S" to
return from the \xcd"async S" should use the idiom 
\xcd"val x=()=>S;async x();" instead (for some new variable \xcd"x". 
(Note that it does not make sense for code executing in the body of an
\xcd"async" to attempt to return from the enclosing method -- the
method may already have returned asynchronously.)


\section{Place changes}\label{AtStatement}

An activity may change place using the \xcd"at" statement or
\xcd"at" expression:

\begin{grammar}
Statement \: AtStatement \\
AtStatement \: \xcd"at" PlaceExpressionSingleList Statement \\
Expression \: AtExpression \\
AtExpression \: \xcd"at" PlaceExpressionSingleList ClosureBody \\
\end{grammar}

The statement \xcd"at (p) S" executes the statement \xcd"S"
synchronously at place \xcd"p".
The expression \xcd"at (p) E" executes the statement \xcd"E"
synchronously at place \xcd"p", returning the result to the
originating place.

\section{Finish}\index{finish}\label{finish}
The statement \xcd"finish S" converts global termination to local
termination and introduces a root activity. 

\begin{grammar}
Statement \: FinishStatement \\
FinishStatement \: \xcd"finish" Statement \\
\end{grammar}

An activity $A$ executes \xcd"finish S" by executing \xcd"S".  The
execution of \xcd"S" may spawn other asynchronous activities (here or
at other places).  Uncaught exceptions thrown or propagated by any
activity spawned by \xcd"S" are accumulated at \xcd"finish S".
\xcd"finish S" terminates locally when all activities spawned by \xcd"S"
terminate globally (either abruptly or normally). If
\xcd"S" terminates normally, then \xcd"finish S" terminates normally
and $A$ continues execution with the next statement after \xcd"finish S".
If \xcd"S" terminates abruptly, then \xcd"finish S"
terminates abruptly and throws a single exception formed 
from the collection of exceptions accumulated at \xcd"finish S".

Thus a \xcd"finish S" statement serves as a collection point for
uncaught exceptions generated during the execution of \xcd"S".

Note that repeatedly \xcd"finish"ing a statement has no effect after
the first \xcd"finish": \xcd"finish finish S" is indistinguishable
from \xcd"finish S".

\paragraph{Interaction with clocks.}\label{sec:finish:clock-rule}
\xcd"finish S" interacts with clocks (\Sref{XtenClocks}). 

While executing \xcd"S", an activity must not spawn any \xcd"clocked"
asyncs. (Asyncs spawned during the execution of \xcd"S" may spawn
clocked asyncs.) A
\xcd"ClockUseException"\index{clock!ClockUseException} is thrown
if (and when) this condition is violated.

In \XtenCurrVer{} this condition is checked dynamically; future
versions of the language will introduce type qualifiers which permit
this condition to be checked statically.

\futureext{
The semantics of \xcd"finish S" is conjunctive; it terminates when all
the activities created during the execution of \xcd"S" (recursively)
terminate. In many situations (e.g., nondeterministic search) it is
natural to require a statement to terminate when any {\em one} of the
activities it has spawned succeeds. The other activities may then be
safely aborted. Future versions of the language may introduce a
\xcd"finishone S" construct to support such speculative or nondeterministic
computation.
}
%% Need an example here.

\section{Initial activity}\label{initial-computation}\index{initial activity}

An \Xten{} computation is initiated from the command line on the
presentation of a classname \xcd"C". The class must have a
\xcd"public static def main(a: array[String])" method, otherwise an
exception is thrown
and the computation terminates.  The single statement
\begin{xten}
finish async (place.FIRST_PLACE) {
  C.main(s);
}
\end{xten} 
\noindent is executed where \xcd"s" is an array of strings created
from command line arguments. This single activity is the root activity
for the entire computation. (See \Sref{XtenPlaces} for a discussion of
placs.)

%% Say something about configuration information? 

\section{Foreach statements}
\index{foreach@\xcd"foreach"}\label{ForLoop}


\begin{grammar}
Statement \: ForEachStatement \\
ForEachStatement \: 
      \xcd"foreach" \xcd"(" Formal \xcd"in" Expression \xcd")"
          Statement \\
\end{grammar}

The \xcd"foreach" statement is similar to the enhanced \xcd"for"
statement (\Sref{ForAllLoop}).

An activity executes a \xcd"foreach" statement in a similar fashion
except that separate \xcd"async" activities are launched in parallel
in the local place of each object returned by the iteration.
The statement
terminates locally when all the activities have been spawned. It never
throws an exception, though exceptions thrown by the spawned
activities are propagated through to the root activity.

In a common case, the
the collection is intended to be of type
\xcd"Region" and the formal parameter is of type \xcd"Point".  Expressions \xcd"e" of type \xcd"Dist" and
\xcd"Array" are also accepted, and treated as if they were \xcd"e.region".


\section{Ateach statements}

\begin{grammar}
Statement \: AtEachStatement \\
AtEachStatement \:
      \xcd"ateach" \xcd"(" Formal \xcd"in" Expression \xcd")"
         Statement \\
\end{grammar}

The \xcd"ateach" statement is similar to the \xcd"foreach"
statement.  The collection must be of type \xcd"Dist"
and the formal parameter of type \xcd"Point".
Expressions \xcd"e" of type \xcd"Array" are also accepted, and treated
as if they were \xcd"e.dist". The compiler reports a type error
in all other cases.

This statement differs from \xcd"foreach" only in
that each activity is spawned at the place specified by the
distribution for the point. That is, 
\xcd"ateach(p(i1,...,ik): point in D) S" may
be thought of as standing for:
\begin{xten}
foreach (p(i1,...,ik): point in D.region) 
  async (D(p)) S
\end{xten}

\section{Futures}\label{XtenFutures}

\Xten{} provides syntactic support for {\em asynchronous expressions}, also
known as futures:

\begin{grammar}
Primary \: FutureExpression \\
FutureExpression \:
  \xcd"future" PlaceExpressionSingleList\opt ClosureBody
\end{grammar} 

Intuitively such an expression evaluates its body asynchronously at
the given place. The resulting value may be obtained from the future
returned by this expression, by using the \xcd"force" operation.

In more detail, in an expression \xcd"future (Q) e", the place
expression \xcd"Q" is treated as in an \xcd"async" statement. \xcd"e"
is an expression of some type \xcd"T". \xcd"e" may reference only
those variables in the enclosing lexical environment which are
declared to be \xcd"final".

If the type of \xcd"e" is \xcd"T" then the type of
\xcd"future (Q) e" is \xcd"future[T]".  This 
type \xcd"Future[T]" is defined as if by:
\begin{xten}
package x10.lang;
public interface Future[T] implements () => T {
  def forced(): Boolean;
}
\end{xten}

Evaluation of \xcd"future (Q) e" terminates locally with the creation
of a value \xcd"f" of type \xcd"Future[T]".  This value may be
stored in objects, passed as arguments to methods, returned from
method invocation etc. 

At any point, the method \xcd"forced" may be invoked on \xcd"f". This
method returns without blocking, with the value \xcd"true" if the
asynchronous evaluation of \xcd"e" has terminated globally and with
the value \xcd"false" if it has not.

\xcd"Future[T]" is a subtype of the function type \xcd"() => T".
Invoking---\emph{forcing}---the future \xcd"f" blocks until the
asynchronous evaluation of \xcd"e" has terminated globally. If the
evaluation terminates successfully with value \xcd"v", then the method
invocation returns \xcd"v". If the evaluation terminates abruptly with
exception \xcd"z", then the method throws exception \xcd"z". Multiple
invocations of the function (by this or any other activity) do not
result in multiple evaluations of \xcd"e". The results of the first
evaluation are stored in the future \xcd"f" and used to respond to all
queries.

\begin{example}
\begin{xten}
promise: Future[T] = future (a.dist(3)) a(3);
value: T = promise();
\end{xten}
\end{example}

\eat{
\subsection{Implementation notes}
Futures are provided in \Xten{} for convenience; they may be
programmed using latches, \xcd"async" and \xcd"finish" as
described in \Sref{future-imp}.
}

\section{At expressions}

\begin{grammar}
Expression \: \xcd"at" \xcd"(" Expression \xcd")" Expression
\\
\end{grammar}

An at expression evaluates an expression synchronously at a given place.
The expression \xcd"at (p) e" is equivalent to \xcd"future (p) e).force()".

\section{Shared variables}
\label{Shared}

A shared local variable is declared with the annotation
\xcd"shared".  It may be thought of as being accessible by any spawned
activity in its lexical scope.  Final variables are implicitly
shared.  An implementation may consider boxing shared 
variables and making a reference to the boxed value available to
any closures that use the variable.

\section{Atomic blocks}\label{AtomicBlocks}\index{atomic blocks}
Languages such as \java{} use low-level synchronization locks to allow
multiple interacting threads to coordinate the mutation of shared
data. \Xten{} eschews locks in favor of a very simple high-level
construct, the {\em atomic block}.

A programmer may use atomic blocks to guarantee that invariants of
shared data-structures are maintained even as they are being accessed
simultaneously by multiple activities running in the same place.

\subsection{Unconditional atomic blocks}
The simplest form of an atomic block is the {\em unconditional
atomic block}:

\begin{grammar}
Statement \: AtomicStatement \\
AtomicStatement \: \xcd"atomic"  Statement \\
MethodModifier \: \xcd"atomic" \\
\end{grammar}

For the sake of efficient implementation \XtenCurrVer{} requires
that the atomic block be {\em analyzable}, that is, the set of
locations that are read and written by the \grammarrule{BlockStatement} are
bounded and determined statically.\footnote{A static bound is a constant
that depends only on the program text, and is independent 
of any runtime parameters. }
The exact algorithm to be used by
the compiler to perform this analysis will be specified in future
versions of the language.
\tbd{}

Such a statement is executed by an activity as if in a single step
during which all other concurrent activities in the same place are
suspended. If execution of the statement may throw an exception, it is
the programmer's responsibility to wrap the atomic block within a
\xcd"try"/{\xcd"finally" clause and include undo code in the finally
clause. Thus the \xcd"atomic" statement only guarantees atomicity on
successful execution, not on a faulty execution.

%% A compiler is allowed to reorder two atomic blocks that have no
%%data-dependency between them, just as it may reorder any two
%%statements which have no data-dependencies between them. For the
%%purposes of data dependency analysis, an atomic block is deemed to
%%have read and written all data at a single program point, the
%%beginning of the atomic block.
%%%% I dont believe we need to say at some point in the atomic block.
%%
We allow methods of an object to be annotated with \xcd"atomic". Such
a method is taken to stand for a method whose body is wrapped within an
\xcd"atomic" statement.

Atomic blocks are closely related to non-blocking synchronization
constructs \cite{herlihy91waitfree}, and can be used to implement 
non-blocking concurrent algorithms.

\begin{staticrule*}
In \xcd"atomic S", \xcd"S" may include method calls,
conditionals, etc.
It may {\em not} include an \xcd"async" activity.
It may {\em not} include any statement that may potentially block at
runtime (e.g., \xcd"when", \xcd"force" operations, \xcd"next"
operations on clocks, \xcd"finish").

\limitation{Not checked in the current implementation.}
\end{staticrule*}


All locations accessed in an atomic block must reside \xcd"here"
(\Sref{Here}). A
\xcd"BadPlaceException"\index{place!BadPlaceException} is thrown
if (and when) this condition is violated.

All locations accessed in an atomic block must statically satisfy the
{\em locality condition}: they must belong to the place of the current
activity.\index{locality condition}\label{LocalityCondition} The
compiler checks for this condition by checking whether the statement
could be the body of a \xcd"void" method annotated with \xcd"local" at
that point in the code (\Sref{LocalAnnotation}).

\paragraph{Consequences.}
Note an important property of an (unconditonal) atomic block:

\begin{eqnarray}
 \mbox{\xcd"atomic {s1; atomic s2}"} &=& \mbox{\xcd"atomic {s1; s2}"}
\end{eqnarray}

Further, an atomic block will eventually terminate successfully or
thrown an exception; it may not introduce a deadlock.


\subsubsection{Example}

The following class method implements a (generic) compare and swap (CAS) operation:

\begin{xten}
// target defined in lexically enclosing environment.
public atomic def CAS(old: Object, new: Object): Boolean {
   if (target.equals(old)) {
     target = new;
     return true;
   }
   return false;
}
\end{xten}

\subsection{Conditional atomic blocks}

Conditional atomic blocks are of the form:

\begin{grammar}
Statement \:  WhenStatement \\
WhenStatement \:  \xcd"when" \xcd"(" Expression \xcd")" Statement \\
            \| WhenStatement \xcd"or" \xcd"(" Expression \xcd")" Statement \\
\end{grammar}

In such a statement the one or more expressions are called {\em
guards} and must be \xcd"Boolean" expressions. The statements are the
corresponding {\em guarded statements}. The first pair of expression
and statement is called the {\em main clause} and the additional pairs
are called {\em auxiliary clauses}. A statement must have a main
clause and may have no auxiliary clauses.

An activity executing such a statement suspends until such time as any
one of the guards is true in the current state. In that state, the
statement corresponding to the first guard that is true is executed.
The checking of the guards and the execution of the corresponding
guarded statement is done atomically. 


\Xten{} does not guarantee that a conditional atomic block
will execute if its condition holds only intermmitently. For, based on
the vagaries of the scheduler, the precise instant at which a
condition holds may be missed. Therefore the programmer is advised to
ensure that conditions being tested by conditional atomic blocks are
eventually stable, i.e., they will continue to hold until the block
executes (the action in the body of the block may cause the condition
to not hold any more).

%%Fourth, \Xten{} does not guarantees only {\em weak fairness} when executing
%%conditional atomic blocks. Let $c$ be the guard of some conditional
%%atomic block $A$. $A$ is required to make forward progress only if
%%$c$ is {\em eventually stable}. That is, any execution $s_1, s_2,
%%\ldots$ of the program is considered illegal only if there is a $j$
%%such that $c$ holds in all states $s_k$ for $k > j$ and in which $A$
%%does not execute. Specifically, if the system executes in such a way
%%that $c$ holds only intermmitently (that is, for some state in which
%%$c$ holds there is always a later state in which $c$ does not hold),
%%$A$ is not required to be executed (though it may be executed).

\begin{rationale}
The guarantee provided by \xcd"wait"/\xcd"notify" in \java{} is no
stronger. Indeed conditional atomic blocks may be thought of as a
replacement for \java's wait/notify functionality.
\end{rationale} 

We note two common abbreviations. The statement \xcd"when (true) S" is
behaviorally identical to \xcd"atomic S": it never suspends. Second,
\xcd"when (c) {;}" may be abbreviated to \xcd"await(c);"---it
simply indicates that the thread must await the occurrence of a
certain condition before proceeding.  Finally note that a \xcd"when"
statement with multiple branches is behaviorally identical to a
\xcd"when" statement with a single branch that checks the disjunction of
the condition of each branch, and whose body contains an
\xcd"if"/\xcd"then"/\xcd"else" checking each of the branch conditions.

\begin{staticrule*}
For the sake of efficient implementation certain restrictions are
placed on the guards and statements in a conditional atomic
block. 
\end{staticrule*}

Guards are required not to have side-effects, not to spawn
asynchronous activities and to have a statically determinable upper
bound on their execution. These conditions are expected to be checked
statically by the compiler.

The body of a \xcd"when" statement must satisfy the conditions
for the body of an \xcd"atomic" block.
%Second, as for unconditional atomic blocks, the set of memory
%locations accessed by a guarded statements are required to be bounded
%and statically anlayzable.

Note that this implies that guarded statements are required to be {\em
flat}, that is, they may not contain conditional atomic blocks. (The
implementation of nested conditional atomic blocks may require
sophisticated operational techniques such as rollbacks.)

\paragraph{Sample usage.} 
There are many ways to ensure that a guard is eventually
stable. Typically the set of activities are divided into those that
may enable a condition and those that are blocked on the
condition. Then it is sufficient to require that the threads that may
enable a condition do not disable it once it is enabled. Instead the
condition may be disabled in a guarded statement guarded by the
condition. This will ensure forward progress, given the weak-fairness
guarantee.

\begin{example}
The following class shows how to implement a bounded buffer of size
$1$ in \Xten{} for repeated communication between a sender and a
receiver.

\begin{xten}
class OneBuffer {
  datum: Object = null;
  filled: Boolean = false;
  public def send(v: Object) {
    when (!filled) {
      this.datum = v;
      this.filled = true;
    }
  }
  public def receive(): Object {
    when (filled) {
      v: Object = datum;
      datum = null;
      filled = false;
      return v;
    }
  }
}
\end{xten}
\end{example}

\eat{
\paragraph{Implementing a future with a latch.}\label{future-imp}
The following class shows how to implement a {\em latch}. A latch is
an object that is initially created in a state called the {\em
unlatched} state. During its lifetime it may transition once to a {\em
forced} state. Once forced, it stays forced for its lifetime. The
latch may be queried to determine if it is forced, and if so, an
associated value may be retrieved. Below, we will consider a latch set
when some activity invokes a \xcd"setValue" method on it. This method
provides two values, a normal value and an exceptional value. The
method \xcd"force" blocks until the latch is set. If an exceptional
value was specified when the latch was set, that value is thrown on
any attempt to read the latch. Otherwise the normal value is returned.

\begin{xten}
public interface Future[T] {
   def forced(): Boolean;
   def apply(): T;
}
public class Latch implements Future {
  protected var forced: Boolean = false;
  protected var result: Box[T] = null;
  protected var z: Box[Exception] = null;

  public atomic def setValue(val: T): Boolean {
    return setValue(val, null);
  }
  public atomic def setValue(z: Exception): Boolean {
    return setValue(null, z);
  }
  public atomic def setValue(val: T,
                             z: Exception): Boolean {
    if (forced) return false;
    // these assignment happens only once.
    this.result = val;
    this.z = z;
    this.forced = true;
    return true;
  }
  public atomic def forced(): Boolean {
    return forced;
  }
  public def apply(): T {
    when (forced) {
      if (z != null) throw z;
      return result to T;
    }
  }
}
\end{xten}

Latches, \xcd"aync" operations and \xcd"finish" operations may be used
to implement futures as follows. The expression \xcd"future(P) e"
can be translated to:
\begin{xten}
(() => {
    L: Latch = new Latch();
    async (P) {
      X: Object;
      try {
        finish X = e;
        async (L) {
          L.setValue(X); 
        }
      }
      catch (Z: Exception) {
        async (L) {
          L.setValue(Z);
        }
      }
    }
    return L;
  })()
\end{xten}

Here we assume that \xcd"RunnableLatch" is an interface defined by:
\begin{xten} 
public interface RunnableLatch {
  def run(): Latch;
}
\end{xten}

We use the standard \java{} idiom of wrapping the core translation
inside an inner class definition/method invocation pair (i.e.,
\xcd"new RunnableLatch() {....}.run()") so as to keep the resulting
expression completely self-contained, while executing statements
inside the evaluation of an expression.

Execution of a \xcd"future(P) e" causes a new latch to be created,
and an \xcd"async" activity spawned at \xcd"P". The activity attempts
to \xcd"finish" the assigned \xcd"x = e", where \xcd"x" is a local
variable.  This may cause new activities to be spawned, based on
\xcd"e". If the assignment terminates successfully, another activity is
spawned to invoke the \xcd"setValue" method on the latch.  Exceptions
thrown by these activities (if any) are accumulated at the \xcd"finish"
statement and thrown after global termination of all
activities spawned by \xcd"x=e". The exception will be caught by the 
\xcd"catch" clause and stored with the latch. 


\oldtodo{Conditional atomic blocks should be powerful enough to implement clocks as well.}

\paragraph{A future to execute a statement.}
Consider an expression \xcd"onFinish {S}". This should return
a \xcd"Boolean" latch which should be forced when \xcd"S" has terminated
globally. Unlike \xcd"finish S", the evaluation of \xcd"onFinish {S}"
should locally terminate immediately, returning a latch. The
latch may be passed around in method invocations and stored in
objects. An activity may perform \xcd"force"/\xcd"forced" method
invocations on the latch whenever it desires to determine whether \xcd"S"
has terminated.

Such an expression can be written as:
\begin{xten}
(=> {
    L: Latch = new Latch();
    async (here) {
      try {
        finish S;
        L.setValue(true);
      }
      catch (Z: Exception) {
        L.setValue(Z);
      }
    }
    return L;
  }
)()
\end{xten}
}
	\par %0.1
\chapter{Clocks}\label{XtenClocks}\index{clocks}
\cbstart
The standard library for \Xten{}, {\cf x10.lang} defines a {\cf final
value class}, {\tt clock} intended for repeated quiescence detection
of arbitrary, data-dependent collection of activities. Clocks are a
generalization of {\em barriers}. They permit dynamically created
activities to register and deregister. An activity may be registered
with multiple clocks at the same time. In particular, nested clocks
are permitted: an activity may create a nested clock and within one
phase of the outer clock schedule activities to run to completion on
the nested clock.  Neverthless the design of clocks ensures that
deadlock cannot be introduced by using clock operations.

This chapter describes the syntax and semantics of clocks and
statements in the language that have parameters of type {\cf clock}. 

The key invariants associated with clocks are as follows.  At any
stage of the computation, a clock has zero or more {\em registered}
activities. An activity may perform operations only on those clocks it
is registered with (these clocks constitute its {\em clock set}).  An
activity is registered with one or more clocks when it is created.
During its lifetime the only additional clocks it is registered with
are exactly those that it creates. In particular it is not possible
for an activity to register itself with a clock it discovers by
reading a data-structure.

An activity may perform the following operations on a clock {\cf
c}. It may {\em unregister} with {\cf c} by executing {\cf
c.drop();}. After this, it may perform no further actions on {\cf c}
for its lifetime. It may {\em check} to see if it is unregistered on a
clock. It may {\em register} a newly forked activity with {\cf c}.  It
may {\em post} a statement {\cf S} for completion in the current phase
of {\cf c} by executing the statement {\cf now(c) S}. It may {\em
resume} the clock by executing {\cf c.resume();}. This indicates to
{\cf c} that it has finished posting all statements it wishes to
perform in the current phase. Finally, it may {\em block} (by
executing {\cf next;}) on all the clocks that it is registered
with. (This operation implicitly {\cf resume}'s all clocks for the
activity.) It will resume from this statement only when all these
clocks are ready to advance to the next phase.

A clock becomes ready to advance to the next phase when every activity
registered with the clock has executed at least one {\cf resume}
operation on that clock and all statements posted for completion in
the current phase have been completed.

Though clocks introduce a blocking statement ({\cf next}) an important
property of \Xten{} is that clocks cannot introduce deadlocks. That is,
the system cannot reach a quiescent state (in which no activity is
progressing) from which it is unable to progress. For, before blocking
each activity resumes all clocks it is registered with. Thus if a
configuration were to be stuck (that is, no activity can progress) all
clocks will have been resumed. But this implies that all activities
blocked on {\cf next} may continue and the configuration is not stuck.

\section{Clock operations}
The special statements introduced for clock operations are listed below.
\begin{x10}
462 Statement ::= ClockedStatement
472 StatementNoShortIf ::= 
      ClockedStatementNoShortIf
479 NowStatement ::= 
      now ( Clock ) Statement
480 ClockedStatement ::= 
      clocked ( ClockList ) Statement
490 ClockedStatementNoShortIf ::= 
      clocked ( ClockList ) 
         StatementNoShortIf
501 NextStatement ::= next ;
\end{x10}

Note that {\tt x10.lang.clock} provides several useful methods on
clocks (e.g. {\tt drop}).

\subsection{Creating new clocks}\index{clock!creation}
Clocks are created using the nullary constructor for {\cf
x10.lang.clock} via a factory method:

\begin{x10}
clock timeSynchronizer = clock.factory.clock();
\end{x10}

All clocked variables are implicitly final. The initializer for a
local variable declaration of type {\tt clock} must be a new clock
expression. Thus \Xten{} does not permit aliasing of clocks.
Clocks are created in the place global heap and hence outlive the
lifetime of the creating activity.  Clocks are instances of value
classes, hence may be freely copied from place to
place. (Clock instances typically contain references to mutable state
that maintains the current state of the clock.)

The current activity is automatically registered with the newly
created clock.  It may deregister using the {\tt drop} method on
clocks (see the documentation of {\tt x10.lang.clock}). All activities
are automatically deregistered from all clocks they are registered
with on termination (normal or abrupt).

\subsection{Registering new activities on clocks}\index{clock!clocked statements}

The programmer may specify which clocks a new activity is to be registered with using the {\tt clocked} clause:
\begin{x10}
\end{x10}


\paragraph{Static semantics.} An activity may 
transmit only those clocks that is registered with and has not
quiesced on.  (\S~\ref{resume}). The compiler checks this
statically, inserting code to throw a {\tt ClockUseException}
if a violation is detected at runtime.

An activity may check that it is registered on a clock {\tt c} by
executing:
\begin{x10}
c.registered()
\end{x10}
\noindent This call returns a {\cf boolean} value: {\cf true} iff the
activity is registered on {\cf c}.

\paragraph{Note.} 
\Xten{} does not contain a ``register'' statement that would allow an
activity to discover a clock in a datastructure and register itself on
it. Therefore, while clocks may be stored in a datastructure by one
activity and read from that by another, the new activity cannot
``use'' the clock unless it is already registered with it.

\todo{Add text on arrays of clocks.}

\subsection{Resuming clocks}\index{clock!resume}\label{resume}
\Xten{} permits {\em split phase} clocks. An activity may wish
to indicate that it has completed whatever work it wishes to perform
in the current phase of a  clock {\tt c} it is registered with, without
suspending all activity. It may do so  by executing the method
invocation:
\begin{x10}
  c.resume();
\end{x10}
\noindent on a clock {\tt c} it is registered with.  

Nothing happens if the activity invokes this method on a clock it is
not registered with, or if it has already invoked a {\tt resume} on
this clock in the current phase.  Otherwise execution of this
statement indicates that the activity will not transmit {\cf c} to an
async or invoke {\cf now} until it terminates, drops {\cf c} or
executes a {\tt next}. The runtime throws a {\tt ClockUseException} if
it detects a violation of this condition.

\paragraph{Static semantics.} 
The compiler should issue an error if any activity has a potentially
live execution path from a {\cf resume} statement on a clock {\tt c}
to a {\cf now} or async spawn statement (which registers the new
activity on {\cf c}) unless the path goes through a {\cf next}
statement. (A {\cf c.drop()} following a {\cf c.resume()} is legal,
as is {\cf c.resume()} following a {\cf c.resume()}.

\subsection{Advancing clocks}\index{clock!next}
An activity may execute the statement
\begin{x10}
  next;
\end{x10}

\noindent 
Execution of this statement blocks until all the clocks that the
activity is registered with (if any) have advanced. (The activity
implicitly issues a {\cf resume} on all clocks it is registered
with before suspending.)

An \Xten{} computation is said to be {\em quiescent} on a clock {\cf
c} if each activity registered with {\cf c} has continued {\cf c}.
Note that once a computation is quiescent on {\cf c}, it will remain
quiescent on {\cf c} forever (unless the system takes some action),
since no other activity can become registered with {\cf c}.  That is,
quiescence on a clock is a {\em stable property}.

Once the implementation has detected quiecence on {\cf c}, the system
marks all activities registered with {\cf c} as being able to progress
on {\cf c}. An activity blocked on {\cf next} resumes execution once
it is marked for progress by all the clocks it is registered with.

\subsection{Dropping clocks}\index{clock!drop}
An activity may drop a clock by executing:
\begin{x10}
c.drop();
\end{x10}

\noindent{} 
The method does nothing if the activity has already dropped {\cf c}. 

\paragraph{Static semantics.}
The compiler should issue an error if it discovers a potentially live
execution path from a {\tt c.drop()} to a statement using {\tt c}.

\subsection{Posting statements on a clock}\index{clock!now}
\Xten{} provides syntactic support for a common idiom. Often it may be
necessary for an activity $A$ to require that a certain set of
statements be executed to completion before a clock $c$ can move
forward, without $A$ actually waiting for the completion
of the statement. We introduce the syntax:
\begin{x10}
461 Statement ::= NowStatement
471 StatementNoShortIf ::= 
       NowStatementNoShortIf
479 NowStatement ::= 
       now ( Clock ) Statement
489 NowStatementNoShortIf ::= 
       now ( Clock ) StatementNoShortIf
\end{x10}
\noindent 

A statement {\tt now (c) s} may be considered as shorthand for
\begin{x10}
  async clocked(c) \{ 
     finish async s; 
  \}
\end{x10}

\paragraph{Note.} Because of the static semantics of {\tt finish}
it is not possible to nest {\cf now} statements. Instead if it proves
useful, we may introduce a multi-clocked {\tt now} statement,
which permits the statement to be posted on multiple clocks
simultaneously.
\begin{x10}
479' NowStatement ::= 
       now ( ClockList ) Statement
489' NowStatementNoShortIf ::= 
       now ( ClockList ) StatementNoShortIf  
\end{x10}

\subsection{Program equivalences}

From the discussion above it should be clear that the following
equivalences hold:

\begin{eqnarray}
 {\cf c.resume(); next;}       &=& {\cf next;}\\
 {\cf c.resume(); d.resume();} &=& {\cf d.resume(); c.resume();}\\
 {\cf c.resume(); c.resume();} &=&  {\cf c.resume();}
\end{eqnarray}

Note that {\cf next; next;} is not the same as {\cf next;}. The
first will wait for clocks to advance twice, and the second
once.  

\cbend
%%\subsection{Implementation Notes}
%%Clocks may be implemented efficiently with message passing, e.g.{} by
%%using short-circuit ideas in \cite{SaraswatPODC88}.  Recall that every
%%activity is spawned with references to a fixed number of clocks. Each
%%reference should be thought of as a global pointer to a location in
%%some place representing the clock. (We shall discuss a further
%%optimization below.) Each clock keeps two counters: the total number
%%of outstanding references to the clock, and the number of activities
%%that are currently suspended on the clock.
%%
%%When an activity $A$ spawns another activity $B$ that will reference a
%%clock $c$ referenced by $A$, $A$ {\em splits} the reference by sending
%%a message to the clock. Whenever an activity drops a reference to a
%%clock, or suspends on it, it sends a message to the clock. 
%%
%%The optimization is that the clock can be represented in a distributed
%%fashion. Each place keeps a local counter for each clock that is
%%referenced by an activity in that place. The global location for the
%%clock simply keeps track of the places that have references and that
%%are quiescent. This can reduce the inter-place message traffic
%%significantly.

\todo{Reintroduce clocked types}
%%\section{Clocked types}\index{types!clocked}
%%
%%%We allow types to specify clocks, via a {\cf clocked(c)} modifier,
%%%e.g.{}
%%
%%%\begin{x10}
%%%  clocked(c) int r;
%%%\end{x10}
%%
%%%This declaration asserts that {\cf r} is accessible
%%%(readable/writable) only by those statements that are clocked on {\cf
%%%c}. Thus propagation of the clock provides some control over the
%%%``visibility'' of {\cf r}.
%%
%%The declaration 
%%
%%\begin{x10}
%%  clocked(c) final int l = r;
%%\end{x10}
%%
%%\noindent asserts additionally that in each clock instant {\cf l} is final, 
%%i.e.{} the value of {\cf l} may be reset at the beginning of each phase
%%of {\tt c} but stays constant during the phase.
%%
%%This statement terminates when the computation of {\tt r} has
%%terminated and the assignment has been performed.
%%
%%\todo{Generalize the syntax so that aggregate variables can be clocked with an aggregate clock of the same shape.}
%%
%%\subsection{Clocked assignment}\index{assignment!clocked}
%%We expect that most arrays containing application data will be
%%declared to be {\cf clocked final}. We support this very powerful type
%%declaration by providing a new statement:
%%{\footnotesize
%%\begin{verbatim}
%%  next(c) l = r; 
%%\end{verbatim}}
%%
%%
%%\noindent 
%%for a variable $l$ declared to be clocked on $c$. The statement
%%assigns $r$ to the {\em next} value of $l$. There may be multiple such
%%assignments before the clock advances. The last such assignment
%%specifies the value of the variable that will be visible after the
%%clock has advanced.  If the variable is {\cf clocked final} it is
%%guaranteed that {\em all} readers of the variable throughout this
%%phase will see the value $r$.
%%
%%The expression {\tt r} is implicitly treated as {\tt now(c) r}. That
%%is, the clock {\tt c} will not advance until the computation of {\tt r} has
%%terminated.
%%
%%\section{Examples}
%%
%%Consider the core of the ASCI Benchmark Sweep3D program for computing
%%solutions to mass transport problems.
%%
%%In a nutshell the core computation is a triply nested sequential loop
%%in which the value of a variable in the current iteration is dependent
%%on the values of neighboring variables in a past iteration. Such a
%%problem can be parallelized through pipelining. One visualizes a
%%diagonal wavefront sweeping through the array. An MPI version of the
%%program may be described as follows. There is a two dimensional grid
%%of processors which performs the following computation
%%repeatedly. Each processor synchronously receives a value from the
%%processor to its west, then to its north, then computes some function
%%of these values and computes a new value to be sent to the processor
%%to its east and then to its south.  Ignoring the behavior of the
%%boundary processors for the moment such a computation may be described
%%by the following \Xten{} program:
%%
%%\begin{x10}
%%region R = [1..n0,1..m0];
%%clock[R] W,N;
%%clock(W) final double [cyclic(R)] A; 
%%for (int t : 1..TMax) \{
%%  ateach( i,j:A) 
%%    clock (W[i-1,j],N[i,j-1],W[i,j],N[i,j]) \{
%%      double west = now (W[i-1,j]) future\{A[i-1,j]\}; 
%%      W[i-1,j].continue();           
%%      double north = now (N[i,j-1]) future\{A[i,j-1]\}; 
%%      N[i,j-1].continue();
%%      next(W[i,j]) A[i,j] = compute(west, north);
%%      next W[i-1,j],N[i,j-1],W[i,j],N[i,j];
%%  \}
%%\}
%%\end{x10}

	\par  %\vfill\eject %0.1
\chapter{Interfaces}
\label{XtenInterfaces}\index{interfaces}

{}\XtenCurrVer{} interfaces are essentially the same \java{}
interfaces \cite[\S 9]{jls2}. An interface primarily specifies
signatures for public methods. It may extend multiple interfaces. 
%The
%need for magic constants in interfaces is lessened with the
%introduction of {\tt enum} (\S~\ref{XtenEnums}).


Future version of \Xten{} will introduce additional structure in
interface definitions that will allow the programmer to state
additional properties of classes that implement that interface. For
instance a method may be declared {\tt pure} to indicate that its
evaluation cannot have any side-effects. A method may be declared {\tt
local} to indicate that its execution is confined purely to the
current place (no communication with other places). Similarly,
behavioral properties of the method as they relate to the usage of
clocks of the current activity may be specified.

	\par  %\vfill\eject % empty
\chapter{Classes}
\label{XtenClasses}\index{class}

The {\em class declaration} has
a list of type \params,
value properties, 
a constraint (the {\em class invariant}, a single superclass,
one or more interfaces, and a class body containing the
the definition of
fields, methods, and member types.
Each such declaration introduces a class
type (\Sref{ReferenceTypes}).

\begin{grammar}
NormalClassDeclaration \:
      ClassModifiers\opt \xcd"class" Identifier  \\
   && TypeParameterList\opt PropertyList\opt Guard\opt \\
   && Super\opt Interfaces\opt ClassBody \\
\\
TypeParameterList     \:  \xcd"[" TypeParameters \xcd"]" \\
TypeParameters        \:  TypeParameter ( \xcd"," Typearameter )\star \\
TypeParameter         \:  Variance\opt Annotation\star Identifier     \\
Variance \: \xcd"+" \\
         && \xcd"-" \\
\\
PropertyList     \:  \xcd"(" Properties \xcd")" \\
Properties       \:  Property ( \xcd"," Property )\star \\
Property         \:  Annotation\star \xcd"val"\opt Identifier \xcd":" Type \\
\\
Super \: \xcd"extends" ClassType \\
Interfaces \: \xcd"implements" InterfaceType ( \xcd"," InterfaceType)\star \\
\\
ClassBody \: ClassMember\star \\
ClassMember \: ClassDeclaration \\
            \| InterfaceDeclaration \\
            \| FieldDeclaration \\
            \| MethodDeclaration \\
            \| ConstructorDeclaration \\
\end{grammar}

A type parameter declaration is given by an optional variance
tag and an identifier.
A type parameter must be
bound to a concrete type when an instance of the class is created.


A value property has a name and a type.   Value properties
are accessible in the same way as \xcd"public" \xcd"final"
fields.

\begin{staticrule*}
It is a compile-time error for a class
defining a value property \xcd"x: T" to have an ancestor class that defines
a value property with the name \xcd"x".  
\end{staticrule*}

Each class \xcd"C" defining a property \xcd"x: T" implicitly has a field

\begin{xten}
public val x : T;
\end{xten} 

\noindent and a getter method

\begin{xten}
public final def x(): T { return x; }
\end{xten}

\noindent Each interface \xcd"I" defining a property \xcd"x: T"
implicitly has a getter method

\begin{xten}
public def x(): T;
\end{xten}

\begin{staticrule*}
It is a compile-time error for a class or
interface defining a property \xcd"x :T" to have an existing method with
the signature \xcd"x(): T".
\end{staticrule*}

Properties are used to build dependent types from classes, as
described in \Sref{DepType:DepType}.

\label{ClassGuard}

The \grammarrule{Guard} in a class or interface declaration specifies an
explicit condition on the properties of the type, and is discussed further
in \Sref{DepType:Guard}.

\begin{staticrule*}
     Every constructor for a class defining
   properties \xcdmath"x$_1$: T$_1$, $\ldots$, x$_n$: T$_n$" must ensure that each of the fields
   corresponding to the properties is definitely initialized
   (cf. requirement on initialization of final fields in Java) before the
   constructor returns.
\end{staticrule*}

Type \params{}
are used to define generic classes and
interfaces, as described in \Sref{Generics}.

Classes are structured in a single-inheritance code
hierarchy, may implement multiple interfaces, may have static and
instance fields, may have static and instance methods, may have
constructors, may have static and instance initializers, may have
static and instance inner classes and interfaces. \Xten{} does not
permit mutable static state, so the role of static methods and
initializers is quite limited. Instead programmers should use
singleton classes to carry mutable static state.

Method signatures may specify checked exceptions. Method definitions
may be overridden by subclasses; the overriding definition may have a
declared return type that is a subclass of the return type of the
definition being overridden. Multiple methods with the same name but
different signatures may be provided on a class (ad hoc
polymorphism). The public/private/protected/package-protected access
modification framework may be used.

\oldtodo{Add the new rule for preventing leakage of this from a constructor.}

Because of its different concurrency model, \Xten{} does not support
\xcd"transient" and \xcd"volatile" field modifiers.

\oldtodo{Figure out class modifiers. Figure out which new ones need to be added to support IEEE FP.}

\section{Reference classes}\index{class!reference class}\label{ReferenceClasses}
A reference class is declared with the optional keyword \xcd"reference" preceding \xcd"class" in a class declaration. Reference
class declarations may be used to construct reference types
(\Sref{ReferenceTypes}). Reference classes may have mutable
fields. Instances of a reference class are always created in a fixed
place and in \XtenCurrVer{} stay there for the lifetime of the
object. (Future versions of \Xten{} may support object migration.)
Variables declared at a reference type always store a reference to the
object, regardless of whether the object is local or remote.

\section{Value classes}\index{class!value class}\label{ValueClasses}

{}\Xten{} singles out a certain set of classes for additional
support. A class is said to be {\em stateless} if all of its fields
are declared to be \xcd"final" (\Sref{FinalVariable}), otherwise it
is {\em stateful}. (\Xten{} has syntax for specifying an array class
with final fields, unlike \java{}.) A {\em stateless (stateful)
object} is an instance of a stateless (stateful) class.

{}\Xten{} allows the programmer to signify that a class (and all its
descendents) are stateless. Such a class is called a {\em value
class}.  The programmer specifies a value class by prefixing the
modifier \xcd"value" before the keyword \xcd"class" in a class
declaration.  (A class not declared to be a value class will be called
a {\em reference class}.)  Each instance field of a value class is
treated as \xcd"final". It is legal (but neither required nor recommended)
for fields in a value class to be declared final. For brevity, the \Xten{}
compiler allows the programmer to omit the keyword \xcd"class" after
\xcd" value" in a value class declaration.


\begin{grammar}
ValueClassDeclaration \:
      ClassModifiers\opt \xcd"value" \xcd"class"\opt Identifier  \\
   && TypePropertyList\opt PropertyList\opt Guard\opt \\
   && Super\opt Interfaces\opt ValueClassBody \\
\end{grammar}


The \xcd"Box" type constructor (\Sref{BoxType}) can
be used to declare variables whose value may be \xcd"null" or a value
type.

Stable equality for value types is defined through a deep walk,
bottoming out in fields of reference types (\Sref{StableEquality}).

\begin{staticrule*}
It is a compile-time error for a value class to inherit from a
stateful class or for a reference class to inherit from a value
class. All fields of a value class are implicitly declared \xcd"final".
\end{staticrule*}

\subsection{Representation}

Since value objects do not contain any updatable locations, they can
be freely copied from place to place. An implementation may use
copying techniques even within a place to implement value types,
rather than references. This is transparent to the programmer.

More explicitly, \Xten{} guarantees that an implementation must always
behave as if a variable of a reference type takes up as much space as
needed to store a reference that is either null or is bound to an
object allocated on the (appropriate) heap. However, \Xten{} makes no
such guarantees about the representation of a variable of value
type. The implementation is free to behave as if the value is stored
``inline'', allocated on the heap (and a reference stored in the
variable) or use any other scheme (such as structure-sharing) it may
deem appropriate. Indeed, an implementation may even dynamically
change the representation of an object of a value type, or dynamically
use different representations for different instances (that is,
implement automatic box/unboxing of values).

Implementations are strongly encouraged to implement value types as
space-efficiently as possible (e.g., inlining them or passing them in
registers, as appropriate).  Implementations are expected to cache
values of remote final value variables by default. If a value is
large, the programmer may wish to consider spawning a remote activity
(at the place the value was created) rather than referencing the
containing variable (thus forcing it to be cached).

\oldtodo{Need to figure out whether we should let the programmer be
aware of lazy pull vs full-value push of value objects. This is the
idea of introducing a *-annotation. Need to make a decision on
this. Could leave this for 0.7.}

\begin{example}
A functional \xcd"LinkedList" program may be written as follows:


\begin{xten}
value LinkedList { 
  val first: Object;
  val rest: LinkedList;
  public def this(first: Object) {
     this(first, null);
  }
  public def this(first: Object, rest: LinkedList) {
    this.first = first;
    this.rest = rest;
  }
  public def first(): Object {
    return first;
  }
  public def rest(): LinkedList {
    return rest;
  } 
  public def append(l: LinkedList): LinkedList {
    return (this.rest == null) 
        ? new LinkedList(this.first, l) 
        : this.rest.append(l);
  }
}
\end{xten}

Similarly, a \xcd"Complex" class may be implemented as follows:
\begin{xten}
value Complex { 
  re: Double;
  im: Double;
  public def this(re: Double, im: Double) {
     this.re=re;
     this.im=im;
  }
  public def add(other: Complex): Complex {
    return new Complex(this.re+other.re,
                       this.im+other.im);
  }
  public def mult(other: Complex): Complex {
    return new Complex(this.re^2-other.re^2,
                       2*this.im*other.im);
  }
  ...
}
\end{xten}
\end{example}

\section{Type invariants}
\index{type invariants}
\index{guards}

There is a general recipe for constructing a list of parameters or
properties \xcdmath"x$_1$: T$_1${c$_1$}, $\dots$, x$_k$: T$_k${c$_k$}" that must satisfy a given
(satisfiable) constraint \xcd"c". 

\begin{xtenmath}
class Foo(x$_1$: T1{x$_2$: T$_2$; $\dots$; x$_k$: T$_k$; c},
          x$_2$: T2{x$_3$: T$_3$; $\dots$; x$_k$: T$_k$; c},
          $\dots$
          x$_k$: T$_k${c}) {
  $\dots$
}
\end{xtenmath}

The first type \xcdmath"x$_1$: T$_1${x$_2$: T$_2$; $\dots$; x$_k$: T$_k$; c}" is consistent iff
\xcdmath"$\exists$x$_1$: T$_1$, x$_2$: T$_2$, $\dots$, x$_k$: T$_k$. c" is consistent. The second is
consistent iff
\begin{xtenmath}
$\forall$x$_1$: T$_1${x$_2$: T$_2$; $\dots$; x$_k$: T$_k$; c}
$\exists$x$_2$: T$_2$. $\exists$x$_3$: T$_3$, $\dots$, x$_k$: T$_k$. c
\end{xtenmath}
\noindent But this is always true. Similarly for the conditions for the other
properties.

Thus logically every satisfiable constraint \xcd"c" on a list of parameters
\xcdmath"x$_1$", \dots, \xcdmath"x$_k$"
can be expressed using the dependent types of 
\xcdmath"x$_i$", provided
that the constraint language is rich enough to permit existential
quantifiers.

Nevertheless we will find it convenient to permit the programmer to
explicitly specify a depclause after the list of properties, thus:
\begin{xten}
class Point(i: Int, j: Int) { ... }
class Line(start: Point, end: Point){end != start}
  = { ... }
class Triangle (a: Line, b: Line, c: Line)
        {a.end == b.start && b.end == c.start &&
         c.end == a.start} = { ... }
class SolvableQuad(a: Int, b: Int, c: Int)
                   {a*x*x+b*x+c==0} = { ... }
class Circle (r: Int, x: Int, y: Int)
              {r > 0 && r*r==x*x+y*y} = { ... }
class NonEmptyList extends List{n > 0} {...}
\end{xten}

Consider the definition of the class \xcd"Line". This may be thought of as
saying: the class \xcd"Line" has two fields, \xcd"start: Point" and
\xcd"end: Point".
Further, every instance of \xcd"Line" must satisfy the constraint that
\xcd"end != start". Similarly for the other class definitions. 

In the general case, the production for \grammarrule{NormalClassDeclaration}
specifies that the list of properties may be followed by a
\grammarrule{Guard}.

\begin{grammar}
NormalClassDeclaration \:
      ClassModifiers\opt \xcd"class" Identifier  \\
   && TypeParameterList\opt PropertyList\opt Guard\opt \\
   && Extends\opt Interfaces\opt ClassBody \\
\\
NormalInterfaceDeclaration \:
      InterfaceModifiers\opt \xcd"interface" Identifier  \\
   && TypeParameterList\opt PropertyList\opt Guard\opt \\
   && ExtendsInterfaces\opt InterfaceBody \\
\end{grammar}

All the properties in the list, together with inherited properties,
may appear in the \grammarrule{Guard}. A guard \xcd"c" with
value property list \xcdmath"x$_1$: T$_1$, $\dots$, x$_n$: T$_n$"
for a class \xcd"C" is said to be consistent if each of the \xcdmath"T$_i$" are
consistent and the constraint
\begin{xtenmath}
$\exists$x$_1$: T$_1$, $\dots$, x$_n$: T$_n$, self: C. c
\end{xtenmath}
\noindent is valid (always true).

\section{Class definitions}

Consider a class definition
\begin{xtenmath}
$\mbox{\emph{ClassModifiers}}^{\mbox{?}}$
class C(x$_1$: P$_1$, $\dots$, x$_n$: P$_n$) extends D{d}
   implements I$_1${c$_1$}, $\dots$, I$_k${c$_k$}
$\mbox{\emph{ClassBody}}$
\end{xtenmath}

Each of the following static semantics rules must be satisfied:

\begin{staticrule}{Int-implements}
The type invariant \xcdmath"$\mathit{inv}$(C)" of \xcd"C" must entail
\xcdmath"c$_i$[this/self]" for each $i$ in $\{1, \dots, k\}$
\end{staticrule}

\begin{staticrule}{Super-extends}
The return type \xcd"c" of each constructor in \grammarrule{ClassBody}
must entail \xcd"d".
\end{staticrule}

\section{Constructor definitions}

A constructor for a class \xcd"C" is guaranteed to return an object of the
class on successful termination. This object must satisfy i(C), the
class invariant associated with \xcd"C" (\Sref{DepType:TypeInvariant}).
However,
often the objects returned by a constructor may satisfy {\em stronger}
properties than the class invariant. \Xten{}'s dependent type system
permits these extra properties to be asserted with the constructor in
the form of a constrained type (the ``return type'' of the constructor):

\begin{grammar}
ConstructorDeclarator \:
  \xcd"def" \xcd"this" TypeParameterList\opt \xcd"(" FormalParameterList\opt \xcd")" \\
  && ReturnType\opt Guard\opt Throws\opt \\
ReturnType    \: \xcd":" Type \\
Guard   \: "{" DepExpression "}" \\
Throws    \: \xcd"throws" ExceptionType  ( \xcd"," ExceptionType )\star \\
ExceptionType \: ClassBaseType Annotation\star \\
\end{grammar}

\label{ConstructorGuard}

The parameter list for the constructor
may specify a \emph{guard} that is to be satisfied by the parameters
to the list.

\begin{example}
Here is another example.
\begin{xten}
public class List[T](n: Int{n >= 0}) {
    protected head: Box[T];
    protected tail: List[T](n-1);
    public def this(o: T, t: List[T]) : List[T](t.n+1) = {
        n = t.n+1;
        tail = t;
        head = o;
    }
    public def this() : List[T](0) = {
        n = 0;
        head = null;
        tail = null;
    }
    ...
}
\end{xten}
The second constructor returns a \xcd"List" that is guaranteed to have
\xcd"n==0";
the first constructor is guaranteed to return a List with \xcd"n>0"
(in fact, \xcd"n==t.n+1", where the argument to the constructor is \xcd"t"). 
This is recorded by the programmer in the constrained type associated with the
constructor.
\end{example}

\begin{staticrule}{Super-invoke}
   Let \xcd"C" be a class with properties
   \xcdmath"p$_1$: P$_1$, $\dots$, p$_n$: P$_n$", invariant \xcd"c"
   extending the constrained type \xcd"D{d}" (where \xcd"D" is the name of a class).

   For every constructor in \xcd"C" the compiler checks that the call to
   super invokes a constructor for \xcd"D" whose return type is strong enough
   to entail \xcd"d". Specifically, if the call to super is of the form 
     \xcdmath"super(e$_1$, $\dots$, e$_k$)"
   and the static type of each expression \xcdmath"e$_i$" is
   \xcdmath"S$_i$", and the invocation
   is statically resolved to a constructor
\xcdmath"def this(x$_1$: T$_1$, $\dots$, x$_k$: T$_k$){c}: D{d$_1$}"
   then it must be the case that 
\begin{xtenmath}
x$_1$: S$_1$, $\dots$, x$_i$: S$_i$ $\vdash$ x$_i$: T$_i$  (for $i \in \{1, \dots, k\}$)
x$_1$: S$_1$, $\dots$, x$_k$: S$_k$ $\vdash$ c  
d$_1$[a/self] && x$_1$: S$_1$ ... && x$_k$: S$_k$ $\vdash$ d[a/self]      
\end{xtenmath}
\noindent where \xcd"a" is a constant that does not appear in 
\xcdmath"x$_1$: S$_1$ $\wedge$ ... $\wedge$ x$_k$: S$_k$".
  
\end{staticrule}

\begin{staticrule}{Constructor return}
   The compiler checks that every constructor for \xcd"C" ensures that
   the properties \xcdmath"p$_1$,..., p$_n$" are initialized with values which satisfy
   \xcdmath"t(C)", and its own return type \xcd"c'" as follows.  In each constructor, the
   compiler checks that the static types \xcdmath"T$_i$" of the expressions \xcdmath"e$_i$"
   assigned to \xcdmath"p$_i$" are such that the following is
   true:
\begin{xtenmath}
p$_1$: T$_1$, $\dots$, p$_n$: T$_n$ $\vdash$ t(C) $\wedge$ c'     
\end{xtenmath}
\end{staticrule}
(Note that for the assignment of \xcdmath"e$_i$" to \xcdmath"p$_i$"
to be type-correct it must be the
    case that \xcdmath"p$_i$: T$_i$ $\wedge$ p$_i$: P$_i$".) 


\begin{staticrule}{Constructor invocation}
The compiler must check that every invocation \xcdmath"C(e$_1$, $\dots$, e$_n$)" to a
constructor is type correct: each argument \xcdmath"e$_i$" must have a static type
that is a subtype of the declared type \xcdmath"T$_i$" for the $i$th
argument of the
constructor, and the conjunction of static types of the argument must
entail the \grammarrule{Guard} in the parameter list of the constructor.
\end{staticrule}

\section{Field definitions}

Not every instance of a class needs to have every field defined on the
class. In Java-like languages this is ensured by conditionally setting
fields to a default value, such as \xcd"null", in those instances where the
fields are not needed.  

Consider the class \xcd"List" used earlier.  Here all instances of \xcd"List"
returned by the second constructor do not need the fields \xcd"value" and
\xcd"tail"; their value is set to null.

\label{FieldGuard}

\Xten{} permits a much cleaner solution that does not require default
values such as null to be stored in such fields. \Xten{} permits fields to
be {\em guarded} with a constraint.  The field is accessible
only if the \emph{guard} constraint is satisified.

\begin{grammar}
FieldDeclaration  \:
   FieldModifiers\opt \xcd"val" VariableDeclarators \xcd";" \\
   \|
   FieldModifiers\opt \xcd"var" VariableDeclarators \xcd";" \\
VariableDeclarators \:
        VariableDeclarator ( \xcd"," VariableDeclarator )\star \\
VariableDeclarator \:
   Identifier ( Constraint )\opt ( \xcd":" Type )\opt ( \xcd"=" Expression )\opt \\
\end{grammar}

It is illegal for code to access a guarded field through a reference
whose static type does not satisfy the associated guard, even
implicitly (i.e., through an implicit \xcd"this"). Rather the source
program should contain an explict cast, e.g., \xcd"me: C{c} = this as C{c}".

\begin{staticrule*}
Let \xcd"f" be a field defined in class
\xcd"C" with guard \xcd"c".  The compiler declares an error if
field \xcd"f" is accessed through a reference \xcd"o" whose static
type is not a subtype of \xcd"C{c}".
\end{staticrule*}

\begin{example}

We may now rewrite the List example:
\begin{xten}
public class List(n: Int{n>=0}) {
  protected val head{n>0}: Object;
  protected val tail{n>0}: List(n-1);
  public def this(o: Object, t: List): List(t.n+1) {
     property(t.n+1);
     head=o;
     tail=t;
  }
  public def this(): List(0) {
     property(0);
  }
  ...
}
\end{xten}

The fields \xcd"value" and \xcd"tail" do not exist for instances of the class
\xcd"List(0)".
\end{example}

It is a compile-time error for a class to have two fields of the same
name, even if their guards are different. A class \xcd"C" with a field
named \xcd"f" is said to {\em hide} a field in a superclass named \xcd"f".

\begin{staticrule*}
     A class may not declare two fields with the same name.
\end{staticrule*}

To avoid an ambiguity, it is a static error for a class to
declare a field with a function type (\Sref{FunctionTypes}) with
the same name and signature  as a method of the same class.

\subsection{Field hiding}

A subclass that defines a field \xcd"f" hides any field \xcd"f"
declared in a superclass, regardless of their types.  The
superclass field \xcd"f" may be accessed within the body of
the subclass via the reference \xcd"super.f".

\eat{
The definition of field hiding does not take
\grammar{Guard} into
account. Suppose a class \xcd"C" has a field

\begin{xten}
var f{c}: Foo;
\end{xten}
\noindent and a subclass \xcd"D" of \xcd"C" has a field
\begin{xten}
var f{d}: Fum;
\end{xten}

We will say that \xcd"D.f" hides \xcd"C.f", {\em regardless} of the
constraints \xcd"c" and \xcd"d". This is in keeping with \Java, and
permits a naive implementation which always allocates space for a
conditional field.

\begin{rationale}
It might seem attractive to require that \xcd"D.f"
hides \xcd"C.f" only if \xcd"d" implies \xcd"c". This would seem
to necessitate a rather complex implementation structure for classes,
requiring some kind of a heterogenous translation for
constrained types of \xcd"C"
and \xcd"D". This bears further investigation.
\end{rationale}
}

\section{Method definitions}

\Xten{} permits guarded method definitions, similar to guarded
field definitions. Additionally, the parameter list for a method may
contain a \grammarrule{Guard}.

\begin{grammar}
MethodDeclaration \: MethodHeader \xcd";" \\
                  \| MethodHeader \xcd"=" ClosureBody \\
MethodHeader \:  
  MethodModifiers\opt \xcd"def" Identifier TypeParameters\opt \\
&& \xcd"(" 
  FormalParameterList\opt \xcd")" Guard\opt \\
  && ReturnType\opt Throws\opt \\
\end{grammar}

In the formal parameter list, variables may be declared with
\xcd"val" or \xcd"var".  If neither is specified, the variable
is \xcd"val".

\label{MethodGuard}

The guard (specified by \grammarrule{Guard})
specifies a constraint \xcd"c" on the
properties of the class \xcd"C" on which the method is being defined. The
method exists only for those instances of \xcd"C" which satisfy \xcd"c".  It is
illegal for code to invoke the method on objects whose static type is
not a subtype of \xcd"C{c}".

\begin{staticrule*}
    The compiler checks that every method invocation
    \xcdmath"o.m(e$_1$, $\dots$, e$_n$)"
    for a method is type correct. Each each argument
    \xcdmath"e$_i$" must have a
    static type \xcdmath"S$_i$" that is a subtype of the declared type
    \xcdmath"T$_i$" for the $i$th
    argument of the method, and the conjunction of static types
    of the arguments must entail the guard in the parameter list
    of the method.

    The compiler checks that in every method invocation
    \xcdmath"o.m(e$_1$, $\dots$, e$_n$)"
    the static type of \xcd"o", \xcd"S", is a subtype of \xcd"C{c}", where the method
    is defined in class \xcd"C" and the guard for \xcd"m" is equivalent to
    \xcd"c".

    Finally, if the declared return type of the method is
    \xcd"D{d}", the
    return type computed for the call is
    \xcdmath"D{a: S; x$_1$: S$_1$; $\dots$; x$_n$: S$_n$; d[a/this]}",
    where \xcd"a" is a new
    variable that does not occur in
    \xcdmath"d, S, S$_1$, $\dots$, S$_n$", and
    \xcdmath"x$_1$, $\dots$, x$_n$" are the formal
    parameters of the method.
\end{staticrule*}

\begin{example}
Consider the program:
\begin{xten}
public class List(n: Int{n>=0}) {
  protected val head{n>0}: Object;
  protected val tail{n>0}: List(n-1);
  public def this(o: Object, t: List): List(t.n+1) = {
     property(t.n+1);
     head=o;
     tail=t;
  }
  public def this(): List(0) = {
     property(0);
  }
  public def append(l: List): List{self.n==this.n+l.n} = {
      return (n==0)? l
         : new List(head, tail.append(l)); 
  }
  public def nth(k: Int{1 <= k && k <= n}){n > 0}: Object = {
      return k==1 ? head : tail.nth(k-1);
  }
}
\end{xten}

The following code fragment
\begin{xten}
u: List{self.n==3} = ...
t: List{self.n==x} = ...;
s: List{self.n==x+3} = t.append(u);
\end{xten}
\noindent will typecheck. The type of the expression \xcd"t.append(u)" is 
\begin{xten}
List{a: List{self.n==x}; 
     l: List{self.n==3}; self.n==a.n+l.n}  
\end{xten}
\noindent which is equivalent to:
\begin{xten}
List{self.n==x+3}
\end{xten}
\end{example}

The method body is either an expression, a block of statements,
or a block ending with an expression.

\subsection{Property methods}

A method declared with the modifier \xcd"property" may be used
in constraints.  A property method declared in a class must have
a body and must not be \xcd"void".  The body of the method must
consist of only a single \xcd"return" statement or a single
expression.  It is a static error of the expression cannot be
represented in the constraint system.

Property methods in classes are implicitly \xcd"final"; they cannot be
overridden.

A property method definition may omit the formal parameters and
the \xcd"def" keyword.  That is, the following are equivalent:

\begin{xten}
property def rail(): boolean = rect && onePlace == here && zeroBased;
property rail: boolean = rect && onePlace == here && zeroBased;
\end{xten}

\subsection{Method overloading, overriding, hiding, shadowing and obscuring}
\label{MethodOverload}

The definitions of method overloading, overriding, hiding, shadowing
and obscuring in \Xten{} are the same as in \Java, modulo the following
considerations motivated by type parameters and dependent types.

Two or more methods of a class or interface may have the same
name if they have a different number of type parameters, or
they have value parameters of different types.

The definition of a method declaration \xcdmath"m$_1$" ``having the same signature
as'' a method declaration \xcdmath"m$_2$" involves identity of types. Two \Xten{} types
are defined to be identical iff they are equivalent (\Sref{DepType:Equivalence}).
Two methods are said to have {\em the same signature} if (a)
they have the same number of type parameters, (b) they have the
same number of formal (value) parameters, and (c) for each formal parameter
their types are equivalent. It is a compile-time error for there
to be two methods with the same name and same signature in a class
(either defined in that class or in a superclass).

\begin{staticrule*}
  A class \xcd"C" may not have two declarations for a method named \xcd"m"---either
  defined at \xcd"C" or inherited:
\begin{xtenmath}
def m[X$_1$, $\dots$, X$_m$](v$_1$: T$_1${t$_1$}, $\dots$, v$_n$: T$_n${t$_n$}){tc}: T {...}
def m[X$_1$, $\dots$, X$_m$](v$_1$: S$_1${s$_1$}, $\dots$, v$_n$: S$_n${s$_n$}){sc}: S {...}
\end{xtenmath}
\noindent
if it is the case that the types \xcd"C{tc}", \xcdmath"T$_1${t$_1$}",
\dots, \xcdmath"T$_n${t$_n$}" are
equivalent to the types \xcdmath"C{sc}, S$_1${t$_1$}, $\dots$, T$_n${t$_n$}"
respectively.
\end{staticrule*}

In addition, the guard of a overriding method must be 
no stronger than the guard of the overridden method.   This
ensures that any virtual call to the method
satisfies the guard of the callee.

\begin{staticrule*}
  If a class \xcd"C" overrides a method of a class or interface
  \xcd"B", the guard of the method in \xcd"B" must entail
  the guard of the method in \xcd"C".
\end{staticrule*}

A class \xcd"C" inherits from its direct superclass and superinterfaces all
their methods visible according to the access restriction modifiers
public/private/protected/(package) of the superclass/superinterfaces
that are not hidden or overridden. A method \xcdmath"M$_1$" in a class
\xcd"C" overrides
a method \xcdmath"M$_2$" in a superclass \xcd"D" if
\xcdmath"M$_1$" and \xcdmath"M$_2$" have the same signature.
Methods are overriden on a signature-by-signature basis.

A method invocation \xcdmath"o.m(e$_1$, $\dots$, e$_n$)"
is said to have the {\em static signature}
\xcdmath"<T, T$_1$, $\dots$, T$_n$>" where \xcd"T" is the static type of
\xcd"o", and
\xcdmath"T$_1$",
\dots,
\xcdmath"T$_n$"
are the static types of \xcdmath"e$_1$", \dots, \xcdmath"e$_n$",
respectively.  As in
\Java, it must be the case that the compiler can determine a single
method defined on \xcd"T" with argument type
\xcdmath"T$_1$", \dots \xcdmath"T$_n$"; otherwise, a
compile-time error is declared. However, unlike \Java, the \Xten{} type \xcd"T"
may be a dependent type \xcd"C{c}". Therefore, given a class definition for
\xcd"C" we must determine which methods of \xcd"C" are available at a type
\xcd"C{c}". But the answer to this question is clear: exactly those methods
defined on \xcd"C" are available at the type \xcd"C{c}"
whose guard \xcd"d" is implied by \xcd"c".


\begin{example}
  Consider the definitions:
\begin{xten}
class Point(i: Int, j: Int) {...}
class Line(s: Point, e: Point{self != i}) {
  // m1: Both points lie in the right half of the plane
  def draw(){s.i>= 0 && e.i >= 0} = {...}
  // m2: Both points lie on the y-axis
  def draw(){s.i== 0 && e.i == 0} = {...}
  // m3: Both points lie in the top half of the plane
  def draw(){s.j>= 0 && e.j >= 0} = {...}
  // m4: The general method
  def draw() = {...}
} 
\end{xten} 
\noindent  Three different implementations are given for the
\xcd"draw" method, one
  for the case in which the line lies in the right half of the plane,
  one for the case that the line lies on the y-axis and the third for
  the case that the line lies in the top half of the plane.


\noindent  Consider the invocation
\begin{xten}
m: Line{s.i < 0} = ...
m.draw();
\end{xten}

\noindent  This generates a compile time error because there is no applicable
  method definition.

\noindent  Consider the invocation

\begin{xten}
m: Line{s.i>=0 && s.j>=0 && e.i>=0 && e.j>=0} = ...
m.draw();
\end{xten}

\noindent  This generates a compile time error because both
\xcd"m1" and \xcd"m3" are applicable.

\noindent  Consider the invocation
\begin{xten}
m: Line{s.i>=0 && s.j>=0 && e.i>=0} = ...
m.draw();
\end{xten}
  This does not generate any compile-time error since only m1 is
  applicable. 
\end{example}


In the last example, notice that at runtime \xcd"m1" will be invoked
(assuming \xcd"m" contains an instance of the \xcd"Line" class, and not some
subclass of \xcd"Line" that overrides this method). This will be the case
even if \xcd"m" satisfies at runtime the stronger conditions for \xcd"m2" (i.e.,
\xcd"s.i==0 && e.i==0"). That is, dynamic method lookup will not take into
account the  ``strongest'' constraint that the receiver may
satisfy, i.e.,
its ``strongest constrained type''. 

\begin{rationale}
  The design decision that dynamic method lookup should ignore
  dependent type information was made to keep the design and the
  implementation simple and to ensure that serious errors such as
  method invocation errors are captured at compile-time.
 
  Consider the above example and the invocation
\begin{xten}
m: Line = ...
m.draw();    
\end{xten}


   Statically the compiler will not report an error because m4 is the
   only method that is applicable. However, if dynamic method lookup
   were to use constrained types then we would face the problem that if m is a
   line that lives in the upper right quadrant then both \xcd"m2"
   and \xcd"m3"
   are applicable and one does not override the other. Hence we must
   report an error dynamically.

   As discussed above, the programmer can write code with \xcd"instanceof"
   and class casts that perform any application-appropriate
   discrimination.  
\end{rationale}

\subsection{Method annotations}

\subsubsection{\Xcd{atomic} annotation}

A method may be declared \xcd"atomic".

\begin{grammar}
  MethodModifier \: \xcd"atomic"  
\end{grammar}

Such a method is treated as if the statement in its body is wrapped 
implicitly in an \xcd"atomic" statement.

\subsubsection{\Xcd{local} annotation}\label{LocalAnnotation}\index{local!\xcd"local"}

A method may be declared \xcd"local".

\begin{grammar}
  MethodModifier \: \xcd"local"  
\end{grammar}

By declaring a method \xcd"local" the programmer asserts that while
executing this method an activity will only access local memory.

The compiler implements the following rules to guarantee this property.

Let \xcd"o" be any expression occurring in the body of the
method. Assume its static datatype is \xcd"F". 

\begin{itemize}
\item Local methods can only be overridden by local methods. 

\item If the body of the method contains any field access \xcd"o.e", then
the static placetype of \xcd"o" must be \xcd"here". 

The programmer can always ensure that this condition is satisfied
(albeit at the risk of introducing a runtime exception) by replacing
each field access \xcd"o.e" with \xcd"(o as F!here).e".

\item If the body of the method contains any assignments to fields
(e.g. \xcd"o.e Op= t", or \xcd"Op o.e" or \xcd"o.e Op") then the
static placetype of \xcd"o" must be \xcd"here".

The programmer can always ensure that this condition is satisfied by
replacing \xcd"o.e Op= t" by \xcd"o1.e Op=t" and preceding it (in the
same basic block) with the local variable declaration \xcd"o1: F!here = o as F!here" (for some new local variable \xcd"o1"). Similarly for
\xcd"Op o.e" and \xcd"o.e Op".

\item Recall that the static placetype of an array access \xcd"o(e)"
is \xcd"o.dist(e)". Therefore, any read/write array access
\xcd"o(e)" must be guarded by the condition \xcd"o.dist(e) == here".  (Since  \xcd"e" may have side-effects, the compiler must
ensure that the place check uses the value returned by the same
expression evaluation that is used to access the array element.)

\item If the body of the method contains any method invocation
\xcdmath"o.m(t$_1$,$\dots$,t$_k$)" then the method invoked must be local. Additionally,
the static place type of \xcd"o" must be \xcd"here". 
As above, the programmer can always ensure the second
condition is satisfied by writing such a method invocation
as \xcdmath"(o as F!here).m(t$_1$,$\dots$,t$_k$)".
\end{itemize}

Note that reads/writes to local variables or method parameters are
always local, hence the compiler does not have to check any extra
conditions.

A method declared \xcd"atomic" is automatically declared
to be \xcd"local".
	\par % 0.1
\chapter{Arrays}\label{XtenArrays}\index{arrays}

An array is a mapping from a region (set of points) to a range data
type distributed over one or more places.
Multiple arrays may be declared with the same underlying
distribution.
The distribution underlying an array \xcd"a" may be obtained through
the field \xcd"a.dist".
\index{arrays!distribution@{\tt distribution}}

\section{Points}\label{point-syntax}\index{point syntax}

Arrays are indexed by points--$n$-dimensional tuples of
integers, implemented by the class \xcd"x10.lang.Point".
\Xten{} specifies a simple syntax for the construction of points.
A rail constructor (\Sref{RailConstructors}) of type \xcd"ValRail[Int]"
%or
%\xcd"ValRail[Long]" array
of length $n$
can be implicitly coerced to a \xcd"Point" of rank $n$.  For
example, the following code initializes \xcd"p" to a point of
rank two using a rail constructor:

\begin{xten}
p: Point = [1,2];
\end{xten}

The \xcd"Point" constructor can take a rail constructor as
argument.  The assignment above can be written, without
implicit coercion, as:

\begin{xten}
p: Point = new Point([1,2]);
\end{xten}

Points implement the function type \xcd"(Int) => Int"; thus, the
\xcd"i"the element of a point \xcd"p" may be accessed as \xcd"p(i)".
If \xcd"i" is out of range, an
\xcd"ArrayIndexOutOfBoundsException" is thrown.

\input{XtenRegions}
\input{XtenDistributions}
\section{Array initializer}\label{ArrayInitializer}\label{array!creation}

Arrays are instantiated by invoking a factory method for the
class \xcd"Array".

\eat{
An array instantiation may be annotated
\xcd"unsafe"
if it is intended to be
allocated in an unmanaged region (e.g., for communication with native
code). A value array is an immutable array. An array creation
must take either an \xcd"Int" as an argument or a \xcd"Dist". In the first
case an array is created over the distribution \xcd"[0:N-1]->here";
in the second over the given distribution. 
}

An array creation operation may also specify an initializer
function.
The function is applied in parallel
at all points in the domain of the distribution. The array
construction operation terminates locally only when the array has been
fully created and initialized (at all places in the range of the
distribution).

For instance:
\begin{xten}
val data : Array[Int]
    = Array.make[Int](1000->here, Point(i) => i);
val data2 : Array[Int]
    = Array.make[Int]([1:1000,1:1000]->here, Point(i,j) => i*j);
\end{xten}

{}\noindent 
The first declaration stores in \xcd"data" a reference to a
array with \xcd"1000" elements each of which is located in the
same place as the array. Each array component is initialized to \xcd"i".

The second declaration stores in \xcd"data2" a
2-d array over \xcd"[1:1000, 1:1000]" initialized with \xcd"i*j"
at point \xcd"[i,j]". It uses a more abbreviated form to specify 
the array initializer function.

Other examples:
\begin{xten}
val data : Array[Int]
    = Array.make[Int](1000, ((i): Point) => i*i);
val d : Array[Float](D)
    = Array.make[Float](D, ((i): Point) => 10.0*i);
val result : Array[Float](D)
    = Array.make[Float](D, ((i,j): Point) => i+j);
\end{xten}

\section{Operations on arrays}
In the following let \xcd"a" be an array with distribution \xcd"D" and
base type \xcd"T". \xcd"a" may be mutable or immutable, unless
indicated otherwise.

\subsection{Element operations}\index{array!access}
The value of \xcd"a" at a point \xcd"p" in its region of definition is
obtained by using the indexing operation \xcd"a(p)". This operation
may be used on the left hand side of an assignment operation to update
the value. The operator assignments \xcd"a(i) op= e" are also available
in \Xten{}.

For array variables, the right-hand-side of an assignment must
have the same distribution \xcd"D" as an array being assigned. This
assignment involves
control communication between the sites hosting \xcd"D". Each
site performs the assignment(s) of array components locally. The
assignment terminates when assignment has terminated at all
sites hosting \xcd"D".

\subsection{Constant promotion}\label{ConstantArray}\index{arrays!constant promotion}

For a distribution \xcd"D" and a constant or final variable \xcd"v" of
type \xcd"T" the expression \xcd"Array.make[T](D, (p: Point) => v)"
denotes the mutable array with
distribution \xcd"D" and base type \xcd"T" initialized with \xcd"v"
at every point.

\subsection{Restriction of an array}\index{array!restriction}

Let \xcd"D1" be a sub-distribution of \xcd"D". Then \xcd"a | D1"
represents the sub-array of \xcd"a" with the distribution \xcd"D1".

Recall that a rich set of operators are available on distributions
(\Sref{XtenDistributions}) to obtain sub-distributions
(e.g. restricting to a sub-region, to a specific place etc).

\subsection{Assembling an array}
Let \xcd"a1,a2" be arrays of the same base type \xcd"T" defined over
distributions \xcd"D1" and \xcd"D2" respectively. Assume that both
arrays are value or reference arrays. 
\paragraph{Assembling arrays over disjoint regions}\index{array!union!disjoint}

If \xcd"D1" and \xcd"D2" are disjoint then the expression \xcd"a1 || a2" denotes the unique array of base type \xcd"T" defined over the
distribution \xcd"D1 || D2" such that its value at point \xcd"p" is
\xcd"a1(p)" if \xcd"p" lies in \xcd"D1" and \xcd"a2(p)"
otherwise. This array is a reference (value) array if \xcd"a1" is.

\paragraph{Overlaying an array on another}\index{array!union!asymmetric}
The expression
\xcd"a1.overlay(a2)" (read: the array \xcd"a1" {\em overlaid with} \xcd"a2")
represents an array whose underlying region is the union of that of
\xcd"a1" and \xcd"a2" and whose distribution maps each point \xcd"p"
in this region to \xcd"D2(p)" if that is defined and to \xcd"D1(p)"
otherwise. The value \xcd"a1.overlay(a2)(p)" is \xcd"a2(p)" if it is defined and \xcd"a1(p)" otherwise.

This array is a reference (value) array if \xcd"a1" is.

The expression \xcd"a1.update(a2)" updates the array \xcd"a1" in place
with the result of \xcd"a1.overlay(a2)".

\oldtodo{Define Flooding of arrays}

\oldtodo{Wrapping an array}

\oldtodo{Extending an array in a given direction.}

\subsection{Global operations }

\paragraph{Pointwise operations}\label{ArrayPointwise}\index{array!pointwise operations}
The unary \xcd"lift" operation applies a function to each element of
an array, returning a new array with the same distribution.
The \xcd"lift" operation is implemented by the following method
in \xcd"Array[T]":
\begin{xten}
def lift[S](f: (T) => S): Array[S](dist);
\end{xten}

The binary \xcd"lift" operation takes a binary function and
another
array over the same distribution and applies the function
pointwise to corresponding elements of the two arrays, returning
a new array with the same distribution.
The \xcd"lift" operation is implemented by the following method
in \xcd"Array[T]":
\begin{xten}
def lift[S,R](f: (T,S) => R, Array[S](dist)): Array[R](dist);
\end{xten}

\paragraph{Reductions}\label{ArrayReductions}\index{array!reductions}

Let \xcd"f" be a function of type \xcd"(T,T)=>T".  Let
\xcd"a" be a value or reference array over base type \xcd"T".
Let \xcd"unit" be a value of type \xcd"T".
Then the
operation \xcd"a.reduce(f, unit)" returns a value of type \xcd"T" obtained
by performing \xcd"f" on all points in \xcd"a" in some order, and in
parallel.  The function \xcd"f" must be associative and
commutative.  The value \xcd"unit" should satisfy
\xcd"f(unit,x)" \xcd"==" \xcd"x" \xcd"==" \xcd"f(x,unit)".

This operation involves communication between the places over which
the array is distributed. The \Xten{} implementation guarantees that
only one value of type \xcd"T" is communicated from a place as part of
this reduction process.

\paragraph{Scans}\label{ArrayScans}\index{array!scans}

Let \xcd"f" be a reduction operator defined on type \xcd"T". Let
\xcd"a" be a value or reference array over base type \xcd"T" and
distribution \xcd"D". Then the operation \xcd"a||f()" returns an array
of base type \xcd"T" and distribution \xcd"D" whose $i$th element
(in canonical order) is obtained by performing the reduction \xcd"f"
on the first $i$ elements of \xcd"a" (in canonical order).

This operation involves communication between the places over which
the array is distributed. The \Xten{} implementation will endeavour to
minimize the communication between places to implement this operation.

Other operations on arrays may be found in \xcd"x10.lang.Array" and
other related classes.
	\par % 0.1
\chapter{Statements and Expressions}\label{XtenStatements}\index{statements}

\Xten{} inherits all the standard statements of \Java{}, with the expected semantics:

\begin{x10}
\em\tt EmptyStatement      LabeledStatement  
\em\tt ExpressionStatement IfStatement
\em\tt SwitchStatement     WhileDo
\em\tt DoWhile             ForLoop           
\em\tt BreakStatement      ContinueStatement  
\em\tt ReturnStatement   ThrowStatement
\em\tt TryStatement
\end{x10}

We focus on the new statements in \Xten. 

\section{Assignment}\index{assignment}\label{AssignmentStatement}

%It is often the case that an \Xten{} variable is assigned to only
%once. The user may declare such variables as {\cf final}. However,
%this is sometimes syntactically cumbersome.
%
%{}\Xten{} supports the syntax {\cf l := r} for assignment to mutable
%variables.  The user is strongly enouraged to use this syntax to
%assign variables that are intended to be assigned to more than
%once. The \Xten{} compiler may issue a warning if it detects code 
%that uses {\cf =} assignment statements on {\cf mutable} variables.

{}\Xten{} supports assignment {\tt l = r} to array variables. In this
case {\tt r} must have the same distribution {\tt D} as {\tt l}. This
statement involves control communication between the sites hosting
{\tt D}. Each site performs the assignment(s) of array components
locally. The assignment terminates when assignment has terminated at
all sites hosting {\tt D}.

%% TODO: Sectional assignment??

\section{Point and region construction}\label{point-syntax}\index{[] syntax}
\Xten{} specifies a simple syntax for the construction of points and regions.
\begin{x10}
281   ArgumentList ::= Expression
282      | ArgumentList , Expression
512   Primary ::= [ ArgumentList ]
\end{x10}
Each element in the argument list must be either of type {\tt int} or 
of type {\tt region}. In the former case the expression 
{\tt [ a1,..., ak ] } is treated as syntactic shorthand for
\begin{x10}
  point.factory.point(a1,..., ak)
\end{x10}
\noindent and in the latter case as shorthand for
\begin{x10}
  region.factory.region(a1,..., ak)
\end{x10}

\section{Exploded variable declarations}\label{exploded-syntax}\index{variable declarator!exploded}

\Xten{} permits a richer form of specification for variable
declarators in method arguments, local variables and loop variables
(the ``exploded'' or {\em destructuring} syntax).
\begin{x10}
81    VariableDeclaratorId ::= 
           identifier [ IdentifierList ]
82       | [ IdentifierList ]
\end{x10}
In \XtenCurrVer{} the {\tt VariableDeclaratorId} must be declared at
type {\tt x10.lang.point}. Intuitively, this syntax allows a
point to be ``destructured'' into its corresponding {\tt int} 
indices in a pattern-matching style.
The $k$th identifier in the {\tt
IdentifierList} is treated as a {\tt final} variable of type {\tt int}
that is initialized with the value of the $k$th index of the point. 
The second form of the syntax (Rule 82) permits the specification of only
the index variables.

Future versions of the language may allow destructuring syntax for all
value classes.

\paragraph{Example.}
The following example succeeds when executed.
\begin{x10}
public class Array1Exploded \{
  public int select(point p[i,j], point [k,l]) \{
      return i+k;
  \}
  public boolean run() \{
    distribution d =  [1:10, 1:10] -> here;
    int[.] ia = new int[d];
    for(point p[i,j]: [1:10,1:10]) \{
        if(ia[p]!=0) return false;
        ia[p] = i+j;
    \}
    for(point p[i,j]: d) \{
      point q1 = [i,j];
      if (i != q1[0]) return false;
      if ( j != q1[1]) return false;
      if(ia[i,j]!= i+j) return false;
      if(ia[i,j]!=ia[p]) return false;
      if(ia[q1]!=ia[p]) return false;
   \}
    if (! (4 == select([1,2],[3,4]))) return false;
     return true;
   \}
        
  public static void main(String args[]) \{
     boolean b= (new Array1Exploded()).run();
     System.out.println("++++++ "
                        + (b? "Test succeeded."
                           :"Test failed."));
     System.exit(b?0:1);
 \}
\}
\end{x10}

\input{XtenExpressions} \par  % empty



	\par  % 0.05
\chapter{Annotations and Compiler
Plugins}\label{XtenAnnotations}\index{annotations}


X10 provides an 
an annotation system and compiler plugin system for to allow the
compiler to be extended with new static analyses and new
transformations.

Annotations are interface types that decorate the abstract syntax tree
of an X10 program.  The X10 type-checker ensures that an annotation
is a legal interface type.
In X10, interfaces may declare
both methods and properties.  Therefore, like any interface type, an
annotation may instantiate
one or more of its interface's properties.
Unlike with Java
annotations,
property initializers need not be
compile-time constants;
however, a given compiler plugin
may do additional checks to constrain the allowable
initializer expressions.
The X10 type-checker does not check that
all properties of an annotation are initialized,
although this could be enforced by
a compiler plugin.

\section{Annotation syntax}

The annotation syntax consists of an ``\texttt{@}'' followed by an interface type.
\begin{x10}
532   Annotation ::= @ InterfaceType
533   Annotations ::= Annotation
534     | Annotations Annotation
535   Annotationsopt ::=
536     | Annotations
\end{x10}
Annotations can be applied to most syntactic constructs in the language
including class declarations, constructors, methods, field declarations,
local variable declarations and formal parameters, statements,
expressions, and types.
Multiple occurrences of the same annotation (i.e., multiple
annotations with the same interface type) on the same entity are permitted.

\begin{x10}
537   ClassModifier ::= Annotation
538   InterfaceModifier ::= Annotation
539   FieldModifier ::= Annotation
540   MethodModifier ::= Annotation
541   VariableModifier ::= Annotation
542   ConstructorModifier ::= Annotation
543   AbstractMethodModifier ::= Annotation
544   ConstantModifier ::= Annotation
545   Type ::= AnnotatedType
546   AnnotatedType ::= Type Annotations
547   Statement ::= AnnotatedStatement
548   AnnotatedStatement ::= Annotation Statement
549   Expression ::= AnnotatedExpression
550   AnnotatedExpression ::= ( Annotations ) Expression
\end{x10}
\noindent
Recall that interface types may have dependent parameters.

\noindent
The following examples illustrate the syntax:

\begin{itemize}
\item Declaration annotations:
\begin{x10}
  // class annotation
  @Value
  class Cons \{ ... \}

  // method annotation
  @PreCondition(0 <= i \&\& i < this.size)
  public Object get(int i) \{ ... \}

  // constructor annotation
  @Where(x != null)
  C(T x) \{ ... \}

  // constructor return type annotation
  C@Initialized(T x) \{ ... \}

  // variable annotation
  @Unique A x;
\end{x10}
\item Type annotations:
\begin{x10}
  List@Nonempty

  int@Range(1,4)

  double[][]@Size(n * n)
\end{x10}
\item Expression annotations:
\begin{x10}
  (@RemoteCall) m()

  3 == (@Bits(2)) 15
\end{x10}
\item Statement annotations:
\begin{x10}
  @Atomic \{ ... \}

  @MinIterations(1)
  @MaxIterations(n)
  for (int i = 0; i < n; i++) \{ ... \}

  // An annotated empty statement ;
  @Assert(x < y);
\end{x10}
\end{itemize}

\section{Annotation declarations}

Annotations are declared as interfaces.  They must be
subtypes of \texttt{x10.lang.annotation.Annotation}.
Annotations on types, expressions, statements, classes, fields,
methods, constructors, and local variable declarations (or
formal parameters)
must extend
\texttt{ExpressionAnnotation},
\texttt{StatementAnnotation},
\texttt{ClassAnnotation},
\texttt{FieldAnnotation},
\texttt{MethodAnnotation},
\texttt{ConstructorAnnotation}, and
\texttt{VariableAnnotation}, respectively.

\section{Compiler plugins}
\index{plugins}

After the base X10 semantic checking is completed, 
compiler plugins are loaded and run.  Plugins may perform
any number of compiler passes to implement
additional semantic checking and code transformations, including
transformations using the abstract syntax of the annotations
themselves.  Plugins should output valid X10 abstract
syntax trees.

Plugins are implemented in java as
Polyglot~\cite{ncm03} passes applied to the AST
after normal base X10 type checking.
Plugins to run are specified on the command-line.  The order of
execution is determined by the Polyglot pass scheduler.
\index{Polyglot}

To run compiler plugins, add the command-line option:
\begin{x10}
  -PLUGINS=P1,P2,...,Pn
\end{x10}
where \texttt{P1}, \texttt{P2}, \dots, \texttt{Pn} are classes that implement the
\texttt{CompilerPlugin} interface:
\index{CompilerPlugin}

\begin{x10}
  package polyglot.ext.x10.plugin;

  import polyglot.ext.x10.ExtensionInfo;
  import polyglot.frontend.Job;
  import polyglot.frontend.goals.Goal;

  public interface CompilerPlugin \{
      public Goal register(ExtensionInfo extInfo, Job job);
  \}
\end{x10}

\index{Goal}
The \texttt{Goal} object returned by the \texttt{register} method specifies dependencies on other passes.
Documentation for Polyglot can be found at:
\begin{x10}
  http://www.cs.cornell.edu/Projects/polyglot
\end{x10}
Most plugins should implement either \texttt{SimpleOnePassPlugin} or
\texttt{SimpleVisitorPlugin}.

The compiler loads plugin classes from the x10c classpath.

Plugins are given access to a Polyglot AST and type system.  Annotations are
represented in the AST as \texttt{Node}s with the following interface:
\index{Node}

\begin{x10}
  package polyglot.ext.x10.ast;

  public interface AnnotationNode extends Node \{
    X10ClassType annotation();
  \}
\end{x10}

Annotations for a \texttt{Node} object \texttt{n} can be accessed through the
node's extension object as follows:
\index{AnnotationNode}

\begin{x10}
  List<AnnotationNode> annotations =
    ((X10Ext) n.ext()).annotations();
  List<X10ClassType> annotationTypes =
    ((X10Ext) n.ext()).annotationInterfaces();
\end{x10}
In the type system, \texttt{X10TypeObject} has the following
method for accessing annotations:
\begin{x10}
  List<X10ClassType> annotations();
\end{x10}


%\balance
\bibliography{pm.bib,db.bib}

% \clearpage

\end{document}


	\par  % 0.05
\section{Linking with native code}\label{extern}\index{extern}
\XtenCurrVer{} supports a simple facility to permit the efficient
intra-thread communication of an array of primitive type to code
written in the language {\tt C}.  The array must be a ``local''
array. The primary intent of this design is to permit the reuse of
native code that efficiently implements some numeric array/matrix
calculation.

Future language releases are expected to support similar bindings to
{\sc Fortran}, and to support parallel native processing of
distributed \Xten{} arrays. 

The interface consists of two parts. First, an array intended to be
communicated to native code must be created as an {\tt unsafe} array:
\begin{x10}
450 ArrayCreationExpression ::= 
      new ArrayBaseType Unsafeopt [ ] 
        ArrayInitializer
451   | new ArrayBaseType Unsafeopt [ Expression ]
452   | new ArrayBaseType Unsafeopt 
          [ Expression ] Expression
453   | new ArrayBaseType Unsafeopt [ Expression ] 
          ( FormalParameter ) MethodBody
454   | new ArrayBaseType value 
           Unsafeopt [ Expression ]
455   | new ArrayBaseType value 
           Unsafeopt [ Expression ] Expression
456   | new ArrayBaseType value 
        Unsafeopt [ Expression ] 
          ( FormalParameter ) MethodBody
530   Unsafeopt ::=
531     | unsafe
\end{x10}
Unsafe arrays can be of any dimension. However, \XtenCurrVer{}
requires that unsafe arrays be of a primitive type, and local (i.e.{}
with an underlying distribution that maps all elements in its region
to {\tt here}).

Unsafe arrays are allocated in a special array of memory that permits
their efficient transmission to natively linked code.
%% Comment about when this memory is freed.

Second, the \Xten{} programmer may specify that certain methods are to
be implemented natively by using the keyword {\tt extern}:
\begin{x10}
446   MethodModifier ::= extern
\end{x10}
Such a method must have the statement ``{\tt ;}'' as its body.
\XtenCurrVer{} requires that the method be {\tt static}; this
restriction is likely to be lifted in the future.  Primitive types in
the method argument are translated to their corresponding JNI type
(e.g.{} {\tt float} is translated to {\tt jfloat}, {\tt double} to
{\tt jdouble} etc).  The only non-primitive type permitted in an {\tt
extern} method is an (unsafe) array. This is passed at type {\tt
jlong} as an eight byte address into the unsafe region which contains
the data for the array. ({\tt jlong} is not the same as {\tt long} on
32-bit machines.)


Since only the starting address of an array is passed, if the array is
multidimensional, the user must explicitly communicate (or have a
guarantee of) the rank of the passed array, and must either typecast
or explicitly code the address calculation.  Note that all \Xten{}
arrays are created in row-major order, and so any native routine must
also access them in the same order.

For each class {\tt C} that contains an {\tt extern} method, the
\Xten{} compiler generates a text file {\tt C\_x10stub.c}.  This file
contains generated {\tt C} stub functions which are called from the
{\tt extern} routines.  The name of the stub function is derived from
the name of the {\tt extern} method. If the method is {\tt
C.process()}, the stub function will be {\tt
Java\_C\_C\_process()}. The name is suffixed with the signature of the
method if the method is overloaded.

The programmer must write {\tt C} code to implement the native method,
using the methods in the {\tt C} stub file to call the actual native
method.  The programmer must compile these files and link them into a
dynamically linked library (DLL).  Note that the {\tt jni.h} header file
must be in the include path.  The programmer must ensure this library
is loaded by the program before the method is called e.g.{} add a {\tt
System.loadlibrary} call (in a static initializer of the
\Xten{} class).

\paragraph{Example.}
The following class illustrates the use of {\tt unsafe} and native
linking. 
\begin{x10}
public class IntArrayExternUnsafe \{
  public static extern 
      void process(int [.] yy, int size);
  static {System.loadLibrary("IntArrayExternUnsafe");}
  public static void main(String args[]) \{
     boolean b= (new IntArrayExternUnsafe()).run();
     System.out.println("++++++ Test "
                         +(b?"succeeded.":"failed."));
     System.exit(b?0:1);
  \}
  public boolean run()\{
    int high = 10;
    boolean verified=false;
    distribution d= (0:high) -> here;
    int [.] y = new int unsafe[d]; 
    for( int j=0;j < 10;++j)
        y[j] = j;
    process(y,high);
    for(int j=0;j < 10;++j)\{
      int expected = j+100;
      if(y[j] != expected)\{
        System.out.println("y["+j+"]="
                           +y[j]+" != "+expected);
        return false;
       \}
    \}
    return true;
  \}
\}
\end{x10}

The programmer may then write the {\tt C} code thus:
\begin{x10}
void IntArrayExternUnsafe\_process(jlong yy, 
                                signed int size)\{
  int i;
  int* array = (int *)(long)yy;
  for(i = 0;i < size;++i)\{
    array[i] += 100;
  \}
\}
/* automatically generated in \_x10stub.c*/
void 
 Java\_IntArrayExternUnsafe\_IntArrayExternUnsafe\_process
 (JNIEnv *env,  jobject obj,jlong yy,jint size)\{
   IntArrayExternUnsafe\_process(yy,size);
\}
\end{x10}

This code may be linked with the stub file (or textually placed in
it). The programmer must then compile and link the {\tt C} code and
ensure that the DLL is on the appropriate classpath. 


%\chapter{Performance Model}\label{PerformanceModel}
 \par \vfill\eject % empty
\extrapart{Example}

This example illustrates 2-d Jacobi iteration.

\begin{xten}
public class Jacobi {
   const N: int = 6;
   const epsilon: double = 0.002;
   const epsilon2: double = 0.000000001;
   const R: region = [0:N+1, 0:N+1];
   const RInner: region = [1:N, 1:N];
   const D: distribution = distribution.factory.block(R);
   const DInner: distribution = D | RInner;
   const DBoundary: distribution = D - RInner;
   const EXPECTED_ITERS: int  = 97;
   const double EXPECTED_ERR: double = 0.0018673382039402497;
     
   val B: Array[double](D) = Array.make[double](D,
        (p(i,j): point) => DBoundary.contains(p) ? (N-1)/2 : N*(i-1)+(j-1));
    
   public def run(): boolean {
      var iters: int = 0;
      var err: double;
      while (true) {
        val Temp: Array[double] = 
           new array[double](DInner, ((i,j): point) =>
             (read(i+1,j)+read(i-1,j) +read(i,j+1)+read(i,j-1))/4.0);
        if((err=((B | DInner) - Temp).abs().sum()) < epsilon)
           break; 
        B.update(Temp);
        iters++; 
      }
      Console.OUT.println("Error="+err);
      Console.OUT.println("Iterations="+iters);
      return Math.abs(err-EXPECTED_ERR)<epsilon2 
          && iters==EXPECTED_ITERS;
   }
   public def read(i: int, j: int): double {
      return (future(D(i,j)) => B(i,j)).force();
   }
   public static def main(args: Array[String]) {
      val b = new Jacobi().run();
      Console.OUT.println("++++++ "
                          + (b? "Test succeeded."
                             :"Test failed."));
      System.exit(b?0:1);
   }
}
\end{xten}
	\par  \vfill\eject % have an example
\notinfouro{\onecolumn
\extrapart{\Xten{} syntax}\label{X10 syntax}\index{X10 productions}

This section contains the complete grammar for \Xten{}. This includes
all the new constructs in \Xten{} discussed in the main body of this
reference manual, as well as constructs obtained from \java{} which
behave essentially identically to the corresponding {\tt java} constructs.

Note that in this version of the grammar productions for the same
non-terminal may occur non-contiguously. For instance 
{\tt MethodModifier} is defined on lines {\tt 111--119} and
{\tt 445-446}. This will be corrected in future versions of the grammar.

{\footnotesize
\begin{verbatim}
0     $accept ::= CompilationUnit
1     identifier ::= IDENTIFIER
2     PrimitiveType ::= NumericType
3      | boolean
4     NumericType ::= IntegralType
5      | FloatingPointType
6     IntegralType ::= byte
7      | char
8      | short
9      | int
10     | long
11    FloatingPointType ::= float
12     | double
13    ClassType ::= TypeName
14    InterfaceType ::= TypeName
15    TypeName ::= identifier
16     | TypeName . identifier
17    ClassName ::= TypeName
18    ArrayType ::= Type [ ]
19    PackageName ::= identifier
20      | PackageName . identifier
21    ExpressionName ::= identifier
22      | here
23      | AmbiguousName . identifier
24    MethodName ::= identifier
25      | AmbiguousName . identifier
26    PackageOrTypeName ::= identifier
27      | PackageOrTypeName . identifier
28    AmbiguousName ::= identifier
29      | AmbiguousName . identifier
30    CompilationUnit ::= PackageDeclarationopt ImportDeclarationsopt TypeDeclarationsopt
31    ImportDeclarations ::= ImportDeclaration
32      | ImportDeclarations ImportDeclaration
33    TypeDeclarations ::= TypeDeclaration
34      | TypeDeclarations TypeDeclaration
35    PackageDeclaration ::= package PackageName ;
36    ImportDeclaration ::= SingleTypeImportDeclaration
37      | TypeImportOnDemandDeclaration
38      | SingleStaticImportDeclaration
39      | StaticImportOnDemandDeclaration
40    SingleTypeImportDeclaration ::= import TypeName ;
41    TypeImportOnDemandDeclaration ::= import PackageOrTypeName . * ;
42    SingleStaticImportDeclaration ::= import static TypeName . identifier ;
43    StaticImportOnDemandDeclaration ::= import static TypeName . * ;
44    TypeDeclaration ::= ClassDeclaration
45      | InterfaceDeclaration
46      | ;
47    ClassDeclaration ::= NormalClassDeclaration
48    NormalClassDeclaration ::= ClassModifiersopt class identifier Superopt Interfacesopt ClassBody
49    ClassModifiers ::= ClassModifier
50      | ClassModifiers ClassModifier
51    ClassModifier ::= public
52      | protected
53      | private
54      | abstract
55      | static
56      | final
57      | strictfp
58    Super ::= extends ClassType
59    Interfaces ::= implements InterfaceTypeList
60    InterfaceTypeList ::= InterfaceType
61      | InterfaceTypeList , InterfaceType
62    ClassBody ::= { ClassBodyDeclarationsopt }
63    ClassBodyDeclarations ::= ClassBodyDeclaration
64      | ClassBodyDeclarations ClassBodyDeclaration
65    ClassBodyDeclaration ::= ClassMemberDeclaration
66      | InstanceInitializer
67      | StaticInitializer
68      | ConstructorDeclaration
69    ClassMemberDeclaration ::= FieldDeclaration
70      | MethodDeclaration
71      | ClassDeclaration
72      | InterfaceDeclaration
73      | ;
74    FieldDeclaration ::= FieldModifiersopt Type VariableDeclarators ;
75    VariableDeclarators ::= VariableDeclarator
76      | VariableDeclarators , VariableDeclarator
77    VariableDeclarator ::= VariableDeclaratorId
78      | VariableDeclaratorId = VariableInitializer
79    VariableDeclaratorId ::= identifier
80      | VariableDeclaratorId [ ]
81      | identifier [ IdentifierList ]
82      | [ IdentifierList ]
83    VariableInitializer ::= Expression
84      | ArrayInitializer
85    FieldModifiers ::= FieldModifier
86      | FieldModifiers FieldModifier
87    FieldModifier ::= public
88      | protected
89      | private
90      | static
91      | final
92      | transient
93      | volatile
94    MethodDeclaration ::= MethodHeader MethodBody
95    MethodHeader ::= MethodModifiersopt ResultType MethodDeclarator Throwsopt
96    ResultType ::= Type
97      | void
98    MethodDeclarator ::= identifier ( FormalParameterListopt )
99      | MethodDeclarator [ ]
100   FormalParameterList ::= LastFormalParameter
101     | FormalParameters , LastFormalParameter
102   FormalParameters ::= FormalParameter
103     | FormalParameters , FormalParameter
104   FormalParameter ::= VariableModifiersopt Type VariableDeclaratorId
105   VariableModifiers ::= VariableModifier
106     | VariableModifiers VariableModifier
107   VariableModifier ::= final
108   LastFormalParameter ::= VariableModifiersopt Type ...opt VariableDeclaratorId
109   MethodModifiers ::= MethodModifier
110     | MethodModifiers MethodModifier
111   MethodModifier ::= public
112     | protected
113     | private
114     | abstract
115     | static
116     | final
117     | synchronized
118     | native
119     | strictfp
120   Throws ::= throws ExceptionTypeList
121   ExceptionTypeList ::= ExceptionType
122     | ExceptionTypeList , ExceptionType
123   ExceptionType ::= ClassType
124   MethodBody ::= Block
125     | ;
126   InstanceInitializer ::= Block
127   StaticInitializer ::= static Block
128   ConstructorDeclaration ::= ConstructorModifiersopt ConstructorDeclarator Throwsopt ConstructorBody
129   ConstructorDeclarator ::= SimpleTypeName ( FormalParameterListopt )
130   SimpleTypeName ::= identifier
131   ConstructorModifiers ::= ConstructorModifier
132     | ConstructorModifiers ConstructorModifier
133   ConstructorModifier ::= public
134     | protected
135     | private
136   ConstructorBody ::= { ExplicitConstructorInvocationopt BlockStatementsopt }
137   ExplicitConstructorInvocation ::= this ( ArgumentListopt ) ;
138     | super ( ArgumentListopt ) ;
139     | Primary . this ( ArgumentListopt ) ;
140     | Primary . super ( ArgumentListopt ) ;
141   Arguments ::= ( ArgumentListopt )
142   InterfaceDeclaration ::= NormalInterfaceDeclaration
143   NormalInterfaceDeclaration ::= InterfaceModifiersopt interface identifier ExtendsInterfacesopt InterfaceBody
144   InterfaceModifiers ::= InterfaceModifier
145     | InterfaceModifiers InterfaceModifier
146   InterfaceModifier ::= public
147     | protected
148     | private
149     | abstract
150     | static
151     | strictfp
152   ExtendsInterfaces ::= extends InterfaceType
153     | ExtendsInterfaces , InterfaceType
154   InterfaceBody ::= { InterfaceMemberDeclarationsopt }
155   InterfaceMemberDeclarations ::= InterfaceMemberDeclaration
156     | InterfaceMemberDeclarations InterfaceMemberDeclaration
157   InterfaceMemberDeclaration ::= ConstantDeclaration
158     | AbstractMethodDeclaration
159     | ClassDeclaration
160     | InterfaceDeclaration
161     | ;
162   ConstantDeclaration ::= ConstantModifiersopt Type VariableDeclarators
163   ConstantModifiers ::= ConstantModifier
164     | ConstantModifiers ConstantModifier
165   ConstantModifier ::= public
166     | static
167     | final
168   AbstractMethodDeclaration ::= AbstractMethodModifiersopt ResultType MethodDeclarator Throwsopt ;
169   AbstractMethodModifiers ::= AbstractMethodModifier
170     | AbstractMethodModifiers AbstractMethodModifier
171   AbstractMethodModifier ::= public
172     | abstract
173   ArrayInitializer ::= { VariableInitializersopt ,opt }
174   VariableInitializers ::= VariableInitializer
175     | VariableInitializers , VariableInitializer
176   Block ::= { BlockStatementsopt }
177   BlockStatements ::= BlockStatement
178     | BlockStatements BlockStatement
179   BlockStatement ::= LocalVariableDeclarationStatement
180     | ClassDeclaration
181     | Statement
182   LocalVariableDeclarationStatement ::= LocalVariableDeclaration ;
183   LocalVariableDeclaration ::= VariableModifiersopt Type VariableDeclarators
184   Statement ::= StatementWithoutTrailingSubstatement
185     | LabeledStatement
186     | IfThenStatement
187     | IfThenElseStatement
188     | WhileStatement
189     | ForStatement
190   StatementWithoutTrailingSubstatement ::= Block
191     | EmptyStatement
192     | ExpressionStatement
193     | AssertStatement
194     | SwitchStatement
195     | DoStatement
196     | BreakStatement
197     | ContinueStatement
198     | ReturnStatement
199     | SynchronizedStatement
200     | ThrowStatement
201     | TryStatement
202   StatementNoShortIf ::= StatementWithoutTrailingSubstatement
203     | LabeledStatementNoShortIf
204     | IfThenElseStatementNoShortIf
205     | WhileStatementNoShortIf
206     | ForStatementNoShortIf
207   IfThenStatement ::= if ( Expression ) Statement
208   IfThenElseStatement ::= if ( Expression ) StatementNoShortIf else Statement
209   IfThenElseStatementNoShortIf ::= if ( Expression ) StatementNoShortIf else StatementNoShortIf
210   EmptyStatement ::= ;
211   LabeledStatement ::= identifier : Statement
212   LabeledStatementNoShortIf ::= identifier : StatementNoShortIf
213   ExpressionStatement ::= StatementExpression ;
214   StatementExpression ::= Assignment
215     | PreIncrementExpression
216     | PreDecrementExpression
217     | PostIncrementExpression
218     | PostDecrementExpression
219     | MethodInvocation
220     | ClassInstanceCreationExpression
221   AssertStatement ::= assert Expression ;
222     | assert Expression : Expression ;
223   SwitchStatement ::= switch ( Expression ) SwitchBlock
224   SwitchBlock ::= { SwitchBlockStatementGroupsopt SwitchLabelsopt }
225   SwitchBlockStatementGroups ::= SwitchBlockStatementGroup
226     | SwitchBlockStatementGroups SwitchBlockStatementGroup
227   SwitchBlockStatementGroup ::= SwitchLabels BlockStatements
228   SwitchLabels ::= SwitchLabel
229     | SwitchLabels SwitchLabel
230   SwitchLabel ::= case ConstantExpression :
231     | default :
232   WhileStatement ::= while ( Expression ) Statement
233   WhileStatementNoShortIf ::= while ( Expression ) StatementNoShortIf
234   DoStatement ::= do Statement while ( Expression ) ;
235   ForStatement ::= BasicForStatement
236     | EnhancedForStatement
237   BasicForStatement ::= for ( ForInitopt ; Expressionopt ; ForUpdateopt ) Statement
238   ForStatementNoShortIf ::= for ( ForInitopt ; Expressionopt ; ForUpdateopt ) StatementNoShortIf
239     | EnhancedForStatementNoShortIf
240   ForInit ::= StatementExpressionList
241     | LocalVariableDeclaration
242   ForUpdate ::= StatementExpressionList
243   StatementExpressionList ::= StatementExpression
244     | StatementExpressionList , StatementExpression
245   BreakStatement ::= break identifieropt ;
246   ContinueStatement ::= continue identifieropt ;
247   ReturnStatement ::= return Expressionopt ;
248   ThrowStatement ::= throw Expression ;
249   SynchronizedStatement ::= synchronized ( Expression ) Block
250   TryStatement ::= try Block Catches
251     | try Block Catchesopt Finally
252   Catches ::= CatchClause
253     | Catches CatchClause
254   CatchClause ::= catch ( FormalParameter ) Block
255   Finally ::= finally Block
256   Primary ::= PrimaryNoNewArray
257     | ArrayCreationExpression
258   PrimaryNoNewArray ::= Literal
259     | Type . class
260     | void . class
261     | this
262     | ClassName . this
263     | ( Expression )
264     | ClassInstanceCreationExpression
265     | FieldAccess
266     | MethodInvocation
267     | ArrayAccess
268   Literal ::= IntegerLiteral
269     | LongLiteral
270     | FloatingPointLiteral
271     | DoubleLiteral
272     | BooleanLiteral
273     | CharacterLiteral
274     | StringLiteral
275     | null
276   BooleanLiteral ::= true
277     | false
278   ClassInstanceCreationExpression ::= new ClassOrInterfaceType ( ArgumentListopt ) ClassBodyopt
279     | Primary . new identifier ( ArgumentListopt ) ClassBodyopt
280     | AmbiguousName . new identifier ( ArgumentListopt ) ClassBodyopt
281   ArgumentList ::= Expression
282     | ArgumentList , Expression
283   FieldAccess ::= Primary . identifier
284     | super . identifier
285     | ClassName . super . identifier
286   MethodInvocation ::= MethodName ( ArgumentListopt )
287     | Primary . identifier ( ArgumentListopt )
288     | super . identifier ( ArgumentListopt )
289     | ClassName . super . identifier ( ArgumentListopt )
290   PostfixExpression ::= Primary
291     | ExpressionName
292     | PostIncrementExpression
293     | PostDecrementExpression
294   PostIncrementExpression ::= PostfixExpression ++
295   PostDecrementExpression ::= PostfixExpression --
296   UnaryExpression ::= PreIncrementExpression
297     | PreDecrementExpression
298     | + UnaryExpression
299     | - UnaryExpression
300     | UnaryExpressionNotPlusMinus
301   PreIncrementExpression ::= ++ UnaryExpression
302   PreDecrementExpression ::= -- UnaryExpression
303   UnaryExpressionNotPlusMinus ::= PostfixExpression
304     | ~ UnaryExpression
305     | ! UnaryExpression
306     | CastExpression
307   MultiplicativeExpression ::= UnaryExpression
308     | MultiplicativeExpression * UnaryExpression
309     | MultiplicativeExpression / UnaryExpression
310     | MultiplicativeExpression % UnaryExpression
311   AdditiveExpression ::= MultiplicativeExpression
312     | AdditiveExpression + MultiplicativeExpression
313     | AdditiveExpression - MultiplicativeExpression
314   ShiftExpression ::= AdditiveExpression
315     | ShiftExpression << AdditiveExpression
316     | ShiftExpression >> AdditiveExpression
317     | ShiftExpression >>> AdditiveExpression
318   RelationalExpression ::= ShiftExpression
319     | RelationalExpression < ShiftExpression
320     | RelationalExpression GREATER ShiftExpression
321     | RelationalExpression <_= ShiftExpression
322     | RelationalExpression GREATER = ShiftExpression
323   EqualityExpression ::= RelationalExpression
324     | EqualityExpression == RelationalExpression
325     | EqualityExpression != RelationalExpression
326   AndExpression ::= EqualityExpression
327     | AndExpression AND EqualityExpression
328   ExclusiveOrExpression ::= AndExpression
329     | ExclusiveOrExpression XOR AndExpression
330   InclusiveOrExpression ::= ExclusiveOrExpression
331     | InclusiveOrExpression OR ExclusiveOrExpression
332   ConditionalAndExpression ::= InclusiveOrExpression
333     | ConditionalAndExpression AND_AND InclusiveOrExpression
334   ConditionalOrExpression ::= ConditionalAndExpression
335     | ConditionalOrExpression OR_OR ConditionalAndExpression
336   ConditionalExpression ::= ConditionalOrExpression
337     | ConditionalOrExpression QUESTION Expression : ConditionalExpression
338   AssignmentExpression ::= ConditionalExpression
339     | Assignment
340   Assignment ::= LeftHandSide AssignmentOperator AssignmentExpression
341   LeftHandSide ::= ExpressionName
342     | FieldAccess
343     | ArrayAccess
344   AssignmentOperator ::= =
345     | *=
346     | /=
347     | %=
348     | +=
349     | -=
350     | <<=
351     | >>=
352     | >>>=
353     | &=
354     | ^=
355     | |=
356   Expression ::= AssignmentExpression
357   ConstantExpression ::= Expression
358   Catchesopt ::=
359     | Catches
360   identifieropt ::=
361     | identifier
362   ForUpdateopt ::=
363     | ForUpdate
364   Expressionopt ::=
365     | Expression
366   ForInitopt ::=
367     | ForInit
368   SwitchLabelsopt ::=
369     | SwitchLabels
370   SwitchBlockStatementGroupsopt ::=
371     | SwitchBlockStatementGroups
372   VariableModifiersopt ::=
373     | VariableModifiers
374   VariableInitializersopt ::=
375     | VariableInitializers
376   AbstractMethodModifiersopt ::=
377     | AbstractMethodModifiers
378   ConstantModifiersopt ::=
379     | ConstantModifiers
380   InterfaceMemberDeclarationsopt ::=
381     | InterfaceMemberDeclarations
382   ExtendsInterfacesopt ::=
383     | ExtendsInterfaces
384   InterfaceModifiersopt ::=
385     | InterfaceModifiers
386   ClassBodyopt ::=
387     | ClassBody
388   ,opt ::=
389     | ,
390   ArgumentListopt ::=
391     | ArgumentList
392   BlockStatementsopt ::=
393     | BlockStatements
394   ExplicitConstructorInvocationopt ::=
395     | ExplicitConstructorInvocation
396   ConstructorModifiersopt ::=
397     | ConstructorModifiers
398   ...opt ::=
399     | ...
400   FormalParameterListopt ::=
401     | FormalParameterList
402   Throwsopt ::=
403     | Throws
404   MethodModifiersopt ::=
405     | MethodModifiers
406   FieldModifiersopt ::=
407     | FieldModifiers
408   ClassBodyDeclarationsopt ::=
409     | ClassBodyDeclarations
410   Interfacesopt ::=
411     | Interfaces
412   Superopt ::=
413     | Super
414   ClassModifiersopt ::=
415     | ClassModifiers
416   TypeDeclarationsopt ::=
417     | TypeDeclarations
418   ImportDeclarationsopt ::=
419     | ImportDeclarations
420   PackageDeclarationopt ::=
421     | PackageDeclaration
422   Type ::= DataType PlaceTypeSpecifieropt
423     | nullable Type
424     | future < Type GREATER
425   DataType ::= PrimitiveType
426   DataType ::= ClassOrInterfaceType
427     | ArrayType
428   PlaceTypeSpecifier ::= AT PlaceType
429   PlaceType ::= placelocal
430     | activitylocal
431     | current
432     | PlaceExpression
433   ClassOrInterfaceType ::= TypeName DepParametersopt
434   DepParameters ::= ( DepParameterExpr )
435   DepParameterExpr ::= ArgumentList WhereClauseopt
436     | WhereClause
437   WhereClause ::= : Expression
438   ArrayType ::= X10ArrayType
439   X10ArrayType ::= Type [ . ]
440     | Type reference [ . ]
441     | Type value [ . ]
442     | Type [ DepParameterExpr ]
443     | Type reference [ DepParameterExpr ]
444     | Type value [ DepParameterExpr ]
445   MethodModifier ::= atomic
446     | extern
447   ClassDeclaration ::= ValueClassDeclaration
448   ValueClassDeclaration ::= ClassModifiersopt value identifier Superopt Interfacesopt ClassBody
449     | ClassModifiersopt value class identifier Superopt Interfacesopt ClassBody
450   ArrayCreationExpression ::= new ArrayBaseType Unsafeopt [ ] ArrayInitializer
451     | new ArrayBaseType Unsafeopt [ Expression ]
452     | new ArrayBaseType Unsafeopt [ Expression ] Expression
453     | new ArrayBaseType Unsafeopt [ Expression ] ( FormalParameter ) MethodBody
454     | new ArrayBaseType value Unsafeopt [ Expression ]
455     | new ArrayBaseType value Unsafeopt [ Expression ] Expression
456     | new ArrayBaseType value Unsafeopt [ Expression ] ( FormalParameter ) MethodBody
457   ArrayBaseType ::= PrimitiveType
458     | ClassOrInterfaceType
459   ArrayAccess ::= ExpressionName [ ArgumentList ]
460     | PrimaryNoNewArray [ ArgumentList ]
461   Statement ::= NowStatement
462     | ClockedStatement
463     | AsyncStatement
464     | AtomicStatement
465     | WhenStatement
466     | ForEachStatement
467     | AtEachStatement
468     | FinishStatement
469   StatementWithoutTrailingSubstatement ::= NextStatement
470     | AwaitStatement
471   StatementNoShortIf ::= NowStatementNoShortIf
472     | ClockedStatementNoShortIf
473     | AsyncStatementNoShortIf
474     | AtomicStatementNoShortIf
475     | WhenStatementNoShortIf
476     | ForEachStatementNoShortIf
477     | AtEachStatementNoShortIf
478     | FinishStatementNoShortIf
479   NowStatement ::= now ( Clock ) Statement
480   ClockedStatement ::= clocked ( ClockList ) Statement
481   AsyncStatement ::= async PlaceExpressionSingleListopt Statement
482   AtomicStatement ::= atomic PlaceExpressionSingleListopt Statement
483   WhenStatement ::= when ( Expression ) Statement
484     | WhenStatement or ( Expression ) Statement
485   ForEachStatement ::= foreach ( FormalParameter : Expression ) Statement
486   AtEachStatement ::= ateach ( FormalParameter : Expression ) Statement
487   EnhancedForStatement ::= for ( FormalParameter : Expression ) Statement
488   FinishStatement ::= finish Statement
489   NowStatementNoShortIf ::= now ( Clock ) StatementNoShortIf
490   ClockedStatementNoShortIf ::= clocked ( ClockList ) StatementNoShortIf
491   AsyncStatementNoShortIf ::= async PlaceExpressionSingleListopt StatementNoShortIf
492   AtomicStatementNoShortIf ::= atomic StatementNoShortIf
493   WhenStatementNoShortIf ::= when ( Expression ) StatementNoShortIf
494     | WhenStatement or ( Expression ) StatementNoShortIf
495   ForEachStatementNoShortIf ::= foreach ( FormalParameter : Expression ) StatementNoShortIf
496   AtEachStatementNoShortIf ::= ateach ( FormalParameter : Expression ) StatementNoShortIf
497   EnhancedForStatementNoShortIf ::= for ( FormalParameter : Expression ) StatementNoShortIf
498   FinishStatementNoShortIf ::= finish StatementNoShortIf
499   PlaceExpressionSingleList ::= ( PlaceExpression )
500   PlaceExpression ::= Expression
501   NextStatement ::= next ;
502   AwaitStatement ::= await Expression ;
503   ClockList ::= Clock
504     | ClockList , Clock
505   Clock ::= identifier
506   CastExpression ::= ( Type ) UnaryExpressionNotPlusMinus
507   MethodInvocation ::= Primary ARROW identifier ( ArgumentListopt )
508   RelationalExpression ::= RelationalExpression instanceof Type
509   IdentifierList ::= IdentifierList , identifier
510     | identifier
511   Primary ::= FutureExpression
512   Primary ::= [ ArgumentList ]
513   AssignmentExpression ::= Expression ARROW Expression
514   Primary ::= Expression : Expression
515   FutureExpression ::= future PlaceExpressionSingleListopt { Expression }
516   FieldModifier ::= mutable
517     | const
518   PlaceTypeSpecifieropt ::=
519     | PlaceTypeSpecifier
520   DepParametersopt ::=
521     | DepParameters
522   WhereClauseopt ::=
523     | WhereClause
524   PlaceExpressionSingleListopt ::=
525     | PlaceExpressionSingleList
526   ArgumentListopt ::=
527     | ArgumentList
528   DepParametersopt ::=
529     | DepParameters
530   Unsafeopt ::=
531     | unsafe
\end{verbatim}
}
\twocolumn\par  \vfill\eject} % syntax
\extrapart{Changes from v0.32}

This is the first reference manual that corresponds to a working
implementation. As such a number of details missing from v0.32 have
been spelt out. A number of mistakes have been corrected, and
clarifications added.

The semantics of exception handling across asynchronous activities has
been clarified.

Exploded syntax has been introduced to make it convenient to
destructure points. 

\subsection{Limitations}

Exception propagation from an activity to its invoking activity is not
yet implemented.

All the type checking rules are not implemented. Thus if your program
is already correct, it will exeute correctly. If it is not correct, it
may still execute and give a result.

The predicate {\tt ==} for value types is not yet implemented.

\todo{Update this list from Mantisa.}\par  \vfill\eject % changes
%\newpage                   %  Put bib on it's own page (it's just one)
%\twocolumn[\vspace{-.18in}]%  Last bib item was on a page by itself.
\renewcommand{\bibname}{References}
\newpage\bibliographystyle{plain}
\bibliography{master}

%%\extrapart{Bibliography and references}

% My reference for proper reference format is:
%    Mary-Claire van Leunen.
%    {\em A Handbook for Scholars.}
%    Knopf, 1978.
% I think the references list would look better in ``open'' format,
% i.e. with the three blocks for each entry appearing on separate
% lines.  I used the compressed format for SIGPLAN in the interest of
% space.  In open format, when a block runs over one line,
% continuation lines should be indented; this could probably be done
% using some flavor of latex list environment.  Maybe the right thing
% to do in the long run would be to convert to Bibtex, which probably
% does the right thing, since it was implemented by one of van
% Leunen's colleagues at DEC SRC.
%  -- Jonathan

% This is just a personal remark on your question on the RRRS:
% The language CUCH (Curry-Church) was implemented by 1964 and 
% is a practical version of the lambda-calculus (call-by-name).
% One reference you may find in Formal Language Description Languages
% for Computer Programming T.~B.~Steele, 1965 (or so).
%  -- Matthias Felleisen


\begin{thebibliography}{99}

\bibitem{SICP}
Harold Abelson and Gerald Jay Sussman with Julie Sussman.
{\em Structure and Interpretation of Computer Programs.}
MIT Press, Cambridge, 1985.

\bibitem{readfloat}
William Clinger.
How to read floating point numbers accurately.
In {\em Proceedings of the 1990 ACM SIGPLAN Conference on Programming
  Language Design and Implementation}.  Forthcoming.

University of Oregon Technical Report CIS-TR-90-01.

\bibitem{RRRS}
William Clinger, editor.
The revised revised report on Scheme, or an uncommon Lisp.
MIT Artificial Intelligence Memo 848, August 1985.
Also published as Computer Science Department Technical Report 174,
  Indiana University, June 1985.

\bibitem{R4RS}
William Clinger and Jonathan Rees, editors.
The revised$^4$ report on the algorithmic language Scheme.
University of Oregon Technical Report CIS-TR-90-02.

\bibitem{Scheme311}
Carol Fessenden, William Clinger, Daniel P.~Friedman, and Christopher Haynes.
Scheme 311 version 4 reference manual.
Indiana University Computer Science Technical Report 137, February 1983.
Superceded by~\cite{Scheme84}.

\bibitem{Scheme84}
D.~Friedman, C.~Haynes, E.~Kohlbecker, and M.~Wand.
Scheme 84 interim reference manual.
Indiana University Computer Science Technical Report 153, January 1985.

\bibitem{CFractions}
G.~H.~Hardy and E.~M.~Wright.
{\em An Introduction to the Theory of Numbers.} 5th ed.
Oxford University Press, New York NY, 1979.

\bibitem{Haskell}
Paul Hudak and Philip Wadler, editors.
Report on the Functional Programming Language Haskell.
Yale University Research Report YALEU/DCS/RR-666, December 1988.

\bibitem{IEEE}
{\em IEEE Standard 754-1985.  IEEE Standard for Binary Floating-Point
Arithmetic.}  IEEE, New York, 1985.

\bibitem{Knuth}
Donald E. Knuth.
The Art of Computer Programming, volume 2: Seminumerical Algorithms.
Addison-Wesley, Reading MA, 1969.

\bibitem{Landin65}
Peter Landin.
A correspondence between Algol 60 and Church's lambda notation: Part I.
{\em Communications of the ACM} 8(2):89--101, February 1965.

\bibitem{Matula68}
David W. Matula.
In-and-Out Conversions.
{\em Communications of the ACM} 11(1):47--50, January 1968.

\bibitem{Matula70}
David W. Matula.
A Formalization of Floating-Point Numeric Base Conversion.
{\em IEEE Transactions on Computers} C-19, 8:681-692, August 1970.

\bibitem{MITScheme}
MIT Department of Electrical Engineering and Computer Science.
Scheme manual, seventh edition.
September 1984.

\bibitem{Penfield81}
Paul Penfield, Jr.
Principal values and branch cuts in complex APL.
In {\em APL '81 Conference Proceedings,} pages 248--256.
ACM SIGAPL, San Francisco, September 1981.
Proceedings published as {\em APL Quote Quad} 12(1), ACM, September 1981.

\bibitem{Pitman83}
Kent M.~Pitman.
The revised MacLisp manual (Saturday evening edition).
MIT Laboratory for Computer Science Technical Report 295, May 1983.

\bibitem{Rees82}
Jonathan A.~Rees and Norman I.~Adams IV.
T: A dialect of Lisp or, lambda: The ultimate software tool.
In {\em Conference Record of the 1982 ACM Symposium on Lisp and
  Functional Programming}, pages 114--122.

\bibitem{R3RS}
Jonathan Rees and William Clinger, editors.
The revised$^3$ report on the algorithmic language Scheme.
In {\em ACM SIGPLAN Notices} 21(12), ACM, December 1986.

\bibitem{Reynolds72}
John Reynolds.
Definitional interpreters for higher order programming languages.
In {\em ACM Conference Proceedings}, pages 717--740.
ACM, \todo{month?}~1972.

\bibitem{Rabbit}
Guy Lewis Steele Jr.
Rabbit: a compiler for Scheme.
MIT Artificial Intelligence Laboratory Technical Report 474, May 1978.

\bibitem{CLtL}
Guy Lewis Steele Jr.
{\em Common Lisp: The Language.}
Digital Press, Burlington MA, 1984.

\bibitem{CLtL2}
Guy Lewis Steele Jr.
{\em Common Lisp: The Language.} 2d ed.
Digital Press, Bedford MA, 1990.

\bibitem{Scheme78}
Guy Lewis Steele Jr.~and Gerald Jay Sussman.
The revised report on Scheme, a dialect of Lisp.
MIT Artificial Intelligence Memo 452, January 1978.

\bibitem{Heuristic}
Guy Lewis Steele Jr.~and Jon L White.
How to Print Floating-Point Numbers Accurately.
In {\em Proceedings of the 1990 ACM SIGPLAN Conference on Programming
  Language Design and Implementation}.  Forthcoming.

\bibitem{Stoy77}
Joseph E.~Stoy.
{\em Denotational Semantics: The Scott-Strachey Approach to
  Programming Language Theory.}
MIT Press, Cambridge, 1977.

\bibitem{Scheme75}
Gerald Jay Sussman and Guy Lewis Steele Jr.
Scheme: an interpreter for extended lambda calculus.
MIT Artificial Intelligence Memo 349, December 1975.

\bibitem{Vuillemin}
Jean Vuillemin.
Exact real computer arithmetic with continued fractions.
In {\em Proceedings of the 1988 ACM Conference on Lisp and
  Functional Programming}, pages 14--27.

\end{thebibliography}
	\par

\vfill\eject

% Adjustment to avoid having the last index entry on a page by itself.
%\addtolength{\baselineskip}{-0.1pt}

\bigskip

\documentclass[orivec,twoside,twocolumn]{algol60}
\usepackage{changebar}
%\documentclass[twoside]{algol60}
\def\Hat{{\tt \char`\^}}
\def\ccfont{\sf}
\usepackage{url}
\usepackage{times}
\pagestyle{headings}
\showboxdepth=0
\makeindex
\input{commands}

\def\headertitle{The \XtenCurrVer{} Report }
\def\integerversion{1.0}

% Sizes and dimensions

\topmargin -.375in       %    Nominal distance from top of page to top of
                         %    box containing running head.
\headsep 15pt            %    Space between running head and text.

\textheight 663pt        % Height of text (including footnotes and figures, 
                         % excluding running head and foot).

\textwidth 523pt         % Width of text line.
\columnsep 15pt          % Space between columns 
\columnseprule 0pt       % Width of rule between columns.

\parskip 5pt plus 2pt minus 2pt % Extra vertical space between paragraphs.
\parindent 0pt                  % Width of paragraph indentation.
\topsep 0pt plus 2pt            % Extra vertical space, in addition to 
                                % \parskip, added above and below list and
                                % paragraphing environments.

\oddsidemargin  -.5in    % Left margin on odd-numbered pages.
\evensidemargin -.5in    % Left margin on even-numbered pages.

%% End of sizes and dimensions
\makeatletter
\newsavebox{\eStop}
\savebox{\eStop}{\raisebox{0.6ex}{\framebox[0.5em]\relax}}

\def\newtenv#1{\@ifnextchar[{\@otxm{#1}}{\@ntxm{#1}}}

\def\@ntxm#1#2{\@ifnextchar[{\@xntxm{#1}{#2}}{\@yntxm{#1}{#2}}}

\def\@xntxm#1#2[#3]{\expandafter\@ifdefinable\csname #1\endcsname
{\@definecounter{#1}\@addtoreset{#1}{#3}%
\expandafter\xdef\csname the#1\endcsname{\expandafter\noexpand
  \csname the#3\endcsname \@thmcountersep \@thmcounter{#1}}%
\global\@namedef{#1}{\@txm{#1}{#2}}\global\@namedef{end#1}{\@endtenv}}}

\def\@yntxm#1#2{\expandafter\@ifdefinable\csname #1\endcsname
{\@definecounter{#1}%
\expandafter\xdef\csname the#1\endcsname{\@thmcounter{#1}}%
\global\@namedef{#1}{\@txm{#1}{#2}}\global\@namedef{end#1}{\@endtenv}}}

\def\@otxm#1[#2]#3{\expandafter\@ifdefinable\csname #1\endcsname
  {\global\@namedef{the#1}{\@nameuse{the#2}}%
\global\@namedef{#1}{\@txm{#2}{#3}}%
\global\@namedef{end#1}{\@endtenv}}}

\def\@txm#1#2{\refstepcounter
    {#1}\@ifnextchar[{\@ytxm{#1}{#2}}{\@xtxm{#1}{#2}}}

\def\@xtxm#1#2{\@begintenv{#2}{\csname the#1\endcsname}\ignorespaces}
\def\@ytxm#1#2[#3]{\@opargbegintenv{#2}{\csname
       the#1\endcsname}{#3}\ignorespaces}

%DEFAULT VALUES
\def\@begintenv#1#2{\trivlist \item[\hskip \labelsep{\bf #1\ #2}]}
\def\@opargbegintenv#1#2#3{\trivlist
      \item[\hskip \labelsep{\bf #1\ #2\ (#3)}]}
\def\@endtenv{\hfill\usebox{\eStop}\endtrivlist}
\makeatother

\newtenv{example}{Example}[section]

\begin{document}

\parindent 0pt %!! 15pt                    % Width of paragraph indentation.

%\hfil {\bf 7 Feb 2005}
%\hfil \today{}

\input{first}   \par  % vj: first page
\input{XtenIntro}   \par  % 0.1
\input{XtenOverview}  \par % Semantics section. What else?
\vskip 2ex
\clearchapterstar{Description of the language} %\unskip\vskip -2ex
\input{XtenLex}	\par % 0.1
\input{XtenTypes}	\par % empty
\input{XtenDepTypes}	\par % empty
\input{XtenPackages}	\par % \vfill\eject % empty
\input{XtenPlaces}	\par %0.1
\input{XtenActivities}	\par %0.1
\input{XtenClocks}	\par  %\vfill\eject %0.1
\input{XtenInterfaces}	\par  %\vfill\eject % empty
\input{XtenClasses}	\par % 0.1
\input{XtenArrays}	\par % 0.1
\input{XtenStatements}	\par  % 0.05
\input{XtenAnnotations}	\par  % 0.05
\input{extern}
%\input{PerformanceModel} \par \vfill\eject % empty
\input{example}	\par  \vfill\eject % have an example
\notinfouro{\input{x10-syntax}\par  \vfill\eject} % syntax
\input{x10-changes}\par  \vfill\eject % changes
%\newpage                   %  Put bib on it's own page (it's just one)
%\twocolumn[\vspace{-.18in}]%  Last bib item was on a page by itself.
\renewcommand{\bibname}{References}
\newpage\bibliographystyle{plain}
\bibliography{master}

%\input{bib}	\par

\vfill\eject

% Adjustment to avoid having the last index entry on a page by itself.
%\addtolength{\baselineskip}{-0.1pt}

\bigskip

\input{x10.ind}

\vfill\eject
{\em The \Xten{} language has been developed as part of the IBM PERCS
Project, which is supported in part by the Defense Advanced Research
Projects Agency (DARPA) under contract No. NBCH30390004.}

{\em Java and all Java-based trademarks are trademarks of Sun Microsystems,
Inc. in the United States, other countries, or both.}
\end{document}



\vfill\eject
{\em The \Xten{} language has been developed as part of the IBM PERCS
Project, which is supported in part by the Defense Advanced Research
Projects Agency (DARPA) under contract No. NBCH30390004.}

{\em Java and all Java-based trademarks are trademarks of Sun Microsystems,
Inc. in the United States, other countries, or both.}
\end{document}



\vfill\eject
{\em The \Xten{} language has been developed as part of the IBM PERCS
Project, which is supported in part by the Defense Advanced Research
Projects Agency (DARPA) under contract No. NBCH30390004.}

{\em Java and all Java-based trademarks are trademarks of Sun Microsystems,
Inc. in the United States, other countries, or both.}
\end{document}



\appendix

\chapter{Change Log}

\section{Changes from \Xten{} v2.0}

\begin{itemize}
\item \Xcd{Any} is now the top of the type hierarchy (every object,
  struct and function has a type that is a subtype of
  \Xcd{Any}). \Xcd{Any} defines \Xcd{home}, \Xcd{at}, \Xcd{toString},
  \Xcd{typeName}, \Xcd{equals} and \Xcd{hashCode}. \Xcd{Any} also defines the methods
  of \Xcd{Equals}, so \Xcd{Equals} is not needed any more.
\item Revised discussion of incomplete types (\Sref{ProtoRules}).
\item The manual has been revised and brought into line with the current implementation. 
\end{itemize}
\section{Changes from \Xten{} v1.7}

The language has changed in the following way:
\begin{itemize}

\item {\bf Type system changes}: There are now three kinds of entities
  in an \Xten{} computation: objects, structs and functions. Their
  associated types are class types, struct types and function
  types. 

  Class and struct types are called {\em container types} in that they
  specify a collection of fields and methods. Container types have a
  name and a signature (the collection of members accessible on that
  type). Collection types support primitive equality \Xcd{==} and may
  support user-defined equality if they implement the {\tt
    x10.lang.Equals} interface. 

  Container types (and interface types) may be further qualified with
  constraints.

  A function type specifies a set of arguments and their type, the
  result type, and (optionally) a guard. A function application
  type-checks if the arguments are of the given type and the guard is
  satisfied, and the return value is of the given type.  A function
  type does not permit \Xcd{==} checks. Closure literals create
  instances of the corresponding function type.

  Container types may implement interfaces and zero or more function
  types.

  All types support a basic set of operations that return a string
  representation, a type name, and specify the home place of the entity.

  The type system is not unitary. However, any type may be used to
  instantiate a generic type. 

  There is no longer any notion of \Xcd{value} classes. \Xcd{value}
  classes must be re-written into structs or (reference) classes. 

\item {\bf Global object model}: Objects are instances of
  classes. Each object is associated with a globally unique
  identifier. Two objects are considered identical \Xcd{==} if their
  ids are identical. Classes may specify \Xcd{global} fields and
  methods. These can be accessed at any place. (\Xcd{global} fields
  must be immutable.)

\item{\bf Proto types.} For the decidability of dependent type
  checking it is necessary that the property graph is acyclic. This is
  ensured by enforcing rules on the leakage of \Xcd{this} in
  constructors. The rules are flexible enough to permit cycles to be
  created with normal fields, but not with properties.

\item{Place types.} Place types are now implemented. This means that
  non-global methods can be invoked on a variable, only if the
  variable's type is either a struct type or a function type, or a
  class type whose constraint specifies that the object is located in
  the current place.

  There is still no support for statically checking array access
  bounds, or performing place checks on array accesses.

\end{itemize}
{\em The \Xten{} language has been developed as part of the IBM PERCS
Project, which is supported in part by the Defense Advanced Research
Projects Agency (DARPA) under contract No. NBCH30390004.}

{\em Java and all Java-based trademarks are trademarks of Sun Microsystems,
Inc. in the United States, other countries, or both.}
\end{document}

