\documentclass[nocopyrightspace,preprint,9pt]{sigplanconf}

\usepackage{times-lite}
\usepackage{mathptm}
\usepackage{txtt}
\usepackage{stmaryrd}
\usepackage{code}
%\usepackage{ttquot}
\usepackage{amsmath}
\usepackage{amssymb}
\usepackage{afterpage}
\usepackage{balance}
\usepackage{defs}
\usepackage{pldefs}
\usepackage{utils}
\usepackage[pdftex]{graphicx}
\usepackage{ifpdf}

\ifpdf
\setlength{\pdfpagewidth}{8.5in}
\setlength{\pdfpageheight}{11in}
\fi


% \input{../../../../vj/res/pagesizes}
% \input{../../../../vj/res/vijay-macros}
\newcommand\alt{\bnf}

\newcommand\Implies{\Rightarrow}

\newtheorem{example}{Example}[section]
\newtheorem{theorem}{Theorem}[section]
\newtheorem{lemma}[theorem]{Lemma}

\newcommand\Xten{{\sf X10}}
\newcommand\DML{{ DML}}
\newcommand\FJ{{FJ}}
\newcommand\CFJ{{CFJ}}
\newcommand\java{{Java}}
\newcommand\Java{{\java}}
\newcommand\csharp{{C$\sharp$}}
\newcommand\FXten{{FX10}}

\begin{document}
\title{Constrained Types for Object-Oriented Languages}
\authorinfo{Vijay Saraswat\titlenote{IBM T.~J. Watson Research
Center, P.O. Box 704, Yorktown Heights NY 10598 USA}}{}
  {vsaraswa@us.ibm.com}
\authorinfo{Nathaniel Nystrom$^{\;*}$}{}
  {nystrom@us.ibm.com}
\authorinfo{Radha Jagadeesan\titlenote{School of CTI, DePaul
University, 243 S. Wabash Avenue, Chicago IL 60604 USA}}{}
  {rjagadeesan@cs.depaul.edu}
\authorinfo{Jens Palsberg\titlenote{UCLA Computer Science
Department, Boelter Hall, Los Angeles CA 90095 USA}}{}
  {palsberg@cs.ucla.edu}
\authorinfo{Christian Grothoff\titlenote{
Department of Computer Science, University of Denver,
2360 S. Gaylord Street, John
Green Hall, Room 214, Denver CO, 80208 USA}}{}
  {christian@grothoff.org}

\maketitle

% \conferenceinfo{POPL'08}{XXX}
% \copyrightyear{2008}
% \copyrightdata{[to be supplied]}

\begin{abstract}

\Xten{} is a modern object-oriented language designed for productivity
and performance in concurrent and distributed systems, such as
(heterogenous) multicores and clusters. In this context, dependent
types arise naturally: objects may be located at one of many places,
arrays may be multidimensional, activities may be associated with one
or more clocks, variables may be marked as shared or private following
an ownership discipline, etc.  A framework for dependent types offers
significant opportunities for detecting design errors statically,
documenting design decisions, eliminating costly runtime checks
(e.g.{} for array bounds, null values), and improving the quality of
generated code.

We present a simple, general framework for adding constraint-based
dependent types to nominally typed OO languages such as Java, \Xten{}
and Scala. The framework is parametric on an underlying constraint
system {\cal C}. Classes and interfaces are associated with {\em
properties}(= final instance fields). A type {\tt C(:c)} names a class
or interface {\tt C} and a {\em constraint} {\tt c} on
properties of {\tt C} and in-scope final variables.  Constraints
may also be associated with class definitions (representing
class invariants) and with method and constructor definitions
(representing preconditions). Dynamic casting is permitted.

We present many examples to illustrate that many common OO idioms and
OO type systems proposed recently can be naturally captured by
constrained types: specifically we discuss types for places, aliases,
ownership, arrays and clocks. We have implemented the type system (for
a simple equality-based constraint system) in \Xten{} 1.0 (available
at {\tt x10.sf.net}). We present a simple \FJ{} extension, {\tt
Constrained FJ}, and establish fundamental properties such as type
soundess. We compare this approach with relevant work in dependent
types, specifically, DML, and outline many areas of future work.

In conclusion, we believe that constrained types offer a natural,
simple, clean, and expressive extension to OO programming.
\end{abstract}

\section{Introduction}\label{sec:intro}
\section{Introduction}

The industry's shift to multicore processors has sparked a trend
toward pushing parallel programming into the mainstream.  This trend
poses a significant challenge, since creating and maintaining parallel
programs that are both efficient and reliable is notoriously
difficult.  One response to this challenge
by the programming languages community has been to create new (and
revisit old) language
abstractions and programming models 
for parallel programming and to
develop new languages based on these abstractions.  % For example, there
% has been lots of recent work on support for software transactional
% memory~\cite{}, and the DARPA PERCS project is funding several new
% parallel programming languages~\cite{}.

While new languages can greatly aid programmers in developing 
parallel programs, we believe that new languages cannot
achieve mainstream success without associated tooling support.
Modern integrated development environments
(IDEs) such as Eclipse~\cite{eclipse} provide many benefits that programmers
have come to rely upon, helping
them to more easily navigate through a program, understand
dependencies among parts of a program, and safely evolve a program to
improve its quality along some dimension.  The latter benefit is typically
provided through code {\em refactorings}.  
% Part of the difficulty in programming for concurrent systems is
% that they have often lagged behind in tooling support. Practitioners of most
% modern sequential languages have the advantage of employing a wide variety of
% analyses and refactorings in order to improve the quality of their code. While
% analysis research has continued to show forward momentum in the parallel
% language context, we believe that similar tooling support for refactorig in
% parallel languages will help users improve the quality of their parallel code;
% a belief that a number of others also share~\cite{Kennedy91, Liao99, Overbey05}.

We believe that
specialized refactoring support will be a critical tool to help programmers improve the
quality of parallel code.  Such refactorings could be used
to improve efficiency while preserving program behavior and key
concurrency invariants (e.g., atomicity, deadlock-freedom).
%Such
%refactorings could also be used to improve a parallel program's readability and
%extensibility.

Any set of refactorings will of necessity be tailored to the needs of a particular
parallel programming model and language.
We focus on providing refactoring support for the
\emph{partitioned global address space} (PGAS) memory model as
embodied in the X10~\cite{X10,Charles05}, UPC~\cite{ElGhazawi03} and
Titanium~\cite{Yelick98} programming languages.  In this model, the
programmer sees a uniform 
representation of data and data structures over distributed nodes, regardless
of the physical location of the data.  However, each piece of data is
assigned to a fixed partition (or {\em place} in X10 terminology)
and can only be accessed by {\em activities}
that run at that place.
% That is, a reference to a piece of
% data is treated as a first-class entity in this system and can be treated
% similarly to a local reference even if the owner of said data is at a
% different location. In such cases, the language's compiler or virtual machine
% will handle the messy details of handling message and data passing. The PGAS
% model attempts to minimize background communication overhead by
% locating data at a place where it has good data or processing affinity.

In this presentation, we focus on the X10 language, an object-oriented language
providing first-class, high-level constructs for asynchronous activities, synchronization,
phased computations, data distribution, and atomicity.
%Further, as stated in the introduction, the X10 language is in the family of
%concurrent languages with a {\em partitioned global address space} (PGAS)
%memory model. In the case of X10, the set of partitions in the global address
%space is fixed at the start of the program.
By incorporating such abstractions as first-class constructs in the language,
the burden of reasoning about various program properties is often reduced
from global reasoning involving complex control flow to simple and modular
reasoning about lexical containment.
In fact, many interesting properties (e.g. deadlock freedom) can often
be ensured statically.
In this way, X10's constructs simplify both the programmer's task of
understanding and tools' tasks of static program analysis.
% make all asynchronous and atomic code apparent
%to both the analysis engine and the programmer.

The PGAS model provides particular opportunities and challenges for
automated refactorings.  On the one hand, code transformations are
simplified since the code need not explicitly handle inter-processor
communication.  
On the other hand, transformations must properly handle the
asynchronicity that arises among activities and must respect
the synchronization constraints imposed on these activities by the
semantics of the various language constructs. 
% Transformations must also
% ensure that an activity only accesses data from the current place.
In the rest of this paper we describe an initial concurrency
refactoring for X10 that we
have been developing inside the X10DT plugin for Eclipse.

%%%%%%%%%%%%%%%%%%%%%%%%%%%%%%%%%%%%%%%%%%%%%%%%%%%%%%%%%%%%%%%%%%%%
% 
% Previous Version: for comparison
% 
% \section{Introduction}
% 
% Many modern parallel languages, particularly high-performance computing (HPC)
% languages, have trended towards a \emph{global address
% space} (GAS) memory model. In fact, all three of the remaining DARPA HPC
% Challenge programming languages have built their languages around a GAS
% model. A GAS model simplifies the development of code since the programmer
% does not have to deal with the explicit passing of data from one
% place of execution to another via interfaces such as MPI or OpenMP. Instead, a
% programmer can directly reference any desired data and the language's compiler
% or virtual machine will handle the messy details. This surface simplicity comes
% with a small caveat, though: a lot of extra communication may be incurred in
% order to actually determine which place actually owns some piece of data, all
% of which is hidden at the source level. One solution to this issue is to
% use a \emph{partitioned global address space} (PGAS) model in which data is
% created and kept in a single place for its lifetime, thus minimizing such
% communication overhead. The PGAS model itself is employed in a number of
% languages, such as UPC, Titanium, and X10~\cite{ElGhazawi03, Yelick98, Charles05}.
% 
% Even in languages with an abstracted memory model like the GAS model, creating
% efficient code poses a challenge. Programmers still struggle with such tasks
% as introducing maximal concurrency, avoiding deadlock, minimizing transaction
% conflicts, and numerous other plagues of the parallel world. Making
% such code readable and extensible is
% even more daunting. Most modern sequential languages often employ
% refactoring and other transformation engines to make code more efficient
% without sacrificing readability or introducing errors. We believe that
% similar tooling support for parallel languages can provide these benefits to the
% parallel programming community; a belief that a number of others also
% share~\cite{Kennedy91, Liao99, Overbey05}. 
% 
% We propose a framework in which a refactoring engine, standing alone or coupled
% with an IDE, is available to aid in the creation of parallel programs.
% The refactoring engine should be capable of performing its own program analysis
% to guarantee that transforming the code will not introduce new and subtle
% errors. Its associated transformation can also use the results of the analysis
% to transform the code appropriately. We feel that such sanity checks are vitally
% important to ensuring that new bugs are not introduced, in contrast to current
% commonly-used refactorings which are mostly unchecked but relatively
% cosmetic such as renaming variables or abstracting blocks as methods. Having
% such a refactoring engine allows a developer to write simple and readable
% code with the idea that the refactoring engine will improve the performance of
% the areas in the code that she has identified as good candidates for increased
% efficiency.

%%%%%%%%%%%%%%%%%%%%%%%%%%%%%%%%%%%%%%%%%%%%%%%%%%%%%%%%%%%%%%%%%%%%%


%%Since the advent of inter-processor communication, the desire to
%%create programs which take advantage of parallel execution has
%%existed. 
% The desire to take advantage of parallel execution has
% existed since the advent of the computer processor, and this desire is only
% more fervent in this age of processor ubiquity. However, the creation and
% maintenance of parallel programs involves the difficult challenges of
% enforcing proper program and memory consistency and synchronization.
%%To compound the issue,
%%A. J. Bernstein has already shown that, in the general case,
%%automatically determining parallelism of sequential code is an
%%undecidable problem~\cite{Bernstein66}.
% In the programming language community, two main schools of thought exist on
% solving these problems. One school of thought is to simply develop languages
% that included explicit constructs or compiler optimizations for parallelism.
%% These include variants of Fortran and Lisp and languages built with parallel
%% execution or data distribution support like Ada, Linda, and
%% Emerald~\cite{Allen87, Hutchinson87, Griss82, Ada, Gelernter85}. 
% Even with language level support, though, it is still difficult for
% programmers to quickly create error-free code which efficiently uses
% parallelism. Another approach is to automate the parallelization of
% sequential code via extensive program analysis at points in a program, such as
% loops, that seem to provide natural boundaries for parallel execution. However, this
% is complicated by the difficulty of
%%Fortunately, certain sequential programming language constructs do
%%seem to provide natural boundaries for automatic parallel compiler
%%optimizations: loops~\cite{Lamport74, Allen84, Allen87}. By
%%recognizing that the majority of time in running programs is spent
%%executing the same set of calculations over a different set of data,
%%it is natural to establish parallel execution of loop bodies on
%%different processors or using vector supercomputers. One drawback to
%%this approach, though, is that 
% precise static analysis of the parallelism targets.
% %% is a very difficult task. 
% Dependence on structured memory usage, such as
% arrays, which benefit the most from parallel execution, complicates
% loop analysis further.

% We believe that 
% %%Many research projects found that 
% the key for better parallelism detection is to take advantage of user
% interaction, 


%%As a result, user interactivity in the form of 
% In particular, we hold that
% source code refactoring support in IDEs such as Eclipse can be a
% powerful ally in aiding parallelization.
%%in the compiler optimization and program transformation process
%%A typical way of involving the user in this process
%%solution for these systems 
%%is to build support for some parallel transformations into an IDE and display
%%the
%%collected memory dependence information for a given line of code to the user. If
%%the user determines that the dependences would not prevent parallel
%%execution, then the user can choose to transform the code. For a whole
%%program analysis, such a presentation might be overwhelming and difficult to
%%decipher, and is probably not the best way to involve the user. Our approach
%%is to 

% We think that programmers often know which statements in their code could
% benefit the most from
% added efficiency, even if they do not know how best to introduce this
% efficiency on their own.
% %  and where they would like to insert more concurrency.
% Static and dynamic analysis tools can also help users identify hotspots
% in their programs.
% %In either case, the
% The tasks of identification of suitable parallelism candidates and of
% transforming the code to introduce/manipulate parallelism are separable.
%%Thus, programmers can specify which lines of code
%%would like to refactor into a concurrent form and 


% provides the link
% between these two vectors of concurrency introduction. The refactoring
% engine should determine if refactoring at these points is safe
% and how the code should be transformed to introduce more concurrency. A developer can then

% of which
% statements could benefit from added parallelism or use a tool to identify possible target sites
% %%, such as the code in Figure~\ref{fig:CHM-X10}, 
% and then use 


% Moreover,
% we believe that this approach can be applied to other application concerns of
% parallel programmers, such as introducing new synchronization points or
% refitting blocks as atomic.
%%, such as the code in
%%Figures~\ref{fig:CHM-X10-future} and~\ref{fig:CHM-X10-async}.
%%information is often difficult for a user to decipher for a whole program, though,
%%and is not Because following the flow of data for parallel programs is
%%difficult in these IDEs, advanced concurrency slicing techniques were
%%developed~\cite{Zhao99,Chen01,Chen02,Krinke03}. We believe that
%%including user input is an essential part of making parallelism
%%apparent to the programmer

%%\bug{Now introduce we will use X10 and why we like it: explicit higher-level
%%constructs and PGAS model} 

%%The target of our research is to develop practical refactoring schemes for
%%introducing increased parallelism in the X10 language. X10 includes explicit
%higher-level concurrency constructs such as asynchronous blocks and future
%%expressions. It also incorporates a {\em partitioned global address space}
%%(PGAS) model of data consistency. The combination of the explicit constructs
%%and the PGAS model in X10 allows data dependency analysis to be simplified.

%%Many recent languages have been developed with a focus on
%%the {\em partitioned global address space} (PGAS) model of data consistency
%%including UPC, Titanium, Co-Array Fortran, and X10~\cite{ElGhazawi03,
%%Yelick98, Numrich98, Charles05}. The target of our research is to develop
%%a practical refactoring scheme for X10, although this scheme should be
%%adaptable to other languages based on the PGAS model. 

% To deal with some of the static analysis issues associated with parallelization,
% we focus focus our refactorings on languages with a {\em partitioned global 
% address space} (PGAS) model of data consistency. In the PGAS model,
% %%The primary goal of the PGAS model is to allow parallel asynchronous activity to
% %%occur among multiple concurrent platforms, each having its own local memory and
% %%potentially lacking a joint shared memory, without losing the ability to read
% %%and write to global data. Global 
% shared data is actually owned by local processors but is globally addressable.
% Other processors which would like
% to access or update global data then communicate directly with the owning
% location to perform any necessary actions. Thus, programmers take an active role
% in defining how data in arrays or data structures is distributed
% over the address space so as to maximize locality, thus taking maximum advantage
% of their concurrent environments. This removes the programmer's (or analysis')
% burden of determining
% how complicated structures should be partitioned to the various execution sites at
% refactoring points in the code where parallelism is desired (e.g., inside 
% loops). As a result, the required static analysis is reduced to determining
% local and loop-carried dependencies that prevent a statement from being
% asynchronously executed.\bug{This statement really applies to this particular
% transformation. I.e., the programmer does have to determine the partitioning at
% some point, but doesn't need to worry about it while applying this
% transformation. So perhaps we should say that we envision different kinds of
% {\em lateral} moves, e.g.: manipulate concurrency while keeping the distribution
% fixed, and manipulating the distribution while keeping concurrency relatively
% unchanged.}
% 
% We chose to implement our refactoring scheme in the X10 language. X10 not only
% has a PGAS data consistency model, but also includes explicit higher-level
% concurrency constructs such as asynchronous blocks and future
% expressions. These concurrency constructs further simplify the analysis of X10
% code by making all asynchronous and atomic code apparent to both the analysis
% engine and the programmer.
% 
% %%\begin{figure}[tp]
% %%  \begin{code}
% %%    int mcsum=0; \\
% %%    fo\=r (i=0; i<segments.length; i++)\{ \\
% %%    \>  mcsum += mc[i] = segments[i].modCount(); \\
% %%    \>  if\=(segments[i].containsValue(value)) \\
% %%    \>\>    return true; \\
% %%    \} \\
% %%  \end{code}
% %%\caption{\label{fig:CHM} A Java code excerpt from the library class {\tt
% %%java.util.concurrent.ConcurrentHashMap} which illustrates a loop that
% %%would not be parallelizable via traditional automatic loop
% %%parallelization methods.}
% %%\end{figure}
% 
% %%Because languages designed for the PGAS model do not focus on
% %%providing support for the parallel execution of the same code over
% % different sets of data, they can provide a more flexible alternative
% % to whole loop parallelization: the loops can be executed sequentially
% % and updates to distributed data structures can occur in parallel. 
% As an example, consider the X10 code in Figure~\ref{fig:CHM-X10}, an excerpt 
% from an X10 implementation of the {\tt java.util.concurrent.ConcurrentHashMap} 
% class. In this example, it is possible the {\tt modCount} and {\tt 
% containsValue} method invocations are a both very expensive operations. Because 
% of the PGAS model, the individual elements of the array {\tt segments} could 
% exist anywhere in the global address space. However, automatic whole loop 
% parallelization techniques developed previously would fail on this code because 
% of the conditional return statement and the loop carried dependency summing
% the results of the calls to {\tt modCount}. However, the programmer might still
% be able to take advantage of parallel execution by making the invocations of
% {\tt modCount} asynchronous and caching the results.
% Figures~\ref{fig:CHM-X10-async} and~\ref{fig:CHM-X10-future} show examples of
% adding concurrency in this fashion.
% 
% We present in this paper the source code transformation {\em extract concurrent} for
% the X10 language which allows a programmer to transform loop code
% like that in Figure~\ref{fig:CHM-X10} to one that takes maximum advantage of
% asynchronous execution. We acheive this refactoring through two means:
% 
% \begin{enumerate}
% \item {\em Loop dependence analysis.} Since introducing parallelism in
% the middle of a loop might affect the ability of other statements in a
% loop to properly evaluate, it is important that loops do not carry
% dependence on the results of any asynchronously evaluated statements. We
% have developed a series of analyses to determine whether {\em extract
% concurrent} will adversely affect the execution of the code and
% violate its sequential consistency.
% 
% \item {\em Transformation pattern.} We have developed a general
% pattern for the {\em extract concurrent} transformation on viable
% sequential loops. This pattern uses program slicing
% techniques to split a loop in two. The first loop will allow introduction of
% parallelism while the second loop utilizes the results of the asynchronous
% execution. Because this split requires some code duplication, we
% present the results of a formal analysis on how the transformation
% affects the runtime of the loop and define the conditions under which
% the transformation provides the potential for better runtime. In practice, this
% transformation is more widely applicable to multiple statements or expressions.
% We present here only the single statement or expression case.
% \end{enumerate}
% 
% We have implemented the transformation as refactoring in the X10DT, a
% development framework for the X10 language in Eclipse, however we believe that
% the transformation is adapatable to all languages with a PGAS consistency model.
% \bug{We will insert more about the implementation when it's actually been done.}

% The rest of the paper is organized as follows. Section 2 presents an
% overview of the X10 programming language. Section 3 contains a
% description and analysis of the the algorithm for determining loop
% candidacy for the transformation. Section 4 details the {\em extract
% concurrent} transformation and discusses the impact of the
% transformation on running time. Section 5 presents the implementation
% and evaluation of the transformation in X10DT. Section 6 describes the
% related work and Section 7 contains the conclusion and a summary of
% the paper.


\subsection{Related work}\label{sec:related}
Constraint-based type systems enjoy a long history.
The use of constraints for type inference and subtyping were
developed by
Mitchell~\cite{mitchell84}
and
Reynolds~\cite{reynolds85}.
%
Trifonov and Smith~\cite{trifonov96}
proposed a type system where types are refined by subtyping
constraints.  Dependent types are not supported.
%
Pottier~\cite{pottier96simplifying,pottier01b}
presents a constraint-based type system for an ML-like language with
subtyping.

HM(X)~\cite{sulzmann97type,pottier01a,pottier-remy-attapl}
is a constraint-based framework
for Hindley--Milner style type systems.
The framework is parameterized on the specific constraint system
X; instantiating X yields extensions of the HM type system.
Constraints in HM(X) are over types, not values.
%
% XXX HM(X) introduced {\em term constraint systems}; constraints in
% CFJ are term constraints?

% Cardelli~\cite{cardelli86}, type checking dependent types and
% subtypes.

% Russell
% \cite{fuh88}
% \cite{curtis90}
% \cite{aiken93}

% Aiken, Wimmers, and Lakshman proposed {\em conditional
% types}~\cite{conditional-types}, which have the ability to
% encode control-flow analysis of {\tt case} expressions.
% Conditional types are not dependent.

% \cite{smith94}



% \cite{palsberg95}
% constraint-based inference algorithm for object calculus, 

% Henglein (TAPOS) set constraints for OO language type-inference.

% Bane~\cite{fahndrich99}

% Pottier

% CLP(X) framework in constraint logic programming (JM94)
% HM(X)

Several systems have been proposed that refine types in a base
type system through constraints.
%
{\em Refinement types}~\cite{refinement-types} extend the 
Hindley--Milner type system with intersection, union, and
constructor types, enabling specification and inference of
more precise type information.
%
{\em Conditional
types}~\cite{conditional-types} extend refinement types to
encode control-flow information in the types.
%
Jones introduced {\em qualified types}, which permit
types to be constrained by a finite set of
predicates~\cite{jones94}.
%
{\em Sized types}~\cite{sized-types}
annotate types with the sizes of recursive data structures.
Sizes are linear functions of size variables.
Size inference is performed using a constraint solver for
Presburger arithmetic~\cite{omega}.
% constraints on types, support primitive recursion only

% Indexed types~\cite{indexed-types}

% Index objects must be pure.
% Singleton types int(n).
% ML$^{\Pi}_0$:
% Refinement of the ML type system: does not affect the
% operational semantics.  Can erase to ML$_0$.

% Jay and Sekanina 1996: array bounds checking based on shape
% types.

With hybrid type-checking~\cite{flanagan-popl06,flanagan-fool06},
types can be constrained by arbitrary boolean expressions.
While typing is undecidable, dynamic checks are inserted into
the program when necessary if the type-checker cannot determine
type safety statically.
In \Xten{} dynamic type checks, including tests of dependent
clauses, are inserted only at explicit casts.

% Ada dependent types.
% Ada has constrained array definitions.  A constraint
% \cite{ada-ref-man}.  Not clear if they're dependent.  Are
% there other dependent types?  Generics are dependent?

Singleton types~\cite{aspinall-singletons,stone00} are dependent
types containing only one value.  
In Stone's formulation~\cite{stone00},
$S(e : \tau)$
is the type of all values of type $\tau$ that are equal to $e$.
Term equivalence is
constructed so that type-checking is decidable.
The singleton $S(e: \tau)$ can be encoded in \Xten{} as
$\tau$\xcd{(:self ==}~$e$\xcd{)}.

        % Used for array bounds by Morrisett et al (I think--need
        % to find paper)

% Singleton types~\cite{aspinall-singletons}.

Several languages---gbeta~\cite{ernst99-gbeta},
Scala~\cite{scala-overview,scala-oopsla05}, J\&~\cite{nqm06}, and
others~\cite{oz01,ocrz-ecoop03,dependent-classes}---provide {\em path-dependent
types}.  For a final access path \xcd{p}, {\tt p.type}
in Scala is the singleton type containing the object \xcd{p}.
In J\&, {\tt p.class} is a type containing all objects
whose run-time class is the same as \xcd{p}'s.
Scala's {\tt p.type} can be encoded in \Xten{} using an equality
constraint \xcd{C(:self == p)}, where \xcd{C} is a supertype of
\xcd{p}'s static type.
\eat{
These types can be encoded in CFJ by introducing a
\xcd{type} property.
\rn{T-constr}, as
described in Section~\ref{sec:examples}.
}

% Where clauses for F-bounded polymorphism~\cite{where-clauses}
% Bounded quantification: Cardelli and Wegner.  Bound T with T'
% In F-bounded polymorphism~\cite{f-bounds}, type variables are bounded by a function of 
% the type variable. 
% Not dependent types.

\eat{
conditional types:

type of an expression can be constrained using information about
the results of run-time tests in the context surrounding the
expression.

e.g., can express that e2 is evaluated only if e1 is true

\begin{verbatim}
\y. case y of true => e1 | false => e2 :
        'a -> (true ? ('a ^ typeof(e1)) U (false ? ('a ^ typeof(e2))
\end{verbatim}

Types include type constructors applied to types.

\begin{verbatim}
        so,  true      : true
        but, (\x . x)  : 'a -> 'a
             node(l,r) : node('a tree, 'a tree)
\end{verbatim}


when checking a case branch, type of the expression being
matched refined to the include the type constructor for that
branch

captures some control flow analysis in the types

types
\begin{verbatim}
        t ::= t1 -> t2
                | c(..ti..) <-- type constructor
                | alpha
                | t1 U t2
                | t1 ^ t2
                | t1 ? t2
                | 0
                | 1

        sigma ::= t | \forall ..alpha.. t where ..ti <= tj..
\end{verbatim}
}


Cayenne~\cite{cayenne} is a Haskell-like language with fully dependent types.
There is no distinction between static and dynamic types.
Type-checking is undecidable.
There is no notion of datatype refinement as in DML.

Epigram~\cite{epigram,epigram-matter}
is a dependently typed functional programming language based on
a type theory with inductive families.
Epigram does not have a phase distinction between values and
types.

\eat{
$\lambda^{\sf Con}$ is a lambda calculus with assertions.
Findler, Felleisen, Contracts for higher-order functions (ICFP02)

  example: int[> 9]

contracts are either simple predicates or function contracts.
defined by (define/contract ...)

enforced at run-time.
}

% Jif~\cite{jif,jflow} is an extension of Java in which
% types are labeled with security policies enforced by the
% compiler.

ESC/Java~\cite{esc-java}
allow programmers to write object invariants and pre- and
post-conditions that are enforced statically
by the compiler using an automated theorem prover.
Static checking is undecidable and, in the presence of loops,
is unsound (but still useful) unless the programmer supplies loop invariants.
ESC/Java can enforce invariants on mutable state.

% and Spec$\sharp$~\cite{specsharp}

Pluggable and optional type systems were proposed by
Bracha~\cite{bracha04-pluggable} and provide another means of
specifying refinement types.
Type annotations, implemented in compiler plugins, serve only to
reject programs statically that might otherwise have dynamic
type errors.
CQual~\cite{foster-popl02} extends C with user-defined type
qualifiers.  These
qualifiers may be flow-sensitive and may be inferred. 
CQual supports only a fixed set of typing rules
for all qualifiers.
In contrast, the {\em semantic type qualifiers} of
Chin, Markstrum, and Millstein~\cite{chin05-qualifiers}
allow programmers to define typing rules for qualifiers
in a meta language that allows type-checking rules to be
specified declaratively.
JavaCOP~\cite{javacop-oopsla06} is a pluggable type system
framework for Java.  Annotations are defined in a meta language
that allows type-checking rules to be specified declaratively.
JSR 308~\cite{jsr308} is a proposal for adding user-defined type qualifiers
to Java.

% Holt, Cordy, the Turing programming language

% Ou, Tan, Mandelbaum, Walker, Dynamic typing with dependent types
% Separate dependent and simple parts of the program.
% Statically type the dependent parts.
% Dynamic checks when passing values into dependent part.

Our work is most closely related to \DML{}, \cite{xi99dependent}, an
extension of ML with dependent types. \DML{} is also built
parametrically on a constraint solver. Types are refinement types;
they do not affect the operational semantics and erasing the
constraints yields a legal ML program.

At a conceptual level the intuitions behind the development of \DML{}
and constrained types are similar. Both are intended for practical
programming by mainstream programmers, both introduce a strict
separation between compile-time and run-time processing, are
parametric on a constraint solver, and deal with mutually recursive
data-structures, mutable state, and higher-order functions (encoded as
objects in the case of constrained types). Both support existential
types.

The most obvious distinction between the two lies in the target
domain: \DML{} is designed for functional programming, specifically
ML, whereas constrained types are designed for imperative, concurrent
OO languages. Hence technically our development of constrained types
takes the route of an extension to \FJ. But there are several other
crucial differences as well.

\lstnewenvironment{displayml}
  {\lstset{language=ML,basicstyle=\tt,tabsize=4,columns=fullflexible,captionpos=b,xleftmargin=1em,xrightmargin=1em,keywordstyle=,keepspaces}}
  {}

First, \DML{} achieves its separation by not permitting program
variables to be used in types. Instead, a parallel set of (universally
or existentially quantified) ``index'' variables are
introduced. For instance the typing of the \xcd{append} operation on
lists is provided by:
\begin{displayml}
fun('a)
  append(nil, ys) = ys
| append(cons(x, xs), ys)
    = cons(x, append(xs,ys))
where append <| {m:nat}{n:nat} 
    'a list(m) * 'a list(n) -> 'a list(m+n)  
\end{displayml}
\noindent in contrast with the direct embedded expression with constrained types:
\begin{xten}
class List(int(:self >= 0) n) {
  Object item;
  List(n-1) tail;
  List(n+a.n) append(final List a) { 
    return n==0
      ? a : new List(item, tail.app(a));
  }
  ...
}
\end{xten}

Second, \DML{} permits only variables of basic index sorts known to
the constraint solver (e.g., \xcd{bool}, \xcd{int}, \xcd{nat}) to
occur in types. In contrast, constrained types permit program
variables at any type to occur in constrained types. As with \DML,
only operations specified by the constraint system are permitted in
types. However, these operations always include field selection and
equality on object references.  (As we have seen permitting arbitrary
type/property graphs may lead to undecidability.) Note that \DML{}
style constraints are easily encoded in constrained types.

% A reviewer says:
% The third criticism of DML is technically correct but highly
% misleading.  Instead of casts, DML allows "if tests" or case
% analysis as dynamic tests that then yield static information
% about the type in the appropriate branch of the if or case.
% Either omit this criticism or describe how DML does the same
% thing--or if DML's system is weaker in some way, give a
% particular example to justify that.

% Third, \DML{} does not permit any runtime checking of constraints
% (dynamic casts).


\section{Constrained FJ}\label{sec:lang}
%II. Language design and rules (CFJ) (3.5 pages)

%II. Language design and rules (CFJ) (3.5 pages)

Our basic approach to introducing dependent types into
class-based statically typed OO languages is to
follow the spirit of generic types, but use values instead of
types.

We permit the definition of a class {\tt C} to specify
a list of typed parameters, or {\em properties},
{\tt (T1 x1, \ldots, Tk xk)} similar in syntactic structure to
a method argument list. Each
property in this list is treated as a {\tt public final} instance
field of the class.
We also permit the
specification of a {\em class invariant}, a
{\em where clause}~\cite{where-clauses}
in the class definition. A where
clause is a boolean expression on the properties separated from the
property list with a ``{\tt :}''.  The compiler ensures that all
instances of the class created at runtime satisfy the where clause
associated with the class.
Thus, for instance, we may specify a class {\tt List} with an
{\tt int length} property as follows:
\begin{code}
  class List(int length: length >= 0) \{...\}
\end{code}

Given such a definition for a class {\tt C},
types can be constructed by {\em constraining} the properties of
{\tt C}.
In principle, {\em any} boolean expression over the
properties specifies a type: the type of all instances of the
class which satisfy the boolean expression. Thus,
{\tt List(:length == 3)} is a permissible type, as are
{\tt List(:length <= 41)} and even
{\tt List(:length * f() >= g())}.
If {\tt C} has no properties, the only type that can be
constructed is the type {\tt C}.

Accordingly, a {\em constrained type} is of the form {\tt C(:e)}, the name of
a class or interface {\tt C}, called the {\em base class}, followed by a
where clause {\tt e}, called the {\em condition}, a boolean expression
on the properties of the base class. 
The denotation, or
semantic interpretation, of such a type is the set of all instances
of subclasses of the base class whose properties satisfy the
condition.
Clearly, for the denotation of a constrained type $t$ to be
non-empty the condition of $t$ must be consistent with the class
invariant, if any, of the base class of $t$.  The compiler is required to
ensure that the type of any variable declaration is non-empty.

For brevity, we write {\tt C} as a type as well; it
corresponds to the (vacuously) constrained type {\tt C(:true)}.
We also permit the syntax {\tt C(t1,\ldots, tk)} for
the type {\tt C (:x1 == t1 \&\& \ldots \&\& xk == tk)} (assuming that
the property list for {\tt C} specifies the {\tt k} properties {\tt
x1,\ldots, xk}, and each term {\tt ti} is of the correct
type). Thus, using the definition above, {\tt List(n)} is the type of
all lists of length {\tt n}.

Constrained types naturally come equipped with a {\em subtyping
structure}: type $t_1$ is a subtype of $t_2$ if the denotation of
$t_1$ is a subset of $t_2$. This definition satisfies
Liskov's Substitution Principle~\cite{liskov-behaviors}),
and implies that
{\tt C(:e1)} is a subtype of {\tt C(:e2)} if {\tt e1} implies {\tt e2}.
In particular, for all conditions {\tt e},
{\tt C(:e)} is a subtype of {\tt C}.
{\tt C(:e)} is empty exactly
when {\tt e} conjoined with {\tt C}'s class invariant is inconsistent.

Two constrained types {\tt C1(:e1)} and {\tt C2(:e2)} are considered equivalent
if {\tt C1} and {\tt C2} are the same base type and {\tt e1} and
{\tt e2} are equivalent when considered as logical expressions.

\subsection{Method and constructor preconditions}

Methods and constructors may specify preconditions
on their parameters as where clauses.
For an invocation of a method (or
constructor) to be type-correct, the
associated where clause must be statically known to be satisfied. The
return type of a method may also contain expressions involving the
arguments to the method. However, we will require
that any argument used in this way must be declared {\tt final},
ensuring it is not mutated by the method body.
For instance:
\begin{verbatim}
  List(arg.length-1)
    tail(final List arg : arg.length > 0) {...}
\end{verbatim}
\noindent will be a valid method declaration. It says that
{\tt tail} must be passed a non-empty list, and it returns a list
whose length is one less than its argument.

\subsubsection{Constructors for dependent classes}

Like a method definition,
a constructor may
specify preconditions on its arguments
and a postcondition on the value produced by the constructor.

Postconditions may be specified in a constructor declaration between
the name of the class and the argument list of the constructor using a
where clause. The where clause can reference only the properties of
the class.

For instance, the
nullary constructor for {\tt List} ensures that the property
{\tt length} has the value {\tt 0}:
{\footnotesize
\begin{verbatim}
    public List(0)() { property(0); }
\end{verbatim}}
The {\tt property} statement is used to set all the properties
of the new object simultaneously.  Capturing this assignment in
a single statement simplifies checking that the constructor
postcondition and class invariant are established.  If a class
has properties, every path through the constructor must contain
exactly one {\tt property} statement.

%% Cannot throw an exception.

%% Figure out the real condition. Not sure this is important.

\subsection{Constraints}

In this framework, types may be constrained by any boolean
expression over the properties.  For practical reasons,
restrictions need to be imposed to ensure constraint checking is
decidable.

The condition of a constrained type must be a pure
function only of the properties of the base class.
Because properties are
{\tt final} instance fields of the object,
this requirement
ensures that whether or not an object belongs to a constrained type does
not depend on the {\em mutable} state of the object.
That is, the status of the
predicate ``this object belongs to this type'' does not
change over the lifetime of the object.  Second, by insisting that each
property be a {\em field} of the object, the question of
whether an object is of a given type can be
determined merely by examining the state of the object and evaluating
a boolean expression. Of course, an implementation is free to not {\em
explicitly} allocate memory in the object for such fields. For
instance, it may use some scheme of colored pointers to implicitly
encode the values of these fields~\cite{???}.

Further, by requiring that the programmer distinguish certain {\tt
final} fields of a class as properties, we ensure that the programmer
consciously controls {\em which} {\tt final} fields should be available for
constructing constrained types. A field that is ``accidentally'' {\tt
final} may not be used in the construction of a constrained type. It must be
declared as a property.

\java{}-like languages permit constructors to throw exceptions. This
is necessary to deal with the situation in which the arguments to a
constructor for a class {\tt C} are such that no object can be
constructed which satisfies the invariants for {\tt C}. Dependent
types make it possible to perform some of these checks at
compile-time. The class invariant of a class explicitly captures
conditions on the properties of the class that must be satisfied by
any instance of the class.  Constructor preconditions capture
conditions on the constructor arguments.
The compiler's static check for
non-emptiness of the type of any variable captures these invariant
violations at compile-time.




\subsection{Extending dependent classes}

A class may extend a constrained class.

{\em MetaNote: This should be standard. A class definition may extend
a dependent super class, e.g. class Foo(int i) extends Fum(i*i) \{
\ldots \}. The expressions in the actual parameter list for the super
class may involve only the properties of the class being defined. The
intuition is that these parameters are analogous to explicit arguments
that must be passed in every super-constructor invocation.}

\subsection{Dependent interfaces}

\java{} does not allow interfaces to specify instance fields. Rather all
fields in an interface are final static fields (constants).

X10 supports rich user-definable extensions to the type system by
allowing the user of a type to construct new constrained types: new
types that are predicates on the immutable state of the base type.
For interfaces to support this extension, they must support
user-definable properties, so that constrained types can be
built over interfaces.

As with classes, an interface definition may specify properties
in a 
list after the name of the interface. Similarly, an interface
definition may specify a where clause in its property list. Methods
in the body of an interface may have where clauses
as well.

All classes implementing an interface must have a property
with the same name and
type (either declared in the class or inherited from the superclass)
for each property in the interface. If a class implements
multiple interfaces and more than one of them specify a property
with the same name, then they must all agree on the type of the
property. The class must have a single property with the given name
and type.

The general form of a class declaration is now:
\begin{verbatim}
  class C(T1 x1, ..., Tk xk)
        extends B(:e)
        implements I1(:e1), ..., In(:en) {...}
\end{verbatim}
\noindent
For such a
declaration to type-check, it must be that the class invariant
implies {\tt inv(I) \&\& e}, where {\tt inv(I)} is the invariant associated with
interface {\tt I}.  Again, a constrained class or interface {\tt I} is taken as
shorthand for {\tt I(:true)}.  Further, every method specified in the
interface must have a corresponding method in the class with the same
signature whose precondition, if any, is implied by the precondition
of the method in the interface.


\subsection{XXX more stuff}

CFJ with field assignments.

Discussion of language design issues

-- how should method resolution be done in the presence of constrained
   types?

-- conditional fields.
-- recursive definitions of predicates in the constraint language
   through the use of CLP.

Constraint system (generic presentation).
Design is constraint-system agnostic.

Principal clause



\subsection{Semantics}
\label{sec:semantics}
In this section we formalize a small fragment of \Xten{} to illustrate
the basic concepts behind constrained type-checking. In fact a very
tiny language is chosen---a small extension of \FJ{} with constrained
types. 

The language is functional in that assignment is not
admitted. However, it is not difficult to introduce the notion of
mutable fields, and assignment to such fields. Since constrained types
may onl refer to immutable state, the validity of these types is not
compromised by the introduction of state.

Further, we do not formalize overloading of methods. Rather, with
\FJ{}, we simply require that the input program be such that the class
name {\tt C} and method name {\tt m} uniquely select the associated
method on the class. 

We do model properties, constrained clauses, class invariants, where
clauses in methods and constructors, and dependent type casts.


\begin{figure*}

\paragraph{Syntax for \CFJ.}
The definitions are based on those in Featherweight Java~\cite{FJ}. 

\begin{tabular}{rrcl}
&&&\\
(Class) & {\tt L} &{::=}& $\tt\class \ C(\bar{T}\ \bar{f}:c)\  \extends\ T\ \{\bar{M}\}$ \\
(Method)& {\tt M} &{::=}& $\tt T\ m(\bar{T}\ \bar{x}:c)\{\return\ e;\}$\\
(Expr)& {\tt e} &{::=}& $\tt x \alt e.f \alt e.m(\bar{e})\alt \new\ C(\bar{e})\alt (T)e$ \\
(Type)& {\tt S},{\tt T},{\tt U},{\tt Z}&{::=}& $\tt C(:d)$\\
&&&\\
\end{tabular}

The {\em base type} of a type {\tt C(:c)} (read as {\em {\tt C} with
{\tt c}}) is {\tt C}.  We use the following shorthand for types: For a
type {\tt T} equal to {\tt C(:c)}, we will write {\tt S\ x; T} for
{\tt C(:S\ x; c)}, and {\tt d,T} for {\tt C(:d,c)}.
Application of substitutions is extended to
types by: ${\tt C(:c)\theta}={\tt C(:c\theta)}$.

\paragraph{\CFJ{} subtyping judgment.}\label{CFJ-subtyping}
We add a single rule to the rules of \FJ:
$$
\begin{array}{llll}
 {\tt C} \subtype {\tt C}
&
\from{\class\ {\tt C(\ldots)}\ \extends\ {\tt D(\ldots)}\{\ldots\}}
\infer{{\tt C} \subtype {\tt D}}
& 
\from{{\tt C} \subtype {\tt D} \ \ \ {\tt D} \subtype {\tt E}}
\infer{{\tt C} \subtype {\tt E}} &
\from{
\begin{array}{ll}
{\tt C} \subtype {\tt D} &
\sigma(\Gamma, {\tt C(:c)}\ {\tt x}) \vdash_{\cal C} {\tt d}[{\tt x}/\self] \ \ \mbox{({\tt x} fresh)}
\end{array}}
\infer{\Gamma \vdash {\tt C(:c) \subtype D(:d)}}
\end{array}
$$

(Whenever we state an assumption of the form ``{\tt x} is
fresh'' in a rule we mean it is not free in the consequent of the
rule.)

\paragraph{\CFJ{} typing judgment.}\label{CFJ-typing}
We let $\Gamma$ stand for multisets of type assertions, of the form
${\tt T\ x}$,\footnote{We use the non-standard notation ${\tt T\ x}$
rather than the more familiar ${\tt x} : {\tt T}$ since {\tt :} is
used in the syntax of a type.}  and constraints. Typing judgments are
of the form $\Gamma\vdash {\tt S\ t}$ When $\Gamma$ is empty, it is
omitted. 

Let {\tt C} be a class declared as ${\tt \class\ C(\bar{\tt T}\
\bar{\tt f}:c)\ extends\ D(:d)\{\bar{\tt M}\}}$. Let
$\theta$ be a substitution and the type {\tt T} be based on {\tt C}.
We define $\inv(T,\theta)$
as the conjunction ${\tt c\theta,d\theta}$ and (recursively)
$\inv({\tt D},\theta)$.  We bottom out with $\inv({\tt
Object},\theta)=\true$. For a variable {\tt x}, we use the shorthand
$\inv({\tt C},{\tt x})$ to mean $\inv({\tt C},[{\tt x}/\self])$.

The definition of {\mtype({\tt C},{\tt m})} (the signature of a method
named {\tt m} in class {\tt C}), {\mbody({\tt C},{\tt m})}, (the body
associated with method {\tt m} in type {\tt C}) and \fields(C) (the
sequence of fields and their types inherited or defined at {\tt C}) is
essentially as specified in FJ~\cite{FJ} with the difference that the
method of a signature is taken to be of the form $\bar{\tt S}\ \bar{\tt
x}: {\tt c} \rightarrow {\tt T}$.  The variables {\tt x} are permitted
to occur in the types $\bar{\tt S},{\tt T}$, and are considered bound,
and subject to alpha-renaming.  The definitions of \mtype, \mbody,
\fields{} are extended to apply to constrained types by ignoring the
constraint.  For a substitution $\theta$ we define $\mtype({\tt
T},{\tt m},\theta)$ as the signature obtained by applying $\theta$ to
$\mtype({\tt T},{\tt m})$, renaming bound variables as necessary.
Similarly, for a substitution $\theta$ we define $\fields({\tt
T},\theta)$ to be $\bar{S}\theta\ \bar{f}$, if the sequence of
inherited and defined fields of the class underlying the type {\tt T}
is $\bar{S}\ \bar{f}$. We let $\fields({\tt T},{\tt x})$ stand for
$\fields({\tt T},[{\tt x}/\self]))$.

We define $\sigma(\Gamma)$ to be the set of
constraints obtained from $\Gamma$ by replacing each type assertion
${\tt C(:d)\ x}$ in $\Gamma$ with ${\tt d}[{\tt x}/\self],\inv(C,x)$
and retaining any constraint in $\Gamma$.

$$
\begin{array}{l}
\begin{array}{lll}
\rname{T-Var}%
\from{\sigma(\Gamma, {\tt C(:c)}\ {\tt x}) \vdash_{\cal C} {\tt d}[{\tt x}/\self]}
\infer{\Gamma, {\tt C(:c)\ x} \vdash {\tt C(:d)}\ {\tt x}} &
\rname{T-Field}%
\from{\Gamma \vdash {\tt T}_0\ {\tt e} \ \ \ \fields({\tt T}_0,{\tt z}_0)= \bar{\tt U}\ \bar{\tt f}_i \ \ \ \mbox{(${\tt z}_0$ fresh)}} 
\infer{\Gamma \vdash ({\tt T}_0\ {\tt z}_0; {\tt z}_0.{\tt f}_i=\self;{\tt U}_i)\ {\tt e.f}_i} 
& 
\rname{T-Cast}%
\from{\Gamma \vdash {\tt S}\ {\tt e}}
\infer{\Gamma \vdash {\tt T}\ {\tt (T) e}}
\end{array}
\\  \quad \\
\begin{array}{ll}
\rname{T-Invk}%
\from{\begin{array}{ll}
\Gamma \vdash {\tt T}_{0:n} \ {\tt e}_{0:n}  &
\mtype({\tt T}_0,{\tt m},{\tt z}_0)= \tt {\tt Z}_{1:n}\ {\tt z}_{1:n}:c \rightarrow {\tt S} \\
\Gamma, {\tt T}_{0:n}\ {\tt z}_{0:n} \vdash {\tt T}_{1:n} \subtype {\tt Z}_{1:n}&
\sigma(\Gamma, {\tt T}_{0:n}\ {\tt z}_{0:n}) \vdash_{\cal C} {\tt c} \ \ \ 
\mbox {(${\tt z}_{0:n}$ fresh)}
\end{array}}
\infer{\Gamma \vdash ({\tt T}_{0:n}\ {\tt z}_{0:n}; S)\ {\tt e}_0.{\tt m({\tt e}_{1:n})}}&
\rname{T-New}%
\from{
  \begin{array}{ll}
    \Gamma \vdash \bar{\tt T}\ \bar{\tt e} \ \ \
  \theta=[\bar{\tt f}/\this.\bar{\tt f}] & 
    \fields(C,\theta)=\bar{\tt Z}\ \bar{\tt f} \\
    \Gamma, \bar{\tt T}\ \bar{\tt f} \vdash \bar{\tt T} \subtype \bar{\tt Z} &
    \sigma(\Gamma, \bar{\tt T}\ \bar{\tt f}) \vdash_{\cal C} \inv({\tt C},\theta) 
  \end{array}
}
\infer{\Gamma \vdash {\tt C(:\bar{T}\ \bar{\tt f}{\tt ;\self.\bar{f}}=\bar{\tt f})\ \new\ {\tt C(\bar{\tt e})}}} \\
\end{array}
\end{array}
$$
\paragraph{Method and class typing.}
$$
\begin{array}{ll}
\from{ \bar{\tt T}\ \bar{\tt x}, {\tt C}\ \this, {\tt c} \vdash {\tt S}\ {\tt e}, {\tt S} \subtype {\tt T} }   
\infer{\tt T\ m(\bar{\tt T}\,\bar{\tt x} : c)\{\return\ e;\}\ \mbox{OK in}\ C} &
\from{\bar{M}\ \mbox{OK in}\ C}
\infer{\tt \class\ C(\bar{\tt T}\ \bar{\tt f}:c)\ \extends\ D(:d)\ \{\bar{M}\}\ \mbox{OK}} 
\end{array}
$$

\paragraph{Computation.}
$$
\begin{array}{ccc}
\rname{{\sc R-Field}}%
\from{\fields(C)=\bar{C}\ \bar{f}}
\infer{(\new\ {\tt C}(\bar{\tt e})).{\tt f}_i \derives {\tt e}_i} &
\rname{{\sc R-Invk}}%
\from{mbody({\tt m},{\tt C})=\bar{x}. {\tt e}_0}
\infer{(\new\ {\tt C}(\bar{\tt e})).{\tt m}(\bar{\tt d}) \derives 
[\bar{d}/\bar{x},\new\ C(\bar{e})/\this]{\tt e}_0} &
\rname{{\sc R-Cast}}%
\from{\vdash C \subtype T[\new\ C(\bar{\tt d})/\self]}
\infer{{\tt (T)(\new\ C(\bar{\tt d}))} \derives \new\ C(\bar{\tt d})}
\end{array}
$$
\paragraph{Congruence.}
$$
\begin{array}{l}
\begin{array}{ccc}
\rname{{\sc RC-Field}}%
\from{{\tt e}_0 \derives {{\tt e}_0}'}
\infer{{\tt e}_0.{\tt f} \derives {{\tt e}_0}'.{\tt f}} &
\rname{{\sc RC-Invk-Recv}}%
\from{{\tt e}_0 \derives {{\tt e}_0}'}
\infer{{\tt e}_0.{\tt m}(\bar{\tt e}) \derives {{\tt e}_0}'.{\tt m}(\bar{\tt e})} &
\rname{{\sc RC-Invk-Arg}}%
\from{{\tt e}_i \derives {{\tt e}_i}'}
\infer{{\tt e}_0.{\tt m}(\ldots,{\tt e}_i,\ldots) \derives {{\tt e}_0}.{\tt m}(\ldots,{\tt e}_i',\ldots)} 
\end{array}
\\ \quad \\
\begin{array}{cc}
\rname{{\sc RC-New-Arg}}%
\from{{\tt e}_i \derives {{\tt e}_i}'}
\infer{\new\ {\tt C}(\ldots,{\tt e}_i,\ldots) \derives \new\ {\tt C}(\ldots,{\tt e}_i',\ldots)} &
\rname{{\sc RC-Cast}}%
\from{{\tt e}_0 \derives {{\tt e}_0}'}
\infer{{\tt (C) e}_0 \derives {{\tt (C) e}_0}'}
\end{array}
\end{array}
$$
\caption{The system Constrained \FJ}\label{CFJ-red-rules}
\end{figure*}

\paragraph{Constraint system.}
Constraints are assumed to be drawn from a fixed constraint system,
$\cal C$, with inference relation $\vdash_{\cal C}$ \cite{CCCC}.
All
constraint systems are required to support the trivial constraint
\true, conjunction, existential quantification and equality on
constraint terms. Constraint terms include (final) variables, the
special variable {\tt self} (which may occur only in constraints {\tt
c} which occur in a constrained type {\tt C(:c)}), and field
selections {\tt t.f}. 

We summarize here properties of constraint systems described in
\cite{CCCC} that are needed for the proofs: constraint systems may be
thought of as presented via an intuitionistic Gentzen proof system
supporting identity; affine and exchange on the left; existential
quantification and conjunction on the left and right; and closure
under substitution of terms. We denote the application of the
substitution $\theta=[\bar{{\tt t}}/\bar{{\tt x}}]$ to a constraint ${\tt c}$ by
${\tt c}[\bar{{\tt t}}/\bar{{\tt x}}]$. 

\begin{tabular}{rrcl}
&&&\\
(C Term) & {\tt t} &{::=}& {\tt  x}\alt \self \alt \this \alt {\tt t.f} \\
&&& \alt \new\ {\tt C($\bar{\tt t}) \alt {\tt g}(\bar{\tt t})$}\\
(Constraint) & {\tt c},{\tt d} &{::=}&$\true\alt {\tt p}(\bar{\tt t})$\\
&&& $\alt {\tt t=t}\alt {\tt c,c}\alt{\tt  T\,x;c}$\\
&&&\\
\end{tabular}

All constraint systems are required to satisfy: $\new\ {\tt C(\bar{\tt
t})}.{\tt f}_i={\tt t}_i $ provided that $\fields({\tt C})=\bar{\tt
T}\ \bar{\tt f}$ (for some sequence of types $\bar{\tt T}$).

Above, ``,'' binds tighter than ``;''. We use the syntax {\tt
{\tt T\;x};\;c} for the constraint obtained by existentially quantifying the
variable {\tt x} of type {\tt T} in {\tt c}. {\tt p} ranges over
the collection of predicates supplied by the underlying constraint
system, and {\tt g} over the collection of functions.


\paragraph{Syntax.}
The syntax for the language is specified in Figure~\ref{CFJ-red-rules}.

A type is taken to be of the form {\tt C(:c)} where {\tt C} is the
name of a class or interface and {\tt c} is a constraint; we say that
{\tt C} is the {\em base} of the type {\tt C(:c)}.

A type assertion {\tt C(:c) x} constrains the variable {\tt x} to
contain references to only those objects {\tt o} that are instances of
(subclasses of) {\tt C} and for which the constraint {\tt c} is true
provided that occurrences of {\tt self} in {\tt c} are replaced by
{\tt o}. Thus in the constraint {\tt c} of a constrained type {\tt
C(:c)}, {\tt self} may be used to reference the object whose type is
being specified. Note that {\tt self} is distinct from
{\tt this}---{\tt this} is permitted to occur in the clause of
a type {\tt T} only
if {\tt T} occurs in an instance field declaration or instance method
declaration of a class; as usual, \this{} is considered bound to the
instance of the class to which the field or method declaration
applies.

A {\em class declaration} $\tt\class \ C(\bar{T}\ \bar{f}:c)\ \extends\ D(:d)\
\{\bar{M}\}$ is thought of as declaring a class {\tt C} with the
fields $\bar{\tt f}$ (of type $\bar{\tt T}$), a {\em declared class
invariant} {\tt c}, a {\em super-class invariant} {\tt d} and a
collection of methods $\bar{\tt M}$. The constraints {\tt c} and {\tt
d} are true for all instances of the class {\tt C} (this is verified
in the rule for type-checking constructors, \rn{T-New}).  In these
constraints, {\tt this} may be used to reference the current object;
\self{} does not have any meaning and must not be used.

A {\em method declaration} ${\tt T}_0\ {\tt m(\bar{\tt T}\ \bar{\tt x} :
c)\{\ldots\}}$ specifies the type of the arguments and the result, as
usual.  The method arguments $\bar{\tt x}$ may occur in the argument
types $\bar{\tt T}$ and the return type ${\tt T}_0$.  The constraint
{\tt c} specifies additional constraints on the arguments $\bar{\tt
x}$ and
\this{} that must hold for a method invocation to be legal. Note that
\self{} does not make sense in {\tt c} (no type is being defined), and must not occur in {\tt c}.

\paragraph{Type judgments.}
Typing judgments are of the form $\Gamma \vdash {\tt T}\ {\tt e}$
where $\Gamma$ is a multiset of type assertions ${\tt T}\ {\tt x}$ and
constraints ${\tt c}$. 
%The constraint entailment relation
%$\vdash_{\cal C}$ is lifted to type assertions through the definition:
%$\Gamma \vdash_{\cal C} {\tt D(:d)}\ {\tt x}$ provided that $\{ {\tt
%c}[{\tt x}/\self] \alt {\tt C(:c)}\ x \in \Gamma\} \cup
%\{{\tt c} \alt  {\tt c} \in \Gamma\} \vdash_{\cal C} {\tt d}[{\tt x}/\self]
%$ and $\Gamma \vdash {\tt D}\ {\tt x}$. Intuitively, $\Gamma \vdash
%\tt D(:d)\ {\tt x}$ if $\Gamma$ constrains {\tt x} to be of type {\tt
%D} and there is enough information in the constraints in $\Gamma$ to
%entail {\tt d} for {\tt x}.

%$\sigma(\Gamma)$ is the set of constraints on the variables whose type
%assertions are specified by $\Gamma$, generated by replacing each type
%assertion {\tt C(:c) x} in $\Gamma$ with ${\tt c}[{\tt x}/\self]$.
%The rule T-Constr permits a type {\tt C(:c)} for a variable {\tt x} to
%be strengthened with information entailed per $\cal C$ from the
%information about {\tt x} and other variables specified in $\Gamma$.
\def\TConstr{\mbox{\sc T-Constr}}
\def\TInv{\mbox{\sc T-Inv}}
\def\TVar{\mbox{\sc T-Var}}
\def\TField{\mbox{\sc T-Field}}
\def\TInvk{\mbox{\sc T-Invk}}
\def\TNew{\mbox{\sc T-New}}
\def\TCast{\mbox{\sc T-Cast}}
\def\TUCast{\mbox{\sc T-UCast}}
\def\TDCast{\mbox{\sc T-DCast}}
\def\TSCast{\mbox{\sc T-SCast}}
%\TConstr{} is a form of cut which permits information obtained through
%constraint entailment to enrich the type of an expression.

\TVar{} extends the identity rule ($\Gamma, x:C \vdash x:C$) of {\sf FJ} to take into account the constraint entailment relation.

\TCast{} encapsulates the three inference rules of {\sf FJ}:
\TUCast{}, \TDCast{} and \TSCast{} for upwards cast, downwards cast, and ``stupid'' cast respectively. 

%\TInv{} is a form of contraction that permits the class invariant {\tt c}
%of a class {\tt C} to enrich the type of any variable of type {\tt C}.

In \TField, we postulate the existence of a receiver object {\tt o} of
the given static type (${\tt T}_0$). $\fields({\tt T}_0,{\tt o})$ is
the set of typed fields for ${\tt T}_0$ with all occurrences of 
\this{} replaced  by {\tt o}. We record in the resulting
constraint that ${\tt o.f}_i=\self$.\footnote{A new name {\tt o} is
necessary to name this object since {\tt e} cannot be used. Arbitrary
term expressions {\tt e} are not permitted in constraints; the
functions used in {\tt e} may not be known to the constraint system,
and {\tt e} may have side-effects.}  This permits transfer of
information that may have been recorded in ${\tt T}_0$ about the field
${\tt f}_i$. 

Similarly, in \TInvk{} we postulate the existence of a receiver object
{\tt o} of the given static type. For any type $T$, object {\tt o} of
type $T$ and method name {\tt m}, let $\mtype({\tt T},{\tt m},{\tt
o})$ be a copy of the signature of the method with \this{} replaced by
{\tt o}. We establish (under the assumption that the formals
($\bar{\tt z}$) have the static type of the actuals)\footnote{This is
stronger than assuming $\bar{\tt Z}$.}  that actual types are subtypes
of the formal types, and the method constraint is satisfied. This
permits us to record the constraint {\tt d} on the return type, with
the formal variables $\bar{\tt z}$ existentially
quantified.\footnote{Recall that the $\bar{\tt z}$ may occur in {\tt
d} but must not occur in a type in the calling environment; hence they
must be existentially quantified in the resulting constraint.}

In \TNew, similarly, we establish that the static types of the actual
arguments to the constructor are subtypes of the declared types of the
field, and contain enough information to satisfy the class invariant,
{\tt c}. The declared types (and {\tt c}) contain references to ${\tt
this.\bar{\tt f}}$; these must be replaced by the formals $\bar{\tt
f}$, which carry information about the static type of the
actuals. Note that the object {\tt o} we hypothesized in an analogous
situation in \TInvk{} does not exist; it will exist on successful
invocation of the constructor. The constrained clause of the \new{}
expression contains all the information that can be gleaned from the
static types of the actuals by assigning them to the corresponding
fields of the object being created.

\begin{theorem}[Subject Reduction] 

If $\Gamma \vdash T\ e$ and $e \derives e'$, then for some type $S$,
$\Gamma \vdash S\ e'$ and $\Gamma \vdash S \subtype T$.

\end{theorem}

Let the normal form of expressions be given by {\em values},
i.e.{} expressions:

\begin{tabular}{rrcl}
&&&\\
(Values) & {\tt v} &{::=}& $\new\ {\tt C(\bar{\tt v})}$
\end{tabular}

\begin{theorem}[Progress] If $\vdash {\tt T\ e}$, then one of the following conditions holds:
\begin{enumerate}
\item {\tt e} is a value {\tt v}, 
\item {\tt e} contains a subexpression ${\tt (T)\new\ C(\bar{\tt
v})}$ such that
$\not\vdash {\tt C} \subtype {\tt T}[{\tt \new\ C(\bar{\tt v})}/\self]$,
\item there exists ${\tt e}'$ s.t. ${\tt e} \derives {\tt e}'$.
\end{enumerate}
\end{theorem}

\begin{theorem}[Type Soundness] 

If $\vdash {\tt T\ e}$ and ${\tt e} \starderives {\tt e}'$, with ${\tt
e}'$ in normal form, then ${\tt e}'$ is either (1)~a value {\tt v}
with $\vdash {\tt S\ v}$ and $\vdash {\tt S
\subtype T}$, for some type {\tt S}, or, (2)~ an expression containing
a subexpression ${\tt (T)\new\ {\tt C(\bar{\tt v})}}$ where 
$\not\vdash \tt C\subtype T[\new\ C(\bar{\tt v})/\self]$.

\end{theorem}

\begin{lemma}[Substitution Lemma]
Assume $\Gamma \vdash \bar{\tt A}\ \bar{\tt d}$, $\Gamma \vdash \bar{\tt A}\subtype \bar{\tt B}$, and $\Gamma, \bar{\tt B}\ \bar{\tt x} \vdash {\tt T}\ {\tt e}$. Then for some type ${\tt S}$ s.t. $\Gamma \vdash {\tt S} \subtype \bar{\tt A}\ \bar{\tt x};{\tt T}$ it is the case that $\Gamma \vdash {\tt S}\ {\tt e}[\bar{\tt d}/\bar{\tt x}]$.
\end{lemma}

% Unchanged from FJ
\begin{lemma}[Weakening]
If $\Gamma \vdash {\tt T}\ {\tt e}$, then $\Gamma, {\tt S}\ {\tt x}\vdash {\tt T}\ {\tt e}$.
\end{lemma}

% Unchanged from FJ
\begin{lemma}[Body type]
If $\mtype({\tt T}_0,{\tt m})=\bar{\tt T}\ \bar{\tt x} : {\tt c}
\rightarrow {\tt S}$, and $mbody({\tt m}, {\tt T}_0)=\bar{\tt x}.{\tt
e}$, then for some ${\tt U}_0$ with ${\tt T}_0 \subtype {\tt U}_0$,
there exists ${\tt V}\subtype {\tt S}$ such that
$\bar{\tt T}\ \bar{\tt x},{\tt U}_0\ \this \vdash {\tt V}\ {\tt e}$
\end{lemma}


\subsection{Erasure}

Constrained types in CFJ are a form of {\em refinement
type}~\cite{refinement-types}.  If constraints are erased from a
well-typed program,
the resulting program will behave identically to the unerased
program except that the unerased program might be unable to take
a step on a cast.

Let $\Lb {\tt e} \Rb$ be the erasure of ${\tt e}$ defined as follows:
\begin{align*}
\Lb {\tt x} \Rb &= {\tt x} \\
\Lb {\tt e}.{\tt f} \Rb &= \Lb {\tt e} \Rb.{\tt f} \\
\Lb {\tt e}.{\tt m}(\bar{\tt e}) \Rb &= \Lb {\tt e} \Rb.{\tt m}(\bar{\Lb {\tt e} \Rb}) \\
\Lb {\tt new}~{\tt C}(\bar{\tt e}) \Rb &= {\tt new}~{\tt C}(\bar{\Lb {\tt e} \Rb}) \\
\Lb ({\tt C}(:{\tt c}))~{\tt e} \Rb &= (\tt C(: {\tt true}))~\bar{\Lb {\tt e} \Rb}
\end{align*}

\begin{theorem}[Erasure]

If $\vdash {\tt C}(:{\tt c})\ {\tt e}$ and ${\tt e} \starderives {\tt v}$,
then $\vdash {\tt C}(:{\tt true})\ \Lb {\tt e} \Rb$ and $\Lb
{\tt e} \Rb \starderives \Lb {\tt v} \Rb$.

\end{theorem}



\section{Applied constrained calculii}\label{sec:examples}
\section{Examples}
\todo{Bring in other examples from Concur paper.}
Consider the core of the ASCI Benchmark Sweep3D program for computing
solutions to mass transport problems.

In a nutshell the core computation is a triply nested sequential loop
in which the value of a variable in the current iteration is dependent
on the values of neighboring variables in a past iteration. Such a
problem can be parallelized through pipelining. One visualizes a
diagonal wavefront sweeping through the array. An MPI version of the
program may be described as follows. There is a two dimensional grid
of processors which performs the following computation
repeatedly. Each processor synchronously receives a value from the
processor to its west, then to its north, then computes some function
of these values and computes a new value to be sent to the processor
to its east and then to its south.  Ignoring the behavior of the
boundary processors for the moment such a computation may be described
by the following \Xten{} program:

\begin{x10}
region R = [1..n0,1..m0];
clock[R] W,N;
clock(W) final double [cyclic(R)] A; 
for (int t : 1..TMax) \{
  ateach( i,j:A) 
    clock (W[i-1,j],N[i,j-1],W[i,j],N[i,j]) \{
      double west = now (W[i-1,j]) future\{A[i-1,j]\}; 
      W[i-1,j].continue();           
      double north = now (N[i,j-1]) future\{A[i,j-1]\}; 
      N[i,j-1].continue();
      next(W[i,j]) A[i,j] = compute(west, north);
      next W[i-1,j],N[i,j-1],W[i,j],N[i,j];
  \}
\}
\end{x10}


\section{Implementation}\label{sec:implementation}
%IV. Implementation (0.5 page)
%
%Specify what has been implemented and how. What is interesting about
%the implementation.

The dependent type system is implemented in the \Xten{}
compiler~\cite{X10}, which is implemented as an extension of
Java using the Polyglot compiler framework~\cite{ncm03}.

Polyglot implements a source-to-source base Java compiler 
that is extended to translate \Xten{} to Java.  In describing the
implementation, we ignore the additional statement and expression types
introduced in \Xten{} and treat the language simply as Java
extended with constrained types.

Constraints in \Xten{} are conjunctions of equalities over immutable
side-effect-free expressions.  Expressions used in constrained
types are type-checked as normal non-dependent \Xten{} expressions;
no constraint solving is performed.
A second compiler pass generates and solves constraints via an
ask--tell interface~\cite{my-thesis-book}.
If constraints cannot be solved, an error is reported.

After constraint-checking, the \Xten{} code is translated to Java.
The basic idea behind the translation is simple. Each dependent class
is translated into a single class of the same name without dependent
types. The explicit properties of the dependent class are translated
into {\tt public final} (instance) fields of the target class.
A {\tt property} statement in a constructor is translated to a
sequence of assignments to initialize the property fields.

For each property, a getter method is also generated in the
target Java class.
Properties declared in interfaces are translated into getter
method signatures.  Subclasses implementing these interfaces
thus provide the required properties by implementing the
generated interfaces.

Usually, constrained types are simply translated to
non-constrained types by erasure; constraints are checked
statically and need no run-time representation.
However, dependent types may be used in casts
and {\tt instanceof} expressions.  Because the constraint syntax
in \Xten{} is a subset of the \Xten{} expression syntax, run-time tests
of constrained types are translated to Java
in straightforward manner by evaluating the constraint with
{\tt self} bound to the expression being tested.
For examples, casts are translated as:
\eat{
\begin{code}
  $\Lb$e instanceof C(:c)$\Rb$ = 
    new Object() \{
      boolean check(Object o) \{
        if (o instanceof C) \{
          C self = (C) o;
          return $\Lb$c$\Rb$;
        \}
        return false;
      \}
    \}.check($\Lb$e$\Rb$)
\end{code}
}
\begin{code}
  $\Lb$(C(:c) e$\Rb$ = 
    new Object() \{
      C cast(C self) \{
        if ($\Lb$c$\Rb$) return self;
        throw new ClassCastException(); \}
    \}.cast((C) $\Lb$e$\Rb$)
\end{code}
\noindent Wrapping the evaluation of {\tt c} in an anonymous class
ensures the expression {\tt e} is evaluated only once.



\section{Conclusion and Future work}\label{sec:future}\label{sec:conclusions}
%V. Conclusion and future work. (0.5 page)
%
%state-dependent constrained types.
%
%use of dependent types for optimization. 
%
%type-inference.
%
%Bibliography (1.5 page)

State-dependent constrained types

Use of dependent types for optimization

Constraints on control-flow

Type inference


\section{Related work}\label{sec:related}
%\paragraph{Related work}
Adaptive batching bears some similarities to the steal-half algorithm
of Shavit et al, and its variants. Both approaches attempt to cope
with non-hierarchical workloads for graph problems. In the steal-half
algorithm, each node is queued as its own task; and thieves take half
(or some other percentage) of the nodes available per steal
attempt. In contrast, in our approach, the tasks are pre-batched, so
only one batch is stolen at a time. This can substantially reduce
queue overhead, contention and data movement costs, but comes with
potential disadvantages because nodes cannot be stolen while they are
being batched, and batches cannot be re-split.  For example, our
approach does not allow for a subset of the nodes from a stolen batch
to themselves be re-stolen by other threads (as does
steal-half). However, queue-sensing adaptation makes consequent
impediments to global progress highly unlikely.  Because we adaptively
choose batch sizes so that there are always (during steady state
processing) some nodes available to be stolen from each active thread,
imbalanced progress by any one of them has little impact on the
ability of others to find and steal new work.  Additionally, by
relating batching rules to sequential processing thresholds needed for
any work-stealing program, our approach supports simpler empirically
guided performance tuning.
\section{Conclusion}\label{sec:concl}
%\paragraph{Conclusion}
In this paper we have shown how several graph algorithms can be
expressed concisely and elegantly in \Xten. These algorithms rely
heavily on support for fine-grained concurrency. The \Xten{} runtime
(\XWS) implements fine-grained concurrency through an enhanced
work-stealing scheduler. Specifically the scheduler supports
improperly nested tasks, detection of global termination, and phased
work-stealing.  We measure the performance of spanning tree algorithms
implemented with pseudo-depth-first search and breadth-first search on
two multicore systems. We also present a strategy to adaptively control the granularity of parallel tasks in the work-stealing scheme. We show that the \XWS{} programs scale and
exhibit performance comparable with hand-written C programs.

\paragraph{Acknowledgements} 
We thank Raj Barik for his contributions to the implementation of the
C++ version of \XWS. We thank the rest of the \Xten{} team for many
discussions of these issues. This material is based upon work
supported by the Defense Advanced Research Projects Agency under its
Agreement No.  HR0011-07-9-0002.


\section*{Acknowledgments}

Igor Peshansky,
Lex Spoon,
Vincent Cave.

This material is based upon work supported by the Defense
Advanced Research Projects Agency under its Agreement No.
HR0011-07-9-0002.


\bibliographystyle{plain}
\bibliography{master}
\balance

% \appendix
% \onecolumn

% \section{An extended example}
% {\footnotesize
\begin{verbatim}
/**
   A distributed binary tree.
   @author Satish Chandra 4/6/2006
   @author vj
 */
//                             ____P0
//                            |     |
//                            |     |
//                          _P2  __P0
//                         |  | |   |
//                         |  | |   |
//                        P3 P2 P1 P0
//                         *  *  *  *
// Right child is always on the same place as its parent;
// left child is at a different place at the top few levels of the tree,
// but at the same place as its parent at the lower levels.

class Tree(localLeft: boolean,
           left: nullable Tree(& localLeft => loc=here),
           right: nullable Tree(& loc=here),
           next: nullable Tree) extends Object {
    def postOrder:Tree = {
        val result:Tree = this;
        if (right != null) {
            val result:Tree = right.postOrder();
            right.next = this;
            if (left != null) return left.postOrder(tt);
        } else if (left != null) return left.postOrder(tt);
        this
    }
    def postOrder(rest: Tree):Tree = {
        this.next = rest;
        postOrder
    }
    def sum:int = size + (right==null => 0 : right.sum()) + (left==null => 0 : left.sum)
}
value TreeMaker {
    // Create a binary tree on span places.
    def build(count:int, span:int): nullable Tree(& localLeft==(span/2==0)) = {
        if (count == 0) return null;
        {val ll:boolean = (span/2==0);
         new Tree(ll,  eval(ll => here : place.places(here.id+span/2)){build(count/2, span/2)},
           build(count/2, span/2),count)}
    }
}
\end{verbatim}}

\subsection{Places}
{\footnotesize
\begin{verbatim}
/**

 * This class implements the notion of places in X10. The maximum
 * number of places is determined by a configuration parameter
 * (MAX_PLACES). Each place is indexed by a nat, from 0 to MAX_PLACES;
 * thus there are MAX_PLACES+1 places. This ensures that there is
 * always at least 1 place, the 0'th place.

 * We use a dependent parameter to ensure that the compiler can track
 * indices for places.
 *
 * Note that place(i), for i <= MAX_PLACES, can now be used as a non-empty type.
 * Thus it is possible to run an async at another place, without using arays---
 * just use async(place(i)) {...} for an appropriate i.

 * @author Christoph von Praun
 * @author vj
 */

package x10.lang;

import x10.util.List;
import x10.util.Set;

public value class place (nat i : i <= MAX_PLACES){

    /** The number of places in this run of the system. Set on
     * initialization, through the command line/init parameters file.
     */
    config nat MAX_PLACES;

    // Create this array at the very beginning.
    private constant place value [] myPlaces = new place[MAX_PLACES+1] fun place (int i) {
	return new place( i )(); };

    /** The last place in this program execution.
     */
    public static final place LAST_PLACE = myPlaces[MAX_PLACES];

    /** The first place in this program execution.
     */
    public static final place FIRST_PLACE = myPlaces[0];
    public static final Set<place> places = makeSet( MAX_PLACES );

    /** Returns the set of places from first place to last place.
     */
    public static Set<place> makeSet( nat lastPlace ) {
	Set<place> result = new Set<place>();
	for ( int i : 0 .. lastPlace ) {
	    result.add( myPlaces[i] );
	}
	return result;
    }

    /**  Return the current place for this activity.
     */
    public static place here() {
	return activity.currentActivity().place();
    }

    /** Returns the next place, using modular arithmetic. Thus the
     * next place for the last place is the first place.
     */
    public place(i+1 % MAX_PLACES) next()  { return next( 1 ); }

    /** Returns the previous place, using modular arithmetic. Thus the
     * previous place for the first place is the last place.
     */
    public place(i-1 % MAX_PLACES) prev()  { return next( -1 ); }

    /** Returns the k'th next place, using modular arithmetic. k may
     * be negative.
     */
    public place(i+k % MAX_PLACES) next( int k ) {
	return places[ (i + k) % MAX_PLACES];
    }

    /**  Is this the first place?
     */
    public boolean isFirst() { return i==0; }

    /** Is this the last place?
     */
    public boolean isLast() { return i==MAX_PLACES; }
}
\end{verbatim}}
\subsection{$k$-dimensional regions}
{\footnotesize
\begin{verbatim}
package x10.lang;

/** A region represents a k-dimensional space of points. A region is a
 * dependent class, with the value parameter specifying the dimension
 * of the region.
 * @author vj
 * @date 12/24/2004
 */

public final value class region( int dimension : dimension >= 0 )  {

    /** Construct a 1-dimensional region, if low <= high. Otherwise
     * through a MalformedRegionException.
     */
    extern public region (: dimension==1) (int low, int high)
        throws MalformedRegionException;

    /** Construct a region, using the list of region(1)'s passed as
     * arguments to the constructor.
     */
    extern public region( List(dimension)<region(1)> regions );

    /** Throws IndexOutOfBoundException if i > dimension. Returns the
        region(1) associated with the i'th dimension of this otherwise.
     */
    extern public region(1) dimension( int i )
        throws IndexOutOfBoundException;


    /** Returns true iff the region contains every point between two
     * points in the region.
     */
    extern public boolean isConvex();

    /** Return the low bound for a 1-dimensional region.
     */
    extern public (:dimension=1) int low();

    /** Return the high bound for a 1-dimensional region.
     */
    extern public (:dimension=1) int high();

    /** Return the next element for a 1-dimensional region, if any.
     */
    extern public (:dimension=1) int next( int current )
        throws IndexOutOfBoundException;

    extern public region(dimension) union( region(dimension) r);
    extern public region(dimension) intersection( region(dimension) r);
    extern public region(dimension) difference( region(dimension) r);
    extern public region(dimension) convexHull();

    /**
       Returns true iff this is a superset of r.
     */
    extern public boolean contains( region(dimension) r);
    /**
       Returns true iff this is disjoint from r.
     */
    extern public boolean disjoint( region(dimension) r);

    /** Returns true iff the set of points in r and this are equal.
     */
    public boolean equal( region(dimension) r) {
        return this.contains(r) && r.contains(this);
    }

    // Static methods follow.

    public static region(2) upperTriangular(int size) {
        return upperTriangular(2)( size );
    }
    public static region(2) lowerTriangular(int size) {
        return lowerTriangular(2)( size );
    }
    public static region(2) banded(int size, int width) {
        return banded(2)( size );
    }

    /** Return an \code{upperTriangular} region for a dim-dimensional
     * space of size \code{size} in each dimension.
     */
    extern public static (int dim) region(dim) upperTriangular(int size);

    /** Return a lowerTriangular region for a dim-dimensional space of
     * size \code{size} in each dimension.
     */
    extern public static (int dim) region(dim) lowerTriangular(int size);

    /** Return a banded region of width {\code width} for a
     * dim-dimensional space of size {\code size} in each dimension.
     */
    extern public static (int dim) region(dim) banded(int size, int width);


}

\end{verbatim}}

\subsection{Point}
{\footnotesize
\begin{verbatim}
package x10.lang;

public final class point( region region ) {
    parameter int dimension = region.dimension;
    // an array of the given size.
    int[dimension] val;

    /** Create a point with the given values in each dimension.
     */
    public point( int[dimension] val ) {
        this.val = val;
    }

    /** Return the value of this point on the i'th dimension.
     */
    public int valAt( int i) throws IndexOutOfBoundException {
        if (i < 1 || i > dimension) throw new IndexOutOfBoundException();
        return val[i];
    }

    /** Return the next point in the given region on this given
     * dimension, if any.
     */
    public void inc( int i )
        throws IndexOutOfBoundException, MalformedRegionException {
        int val = valAt(i);
        val[i] = region.dimension(i).next( val );
    }

    /** Return true iff the point is on the upper boundary of the i'th
     * dimension.
     */
    public boolean onUpperBoundary(int i)
        throws IndexOutOfBoundException {
        int val = valAt(i);
        return val == region.dimension(i).high();
    }

    /** Return true iff the point is on the lower boundary of the i'th
     * dimension.
     */
    public boolean onLowerBoundary(int i)
        throws IndexOutOfBoundException {
        int val = valAt(i);
        return val == region.dimension(i).low();
    }
}
\end{verbatim}}

\subsection{Distribution}
{\footnotesize
\begin{verbatim}
package x10.lang;

/** A distribution is a mapping from a given region to a set of
 * places. It takes as parameter the region over which the mapping is
 * defined. The dimensionality of the distribution is the same as the
 * dimensionality of the underlying region.

   @author vj
   @date 12/24/2004
 */

public final value class distribution( region region ) {
    /** The parameter dimension may be used in constructing types derived
     * from the class distribution. For instance,
     * distribution(dimension=k) is the type of all k-dimensional
     * distributions.
     */
    parameter int dimension = region.dimension;

    /** places is the range of the distribution. Guranteed that if a
     * place P is in this set then for some point p in region,
     * this.valueAt(p)==P.
     */
    public final Set<place> places; // consider making this a parameter?

    /** Returns the place to which the point p in region is mapped.
     */
    extern public place valueAt(point(region) p);

    /** Returns the region mapped by this distribution to the place P.
        The value returned is a subset of this.region.
     */
    extern public region(dimension) restriction( place P );

    /** Returns the distribution obtained by range-restricting this to Ps.
        The region of the distribution returned is contained in this.region.
     */
    extern public distribution(:this.region.contains(region))
        restriction( Set<place> Ps );

    /** Returns a new distribution obtained by restricting this to the
     * domain region.intersection(R), where parameter R is a region
     * with the same dimension.
     */
    extern public (region(dimension) R) distribution(region.intersection(R))
        restriction();

    /** Returns the restriction of this to the domain region.difference(R),
        where parameter R is a region with the same dimension.
     */
    extern public (region(dimension) R) distribution(region.difference(R))
        difference();

    /** Takes as parameter a distribution D defined over a region
        disjoint from this. Returns a distribution defined over a
        region which is the union of this.region and D.region.
        This distribution must assume the value of D over D.region
        and this over this.region.

        @seealso distribution.asymmetricUnion.
     */
    extern public (distribution(:region.disjoint(this.region) &&
                                dimension=this.dimension) D)
        distribution(region.union(D.region)) union();

    /** Returns a distribution defined on region.union(R): it takes on
        this.valueAt(p) for all points p in region, and D.valueAt(p) for all
        points in R.difference(region).
     */
    extern public (region(dimension) R) distribution(region.union(R))
        asymmetricUnion( distribution(R) D);

    /** Return a distribution on region.setMinus(R) which takes on the
     * same value at each point in its domain as this. R is passed as
     * a parameter; this allows the type of the return value to be
     * parametric in R.
     */
    extern public (region(dimension) R) distribution(region.setMinus(R))
        setMinus();

    /** Return true iff the given distribution D, which must be over a
     * region of the same dimension as this, is defined over a subset
     * of this.region and agrees with it at each point.
     */
    extern public (region(dimension) r)
        boolean subDistribution( distribution(r) D);

    /** Returns true iff this and d map each point in their common
     * domain to the same place.
     */
    public boolean equal( distribution( region ) d ) {
        return this.subDistribution(region)(d)
            && d.subDistribution(region)(this);
    }

    /** Returns the unique 1-dimensional distribution U over the region 1..k,
     * (where k is the cardinality of Q) which maps the point [i] to the
     * i'th element in Q in canonical place-order.
     */
    extern public static distribution(:dimension=1) unique( Set<place> Q );

    /** Returns the constant distribution which maps every point in its
        region to the given place P.
    */
    extern public static (region R) distribution(R) constant( place P );

    /** Returns the block distribution over the given region, and over
     * place.MAX_PLACES places.
     */
    public static (region R) distribution(R) block() {
        return this.block(R)(place.places);
    }

    /** Returns the block distribution over the given region and the
     * given set of places. Chunks of the region are distributed over
     * s, in canonical order.
     */
    extern public static (region R) distribution(R) block( Set<place> s);


    /** Returns the cyclic distribution over the given region, and over
     * all places.
     */
    public static (region R) distribution(R) cyclic() {
        return this.cyclic(R)(place.places);
    }

    extern public static (region R) distribution(R) cyclic( Set<place> s);

    /** Returns the block-cyclic distribution over the given region, and over
     * place.MAX_PLACES places. Exception thrown if blockSize < 1.
     */
    extern public static (region R)
        distribution(R) blockCyclic( int blockSize)
        throws MalformedRegionException;

    /** Returns a distribution which assigns a random place in the
     * given set of places to each point in the region.
     */
    extern public static (region R) distribution(R) random();

    /** Returns a distribution which assigns some arbitrary place in
     * the given set of places to each point in the region. There are
     * no guarantees on this assignment, e.g. all points may be
     * assigned to the same place.
     */
    extern public static (region R) distribution(R) arbitrary();

}
\end{verbatim}}

\subsection{Arrays}
Finally we can now define arrays. An array is built over a
distribution and a base type.

{\footnotesize
\begin{verbatim}
package x10.lang;

/** The class of all  multidimensional, distributed arrays in X10.

    <p> I dont yet know how to handle B@current base type for the
    array.

 * @author vj 12/24/2004
 */

public final value class array ( distribution dist )<B@P> {
    parameter int dimension = dist.dimension;
    parameter region(dimension) region = dist.region;

    /** Return an array initialized with the given function which
        maps each point in region to a value in B.
     */
    extern public array( Fun<point(region),B@P> init);

    /** Return the value of the array at the given point in the
     * region.
     */
    extern public B@P valueAt(point(region) p);

    /** Return the value obtained by reducing the given array with the
        function fun, which is assumed to be associative and
        commutative. unit should satisfy fun(unit,x)=x=fun(x,unit).
     */
    extern public B reduce(Fun<B@?,Fun<B@?,B@?>> fun, B@? unit);


    /** Return an array of B with the same distribution as this, by
        scanning this with the function fun, and unit unit.
     */
    extern public array(dist)<B> scan(Fun<B@?,Fun<B@?,B@?>> fun, B@? unit);

    /** Return an array of B@P defined on the intersection of the
        region underlying the array and the parameter region R.
     */
    extern public (region(dimension) R)
        array(dist.restriction(R)())<B@P>  restriction();

    /** Return an array of B@P defined on the intersection of
        the region underlying this and the parametric distribution.
     */
    public  (distribution(:dimension=this.dimension) D)
        array(dist.restriction(D.region)())<B@P> restriction();

    /** Take as parameter a distribution D of the same dimension as *
     * this, and defined over a disjoint region. Take as argument an *
     * array other over D. Return an array whose distribution is the
     * union of this and D and which takes on the value
     * this.atValue(p) for p in this.region and other.atValue(p) for p
     * in other.region.
     */
    extern public (distribution(:region.disjoint(this.region) &&
                                dimension=this.dimension) D)
        array(dist.union(D))<B@P> compose( array(D)<B@P> other);

    /** Return the array obtained by overlaying this array on top of
        other. The method takes as parameter a distribution D over the
        same dimension. It returns an array over the distribution
        dist.asymmetricUnion(D).
     */
    extern public (distribution(:dimension=this.dimension) D)
        array(dist.asymmetricUnion(D))<B@P> overlay( array(D)<B@P> other);

    extern public array<B> overlay(array<B> other);

    /** Assume given an array a over distribution dist, but with
     * basetype C@P. Assume given a function f: B@P -> C@P -> D@P.
     * Return an array with distribution dist over the type D@P
     * containing fun(this.atValue(p),a.atValue(p)) for each p in
     * dist.region.
     */
    extern public <C@P, D>
        array(dist)<D@P> lift(Fun<B@P, Fun<C@P, D@P>> fun, array(dist)<C@P> a);

    /**  Return an array of B with distribution d initialized
         with the value b at every point in d.
     */
    extern public static (distribution D) <B@P> array(D)<B@P> constant(B@? b);

}
\end{verbatim}}


\begin{example}
 The code for {\tt List} translates as given in Table~\ref{List-translation}.
\end{example}

\begin{figure*}
{\footnotesize
\begin{verbatim}
  public value class List <Node> {
    public final nat n;   // is a parameter
    nullable Node node = null;
    nullable List<Node> rest = null;  // All assignments must check n = this.n-1.

    /** Returns the empty list. Defined only when the parameter n
        has the value 0. Invocation: new List(0)<Node>().
     */
    public List ( final nat n ) {
      assume n==0;
      this.n = n;
    }

    /** Returns a list of length 1 containing the given node.
        Invocation: new List(1)<Node>( node ).
     */
    public List ( final nat n, Node node ) {
      assume n==1;                         // From the constructor precondition.
      assert 0==0 : "DependentTypeError"; // For the constructor call.
      assert n>=1 : "DependentTypeError"; // For the this call.
      this(n, node, new List<Node>(0));
    }

    public List ( final nat n, Node node, List<Node> rest ) {
      assume n>=1;                               // From the constructor precondition
      assume rest.n==n-1 : "DependentTypeError"; // From the argument type.
      this.n = n;
      this.node = node;
      assert rest.n==n-1 : "DependentTypeError"; // For the field assignment.
      this.rest = rest;
    }

    public  List<Node> append( List<Node> arg ) {
      if (n == 0) {
          final List<Node> result = arg;
          assert n+arg.n == result.n : "DependentTypeError"; // For the return value
          return result;
      } else {
          assume rest.n == n-1;
          final List<Node> argval = rest.append(arg);
          assume argval.n == rest.n+arg.n;
          assert n+arg.n-1== argval.n : "DependentTypeError"; // For the constructor call.
          final List<Node> result = new List<Node>(n+arg.n, node, argval);
          assume result.n == n+arg.n;
          assert n+arg.n == result.n : "DependentTypeError"; // For the return value
          return result;
      }
    }

\end{verbatim}}
\caption{Translation of {\tt List} (contd in Table~\ref{List-translation-2}).}\label{List-translation}
\end{figure*}
\begin{figure*}
{\footnotesize
\begin{verbatim}
    public  List<Node> rev() {
      final List<Node> arg = new List<Node>(0);
      assume arg.n = 0;                           // From the constructor call.
      final List<Node> result = rev( arg );
      assume result.n == n+arg.n;                  // From the method signature
      assert n == result.n : "DependentTypeError"; // For the return value.
      return result;
    }

    public  List(n+arg.n)<Node> rev( final List<Node> arg) {
      if (n==0) {
         assert n+arg.n == arg.n : "DependentTypeError"; // For the return value.
         return arg;
      } else {
        assert 1+arg.n-1=arg.n : "DependentTypeError"; // For the argument to the constructor
        final List<Node> arg2 = new List<Node>(1+arg.n,node, arg));
        assume arg2.n==1+arg.n;                      // From the constructor invocation
        final List<Node> restval = rest;             // Read from a mutable field of parametric type
        assume restval.n == n-1;                     // From the field read.
        final List(restval.n+arg2.n)<Node> result = restval.rev( arg2 );
        assume result.n=restval.n+arg2.n
        assert n+arg.n == result.n                   // For the return value
        return result;
    }

    /** Return a list of compile-time unknown length, obtained by filtering
        this with f. */
    public List<Node> filter(fun<Node, boolean> f) {
         if (n==0) return this;
         if (f(node)) {
           final List<Node> l = rest.filter(f);
           assert l.n+1-1==l.n : "DependentTypeError"; // For the constructor call
           return new List<Node>(l.n+1,node, l);
         } else {
           return rest.filter(f);
         }
    }

    /** Return a list of m numbers from o..m-1. */
    public static  List<nat> gen( final nat m ) {
         assert 0 <= m : "DependentTypeError";        // Precondition for method call.
         final List<nat> result = gen(0,m);
         assume result.n=m-0 : "DependentTypeError";  // From the method signature
         assert m == result.n : "DependentTypeError"; // For the return value
         return result;
    }

    /** Return a list of (m-i) elements, from i to m-1. */
    public static List<nat> gen(final nat i, final nat m) {
      assume i <= m;                                   // Method precondition.
      if (i==m) {
        assert m-i == 0 : "DependentTypeError";        // For the constructor call
        final List result = new List<nat>(m-i);
        assume result.n == 0;                          // From the constructor call.
        assert m-i == result.n : "DependentTypeError"; // For the return value.
        return result;
      } else {
        assert i+1 <= m : "DependentTypeError";        // For the method call.
        final List<nat> arg = gen(i+1,m);
        assume arg.n = m-(i+1);                        // From the method call.
        assert m-i-1 = arg.n;                          // For the constructor invocation.
        final List result = new List<nat>(m-i, i, arg);
        assume result.n = m-i;                         // From the constructor invocation.
        assert m-i == result.n : "DependentTypeError"; // For the return value
        return result;
    }
  }
\end{verbatim}}
\caption{Translation of {\tt List} (continued).}\label{List-translation-2}
\end{figure*}

\section{Type-checking dependent classes}

Each programming language---such as \Xten{}---will specify the base
underlying classes (and the operations on them) which can occur as
types in parameter lists. For instance, in the code for {\tt List}
above, the only type that appears in parameter lists is {\tt int}, and
the only operations on {\tt int} are addition, subtraction, {\tt >=},
{\tt ==}, and the only constants are {\tt 0} and {\tt 1}.  (This
language falls within Presburger arithmetic, a decidable fragment of
arithmetic.)  The compiler must come equipped with a constraint solver
(decision procedure) that can answer questions of the form: does one
constraint entail another?  Constraints are atomic formulas built up
from these operations, using variables. For instance, the compiler
must answer each one of:
{\footnotesize
\begin{verbatim}
  n >= 2 |- n-1 >= 0
  n >= 0, m >= 0 |- m+n >= 0
\end{verbatim}}

Ultimately, the only variables that will occur in constraints are
those that correspond to {\tt config} parameters and those that are
defined by implicit parameter definitions. We need to establish that
the verification of any class will generate only a finite number of
constraints, hence only a finite constraint problem for the constraint
solver.

Second, it should be possible for instances of user-defined classes
(and operations on them) to occur as type parameters. For the compiler
to check conditions involving such values, it is necessary that the
underlying constraint solver be extended.

There are two general ways in which the constraint solver may be
extended.  Both require that the programmer single out some classes
and methods on those classes as {\em pure}. (We shall think of
constants as corresponding to zero-ary methods.) Only instances of
pure classes and expressions involving pure methods on these instances
are allowed in parameter expressions.

How shall constraints be generated for such pure methods? First, the
programmer may explicitly supply with each pure method {\tt T m(T1 x1,
..., Tn xn)} a constraint on {\tt n+2} variables in the constraint
system of the underlying solver that is entailed by {\tt y =
o.m(x1,..., xn)}. Whenever the compiler has to perform reasoning on an
expression involving this method invocation, it uses the constraint
supplied by the programmer. A second more ambitious possibility is
that a symbolic evaluator of the language may be run on the body of
the method to automatically generate the corresponding constraint.

Finally an additional possibility is that the constraint solver itself
be made extensible. In this case, when a user writes a class which is
intended to be used in specifying parameters, he also supplies an
additional program which is used to extend the underlying constraint
solver used by the compiler. This program adds more primitive
constraints and knows how to perform reasoning using these
constraints. This is how I expect we will initially implement the
\Xten{} language. As language designers and implementers we will
provide constraint solvers for finite functions and {\tt Herbrand}
terms on top of arithmetic.





\end{document}
