\documentclass[preprint,nocopyrightspace,9pt]{sigplanconf}
%\documentclass{llncs}

\newif\iflncs
\lncsfalse

\usepackage{times-lite}
\usepackage{mathptm}
\usepackage{txtt}
\usepackage{stmaryrd}
\usepackage{code}
\usepackage{bcprules}
%\usepackage{ttquot}
\usepackage{amsmath}
\usepackage{amssymb}
\usepackage{afterpage}
\usepackage{balance}
\usepackage{floatflt}
\usepackage{defs}
\usepackage{utils}
\usepackage{graphicx}
\usepackage{xspace}
\usepackage{ifpdf}
\usepackage{listings}
\usepackage{x10}
\usepackage{color}

\ifpreprint
\newcommand\todo[1]{\textcolor{red}{#1}}
\else
\newcommand\todo[1]{}
\fi
\renewcommand\todo[1]{}

  %\hyphenpenalty=1000


\newif\ifsemantics
%\semanticsfalse
\semanticstrue

\hfuzz=1pt

\pagestyle{plain}


\ifpdf
\setlength{\pdfpagewidth}{8.5in}
\setlength{\pdfpageheight}{11in}
\fi

\newcommand\val{\mbox{\tt val}}
\newcommand\klass{\mbox{\sf class}}
\newcommand\var{\mbox{\tt var}}
\newcommand\self{\mbox{\tt self}}
\newcommand\this{\mbox{\tt this}}
\newcommand\new{\mbox{\tt new}}
%\newcommand\extends{\unlhd}
%\newcommand\super{\unrhd}
\newcommand\extends{\mathrel{\mbox{\tt \textlt=}}}
\newcommand\super{\mathrel{\mbox{\tt \textgt=}}}
\newcommand\return{\mbox{\tt return}}
\newcommand\true{\mbox{\tt true}}
\newcommand\as{\mbox{\tt as}}
\newcommand\fields{\mbox{\sf fields}}
\newcommand\type{\mbox{\tt type}}
\newcommand\mtype{\mbox{\sf mtype}}
\newcommand\field{\mbox{\sf field}}
\newcommand\CFJ{{\sf CFJ}\xspace}
\newcommand\FJ{{\sf FJ}\xspace}
\newcommand\Java{Java\xspace}

\newcommand\Xten{{\sf X10}\xspace}
\newcommand\csharp{C$^{\sharp}$\xspace}
\newcommand\hmx{$\mathrm{HM}(X)$\xspace}
\newcommand\clpx{$\mathrm{CLP}(X)$\xspace}

\newcommand\xbar[1]{\ensuremath{\bar{\Xcd{#1}}}}
\newcommand\tbar[1]{\ensuremath{\bar{\tt {#1}}}}
\newcommand\exc[2]{\ensuremath{\exists}#1.~#2}
\newcommand\exty[3]{\ensuremath{\exists}#1\ty#2.~#3}
\newcommand\extyty[5]{\ensuremath{\exists}#1\ty#2,#3\ty#4.~#5}
\newcommand\extytyty[7]{\ensuremath{\exists}#1\ty#2,#3\ty#4,#5\ty#6.~#7}
\def\inv{\mathit{inv}}

\def\FGJ{{\sf FGJ}\xspace}
\def\FX{{\sf FX}\xspace}
\def\FXZ{{\sf FX($\cdot$)}\xspace}
\def\FXG{{\sf FX($\cal G$)}\xspace}
\def\FXD{{\sf FX($\cal A$)}\xspace}
\def\FXGD{{\sf FX(${\cal G},{\cal A}$)}\xspace}
\def\has{\mbox{\tt has}}
\def\TConstr{\mbox{\sc T-Constr}}
\def\TInv{\mbox{\sc T-Inv}}
\def\TVar{\mbox{\sc T-Var}}
\def\TField{\mbox{\sc T-Field}}
\def\TInvk{\mbox{\sc T-Invk}}
\def\TNew{\mbox{\sc T-New}}
\def\TCast{\mbox{\sc T-Cast}}
\def\TUCast{\mbox{\sc T-UCast}}
\def\TDCast{\mbox{\sc T-DCast}}
\def\TSCast{\mbox{\sc T-SCast}}

\def\RField{\mbox{\sc R-Field}}
\def\RCField{\mbox{\sc RC-Field}}
\def\RInvk{\mbox{\sc R-Invk}}
\def\RCInvkRecv{\mbox{\sc RC-Invk-Recv}}
\def\RCInvkArg{\mbox{\sc RC-Invk-Arg}}
\def\RCNewArg{\mbox{\sc RC-New-Arg}}
\def\RCast{\mbox{\sc R-Cast}}
\def\RCCast{\mbox{\sc RC-Cast}}


% \input{../../../../vj/res/pagesizes}
% \input{../../../../vj/res/vijay-macros}
\newcommand\alt{\bnf}

\newcommand\Implies{\Rightarrow}

\iflncs
\else
\newtheorem{example}{Example}[section]
\newtheorem{theorem}{Theorem}[section]
\newtheorem{lemma}[theorem]{Lemma}
\newtheorem{definition}[theorem]{Definition}
\newenvironment{proof}{
\trivlist
\item[\hskip \labelsep \textsc{Proof.}]
\selectfont
\ignorespaces}{$\Box$}

%\newtheorem{proof}[theorem]{Proof}
\fi

\begin{document}

\title{Genericity through Constrained Types}

\iflncs

\author{
Nathaniel Nystrom\inst{1}
\and
Olivier Tardieu\inst{1}
\and
Igor Peshansky\inst{1}
\and
Vijay Saraswat\inst{1}
}

\institute{IBM T.~J. Watson Research~Center,
P.O.~Box~704, Yorktown~Heights NY 10598 USA,
\email{\{nystrom,igorp,vsaraswa,tardieu\}@us.ibm.com}}

\else

\authorinfo{Nathaniel Nystrom
        \and Olivier Tardieu
        \and Igor Peshansky
        \and Vijay Saraswat
        }{IBM T.~J. Watson Research
Center, P.O. Box 704, Yorktown Heights NY 10598 USA}
  {\{nystrom,tardieu,igorp,vsaraswa\}@us.ibm.com}

% \conferenceinfo{POPL'08}{XXX}
% \copyrightyear{2008}
% \copyrightdata{[to be supplied]}

\fi

\maketitle

\begin{abstract}
\Xten{} is a modern object-oriented language designed for productivity
and performance in concurrent and distributed systems, such as
(heterogeneous) multicores and clusters. In this context, dependent
types arise naturally: objects may be located at one of many places,
arrays may be multidimensional, activities may be associated with one
or more clocks, variables may be marked as shared or private following
an ownership discipline, etc.  A framework for dependent types offers
significant opportunities for detecting design errors statically,
documenting design decisions, eliminating costly runtime checks
(e.g., for array bounds, null values), and improving the quality of
generated code.

We present a simple, general framework for adding constraint-based
dependent types to nominally typed OO languages such as Java, \Xten{}
and Scala. The framework is parametric on an underlying constraint
system {\cal C}. Classes and interfaces are associated with {\em
properties} (= final instance fields). A type {\tt C(:c)} names a class
or interface {\tt C} and a {\em constraint} {\tt c} on the
properties of {\tt C} and in-scope final variables.  Constraints
may also be associated with class definitions (representing
class invariants) and with method and constructor definitions
(representing preconditions). Dynamic casting is permitted.

We present many examples to illustrate that many common OO idioms and
OO type systems proposed recently can be naturally captured by
constrained types: specifically we discuss types for places, aliases,
ownership, arrays and clocks. We have implemented the type system (for
a simple equality-based constraint system) in \Xten{} 1.0 (available
at {\tt x10.sf.net}). We present a simple \FJ{} extension,
Constrained FJ, and establish fundamental properties such as type
soundness. We compare this approach with relevant work in dependent
types, specifically, DML, and outline many areas of future work.

In conclusion, we believe that constrained types offer a natural,
simple, clean, and expressive extension to OO programming.

\end{abstract}


\section{Introduction}
\label{sec:intro}
%
\section{Introduction}
\label{s:intr}

 Graph theoretic problems arise in several traditional and emerging scientific disciplines such as VLSI design, optimization, databases, and computational biology. There are plenty of theoretically fast parallel algorithms, for example, work-time optimal PRAM algorithms, for graph problems; however, in
 practice few parallel implementations beat the best sequential implementations for arbitrary, sparse
 graphs. The mismatch between theory and practice suggests a large gap between algorithmic model and the actual architecture. We observe that the gap is increasing as new diversified architectures emerge. Elegant solutions seem hard to come by from even combined efforts of algorithmic and architectural improvement. What is lacking is an effient way of mapping fine-grained parallelism expressed by the algorithm to target architectures with good performance. X10 is a new parallel programming language that provides expressive programming constructs and efficient runtime support that effectively helps reduce the gap between theory and practice in solving graph problems. In this paper we show that with X10 the fine-grained parallelism for a graph problem can be expressed much easier at a high algorithmic level, and the X10 program, compared with native C implementation, is much simpler and more elegant, and achieves comparable, and sometimes, even better performance. 

 The challenges of solving large-scale graph problems on current and emerging systems come from the irregular and combinatorial nature of the problem. Many of the important real world graphs, for example, internet topology, social interaction network, transfortation network, protein-protein interaction network, and etc., exhibit a ``small-world'' nature, and can be modeled as the so-called ``scale-free'' graph. There is no known efficient technique to partion such graph, which makes it hard to solve on distributed-memory systems. Also compared with the well-known sequential algorithms, for example, depth-first search (DFS) or breadth-first search (BFS) for the spanning tree problem, the parallel graph algorithms take exotic approaches such as ``graft-and-shortcut''. In the absence of efficient scheduling support of parallel activities, fine-grained parallelism incurs large overhead on current systems and oftentimes do not show practical parallel performance advantage. Lastly, graph algorithms tend to be load/store intensive compared with other scientific problems. For example,  They put great pressure on the memory subsystem. The problem obviously gets worse on distributed-memory architectures if necessary task management and memory affinity scheduling are not provided.  
 
 There are several features of X10 that make it extremely helpful in soving large-scale graph problems. X10 provides a shared virtual address space that obviates the need to partition a graph and issue message passing commands explicitly to access remote data. The irregular nature of the graph is also the reason why no SSCA benchmark has been implemented in MPI. X10 provides a wide range of constructs that are de. X10 has a lot of balancing.

 Our target architecure is a cluster of symmetric multiprocessor(SMP) nodes. Each SMP node may further comprise of chip multiprocessors (CMPs).  SMPs and CMPs are becoming very powerful and common place. Most of the high performance
computers are clusters of SMPs and/or CMPs. It is important to solve for them.
It is important to show flexibility but also good support of  PRAM algorithms for graph problems can be emulated much easier and more. 

The problem we consider if the spanning tree problem. It is notoriously hard to achieve good parallel performance.  Several good ones, we show X10 support that can do better. 

 The rest of the paper is organized as follows. Sections~\ref{s:design} describes algorithm design with the X10 language.
 Section~\ref{s:runtime} presents the workstealing runtime support for load-balancing in X10, and compare with other runtime systems, for example, CILK. 
 Section~\ref{s:results} provides our experimental results on current main-stream SMPs.
 In Section~\ref{s:concl} we conclude and give future work. 
 Throughout the paper, we
 use $n$ and $m$ to denote the number of vertices and the number of
 edges of an input graph $G=(V,E)$, respectively. 
  



\section{Introduction}

The industry's shift to multicore processors has sparked a trend
toward pushing parallel programming into the mainstream.  This trend
poses a significant challenge, since creating and maintaining parallel
programs that are both efficient and reliable is notoriously
difficult.  One response to this challenge
by the programming languages community has been to create new (and
revisit old) language
abstractions and programming models 
for parallel programming and to
develop new languages based on these abstractions.  % For example, there
% has been lots of recent work on support for software transactional
% memory~\cite{}, and the DARPA PERCS project is funding several new
% parallel programming languages~\cite{}.

While new languages can greatly aid programmers in developing 
parallel programs, we believe that new languages cannot
achieve mainstream success without associated tooling support.
Modern integrated development environments
(IDEs) such as Eclipse~\cite{eclipse} provide many benefits that programmers
have come to rely upon, helping
them to more easily navigate through a program, understand
dependencies among parts of a program, and safely evolve a program to
improve its quality along some dimension.  The latter benefit is typically
provided through code {\em refactorings}.  
% Part of the difficulty in programming for concurrent systems is
% that they have often lagged behind in tooling support. Practitioners of most
% modern sequential languages have the advantage of employing a wide variety of
% analyses and refactorings in order to improve the quality of their code. While
% analysis research has continued to show forward momentum in the parallel
% language context, we believe that similar tooling support for refactorig in
% parallel languages will help users improve the quality of their parallel code;
% a belief that a number of others also share~\cite{Kennedy91, Liao99, Overbey05}.

We believe that
specialized refactoring support will be a critical tool to help programmers improve the
quality of parallel code.  Such refactorings could be used
to improve efficiency while preserving program behavior and key
concurrency invariants (e.g., atomicity, deadlock-freedom).
%Such
%refactorings could also be used to improve a parallel program's readability and
%extensibility.

Any set of refactorings will of necessity be tailored to the needs of a particular
parallel programming model and language.
We focus on providing refactoring support for the
\emph{partitioned global address space} (PGAS) memory model as
embodied in the X10~\cite{X10,Charles05}, UPC~\cite{ElGhazawi03} and
Titanium~\cite{Yelick98} programming languages.  In this model, the
programmer sees a uniform 
representation of data and data structures over distributed nodes, regardless
of the physical location of the data.  However, each piece of data is
assigned to a fixed partition (or {\em place} in X10 terminology)
and can only be accessed by {\em activities}
that run at that place.
% That is, a reference to a piece of
% data is treated as a first-class entity in this system and can be treated
% similarly to a local reference even if the owner of said data is at a
% different location. In such cases, the language's compiler or virtual machine
% will handle the messy details of handling message and data passing. The PGAS
% model attempts to minimize background communication overhead by
% locating data at a place where it has good data or processing affinity.

In this presentation, we focus on the X10 language, an object-oriented language
providing first-class, high-level constructs for asynchronous activities, synchronization,
phased computations, data distribution, and atomicity.
%Further, as stated in the introduction, the X10 language is in the family of
%concurrent languages with a {\em partitioned global address space} (PGAS)
%memory model. In the case of X10, the set of partitions in the global address
%space is fixed at the start of the program.
By incorporating such abstractions as first-class constructs in the language,
the burden of reasoning about various program properties is often reduced
from global reasoning involving complex control flow to simple and modular
reasoning about lexical containment.
In fact, many interesting properties (e.g. deadlock freedom) can often
be ensured statically.
In this way, X10's constructs simplify both the programmer's task of
understanding and tools' tasks of static program analysis.
% make all asynchronous and atomic code apparent
%to both the analysis engine and the programmer.

The PGAS model provides particular opportunities and challenges for
automated refactorings.  On the one hand, code transformations are
simplified since the code need not explicitly handle inter-processor
communication.  
On the other hand, transformations must properly handle the
asynchronicity that arises among activities and must respect
the synchronization constraints imposed on these activities by the
semantics of the various language constructs. 
% Transformations must also
% ensure that an activity only accesses data from the current place.
In the rest of this paper we describe an initial concurrency
refactoring for X10 that we
have been developing inside the X10DT plugin for Eclipse.

%%%%%%%%%%%%%%%%%%%%%%%%%%%%%%%%%%%%%%%%%%%%%%%%%%%%%%%%%%%%%%%%%%%%
% 
% Previous Version: for comparison
% 
% \section{Introduction}
% 
% Many modern parallel languages, particularly high-performance computing (HPC)
% languages, have trended towards a \emph{global address
% space} (GAS) memory model. In fact, all three of the remaining DARPA HPC
% Challenge programming languages have built their languages around a GAS
% model. A GAS model simplifies the development of code since the programmer
% does not have to deal with the explicit passing of data from one
% place of execution to another via interfaces such as MPI or OpenMP. Instead, a
% programmer can directly reference any desired data and the language's compiler
% or virtual machine will handle the messy details. This surface simplicity comes
% with a small caveat, though: a lot of extra communication may be incurred in
% order to actually determine which place actually owns some piece of data, all
% of which is hidden at the source level. One solution to this issue is to
% use a \emph{partitioned global address space} (PGAS) model in which data is
% created and kept in a single place for its lifetime, thus minimizing such
% communication overhead. The PGAS model itself is employed in a number of
% languages, such as UPC, Titanium, and X10~\cite{ElGhazawi03, Yelick98, Charles05}.
% 
% Even in languages with an abstracted memory model like the GAS model, creating
% efficient code poses a challenge. Programmers still struggle with such tasks
% as introducing maximal concurrency, avoiding deadlock, minimizing transaction
% conflicts, and numerous other plagues of the parallel world. Making
% such code readable and extensible is
% even more daunting. Most modern sequential languages often employ
% refactoring and other transformation engines to make code more efficient
% without sacrificing readability or introducing errors. We believe that
% similar tooling support for parallel languages can provide these benefits to the
% parallel programming community; a belief that a number of others also
% share~\cite{Kennedy91, Liao99, Overbey05}. 
% 
% We propose a framework in which a refactoring engine, standing alone or coupled
% with an IDE, is available to aid in the creation of parallel programs.
% The refactoring engine should be capable of performing its own program analysis
% to guarantee that transforming the code will not introduce new and subtle
% errors. Its associated transformation can also use the results of the analysis
% to transform the code appropriately. We feel that such sanity checks are vitally
% important to ensuring that new bugs are not introduced, in contrast to current
% commonly-used refactorings which are mostly unchecked but relatively
% cosmetic such as renaming variables or abstracting blocks as methods. Having
% such a refactoring engine allows a developer to write simple and readable
% code with the idea that the refactoring engine will improve the performance of
% the areas in the code that she has identified as good candidates for increased
% efficiency.

%%%%%%%%%%%%%%%%%%%%%%%%%%%%%%%%%%%%%%%%%%%%%%%%%%%%%%%%%%%%%%%%%%%%%


%%Since the advent of inter-processor communication, the desire to
%%create programs which take advantage of parallel execution has
%%existed. 
% The desire to take advantage of parallel execution has
% existed since the advent of the computer processor, and this desire is only
% more fervent in this age of processor ubiquity. However, the creation and
% maintenance of parallel programs involves the difficult challenges of
% enforcing proper program and memory consistency and synchronization.
%%To compound the issue,
%%A. J. Bernstein has already shown that, in the general case,
%%automatically determining parallelism of sequential code is an
%%undecidable problem~\cite{Bernstein66}.
% In the programming language community, two main schools of thought exist on
% solving these problems. One school of thought is to simply develop languages
% that included explicit constructs or compiler optimizations for parallelism.
%% These include variants of Fortran and Lisp and languages built with parallel
%% execution or data distribution support like Ada, Linda, and
%% Emerald~\cite{Allen87, Hutchinson87, Griss82, Ada, Gelernter85}. 
% Even with language level support, though, it is still difficult for
% programmers to quickly create error-free code which efficiently uses
% parallelism. Another approach is to automate the parallelization of
% sequential code via extensive program analysis at points in a program, such as
% loops, that seem to provide natural boundaries for parallel execution. However, this
% is complicated by the difficulty of
%%Fortunately, certain sequential programming language constructs do
%%seem to provide natural boundaries for automatic parallel compiler
%%optimizations: loops~\cite{Lamport74, Allen84, Allen87}. By
%%recognizing that the majority of time in running programs is spent
%%executing the same set of calculations over a different set of data,
%%it is natural to establish parallel execution of loop bodies on
%%different processors or using vector supercomputers. One drawback to
%%this approach, though, is that 
% precise static analysis of the parallelism targets.
% %% is a very difficult task. 
% Dependence on structured memory usage, such as
% arrays, which benefit the most from parallel execution, complicates
% loop analysis further.

% We believe that 
% %%Many research projects found that 
% the key for better parallelism detection is to take advantage of user
% interaction, 


%%As a result, user interactivity in the form of 
% In particular, we hold that
% source code refactoring support in IDEs such as Eclipse can be a
% powerful ally in aiding parallelization.
%%in the compiler optimization and program transformation process
%%A typical way of involving the user in this process
%%solution for these systems 
%%is to build support for some parallel transformations into an IDE and display
%%the
%%collected memory dependence information for a given line of code to the user. If
%%the user determines that the dependences would not prevent parallel
%%execution, then the user can choose to transform the code. For a whole
%%program analysis, such a presentation might be overwhelming and difficult to
%%decipher, and is probably not the best way to involve the user. Our approach
%%is to 

% We think that programmers often know which statements in their code could
% benefit the most from
% added efficiency, even if they do not know how best to introduce this
% efficiency on their own.
% %  and where they would like to insert more concurrency.
% Static and dynamic analysis tools can also help users identify hotspots
% in their programs.
% %In either case, the
% The tasks of identification of suitable parallelism candidates and of
% transforming the code to introduce/manipulate parallelism are separable.
%%Thus, programmers can specify which lines of code
%%would like to refactor into a concurrent form and 


% provides the link
% between these two vectors of concurrency introduction. The refactoring
% engine should determine if refactoring at these points is safe
% and how the code should be transformed to introduce more concurrency. A developer can then

% of which
% statements could benefit from added parallelism or use a tool to identify possible target sites
% %%, such as the code in Figure~\ref{fig:CHM-X10}, 
% and then use 


% Moreover,
% we believe that this approach can be applied to other application concerns of
% parallel programmers, such as introducing new synchronization points or
% refitting blocks as atomic.
%%, such as the code in
%%Figures~\ref{fig:CHM-X10-future} and~\ref{fig:CHM-X10-async}.
%%information is often difficult for a user to decipher for a whole program, though,
%%and is not Because following the flow of data for parallel programs is
%%difficult in these IDEs, advanced concurrency slicing techniques were
%%developed~\cite{Zhao99,Chen01,Chen02,Krinke03}. We believe that
%%including user input is an essential part of making parallelism
%%apparent to the programmer

%%\bug{Now introduce we will use X10 and why we like it: explicit higher-level
%%constructs and PGAS model} 

%%The target of our research is to develop practical refactoring schemes for
%%introducing increased parallelism in the X10 language. X10 includes explicit
%higher-level concurrency constructs such as asynchronous blocks and future
%%expressions. It also incorporates a {\em partitioned global address space}
%%(PGAS) model of data consistency. The combination of the explicit constructs
%%and the PGAS model in X10 allows data dependency analysis to be simplified.

%%Many recent languages have been developed with a focus on
%%the {\em partitioned global address space} (PGAS) model of data consistency
%%including UPC, Titanium, Co-Array Fortran, and X10~\cite{ElGhazawi03,
%%Yelick98, Numrich98, Charles05}. The target of our research is to develop
%%a practical refactoring scheme for X10, although this scheme should be
%%adaptable to other languages based on the PGAS model. 

% To deal with some of the static analysis issues associated with parallelization,
% we focus focus our refactorings on languages with a {\em partitioned global 
% address space} (PGAS) model of data consistency. In the PGAS model,
% %%The primary goal of the PGAS model is to allow parallel asynchronous activity to
% %%occur among multiple concurrent platforms, each having its own local memory and
% %%potentially lacking a joint shared memory, without losing the ability to read
% %%and write to global data. Global 
% shared data is actually owned by local processors but is globally addressable.
% Other processors which would like
% to access or update global data then communicate directly with the owning
% location to perform any necessary actions. Thus, programmers take an active role
% in defining how data in arrays or data structures is distributed
% over the address space so as to maximize locality, thus taking maximum advantage
% of their concurrent environments. This removes the programmer's (or analysis')
% burden of determining
% how complicated structures should be partitioned to the various execution sites at
% refactoring points in the code where parallelism is desired (e.g., inside 
% loops). As a result, the required static analysis is reduced to determining
% local and loop-carried dependencies that prevent a statement from being
% asynchronously executed.\bug{This statement really applies to this particular
% transformation. I.e., the programmer does have to determine the partitioning at
% some point, but doesn't need to worry about it while applying this
% transformation. So perhaps we should say that we envision different kinds of
% {\em lateral} moves, e.g.: manipulate concurrency while keeping the distribution
% fixed, and manipulating the distribution while keeping concurrency relatively
% unchanged.}
% 
% We chose to implement our refactoring scheme in the X10 language. X10 not only
% has a PGAS data consistency model, but also includes explicit higher-level
% concurrency constructs such as asynchronous blocks and future
% expressions. These concurrency constructs further simplify the analysis of X10
% code by making all asynchronous and atomic code apparent to both the analysis
% engine and the programmer.
% 
% %%\begin{figure}[tp]
% %%  \begin{code}
% %%    int mcsum=0; \\
% %%    fo\=r (i=0; i<segments.length; i++)\{ \\
% %%    \>  mcsum += mc[i] = segments[i].modCount(); \\
% %%    \>  if\=(segments[i].containsValue(value)) \\
% %%    \>\>    return true; \\
% %%    \} \\
% %%  \end{code}
% %%\caption{\label{fig:CHM} A Java code excerpt from the library class {\tt
% %%java.util.concurrent.ConcurrentHashMap} which illustrates a loop that
% %%would not be parallelizable via traditional automatic loop
% %%parallelization methods.}
% %%\end{figure}
% 
% %%Because languages designed for the PGAS model do not focus on
% %%providing support for the parallel execution of the same code over
% % different sets of data, they can provide a more flexible alternative
% % to whole loop parallelization: the loops can be executed sequentially
% % and updates to distributed data structures can occur in parallel. 
% As an example, consider the X10 code in Figure~\ref{fig:CHM-X10}, an excerpt 
% from an X10 implementation of the {\tt java.util.concurrent.ConcurrentHashMap} 
% class. In this example, it is possible the {\tt modCount} and {\tt 
% containsValue} method invocations are a both very expensive operations. Because 
% of the PGAS model, the individual elements of the array {\tt segments} could 
% exist anywhere in the global address space. However, automatic whole loop 
% parallelization techniques developed previously would fail on this code because 
% of the conditional return statement and the loop carried dependency summing
% the results of the calls to {\tt modCount}. However, the programmer might still
% be able to take advantage of parallel execution by making the invocations of
% {\tt modCount} asynchronous and caching the results.
% Figures~\ref{fig:CHM-X10-async} and~\ref{fig:CHM-X10-future} show examples of
% adding concurrency in this fashion.
% 
% We present in this paper the source code transformation {\em extract concurrent} for
% the X10 language which allows a programmer to transform loop code
% like that in Figure~\ref{fig:CHM-X10} to one that takes maximum advantage of
% asynchronous execution. We acheive this refactoring through two means:
% 
% \begin{enumerate}
% \item {\em Loop dependence analysis.} Since introducing parallelism in
% the middle of a loop might affect the ability of other statements in a
% loop to properly evaluate, it is important that loops do not carry
% dependence on the results of any asynchronously evaluated statements. We
% have developed a series of analyses to determine whether {\em extract
% concurrent} will adversely affect the execution of the code and
% violate its sequential consistency.
% 
% \item {\em Transformation pattern.} We have developed a general
% pattern for the {\em extract concurrent} transformation on viable
% sequential loops. This pattern uses program slicing
% techniques to split a loop in two. The first loop will allow introduction of
% parallelism while the second loop utilizes the results of the asynchronous
% execution. Because this split requires some code duplication, we
% present the results of a formal analysis on how the transformation
% affects the runtime of the loop and define the conditions under which
% the transformation provides the potential for better runtime. In practice, this
% transformation is more widely applicable to multiple statements or expressions.
% We present here only the single statement or expression case.
% \end{enumerate}
% 
% We have implemented the transformation as refactoring in the X10DT, a
% development framework for the X10 language in Eclipse, however we believe that
% the transformation is adapatable to all languages with a PGAS consistency model.
% \bug{We will insert more about the implementation when it's actually been done.}

% The rest of the paper is organized as follows. Section 2 presents an
% overview of the X10 programming language. Section 3 contains a
% description and analysis of the the algorithm for determining loop
% candidacy for the transformation. Section 4 details the {\em extract
% concurrent} transformation and discusses the impact of the
% transformation on running time. Section 5 presents the implementation
% and evaluation of the transformation in X10DT. Section 6 describes the
% related work and Section 7 contains the conclusion and a summary of
% the paper.


\section{X10 language overview}
\label{sec:lang}

This section presents an informal description of 
dependent and generic types in \Xten{}.  The type system is formalized
in a simplified version of \Xten{}, \FX{}, in
Section~\ref{sec:semantics}.

When presenting syntax, we follow the usual conventions for
Featherweight Java:
we write \xbar{t} for the list
\xcdmath"t$_1$,$\dots$,t$_n$"; 
terms with list subterms are considered 
a single list of terms (e.g., 
we write \xbar{x}\Xcd{:}\xbar{T} for the list
\xcdmath"x$_1$:T$_1$,$\dots$,x$_n$:T$_n$").

\Xten{} is a class-based object-oriented language.
The language has a sequential core similar to Java or Scala, but 
constructs
for concurrency and distribution, as well as constrained types,
described here.
Like Java, the language provides single class inheritance and
multiple interface inheritance.

A constrained type in \Xten{} is written \xcd"C{e}", where \xcd"C" is the
name of a class and \xcd"e" is a constraint on the properties of
\xcd"C" and the final variables in scope at the type.  The
constraint \xcd"e" may refer to the value being constrained
through the special variable \xcd"self", which has type \xcd"C"
in the constraint.  Constraints are drawn from a constraint
language that, syntactically, is a subset of the boolean
expressions of \Xten{}.

The compiler checks that constraints are expressions
of type \xcd"boolean" and that they can be statically checked by
the compiler's constraint solver.  
\Xten{} supports conjunctions of equalities
over final variables and compile-time constants, and equalities
and subtyping constraints over types.
Compiler plugins may be installed to
handle richer constraint systems such as Presburger arithmetic
or set constraints.

For brevity, the constraint may be omitted and
interpreted as \xcd"true".
The syntax
\xcdmath"C[$\tbar{T}$]($\tbar{e}$)" is shorthand for
\xcdmath"C[$\tbar{X}$==$\tbar{T}$]($\tbar{x}$==$\tbar{e}$)"
where \xcdmath{X$_i$} are the type properties and \xcdmath"x$_i$" are the
value properties of \xcd{C}.
If either list of properties is empty, it may be omitted.

To illustrate the features of dependent types in \Xten{}, we develop a \xcd"List"
class.  We will present several versions of \xcd"List" as we
introduce new features.
A \xcd"List" class with a type property \xcd"T" and an \xcd"int"
property \xcd"length" is declared as in Figure~\ref{fig:list0}.
Classes in \Xten{} may be declared with any number of type properties and
value properties.

\begin{figure}
{\footnotesize
\begin{xtennoindent}
class List[T](length: int) {
  var head: T;
  val tail: List[T];
  def get(i: int) = {
    if (i == 0) return head;
    else return tail.get(i-1);
  }
  def this[S](hd: S, tl: List[S]): List[S](tl.length+1) = {
    property[S](tl.length+1);
    head = hd; tail = tl;
  }
}
\end{xtennoindent}
}
\caption{List example, simplified}
\label{fig:list0}
\end{figure}

Like in Scala, fields are declared using the keywords \xcd"var"
or \xcd"val".  The \xcd"List" class has a mutable \xcd"head"
field with type \xcd"T" (which resolves to \xcd"this.T"), and an
immutable (final) \xcd"tail" field with type \xcd"List[T]", that
is, with type \xcd"List{self.T==this.T}".  Note that \xcd"this" occurring
in the constraint refers to the instance of the enclosing
\xcd"List" class,
and \xcd"self" refers to the value being
constrained---\xcd"this.tail" in this case.

Methods are declared with the \xcd"def" keyword.
The method \xcd"get" takes a final integer \xcd"i" argument
and returns the element at that position.

Objects in \Xten{} are initialized with constructors, which
must ensure that all properties of the new object
are initialized and that the class invariants of the object's
class and its superclasses and superinterfaces hold.
\Xten{} uses method syntax with the name
\xcd"this" for constructors.
In \Xten{}, constructors have a ``return type'', which constrains
the properties of the new object.  The constructor in
Figure~\ref{fig:list0} takes a type argument \xcd"S"
and two value arguments \xcd"hd" and \xcd"tl".  The constructor
return type specifies that the constructor initializes the
object to have type \xcd"List[S](tl.length+1)", that is,
\xcd"List{self.T==S," \xcd"self.length==tl.length+1}".
The formal parameter types and return types of both methods and
constructors may refer to final parameters of the same
declaration.

The body of the constructor
begins with a \xcd"property" statement that initializes the
properties of the new instance.  All properties are initialized
simultaneously and it is required that the property assignment
entail the constructor return type.
The remainder of the constructor assigns the fields of the
instance with the constructor arguments.

We next present a version of \xcd"List" where we write
invariants to be enforced statically.  Consider the new version
in Figure~\ref{fig:list}.

\begin{figure}
{\footnotesize
\begin{xtennoindent}
class List[T](length: int){length >= 0} {
  var head{length>0}: T;
  val tail{length>1}: List[T](length-1);

  def get(i: int{0 <= i, i < length}){length > 0} = {
    return i==0 ? head : tail.get(i-1);
  }

  def map[S](f: Object{self :> T} => S): List[S] = {
    if (length==0)
      return new List[S](0);
    else if (length==1)
      return new List[S](f(head));
    else
      return new List[S](f(head), tail.map[S](f));
  }

  def this[S](): List[S](0) = property[S](0);
  def this[S](hd: S): List[S](1) = {
    property[S](1); head = hd;
  }
  def this[S](hd: S, tl: List[S]): List[S](tl.length+1) = {
    property[S](tl.length+1);
    head = hd; tail = tl;
  }
}
\end{xtennoindent}}
\caption{List example, with more constraints}
\label{fig:list}
\end{figure}

\subsection{Class invariants}

Properties of a class may be constrained with 
a \emph{class invariant}.   
The \xcd"List" declaration's class invariant specifies that the length of
the list be non-negative.  
The class invariant must be established by all constructors of
the class and can subsequently be assumed for all instances of the class.

For generic types, the invariant is used to provide subtyping
bounds on the type properties.  For instance, a binary
tree class
might require that its elements implement the \xcd"Comparable"
interface:
{\footnotesize
\begin{xten}
class Tree[T]{T <: Comparable[T]} {
  left, right: Tree[T]; ...
}
\end{xten}}

\subsection{Class member invariants}

Class and interface member declarations may have additional
constraints that must be satisfied for access.

The field  declarations in Figure~\ref{fig:list}
each have a \emph{field constraint}.  The field constraint on
\xcd"head" requires that \xcd"this.length>0";
that is \xcd"this.head" may not be dereferenced
unless \xcd"this" has type \xcd"List{length>0}".  Similarly,
\xcd"tail" cannot be accessed unless the list has a non-empty tail.  The
compiler is free to generate optimized representations of
instances of \xcd"List" with a given length: it may remove the
the \xcd"head" and \xcd"tail" fields for empty lists, for
instance.  Similarly, the compiler may specialize instances of
\xcd"List" with a given concrete type for \xcd"T".  This
specialization is described in Section~\ref{sec:impl}.

The method \xcd"get" in Figure~\ref{fig:list}
has a constraint on the type of \xcd"i" that requires
that it be within the list bounds.
The method also has a \emph{method constraint} that
requires that the actual
receiver's
\xcd"length" field must be non-zero---calls to \xcd"get" on empty lists are not
permitted.
A method with a method constraint is called a \emph{conditional method}.
The constraint on \xcd"get" ensures that the field constraint on
\xcd"tail" is satisfied in the method body.
In the method body, the \xcd"head" of the list is returned for
position \xcd"0"; otherwise, the call recurses on \xcd"tail".
Note that for this example to type-check, the constraint system
must establish the field constraint on \xcd"tail"
and the method constraint on the recursive call;
that is, it must be able
guarantee that the \xcd"tail" is non-empty and that
\xcd"i-1" is within the bounds of \xcd"tail".
\eat{
The constraint solver must prove
{\footnotesize
$$
\begin{array}[t]{l}
\tt length > 0, tail.length=length-1, 0 \leq i, i < length, i \not= 0 \\
\tt \quad\vdash tail.length > 0, 0 \leq i-1,i-1 < tail.length 
\end{array}
$$}
}

Method overriding is similar to Java: a method of a subclass
with the same name and parameter types overrides a method of the
superclass.  An overridden method may have a return type that is
a subtype of the superclass method's return type.
A method constraint may be weakened by an overriding
method; that is, the method constraint in the superclass must entail the  
method constraint in the subclass.

Methods may also have type parameters.  
For instance, the \xcd"map" method in Figure~\ref{fig:list} 
has a type parameter \xcd"S" and a value parameter that is a
function from a supertype of \xcd"T" to \xcd"S".
A parametrized method is invoked by giving type arguments before the
expression arguments (see recursive call to
\xcd"map").\footnote{Actual type arguments can be inferred from the types
of the value arguments. Type inference is out of the scope of this paper.}

\xcd"List" also defines three constructors: the first
constructor takes no value arguments and initializes
the length to \xcd"0".  Note that \xcd"head" and \xcd"tail" are
not assigned since they are inaccessible.
The second constructor takes an argument for the head of the
list; the third takes both a head and tail.

\subsection{Type constraints and variance}
\label{sec:variance}

Type properties and subtyping constraints may be used in \Xten{} to 
provide use-site variance
constraints~\cite{variant-parametric-types}.

Consider the following subtypes  of \xcd"List" from
Figure~\ref{fig:list}.
\begin{itemize}
\item \xcd"List".  This type has no constraints on the type
property \xcd"T".
Any type that constrains \xcd"T",
is a subtype of \xcd"List".  The type \xcd"List" is equivalent to
\xcd"List{true}".
%
For a \xcd"List" \xcd"l", the return type of the \xcd"get" method
is \xcd"l.T".
Since the property \xcd"T" is unconstrained,
the caller can only assign the return value of \xcd"get"
to a variable of type \xcd"l.T" or of type \xcd"Object".

\item \xcd"List{T==float}".
The type property \xcd"T" is bound to \xcd"float".
For a final expression \xcd"l" of this type,
\xcd"l.T" and \xcd"float" are equivalent types and can be used
interchangeable wherever \xcd"l" is in scope.

\item \xcd"List{T<:Collection}".
This type constrains \xcd"T" to be a subtype of \xcd"Collection".
All instances of this type must bind \xcd"T" to a subtype of
\xcd"Collection"; for example \xcd"List[Set]" (i.e.,
\xcd"List{T==Set}") is a subtype of
\xcd"List{T<:Collection}" because \xcd"T==Set" entails
\xcd"T<:Collection".
%
If \xcd"l" has the type \xcd"List{T<:Collection}",
then the return type of \xcd"get" has type \xcd"l.T", which is an unknown but
fixed subtype of \xcd"Collection"; the return value can be
assigned into a variable of type \xcd"Collection".

\item \xcd"List{T:>String}".  This type bounds the type property
\xcd"T"
from below.  For a \xcd"List" \xcd"l" of this type, any
supertype of \xcd"String" may flow into a variable of type \xcd"l.T".
The return type of the \xcd"get"
method is known to be a
supertype of \xcd"String" (and implicitly a subtype of \xcd"Object").
\end{itemize}

In the shortened syntax for types (e.g., \xcd"List[T](n)"),
an actual type argument \xcd"T" may optionally
be annotated
with
a \emph{use-site variance tag}, either \xcd"+" or \xcd"-":
if \xcd"X" is a type property, then
the syntax \xcd"C[+T]" is shorthand for \xcd"C{X<:T}" and
\xcd"C[-T]" is shorthand for \xcd"C{X:>T}"; of course,
\xcd"C[T]" is shorthand for \xcd"C{X==T}".

\eat{
\subsection{Type properties}

Type properties may be declared invariant, covariant, or
contravariant.
If a property \xcd"X" of a class \xcd"C" is covariant,
then if \xcd"S" is a subtype of
\xcd"T", the type \xcd"C{X==S}" is a subtype of \xcd"C{X==T}".
Similarly, if \xcd"X" is contravariant, 
                  \xcd"C{X==T}" is a subtype of \xcd"C{X==S}".
It is illegal for a covariant property to occur in a negative
position in its class declaration and for a contravariant
property to occur in a positive position.  A position is
negative if it is a formal parameter type, or occurs in a method
where clause.  A position is positive if it is a return type or
occurs in a method constraint.
}

\eat{
\subsection{Methods}

Methods in \Xten{} are declared with the \xcd"def" keyword.
The \xcd"List" class in Figure~\ref{fig:list} declares methods
\xcd"get" and \xcd"map".

Like Java, \Xten{} supports both instance and static methods.
Since a type property is an instance member, a static method may
not refer to a type property of the class.

Interfaces are also permitted to have static methods.  Classes
implementing the interface must provide an implementation of the
static methods of the interface.
This feature is
useful when a type property \xcd"T" is constrained to implement
an interface \xcd"I"; static methods of \xcd"I" can be invoked
through \xcd"T".
}

\eat{
\subsection{Interfaces}

optional interfaces
value properties in interfaces
static methods in interfaces

\subsubsection{Optional methods and interfaces}

Method constraints also provide support for optional methods.

{\footnotesize
\begin{xten}
class List[T] {
    ...
    def print(){T <: Printable} = {
        for (x: T in this)
            x.print();
    }
}
\end{xten}}

\xcd"List.print" may only be called on lists instantiated on
subtypes of the \xcd"Printable" interface.

Optional methods generalize to optional interfaces.

{\footnotesize
\begin{xten}
interface Printable { def print(); }

class List[T] implements Printable if {T <: Printable} {
    ...
    def print(){T <: Printable} = {
        for (x: T in this)
            x.print();
    }
}
\end{xten}}

In this case \xcd"List" implements the \xcd"Printable" interface
only if \xcd"List.T" implements \xcd"Printable".
Thus \xcd"List{T==String}"
and \xcd"List{T==List[String]}"
are subtypes of \xcd"Printable", but
\xcd"List{T" \xcd"==DirtyWord}" is not.


Without optional interfaces, \xcd"List" cannot be a subtype
of \xcd"Printable".  The constraint \xcd"{T <: Printable}" on
the \xcd"print" method is more restrictive than the 
constraint (i.e., \xcd"true") on 
\xcd"Printable.print".
}


\eat{\section{Constraint solver}
\label{sec:solver}

The goal of the constraint solver is 
to check an assertion $\xbar{c} \vdashC \Xcd{d}$.

\eat{
Inference

The first step is to normalize constraints
into a set of constraint judgments
$\xbar{c} \vdashC \Xcd{c}$ where $\Xcd{c}$ contains no conjunctions.


Once in normalized form, the inference proceeds as follows:
Select a constraint $\xbar{c} \vdashC \Xcd{c}$.
If not consistent, fail.
If valid, ok.
If not valid, generate assignment of variables that makes it
true, adding the assignment to the assumptions for all
constraints.

The inference algorithm must specify the criteria for:
\begin{itemize}
\item selecting the next constraint to solve
\item generating the variable assignment consistent with all
other constraints (to avoid backtracking)
\end{itemize}

Pick an unassigned variable, find weakest assignment that makes just
this clause true.  Does the weakest assignment exist?

Question: can we ensure each clause involves only one or two
unknowns?
}

We add the following rules to allow type arguments to calls to
be omitted.

\infrule[T-invk-inferred]{
\xbar{Y}~\mbox{fresh}
\\
\Gamma, \xbar{Y} \ty {\tt type}
\vdash
\Xcd{e}_0.\Xcd{m[}\xbar{Y}\Xcd{](}\xbar{e}\Xcd{)} \ty
\Xcd{T}
}{
\Gamma \vdash
\Xcd{e}_0.\Xcd{m(}\xbar{e}\Xcd{)} \ty
\Xcd{T}
}

\infrule[T-new-inferred]{
\xbar{Y}~\mbox{fresh}
\\
\Gamma, \xbar{Y} \ty {\tt type}
\vdash
\Xcd{new}~\Xcd{C[}\xbar{Y}\Xcd{](}\xbar{e}\Xcd{)} \ty
\Xcd{T}
}{
\Gamma \vdash
\Xcd{new}~\Xcd{C(}\xbar{e}\Xcd{)} \ty
\Xcd{T}
}

\subsection{Constraint representation}

%\newcommand\eqedge{\rightleftharpoons}
\newcommand\eqedge{\sim}
\newcommand\flowedge{\to}
\newcommand\treeedge[1]{\mapsto_{#1}}
\newcommand\typeedge{\mapsto_{\tt type}}

Represent a constraint as a graph $G$.
Each node represents a constraint term for a value or a type.
The node for a path $p$ is written $v_p$;
the node for a type $T$ is written $V_T$.
There are four kinds of edges:
\begin{enumerate}
\item undirected equivalence edges,
        $v_p \eqedge v_q$ and $V_S \eqedge V_T$,
\item type edges, $v_p \typeedge V_T$,
\item tree edges, $v_p \treeedge{f} v_{p.f}$
              and $v_p \treeedge{X} V_{p.X}$, and
\item flow edges, $V_S \flowedge V_T$.
\end{enumerate}

First, each constraint term is mapped to a node in the graph as
follows.
Associate each term $t$ with a node
$v_t$.  For each access path {\tt p.x}, add a tree edge
$v_{{\tt p}} \treeedge{{\tt x}} v_{{\tt p.x}}$.
For each path type {\tt p.X}, add a tree edge
$v_{{\tt p}} \treeedge{{\tt X}} V_{{\tt p.X}}$.
For each atomic formula ${\tt f}(\xbar{t})$, add the tree edge
$v_{{\tt f}(\xbar{t})} \treeedge{i} v_{t_i}$ for all $i$.
If term $t$ has type $T$, add $v_t \typeedge V_{t{\tt .type}}$
and
add $V_T \eqedge V_{t{\tt .type}}$ to $G$.

Type nodes are sets of classes.

Next, constraints are incorporated into the graph:

\begin{itemize}
\item
For constraint {\tt p==q}, add $v_{\tt p} \eqedge v_{\tt q}$ to $G$.

\item
For constraint {\tt S==T}, add $V_{\tt S} \eqedge V_{\tt T}$ to $G$.

\item
For constraint {\tt S<:T},
add $V_{\tt S} \flowedge V_{\tt T}$
to $G$.

\end{itemize}

\subsection{Solving}

A flow-path is a path that follows flow and equivalence edges
only.
A type-path is a path that follows type and equivalence edges
only.

Now, we saturate: 
If there is a type-path $v_t \typeedge^* V_{\tt C\{c\}}$,
add $c[t/\Xcd{self}]$ to the worklist.

        Can saturate lazily when doing a lookup.
        EXCEPT: a type may have an arbitrary constraint
                \xcd"C{self.x==3 && y > 7}", so affect is non-local
        EXCEPT: c is x.f==...
                with x: C{c}
                need to avoid infinite loop

To check:

\begin{itemize}
\item To check
constraint {\tt p==q}, check if $v_{\tt p} \eqedge^* v_{\tt q}$.
\item To check
constraint {\tt S<:T}, check if there is a flow-path from $V_{\tt S}$ to
$V_{\tt T}$.  This requires checking entailment of the type constraints and
adding more edges to the graph.  (XXX details!)
Add the flow edge to memoize.
\end{itemize}


}

\section{Semantics}
\label{sec:semantics}
%\setlength{\afterruleskip}{\smallskipamount}
%\setlength{\afterruleskip}{\medskipamount}

\newcommand{\constraint}{{\tt constraint}}
\newcommand\cj[2]{{#1} \vdash {#2}~\constraint}
\newcommand\cjj[3]{{#1} \vdash {#2}~\constraint, {#3}~\constraint}
\newcommand\wj[2]{{#1} \vdash {#2}~\type}
\newcommand\tj[3]{{#1} \vdash {#2} \ty {#3}}
\newcommand\stj[3]{{#1} \vdash {#2} \subtype {#3}}

We now describe the semantics of languages in the \FX{} family.
We start with a core \FXZ{} language that supports simple
\FJ-like types, then add value-dependent types and type-dependent
types, separately, then finally add both. For uniformity we declare type-valued parameters and properties to be of ``type'' \type, instead of using square brackets to demarcate them.

\subsection{\FXZ}

The grammar for \FXZ{} is shown in Figure~\ref{fig:fx-grammar}.
The syntax is essentially that of \FJ{}.
Following the convention of \FJ{}, we use $\bar{x}$ to denote a
list $x_1, \dots, x_n$, and use $\bullet$ to denote the empty
list.

A program {\tt P} is a set of class declarations \tbar{L}.
Class names {\tt C} range over the declared classes in {\tt P} 
and {\tt Object}.
Classes have
properties (i.e., immutable fields) \tbar{f} and methods \tbar{M}.  We omit constructors
and require that the \new{} expression provide initializers
for all fields, including inherited fields. 
Methods are introduced with the {\tt def} keyword.

Both classes and methods may have constraint clauses
{\tt c}.  In the case of classes, {\tt c} is to be thought of as an
{\em invariant} satisfied by all instances of the class; in the case of
methods, {\tt c} is an additional condition, or {\em guard},
that must be satisfied by
the receiver 
and the actual arguments of the method in order for the method to
be applicable.

Expressions {\tt e} are either parameters {\tt x} (including the implicit
method parameter {\tt this}), field accesses, method invocations, \new{}
expressions, or casts (written {\tt e}~\as~{\tt T}).

The set of types includes classes {\tt C} and is closed under
constrained types ($\tt T\{c\}$) and existential
quantification ($\exists \tt x:T.~U$).
A value {\tt v} is of type {\tt C} if it is an instance of class {\tt C}; it is of type $\tt
T\{c\}$ if it is of type {\tt T} and it satisfies the constraint $\tt
c[v/self]$; it is of type $\exists \tt x:T.~U$
if there is some value {\tt w}
of type {\tt T} such that {\tt v} is of type
$\tt U[w/x]$.

The syntax for constraints in \FXZ{} is specified in
Figure~\ref{fig:fx-grammar}. As expected, constraints
relate property fields of objects. Neither casts
nor method invocations are permitted in constraints.

We distinguish a subset of these constraints as
{\em user constraints}---these are permitted to occur in
programs. For \FXZ{} the only user constraint permitted is the vacuous
{\tt true}. Thus the types occurring in user programs are isomorphic
to class types, and class and method definitions specialize to the
standard class and method definitions of \FJ{}. 

The constraints permitted by the syntax in
Figure~\ref{fig:fx-grammar} that
are not user constraints are used to define the static and
dynamic semantics of \FXZ{} (see, e.g., rule \TField{} in Figure~\ref{fig:FX}).
The use of this richer constraint set as well as constrained and existential types is
not necessary in \FXZ; it simply enables us to present the static and dynamic
semantics once for the entire family of \FX{} languages,
specifying the other members of the family as extensions
to these core semantics.

Existential constraints are introduced for convenience only:
${\tt T}\{\exists {\tt x}:{\tt U}.~{\tt c}\}$ is equivalent to $\exists {\tt y}:{\tt U}.~{\tt T}\{{\tt c}[{\tt y}/{\tt x}]\}$ choosing {\tt y} not free in {\tt T}.

\paragraph{Dynamic semantics.}
The operational semantics, shown in Figure~\ref{fig:sos},
is straightforward and essentially identical
to \FJ \cite{FJ}. It is described in terms of a non-deterministic
reduction relation on expressions.\eat{\footnote{
For simplicity, we enforce call-by-value-like semantics:
\RField, \RInvk, and \RCast{} require receivers of the form
``$\new~{\tt C}(\tbar{t})$'' instead of ``$\new~{\tt C}(\tbar{e})$''.
Otherwise, we would have to distinguish compile-time constraints over constraint terms from run-time constraints over expressions.}}

The only novelty is the use of the
subtyping relation to check that the cast is satisfied.
The typing rule for casts ({\sc T-New}) in Figure~\ref{fig:FX} specifies that if the arguments $\tbar{e}$ have type $\tbar{V}$ then $\new~{\tt C}(\tbar{e})$ has type $\exists\tbar{y}:\tbar{V}.~{\tt C}\{\self==\new~{\tt C}(\tbar{y})\}$, therefore {\RCast} requires this particular type to be a subtype of {\tt T}.
In \FXZ, this
test simply involves checking that the class of which the object is an
instance is a subclass of the class specified in the given type; in
languages with richer notions of type this operation may
involve run-time constraint solving using the properties of the object.
See Section~\ref{sec:casts} for further discussion of the casts,
including decidability issues.

\paragraph{Static Semantics.}
Each language in the family is defined over a given input constraint system $\mathcal{X}$ that is required to support the trivial constraint \true{}, conjunction, existential quantification, and equality on constraint terms. Given a program {\tt P}, we now show how to derive from $\mathcal{X}$ a larger deduction system that captures the object-oriented structure of {\tt P} and lets us decide whether {\tt P} is well typed.

In the following, the context $\Gamma$ is always a
(finite, possibly empty) sequence of formulas $\tt x:T$ and constraints $\tt c$ satisfying:
\begin{enumerate}
  \item for any formula $\phi=\tt x:T$ or constraint $\phi=\tt c$ in the sequence all free variables $\tt y$
  occurring in $\tt T$ or $\tt c$ are declared by a formula $\tt
  y:U$ in the sequence to the left of $\phi$.

  \item for any variable $\tt x$, there is at most one
  formula $\tt x:T$ in $\Gamma$.
\end{enumerate}

\medskip

In the judgments that follow, 
for any formulas $\phi_1$ and
$\phi_2$, we adopt the convention that the rule $\Gamma \vdash \phi_1,~\phi_2$
is shorthand for the rules
$\Gamma \vdash \phi_1$
and
$\Gamma \vdash \phi_2$. Whenever
we state an assumption of the form ``{\tt x} fresh'' in a rule we mean
it is not free in the consequent of the rule.

The judgments of interest are as follows:
\begin{enumerate}
	\item Well-formedness:\\
	  $\cj{\Gamma}{\tt c}$ \hfill constraint {\tt c} is well formed\\
	  $\wj{\Gamma}{\tt T}$ \hfill  type {\tt T} is well formed
	\item Member lookup:\\
	  $\fields({\tt C})=\tbar{f}:\tbar{T}$ \hfill class {\tt C} has fields \tbar{f} of type \tbar{T}\\
	  $\Gamma\vdash {\tt C}~\has~{\tt I}$ \hfill class {\tt C} has member {\tt I}\\
	  $\Gamma\vdash {\tt x}~\underline\has~{\tt I}$ \hfill variable {\tt x} has member {\tt I}\\
	  $~$ \hfill where ${\tt I}::= {\tt f}:{\tt T} \alt {\tt m}(\tbar{x}:\tbar{T})\{{\tt c}\}:{\tt U}={\tt e}$
	\item Constraints:\\
	  $\Gamma\vdash {\tt c}$ \hfill constraint {\tt c} holds
	\item Typing:\\
	  $\Gamma\vdash {\tt e}:{\tt T}$ \hfill expression {\tt e} has type {\tt T}\\
	  $\vdash {\tt def}~{\tt m}(\tbar{x}:\tbar{T})\{{\tt c}\}:{\tt U}={\tt e}~{\rm OK~in}~{\tt C}$ \\ $~$ \hfill method {\tt m} in class {\tt C} type checks\\
	  $\vdash {\tt class}~{\tt C}(\tbar{f}:\tbar{T})\{{\tt c}\}~{\tt extends}~{\tt D}~\{~\tbar{M}~\}~{\rm OK}$ \\ $~$ \hfill class {\tt C} type checks
	\item Subtyping:\\
	  $\Gamma \vdash {\tt S} \subtype {\tt T}$ \hfill type {\tt S} is a subtype of type {\tt T}
\end{enumerate}

\medskip

A program type checks iff all its classes do.  We now describe
in more detail each of these judgments, in turn.

\paragraph{1. Well-formedness.} A constraint {\tt c} is well
formed in context $\Gamma$, written $\cj{\Gamma}{\tt c}$, iff
both
its free variables are declared in $\Gamma$ and it is well
formed for the underlying constraint system. The rules in Figure~\ref{fig:well} extend this requirement to types.

We say a program, context, or judgment is well formed iff all the constraints and types involved are well formed. By design, every judgment derived from a well-formed context is also well formed. As a consequence, if a program type checks, it is well formed.

\paragraph{2. Member lookup.} Figure~\ref{fig:lookup} specifies the fields and methods available on each class. We impose restrictions on inheritance as follows. Classes may only be extended by classes with stronger invariants. Fields cannot be overridden. Methods cannot be overloaded.
A method may only be overridden by a method with the same name,
arity, parameter types, a return type that is a subtype of the
overridden return type, and a weaker guard (i.e., the superclass
method guard must entail the subclass's).
\footnote{To avoid cluttering the inference rules, we
define overriding only informally; a formal definition is
straightforward.}

To prepare for the introduction of generic types later, we
distinguish members that are available on variables from members
available on classes.  For now,
${\tt x}:{\tt C}~\underline\has~{\tt I}[{\tt x}/\this]$ iff ${\tt C}~\has~{\tt I}$.

\begin{figure*}
\centering
\begin{tabular}{r@{\quad}rcl}
  (Program) & {\tt P} &{::=}& $\tbar{L}$ \\
  (Class declaration) & {\tt L} &{::=}& $\tt class~C(\tbar{f}:\tbar{T})\{c\}~extends~D~\{~\tbar{M}~\}$ \\
  (Method declaration)& {\tt M} &{::=}& $\tt def~m(\tbar{x}:\tbar{T})\{c\}:T=e;$ \\
  (Expression)& {\tt e} &{::=}& $\tt x$ \alt $\tt e.f$ \alt $\tt\new~C(\tbar{e})$ \alt $\tt e.m(\tbar{e})$ \alt $\tt e~\as~T$ \\
  (Value)& {\tt v} &{::=}& $\tt\new~C(\tbar{v})$ \\
  (Type)& {\tt T} &{::=}& $\tt C$ \alt $\tt T\{c\}$ \alt $\tt \exists x:T.~T$ \\
  (Constraint term) & {\tt t} &{::=}& $\tt x$ \alt $\tt t.f$ \alt $\tt\new~C(\tbar{t})$ \\
  (Constraint) & {\tt c} &{::=}& $\true$ \alt $\tt t==t$ \alt $\tt c,c$ \alt $\tt \exists x:T.~c$ \\
\end{tabular} 
\caption{\FX{} productions.
{\tt C} ranges over class names, {\tt f} over field names, {\tt m} over method names, {\tt x} over variable names.}
\label{fig:fx-grammar}
\end{figure*}


\begin{figure*}
\vspace{-\bigskipamount}
\begin{minipage}{.4\textwidth}
\quad\typicallabel{XXXXXX}
\infrule[\RField]
	{\fields({\tt C})=\tbar{f}:\tbar{T}}
	{\new~{\tt C}(\tbar{e}).{\tt f}_i \derives {\tt t}_i}

\infrule[\RCField]
	{{\tt e}\derives {\tt e}'}
	{{\tt e}.{\tt f}\derives {\tt e}'.{\tt f}}

\infrule[\RCInvkRecv]
	{{\tt e}\derives {\tt e}'}
	{{\tt e}.{\tt m}(\tbar{a})\derives {\tt e}'.{\tt m}(\tbar{a})}

\infrule[\RCCast]
	{{\tt e}\derives {\tt e}'}
	{{\tt e}~\as~{\tt T}\derives {\tt e}'~\as~{\tt T}}
\end{minipage}%
\begin{minipage}{.6\textwidth}
\quad\typicallabel{XXXXXX}
\infrule[\RCNewArg]
	{{\tt e}_i\derives {\tt e}'_i}
	{\new~{\tt C}(\ldots,{\tt e}_i,\ldots)\derives\new~{\tt C}(\ldots,{\tt e}'_i,\ldots)}

\infrule[\RInvk]
	{{\tt C}~\has~{\tt m}(\tbar{x}:\tbar{T})\{{\tt c}\}:{\tt U}={\tt b}}
	{\new~{\tt C}(\tbar{e}).{\tt m}(\tbar{a})\derives {\tt b}[\new~{\tt C}(\tbar{e}),\tbar{a}/\this,\tbar{x}]}

\infrule[\RCInvkArg]
	{{\tt a}_i\derives {\tt a}'_i}
	{{\tt e}.{\tt m}(\ldots,{\tt a}_i,\ldots)\derives {\tt e}.{\tt m}(\ldots,{\tt a}'_i,\ldots)}

\infrule[\RCast]
	{\vdash \tbar{e}:\tbar{V} \andalso \vdash\exists\tbar{y}:\tbar{V}.~{\tt C}\{\self==\new~{\tt C}(\tbar{y})\}\subtype {\tt T}}
	{\new~{\tt C}(\tbar{e})~\as~{\tt T}\derives\new~{\tt C}(\tbar{e})}
\end{minipage}
\caption{\FX{} operational semantics}
\label{fig:sos}
\end{figure*}


\begin{figure*}
\vspace{-\bigskipamount}
\begin{minipage}{.5\textwidth}
\quad\typicallabel{XXXXXX}
\infax[W-Object]
  {\wj{}{\tt Object}}

\infrule[W-Class]
  {{\tt class}~{\tt C}(\tbar{f}:\tbar{T})\{{\tt c}\}~{\tt extends}~{\tt D}~\{~\tbar{M}~\} \in {\tt P}}
  {\wj{}{\tt C}}
\end{minipage}%
\begin{minipage}{.5\textwidth}
\quad\typicallabel{XXXXXX}
\infrule[W-Dep]
        {\wj{\Gamma}{\tt T} \andalso \cj{\Gamma, \self \ty {\tt T}}{\tt c}}
	{\wj{\Gamma}{{\tt T}\{{\tt c}\}}}

\infrule[W-Exists]
        {\wj{\Gamma}{\tt T} \andalso \wj{\Gamma, {\tt x} \ty {\tt T}}{\tt U}}
        {\wj{\Gamma}{\exists {\tt x} \ty {\tt T}.~{\tt U}}}
\end{minipage}
\caption{\FX{} well-formed types}
\label{fig:well}
\end{figure*}

\begin{figure*}
\vspace{-\bigskipamount}
\begin{minipage}{.32\textwidth}
\quad\typicallabel{XXXXXXXX}
\infax[L-Fields-Obj]
  {\vdash\fields({\tt Obj})=\bullet}
\end{minipage}%
\begin{minipage}{.31\textwidth}
\quad\typicallabel{XXXXXX}
\infrule[L-Field-B]
  {\vdash\fields({\tt C})=\tbar{f}:\tbar{T}}
  {{\tt C}~\has~{\tt f}_i:{\tt T}_i}
\end{minipage}
\begin{minipage}{.37\textwidth}
\quad\typicallabel{XXXXXX}
\infrule[L-Member-B]
  {{\tt C}~\has~{\tt I}}
  {{\tt x}:{\tt C}\vdash {\tt x}~\underline\has~{\tt I}[{\tt x}/\this]}
\end{minipage}%

\begin{minipage}{.5\textwidth}
\quad\typicallabel{XXXXXX}
\infrule[L-Member-C]
  {\Gamma,{\tt x}:{\tt T},{\tt c}\vdash {\tt x}~\underline\has~{\tt I}}
  {\Gamma,{\tt x}:{\tt T}\{{\tt c}\}\vdash {\tt x}~\underline\has~{\tt I}}
\end{minipage}
\begin{minipage}{.5\textwidth}
\quad\typicallabel{XXXXXX}
\infrule[L-Member-E]
  {\Gamma,{\tt y}:{\tt U},{\tt x}:{\tt T}\vdash {\tt x}~\underline\has~{\tt I} \andalso {\tt y}~\rm fresh}
  {\Gamma,{\tt x}:\exists {\tt y}:{\tt U}.~{\tt T}\vdash {\tt x}~\underline\has~{\tt I}}
\end{minipage}

\begin{minipage}{\textwidth}
\quad\typicallabel{XXXXXX}
\infrule[L-Fields-I]
  {{\tt class}~{\tt C}(\tbar{f}:\tbar{T})\{{\tt c}\}~{\tt extends}~{\tt D}~\{~\tbar{M}~\}\in {\tt P}
   \andalso
   \vdash\fields({\tt D})=\tbar{g}:\tbar{U}}
   {\vdash\fields({\tt C})=\tbar{g}: \tbar{U}, \tbar{f}: \tbar{T}}

\infrule[L-Method-B]
  {{\tt class}~{\tt C}(\tbar{f}:\tbar{T})\{{\tt c}\}~{\tt extends}~{\tt D}~\{~\tbar{M}~\}\in {\tt P}
   \andalso
   {\tt def}~{\tt m}(\tbar{x}\ty \tbar{U})\{{\tt d}\}\ty {\tt V}={\tt e}\in \tbar{M}}
  {\vdash {\tt C}~\has~{\tt m}(\tbar{x}:\tbar{U})\{{\tt d}\}:{\tt V}={\tt e}}

\infrule[L-Method-I]
  {{\tt class}~{\tt C}(\tbar{f}:\tbar{T})\{{\tt c}\}~{\tt extends}~{\tt D}~\{~\tbar{M}~\}\in {\tt P}
   \andalso
   \vdash {\tt D}~\has~{\tt m}(\tbar{x}:\tbar{U})\{{\tt d}\}:{\tt V}={\tt e}
   \andalso
   {\tt m}\not\in\tbar{M}}
  {\vdash {\tt C}~\has~{\tt m}(\tbar{x}:\tbar{U})\{{\tt d}\}:{\tt V}={\tt e}}
\end{minipage}

\caption{\FX{} member lookup}
\label{fig:lookup}
\end{figure*}

\begin{figure*}
\begin{minipage}[t]{.2\textwidth}
\quad\typicallabel{XXXXXX}
\infrule[X-Proj]
  {\sigma(\Gamma)\vdashX{\tt c}}
  {\Gamma\vdash {\tt c}}
\end{minipage}
\begin{minipage}[t]{.8\textwidth}
\quad\typicallabel{XXXXXX}
\infrule[X-Sel]
  {\vdash\fields({\tt C})=\tbar{f}:\tbar{T}
  	\andalso
  	\Gamma\vdash\new~{\tt C}(\tbar{t}):{\tt U}
  	\andalso
  	\Gamma, \new~{\tt C}(\tbar{t}).\tbar{f}==\tbar{t}\vdash {\tt c}}
  {\Gamma\vdash {\tt c}}

\infrule[X-Inv]
  {{\tt class}~{\tt C}(\tbar{f}:\tbar{T})\{{\tt c}\}~{\tt extends}~{\tt D}~\{~\tbar{M}~\}\in {\tt P}
   \andalso
   \Gamma\vdash {\tt t}:{\tt U},~{\tt U}\subtype{\tt C}\andalso \Gamma,{\tt c}[{\tt t}/\this]\vdash {\tt d}}
  {\Gamma\vdash {\tt d}}
\end{minipage}%
\caption{\FX{} object constraint system}
\label{fig:object}
\end{figure*}

\begin{figure*}
\vspace{-\bigskipamount}
\begin{minipage}{.4\textwidth}
\quad\typicallabel{XXXXXX}
\infax[T-Var]
  {\Gamma,{\tt x}:{\tt T}\vdash {\tt x}:{\tt T}\{\self=={\tt x}\}}
\end{minipage}
\begin{minipage}{.5\textwidth}
\quad{}\typicallabel{XXXXXX}
\infrule[T-Cast]
	{\Gamma\vdash {\tt e}:{\tt U}\andalso\wj{\Gamma}{\tt T}}
	{\Gamma\vdash {\tt e}~\as~{\tt T}:{\tt T}}
\end{minipage}
\begin{minipage}{\textwidth}
\quad{}\typicallabel{XXXXXX}
\infrule[T-Field]
	{\Gamma\vdash {\tt e}:{\tt T} \andalso \Gamma,{\tt x}:{\tt T}\vdash {\tt x}~\underline\has~{\tt f}:{\tt U} \andalso {\tt x}~\rm fresh}
	{\Gamma\vdash {\tt e}.{\tt f}:\exists {\tt x}:{\tt T}.~{\tt U}\{\self=={\tt x}.{\tt f}\}}

\infrule[T-Invk]
	{\Gamma\vdash {\tt e}:{\tt T},~\tbar{a}:\tbar{U} \andalso 
	  \Gamma,{\tt x}:{\tt T},\tbar{y}:\tbar{U}\vdash {\tt
          x}~\underline\has~{\tt m}(\tbar{y}:\tbar{V})\{{\tt d}\}:{\tt W}={\tt b},~\tbar{U}\subtype\tbar{V},~{\tt d} \andalso {\tt x},\tbar{y}~\rm fresh}
	{\Gamma\vdash {\tt e}.{\tt m}(\tbar{a}):\exists {\tt x}:{\tt T}.~\exists\tbar{y}:\tbar{U}.~{\tt W}}

\infrule[T-New]
	{{\tt class}~{\tt C}(\tbar{f}:\tbar{T})\{{\tt c}\}~{\tt extends}~{\tt D}~\{~\tbar{M}~\}\in {\tt P}
		\andalso
		\fields({\tt C})=\tbar{g}:\tbar{U} \\
	  \Gamma\vdash\tbar{e}:\tbar{V} \andalso
    \Gamma,{\tt x}:{\tt C},\tbar{y}:\tbar{V},{\tt x}.\tbar{g}==\tbar{y}\vdash
    \tbar{V}\subtype\tbar{U}[{\tt x}/\this],~{\tt c}[{\tt x}/\this] \andalso {\tt x},\tbar{y}~\rm fresh}
	{\Gamma\vdash\new~{\tt C}(\tbar{e}):\exists\tbar{y}:\tbar{V}.~{\tt C}\{\self==\new~{\tt C}(\tbar{y})\}}
        
\infrule[OK-Method]
  {{\tt class}~{\tt C}(\tbar{f}:\tbar{T})\{{\tt c}\}~{\tt extends}~{\tt D}~\{~\tbar{M}~\}\in {\tt P} \andalso
    {\tt def}~{\tt m}(\tbar{x}:\tbar{U})\{{\tt d}\}:{\tt V}={\tt e}\in\tbar{M}\\
    \this:{\tt C},\tbar{x}:\tbar{U}\vdash\tbar{U}~\type,~{\tt d}~\constraint
    \andalso
    \this:{\tt C},\tbar{x}:\tbar{U},{\tt d}\vdash {\tt e}:{\tt W},~{\tt W}\subtype {\tt V}}
  {\vdash {\tt def}~{\tt m}(\tbar{x}:\tbar{U})\{{\tt d}\}:{\tt V}={\tt e}~{\rm OK~in}~{\tt C}}

\infrule[OK-Class]
  {\this:{\tt C},\tbar{f}:\tbar{T}\vdash\tbar{T}~\type,~{\tt c}~\constraint \andalso \wj{}{\tt D} \andalso \tbar{M}~{\rm OK~in}~{\tt C}}
  {\vdash {\tt class}~{\tt C}(\tbar{f}:\tbar{T})\{{\tt c}\}~{\tt extends}~{\tt D}~\{~\tbar{M}~\}~\rm OK}
\end{minipage}
\caption{\FX{} typing rules}\label{fig:FX}
\end{figure*}


\begin{figure*}
\vspace{-\bigskipamount}
\begin{minipage}{.25\textwidth}
\quad\typicallabel{XXX}
\infrule[S-Refl]
  {\wj{\Gamma}{\tt T}}
  {\Gamma\vdash {\tt T}\subtype {\tt T}}
\end{minipage}%
\begin{minipage}{.30\textwidth}
\quad\typicallabel{XXX}
\infrule[S-Trans]
	{\Gamma\vdash {\tt T}\subtype {\tt U}, {\tt U}\subtype {\tt V}}
	{\Gamma\vdash {\tt T}\subtype {\tt V}}
\end{minipage}%
\begin{minipage}{.45\textwidth}
\quad\typicallabel{XXXX}
\infrule[S-Const-L]
	{\Gamma\vdash{\tt T}\{{\tt c}\}~\type \andalso\Gamma,{\tt c}\vdash{\tt T}\subtype {\tt U}}
	{\Gamma\vdash {\tt T}\{{\tt c}\}\subtype {\tt U}}
\end{minipage}%

\begin{minipage}{.5\textwidth}
\quad\typicallabel{XXXXX}
\infrule[S-Class]
  {{\tt class}~{\tt C}(\tbar{f}:\tbar{T})\{{\tt c}\}~{\tt extends}~{\tt D}~\{~\tbar{M}~\}\in {\tt P}}
  {\Gamma\vdash {\tt C}\subtype {\tt D}}

\infrule[S-Exists-L]
  {\wj{\Gamma}{\tt T} \andalso \Gamma,{\tt x}:{\tt T}\vdash {\tt U}\subtype {\tt V} \andalso {\tt x}~\rm fresh}
  {\Gamma\vdash\exists {\tt x}:{\tt T}.~{\tt U}\subtype {\tt V}}
\end{minipage}%
\begin{minipage}{.5\textwidth}
\quad\typicallabel{XXXXX}
\infrule[S-Const-R]
	{\wj{\Gamma}{{\tt U}\{{\tt c}\}}\andalso\Gamma,\self:{\tt T}\vdash {\tt c},{\tt T}\subtype {\tt U} }
	{\Gamma\vdash {\tt T}\subtype {\tt U}\{{\tt c}\}}

\infrule[S-Exists-R]
  {\Gamma\vdash {\tt t}:{\tt T},~{\tt U}\subtype {\tt V}[{\tt t}/{\tt x}]}
  {\Gamma\vdash {\tt U}\subtype\exists {\tt x}:{\tt T}.~{\tt V}}
\end{minipage}        
\caption{\FX{} subtyping rules}\label{fig:subtyping}
\end{figure*}


\begin{figure*}
\vspace{-\bigskipamount}
\begin{minipage}{.5\textwidth}
\quad
\typicallabel{XXX}
\infax[W-Type]
	{\wj{}{\type}}

\infrule[S-Extends]
	{\Gamma\vdash {\tt T}\extends {\tt U} \andalso \Gamma\vdash{\tt T}:\type,~{\tt U}:\type}
	{\Gamma\vdash {\tt T}\subtype {\tt U}}

\infrule[L-Extends]
	{\Gamma\vdash {\tt T}\extends{\tt U}
         \andalso
         \Gamma,{\tt x}:{\tt U}\vdash {\tt x}~\underline\has~{\tt I}}
	{\Gamma,{\tt x}:{\tt T}\vdash {\tt x}~\underline\has~{\tt I}}
\end{minipage}%
\begin{minipage}{.5\textwidth}
\quad\typicallabel{XXX}
\infax[W-Var]
	{\wj{\Gamma,{\tt x}:\type}{\tt x}}

\infrule[W-Path]
	{\Gamma\vdash {\tt p}:{\tt T}
         \andalso
         \Gamma,{\tt x}:{\tt T}\vdash {\tt x}~\underline\has~{\tt f}:\type}
	{\wj{\Gamma}{{\tt p}.{\tt f}}}

\infrule[T-Type]
        {\wj{\Gamma}{\tt C\{{\tt c}\}}}
  {\Gamma\vdash{\tt C}\{{\tt c}\}:\type\{\self=={\tt C}\{{\tt c}\}\}}
\end{minipage}
\caption{\FXG}
\label{fig:FXG}
\end{figure*}

\paragraph{3. Constraints.}
In defining these judgments we will use \mbox{$\Gamma \vdashX {\tt c}$}, the judgment corresponding to the underlying constraint system. Formally, $\cal X$ permits judgments of the form $\Gamma \vdashX {\tt c}$ where $\Gamma::=\tbar{c}$ is a (finite, possibly empty) sequence of constraints. We define the {\em constraint
projection}, $\sigma(\Gamma)$ as follows.
%
\begin{center}
\begin{tabular}{l}
$\sigma(\epsilon)={\tt true}$\\
$\sigma({\tt x}:{\tt C}, \Gamma)=\sigma(\Gamma)$\\
$\sigma({\tt x}:{\tt T\{c\}}, \Gamma)={\tt c}[{\tt x}/\self], \sigma({\tt x}:{\tt T},\Gamma)$\\
$\sigma({\tt x}:\exists {\tt y}:{\tt T}.~{\tt U}, \Gamma)=\sigma({\tt z:T}, {\tt x}:{\tt U[{\tt z}/{\tt y}]},\Gamma)$\\
$\sigma({\tt c},\Gamma) = {\tt c}, \sigma(\Gamma)$
\end{tabular}
\end{center}
%
Above, in the fourth rule, 
we assume that alpha-equivalence is used to
choose a variable {\tt z} that does not
occur in the context under construction.

We define $\Gamma\vdash {\tt c}$ in Figure~\ref{fig:object}. {\sc X-Sel} permits the underlying constraint system $\mathcal{X}$ to interpret field accesses. {\sc X-Inv} handles class invariants. Note the use of subtyping here to prepare for the introduction of bounds on generics types later.

We say that a context $\Gamma$ is {\em consistent} if all (finite)
subsets of $\{{\tt c}\alt \sigma(\Gamma) \vdash {\tt c}\}$ are consistent.
In all type judgments presented below ({\sc T-Cast}, {\sc T-Field},
etc.), we make the implicit assumption that the context $\Gamma$ is
consistent; if it is inconsistent, the rule cannot be used and the
type of the given expression cannot be established (type-checking
fails).

\eat{
The following rules govern existential constraints:
\infrule[Exists-R]
  {\Gamma\vdash {\tt t}:{\tt T} \andalso \Gamma\vdash{\tt c}[{\tt t}/{\tt x}]}
  {\Gamma\vdash \exists {\tt x}:{\tt T}.~{\tt c}}

\infrule[Exists-L]
  {\Gamma,{\tt x}:{\tt T},{\tt c}\vdash {\tt d} \andalso {\tt x}~\rm fresh}
  {\Gamma,\exists {\tt x}:{\tt T}.~{\tt c}\vdash {\tt d}}
}

\paragraph{4. Typing.} The typing rules are specified in Figure~\ref{fig:FX}.

{\sc T-Var} is as expected, except that it asserts the constraint {\tt
self==x}, which records that any value of this type is known
statically to be equal to {\tt x}. This constraint is actually very
crucial---as we shall see in the other rules, once we establish that
an expression {\tt e} is of a given type {\tt T}, we ``transfer'' the
type to a freshly chosen variable {\tt z}.  If, in fact, {\tt e} has a
static ``name'' {\tt x} (i.e., {\tt e} is known statically to be
equal to {\tt x}; that is, it has type {\tt T\{self==x\}}), then
{\sc T-Var} lets us assert that {\tt z:T\{self==x\}}, i.e., that {\tt z}
equals {\tt x}.
Thus {\sc T-Var} provides an important base case for
reasoning statically about equality of values in the environment.

We do away with the three cast rules in \FJ{} in favor of a single
cast rule, requiring only that {\tt e} be of some type {\tt U}.  At run time,
{\tt e} will be checked to see if it is actually of type {\tt T} (see
{\sc R-Cast} in Figure~\ref{fig:sos}).

{\sc T-Field} may be understood through ``proxy'' reasoning as
follows:  Given the context $\Gamma$, assume the receiver {\tt e} can
be established to be of type {\tt T}. Now, we do not know the run-time
value of {\tt e}, so we shall assume that it is some fixed but unknown
``proxy'' value {\tt x} (of type {\tt T}) that is ``fresh'' in that it
is not known to be related to any known value (i.e., those recorded
in $\Gamma$).  If we can establish that {\tt x} has a field {\tt f} of
type {\tt U}\footnote{Note from the definition of
\fields{} in Figure~\ref{fig:lookup} that all occurrences of
\this{} in the declared type of the field {\tt f} will have been replaced
by {\tt x}.}, then we can assert that
{\tt e.f} has type {\tt U} and, further, that it equals {\tt x.f}.
Hence, we can assert that {\tt e.f} has type 
$\exists {\tt x}:{\tt T}.~{\tt U}\{\self={\tt x}.{\tt f}\}$.

{\sc T-Invk} has a similar structure to {\sc T-Field}: we use
proxy reasoning for the receiver and the arguments of the method
call. {\sc T-New} also uses the same proxy reasoning: however in this case
we can establish that the resulting value is equal to $\new~{\tt C}(\tbar{y})$
for some values $\bar{\tt y}$ of the given types.

{\sc OK-Method} and {\sc OK-Class} ensure that the types and constraints occurring in a program are well formed. Following from \FJ{}, these rules do not preclude the existence of cycles in the type declarations. We assume they are acyclic. {\sc OK-Method} checks that the actual type {\tt W} of the method body {\tt e} is a subtype of its declared return type {\tt V}. {\sc OK-Class} makes sure all methods of the class type check.


\paragraph{5. Subtyping.} The subtyping relation is defined in Figure~\ref{fig:subtyping}.
Unsurprisingly, it is reflexive ({\sc S-Refl}) and transitive ({\sc S-Trans}).
{\sc S-Exists-L} and {\sc S-Exists-R} handle existential types.
{\sc S-Const-L} and {\sc S-Const-R} handle constraints. The rules ensure all types are well formed.

\subsection{\FXD}

Turning \FXZ{} into a language with value-dependent types is straightforward
since the construction of the previous section is parametric in the underlying constraint system $\mathcal{X}$, and constraint propagation is already built into the typing rules.

First, we assume we are given a constraint system $\cal A$ with a vocabulary of primitive types ${\tt A}$,
predicates ${\tt p}$, functions ${\tt q}$, and literals ${\tt l}$.

Second, we extend the productions of \FXZ{} as follows.
\begin{center}
\begin{tabular}{r@{\quad}rcl}
  (Type)& {\tt T} &{::=}& {\tt A} \\
  (Expression) & {\tt e} &{::=}& ${\tt q}(\tbar{e})$ \\
  (Values) & {\tt v} &{::=}& ${\tt l}$ \\
  (Constraint term) & {\tt t} &{::=}& ${\tt q}(\tbar{t})$ \\
  (Constraint) & {\tt c} &{::=}& ${\tt p}(\tbar{t})$ \\  
\end{tabular}
\end{center}

The following rules are needed to type functions and literals:

\infrule[T-Fun]
	{{\tt q}{\rm~has~type~\tbar{A} \rightarrow {\tt B}{\rm~in~}\mathcal{A}} \andalso \Gamma\vdash\tbar{e}:\tbar{A}} 
	{\Gamma\vdash{{\tt q}(\tbar{e}):\exists\tbar{x}:\tbar{A}.~{\tt B}\{\self=={\tt q}(\tbar{x})\}}}

\infrule[T-Lit]
	{{\tt l}{\rm~has~type~{\tt A} {\rm~in~}\mathcal{A}}} 
	{\Gamma\vdash{{\tt l}:{\tt A}\{\self=={\tt l}\}}}

We extend the operational semantics with rules:
\infrule[RC-Fun]
	{{\tt e}_i\derives {\tt e}'_i}
	{{\tt q}(\ldots,{\tt e}_i,\ldots)\derives {\tt q}(\ldots,{\tt e}'_i,\ldots)}

\infrule[R-Fun]
	{{\tt q}{\rm~has~type~\tbar{A} \rightarrow {\tt B}{\rm~in~}\mathcal{A}} \andalso \Gamma\vdash\tbar{v}:\tbar{A}} 
	{{\tt q}(\tbar{v})\derives \SB{{\tt q}(\tbar{v})}}
where $\SB{{\tt q}(\tbar{v})}$ denotes the result of the evaluation of function {\tt q} on values \tbar{v}.

Finally, we permit users to write constraints in $\cal A$ (except for existential constraints) in programs.
%
For instance, if $\mathcal{A}$ defines the type {\tt int}, integer literals, and the usual arithmetic operators, then we can declare:

\begin{xten}
class Count(n:int) extends Object {
  def inc():Count{self.n==this.n+1} =
  	new Count(this.n+1);
}
\end{xten}

In practice, it makes sense to distinguish the functions of the
constraint language from the functions of the base language.
One would define the {\sc T-Fun} typing judgment on a case-by-case
basis to
relate the interpretation of \xcd"q" as an expression to
its interpretation as a constraint term.

\FXD corresponds to the \CFJ calculus presented
in our prior work on constrained types~\cite{constrained-types}.  As described there, \Xten
supports equality constraints and has been extended with constraint
systems for Presburger arithmetic and for set constraints over
\Xten's array index domains (viz., regions).

\subsection{\FXG}
We now turn to showing how \FGJ{}-style generics can be supported in the \FX{} family.
\FXG{} is the language obtained by adding to \FXZ{} the
following productions:
\begin{center}
\begin{tabular}{r@{\quad}rcl}
  (Expression)& {\tt e} &{::=}& ${\tt C}\{{\tt c}\}$ \\
  (Value)& {\tt v} &{::=}& ${\tt C}\{{\tt c}\}$ \\
  (Path)& {\tt p} &{::=}& ${\tt x}$ \alt {\tt p}.{\tt f} \\
  (Type)& {\tt T} &{::=}& ${\tt p}$ \alt \type \\
  (Constraint term)& {\tt t} &{::=}& ${\tt T}$ \\
  (Constraint) & {\tt c} &{::=}& ${\tt T}\extends {\tt T}$
\end{tabular}
\end{center}
\noindent
and deduction rules of Figure~\ref{fig:FXG}.

First we introduce the ``type'' \type. \FGJ{} method type
parameters are modeled in \FXG{} as normal parameters of type
\type.\footnote{In concrete \Xten{} syntax type parameters are
distinguished from ordinary value parameters through the use of
``square'' brackets. This is particularly useful in implementing type
inference for generic parameters. We abstract these concerns away in
the abstract syntax presented in this section.}  Generic class
parameters are modeled as ordinary fields of type \type, with
parameter bound information recorded as a constraint in the class
invariant. This decision to use fields rather than parameters is
discussed further in Section~\ref{sec:parameters-vs-fields}. In brief,
it permits powerful idioms using fixed but unknown types without
requiring ``wildcards''.

The set of well-formed types is now enhanced to permit some fixed but unknown
types {\tt x} as well as \emph{path types} (cf. \cite{scala}),
i.e., type-valued fields of objects as types.\footnote{But we will not permit invocations of methods with return type \type\ to be 
used as types. This does indeed make sense, but developing
this theory further is beyond the scope of this paper.} We extend $\sigma$ in the obvious way:
%
\begin{center}
\begin{tabular}{l}
$	\sigma({\tt x}:\type, \Gamma)=\sigma(\Gamma)$\\
$\sigma({\tt x}:{\tt y}, \Gamma)=\sigma(\Gamma)$\\
$\sigma({\tt x}:{\tt p}.{\tt f}, \Gamma)=\sigma(\Gamma)$
\end{tabular}
\end{center}
%
Reciprocally, we permit class types ${\tt C}\{{\tt c}\}$ to be
used as expressions. We type them accordingly ({\sc T-Type}). In
contrast, the ``type'' \type{} is neither a valid expression nor
a class type: it has no field, method, subclass, or superclass.
It may however be constrained as usual as, for instance, in rule
{\sc T-Type}; that is to say, we permit equality constraints over types.\footnote{Type equality is just equality over uninterpreted functions.}

The key idea is that information about type-valued expressions can
be accumulated through constraints. Specifically we introduce 
the ``extends'' constraint ${\tt T}\extends{\tt U}$. It may be used, for
instance, to specify upper bounds on type variables or fields (path
types). In \FXG{}, users are permitted to specify ``$==$'' and ``$\extends$'' constraints
about type variables, fields, and class types.

\begin{example}
The \FGJ{} parametric method

\begin{xten} 
<T> T id(T x) { return x; }
\end{xten}
\noindent can be represented as
\begin{xten} 
def id(T: type, x: T): T = x;
\end{xten}
\end{example}

\begin{example}
\noindent The \FGJ{} class 
\begin{xten} 
class Comparator<B> {
  int compare(B y) { ... } }
class SortedList<T extends Comparator<T>> { 
  int m(T x, T y) { return x.compare(y); } }
\end{xten}
\noindent can be represented as
\begin{xtenmath} 
class Comparator(B: type) {
  def compare(y:B):int = ...; }
class SortedList(T: type)
    {T $\extends$ Comparator{self.B==T}} { 
  def m(x:T, y:T):int = x.compare(y); }
\end{xtenmath}
\end{example}

We require the underlying constraint system $\mathcal{G}$ to treat ``$\extends$'' as a partial order relation (reflexive, antisymmetric, and transitive). It is possible for a program to specify constraints incompatible with the class hierarchy, e.g., ${\tt x}\extends{\tt C}$ and ${\tt x}\extends{\tt D}$ if both class {\tt C} and class {\tt D} extend {\tt Object}. We therefore require $\mathcal{G}$ to treat as inconsistent all sets of constraints on type-valued variables that admit no valuations where these variables take on types as values.

The ``$\extends$'' constraint is used in two deduction rules. If
type {\tt T} extends type {\tt U}, then
\begin{itemize}
\item{\sc S-Extends}. {\tt T} is a subtype of {\tt U}. A method or constructor with argument type {\tt U} may be passed a parameter of type {\tt T}.
\item{\sc L-Extends}. If {\tt x} has type {\tt U} then {\tt x} has all the members of type {\tt T}. Note we only extend the ``$\underline\has$'' predicate that is used in typing judgments. On the other hand, the ``$\has$'' predicate used for method lookup in the operational semantics is not affected.
\end{itemize}

The modification of the lookup predicate is
necessary to permit typing method invocations with receivers of
generic types. It has the unfortunate side effect that we can no
longer ensure that type derivations---and hence types---are unique.
For instance, given the class definitions:
%
\begin{xten}
class A() extends Object { def m():A = new A(); }
class B() extends A { def m():B new B(); }
class C(f:type){this.f<=A} extends Object {}
class D(){this.f<=B} extends C { ..this.f.m().. }
\end{xten}
%
occurrences of $\this.{\tt f}$ in {\tt D} are bounded both by {\tt A} and {\tt B} hence 
$\this.{\tt f}.{\tt m}()$ may either be typed using the declaration of {\tt m} in {\tt A} or {\tt B}.

Another property of \FXG{} worth noticing is that casts can ``erase'' typing information.
Consider the program:
\begin{xten}
class C() extends Object {}
class D(f:type, g:this.f) extends Object {}
\end{xten}
Class {\tt D} has a type parameter {\tt f} and a value field {\tt g} of type {\tt f}.
Thanks to constraints, if
${\tt e}=\new~{\tt D}({\tt C},\new~{\tt C}())$,
then expression ${\tt e}.{\tt g}$ can be shown to
have type {\tt C}.
In contrast $({\tt e}~\as~{\tt D}).{\tt g}$ has type
$\exists {\tt x}:{\tt D}.{\tt x}.{\tt f}\{\self=={\tt x.g}\}$.
The type of $({\tt e}~\as~{\tt D}).{\tt g}$ is essentially ``unknown''
because the cast erased all information about it. In \Xten, we choose to shield users from existential types and only permit casts of the form $({\tt e}~\as~{\tt D}\{\self.{\tt f}=={\tt t}\})$ where {\tt t} is a type in scope (class type, type parameter, or path type).


\subsection{\FXGD} 

No additional rules are needed beyond those of \FXG{} and \FXD{}. This
language permits type and value constraints, supporting \FGJ{} style
generics and value-dependent types. All constraints but existential constraints are now user constraints.

\subsection{Results}
The following results hold for \FXGD supposing the program {\tt P} type checks.

\begin{theorem}[Subject Reduction] If $\Gamma$ is well formed and $\Gamma \vdash {\tt e:T}$ and ${\tt e} \derives {\tt e'}$, then
for some type {\tt S}, $\Gamma \vdash {\tt {\tt e}':{\tt S}},~{\tt S} \subtype {\tt T}$.
\end{theorem}

Values are of the form $\tt v ::= \new\ C(\bar{\tt v}) \alt {\tt l} \alt C\{c\}$.

\begin{theorem}[Progress]
If $\vdash {\tt e:T}$ then one of the following conditions holds:
\begin{enumerate}
\item {\tt e} is a value,
\item {\tt e} contains a cast sub-expression that is stuck,
\item there exists an $\tt e'$ s.t. $\tt e\derives e'$.
\end{enumerate}
\end{theorem}

\begin{theorem}[Type soundness]
If $\vdash {\tt e:T}$ and {\tt e}
reduces to a normal form ${\tt e'}$, then
either $\tt e'$ is a value {\tt v} and $\vdash {\tt v:S},{\tt S\subtype T}$ or
${\tt e'}$ contains  a stuck cast sub-expression.
\end{theorem}

\paragraph{Proof sketch.} The proof of the same results for a
formal language essentially equivalent to \FXD{} has been
reported in \cite{constrained-types}. We discuss here the key
insights that permit us to revise this proof in order to encompass \FXGD{}.
\begin{itemize}
\item Subject reduction. Having potentially multiple types for
an expression does not make the proof any harder as the subject
reduction theorem lets us choose {\tt S} among the possible types of ${\tt e}'$.

The main novelty of the \FXG{} type system is that it permits
the $\underline\has$ predicate to look for methods in arbitrary
superclasses or upper bounds of the type under scrutiny. This is
not so much a concern for fields as they cannot be overridden.
Because methods can, we must adapt the proof of subject reduction for the execution step corresponding to a method invocation (\RInvk).

First, we observe that the operational semantics rule for method
invocations (\RInvk) is required to employ the ``correct''
method for objects of class {\tt C}, that is, the first method
{\tt m} found on the inheritance path from class {\tt C} to
class {\tt Object} from the bottom up. Second, thanks
to overriding restrictions, we know that this method must have a
return type that is a subtype of any other method {\tt m}
defined in any superclass of {\tt C}. Finally, because
constraint sets incompatible with the class hierarchy are
inconsistent, we also know that the type of
$\new~{\tt C}(\tbar{e})$ cannot be constrained to have any upper
bound that is not {\tt C} itself or one of its superclasses.

We
therefore derive that any method instance one could use to type
the expression $\new~{\tt C}(\tbar{e}).{\tt m}(\tbar{a})$ has a
return type that is a supertype of the return type of the only
method instance that can be used to make a step of execution. We
assume the program type checks; hence, by {\sc OK-Method}, we
know that the actual residue ${\tt b}[\new~{\tt C}(\tbar{e}),\tbar{a}/\this,\tbar{x}]$ is guaranteed to have a
type that is a subtype of its declared type. Therefore, by
transitivity of the subtyping relation, we can derive that if
{\tt T} is a type of $\new~{\tt C}(\tbar{e}).{\tt m}(\tbar{a})$,
then there exists a type {\tt S} of
${\tt b}[\new~{\tt C}(\tbar{e}),\tbar{a}/\this,\tbar{x}]$
that is a subtype of {\tt T}.

\item Progress. \FXGD{} only differs from \FXD{} in that it
admits a new kind of expressions: {\tt C}\{{\tt c}\}. But these are also values, so the proof of progress is essentially unchanged.
\eat{
 {\tt l} ~$|$~ {\tt q}(\tbar{e}) ~$|$~ {\tt C}\{{\tt c}\}.
Both literals and class types are values, we just have to establish progress in the case of function calls (in the context of a proof by induction on the structure of the expression {\tt e}).
Assume ${\tt q}(tbar{e})$ is well typed then if all \tbar{e} are values progress is possible by rule {\sc R-Fun}. If not, then at leat one of the expressions is not a value and by induction hypothesis applied to this expression we can conclude that it is either stuck on a cast or can make a step of execution step. We conclude using rule {\sc RC-Fun} in the second case.}
\item{Type soundness}. Direct consequence of the previous two theorems.
\end{itemize}



\eat{We proceed by induction on the last rule used in the proof of ${\tt e} \derives {\tt e'}$. The key case is {\sc R-Invk}. Because of rule 


\begin{itemize}
\item {\sc RC-Field}. If $\Gamma\vdash S<:T$ then the fields of $T$ are the first fields of $S$.
\item {\sc RC-Invk-Recv}. If $\Gamma\vdash S<:T$ and $\Gamma,x:T\vdash x~\underline\has~m(y:U)\{c\}:V=a$ then there exists $d$ and $W$ such that $\Gamma,x:S\vdash x~\underline\has~m(y:U)\{d\}:W=b$ and $d$ entails $c$ and $W<:V$.
\item {\sc RC-Cast}. Straightforward.
\item {\sc RC-New-Arg}. If $\Gamma\vdash S<:T$ then $\exists x:T.U<:\exists x:S.U$.
\item {\sc RC-Invk-Arg}. Same.
\item {\sc R-Invk}. If $\Gamma,x:C\vdash x~\has~m(y:U)\{c\}:V=a$ then for all $d$ and $W$ such that $\Gamma,x:C\vdash x~\underline\has~m(y:U)\{d\}:W=b$ it is the case that $c$ entails $d$ and $V<:W$.
\item {\sc R-Field}. We have $\Gamma\vdash t:V$ for some $V$, $\Gamma\vdash \new~C(t).f_i:W_i\{\exists x:(\exists y:V.C\{\self==\new~C(y)\}).\self==x.f_i\}$ where $\Gamma\vdash C~\has~f_i:W_i$. We prove $\Gamma\vdash V_i<:W_i\{\exists x:(\exists y:V.C\{\self==\new~C(y)\}).\self==x.f_i\}$.
\item {\sc R-Cast}. Straightforward.
\end{itemize}
}


\section{Translation}
\label{sec:translation}
\label{sec:impl}
This section describes an implementation approach for generic types in
\Xten{} on a JVM, with bytecode rewriting.

The design is a hybrid design combining techniques of run-time
instantiation from
NextGen~\cite{nextgen,allen03} and type-passing from
PolyJ~\cite{java-popl97}.  Generic classes are translated
into ``template'' classes that are instantiated on demand at run time by
binding the type properties to concrete types.
%
Constraints on values are erased from type references.
Adapter objects are used to represent type
properties and constraints.  
Run-time type tests (e.g., casts) are translated
into code that checks those constraints at run time.
%
This design has been implemented in the \Xten{} compiler, built
on the Polyglot compiler framework~\cite{ncm03}.  The compiler
translates \Xten{} source to Java source, which is then compiled
to Java bytecode using an off-the-shelf Java compiler.\footnote{There is also
a translation from \Xten{} to C++ source, not described here.}
The \Xten{} runtime is augmented with a class loader
implementation that performs run-time instantiation.

\paragraph{Classes.}
Each class is translated into a \emph{template class}.
The template class is compiled by a Java compiler (e.g., javac)
to produce a class file.
At run time, when a constrained type \xcd"C{c}" is first referenced, a
class loader loads the template class for \xcd"C" and then
transforms its bytecode, specializing it to the constraint
\xcd"c".  The implementation specializes code based on type constraints,
not value constraints; we leave value-constraint specialization to
future work.
%
For example, consider the following classes.
{
\begin{xten}
class A[X] {
  var a: X;
}
\end{xten}}
{
\begin{xten}
class C {
  val x: A[int] = new A[int]();
  val y: int = x.a;
}
\end{xten}}

The compiler generates the following Java code:
{
\begin{xten}
@Parameters({"X"})
class A {
  @TypeProperty public static class X { }
  public x10.runtime.Type X;
  X a;
  @Synthetic public A(Class X) { this(); }
}
\end{xten}}
{
\begin{xten}
class C {
  final A x = new A(int.class);
  final int y = Runtime.to$int(x.a);
}
\end{xten}}

The member class \xcd"A.X" is used in place of the
type property \xcd"X".   The field \xcd"X" of type
\xcd"x10.runtime.Type" captures the actual constrained type on which \xcd"A"
is instantiated, and is used for run-time type tests.
The \xcd"@Parameters" annotation on \xcd"A" is used during
run-time instantiation to identify the type properties.
Synthetic constructors with added \xcd"Class" parameters are
used to pass instantiation arguments to the \xcd"new"
expression.
This code is compiled to Java bytecode.

When an expression (e.g., \xcd"new C()") is evaluated,
the class \xcd"C" is loaded.
The class loader transforms the bytecode as if it had
been written as follows:

{
\begin{xten}
class C {
  final A$$int x = new A$$int();
  final int y = x.a;
}
\end{xten}}

The class loader rewrites allocations of template classes
(e.g., \xcd"new A(int.class)") into allocations of the
instantiated classes (i.e., \xcd"new A$$int()").
The template class name and actual type arguments are mangled to
derive the name of the instantiated class.
This code cannot be generated directly because
class \xcd"A$$int" does not yet exist; the Java source compiler
would fail to compile \xcd"C".

Upon evaluation of the constructor,
the class \xcd"A$$int" is loaded.
The class loader intercepts
this, demangles the name, and loads the bytecode for the
template class \xcd"A".
The bytecode is transformed, replacing the type property \xcd"X"
with the concrete type \xcd"int".

Parameter types are coerced to and from the actual type
\xcd"T" (a Java primitive type or \xcd"Object") using
method \xcd"Runtime.to$T(Object)" and \xcd"Runtime.from(T)",
possibly with additional casts.
Both are eliminated from the transformed
bytecode, but are needed for the template class to type-check.

\eat{
Currently, the class loader instantiates the template for
every encountered combination of parameters.  If desired,
it is possible (and relatively easy) to optimize this scheme
to instantiate only for the Java primitive types and Object,
giving nine possible instantiations per parameter.
}

%Instantiations are used for representation.
%Adapter objects are used for run time type information.
%
%Could do instantiation eagerly, but quickly gets out of hand without
%whole-program analysis to limit the number of instantiations: 9
%instantiations for one type property, 81 for two type
%properties, 729 for three.

%Constructors are translated to static methods of their
%enclosing
%class.
%Constructor calls
%are translated to calls to static methods.

\eat{
Consider the code in Figure~\ref{fig:translation1}.  It contains most of the
features of generics that have to be translated.
\begin{figure*}[tp]
{\footnotesize
\begin{xtenmath}
class C[T] {
    var x: T;
    def this[T](x: T) { this.x = x; }
    def set(x: T) { this.x = x; }
    def get(): T { return this.x; }
    def map[S](f: F[T,S]): S { return f._(this.x); }
    def d() { return new D[T](); }
    def t() { return new T(); } // FIXME
    def isa(y: Object): boolean { return y instanceof C[T]; }
}
abstract class F[T,S] { S _(T x); }

val x : C = new C[String]();
val y : C[Int] = new C[Int]();
val z : C{T $\extends$ Array} = new C[Array[Int]]();
val f : F[String,Int] = ...;
x.map[Int](f);
new C[Int{self==3}]() instanceof C[Int{self<4}];
\end{xten}}
\caption{Code to translate}
\label{fig:translation1}
\end{figure*}

The translated version is shown in Figure~\ref{fig:translation2}.
\begin{figure*}[tp]
{\footnotesize
\begin{xten}
@Parameters({"T"})
class C {
    @TypeProperty public static class T { }
    T x;
    C(T x) { this.x = x; }
    @Synthetic C(Class T, T x) { this(x); }
    @Synthetic public static boolean instanceof\$(Object o, String constraint) { assert(false); return true; }
    public static boolean instanceof\$(Object o, String constraint, boolean b) { /*check constraint*/; return b; }
    public static Object cast\$(Object o, String constraint) { /*check constraint*/; return (C)o; }
    void set(T x) { this.x = x; }
    T get() { return this.x; }
    @Synthetic
    @Parameters("S")
    public static class map {
        public static class S { };
        public C c;
        public map(C c) { this.c = c; }
        @Synthetic
        public map(Class S, C c) { this(c); }
        public S apply(@InstantiateClass({"C\$T", "C\$map\$S"}) F f) { return f._(c.x); }
        @Synthetic
        public T apply(Class T, T x, T y) { return apply(x, y); } // We might only need one
    }
    @Synthetic
    @ParametricMethod("T")
    Object make\$map(Class T) { assert(false); return null; }
    @Synthetic
    Object make\$map(Class T, boolean ignored) {
        Object retval = null;
        try {
            X10RuntimeClassloader cl = (X10RuntimeClassloader)C.class.getClassLoader();
            Class<?> c = cl.instantiate(map.class, T); 
            retval = c.getDeclaredConstructor(new Class[] { C.class }).newInstance(this);
        }
        catch (IllegalAccessException e) { }
        catch (NoSuchMethodException e) { }
        catch (InstantiationException e) { }
        catch (InvocationTargetException e) { }
        return retval;
    }
    @InstantiateClass({"C\$T"}) D d() { return new D(T.class); }
    T t() { return new T(); } // FIXME
    boolean isa(Object y) { return Runtime.instanceof\$(C.instanceof\$(y, null), T.class); }
}
@Parameters({"T","S"})
abstract class F { ... }

C x = new C(String.class);
C y = new C(int.class);
C z = new C(((X10RuntimeClassloader)C.class.getClassLoader()).getClass("Array\$\$int"));
F f = ...;
((C.map)(Object)(C.map)x.make\$map(int.class)).apply(int.class, f);

Runtime.instanceof\$(C.instanceof\$(new C(int.class)(), "self<4"), int.class);
\end{xten}}
\caption{Translated code}
\label{fig:translation2}
\end{figure*}
}

\paragraph{Passing type arguments.}

For types visible at run time, annotations are used to
pass actual type arguments to the class loader.
The annotation \xcd"@InstantiateClass"
specifies the type parameter and
is placed on
fields, methods,
method parameters, and classes to
indicate instantiation parameters for field
types, method return types, method parameters, and superclasses,
respectively.
Interface instantiations are similarly handled
by \xcd"@InstantiateInterfaces".
The annotation
\xcd"@Instantiation"
is used for parametrized exceptions.
The class loader uses the arguments of the annotations to
propagate the instantiation information of the enclosing class
to the instantiation of annotated entities.  It then turns these
entities into references to the appropriate dynamically
instantiated classes.

Type arguments are passed to allocation expressions as
synthetic constructor arguments.  Run-time type tests and casts
receive type parameters via the \xcd"Runtime.cast$" and
\xcd"Runtime.instanceof$" helper methods.

\paragraph{Eliminating method type parameters.}

A parametrized
adapter class with an \xcd"apply" method
is generated for each parametrized method,
The adapter class
is annotated with \xcd"@ParametricMethod".
The parametrized method is invoked by instantiating the adapter
class through a generated factory method
and invoking its \xcd"apply()" method.

\paragraph{Parametrized exceptions.}

Parametrized exceptions are treated just like other classes.
Synthetic local classes, annotated with \xcd"@Instantiation",
are generated for each catch block with an instantiated
generic exception class.  Exception tables in the
bytecode are rewritten with the new exception types.

\paragraph{Run-time instantiation.}

The
\xcd"instanceof" and cast operations on
constrained types or type variables
are translated
to
similar operations on the instantiated type followed by calls
to
methods of the adapter object for the type
that evaluate the constraint.
% run-time constraint solving or other
% complex code that cannot be easily substituted in when rewriting
% the bytecode during instantiation.


\section{Discussion}
\label{sec:discussion}
\subsection{Type-valued properties}
\label{sec:type-properties}
\label{sec:variance}

We explored a 
different approach to adding generics to \Xten based on a
generlization of properties.
A \emph{type property}
is a final object member initialized at construction-time with a
concrete type.  Type properties and other type-valued variables
are declared just as normal variables but with type \xcd"*".
Like normal value properties, type properties
can be used in constrained types through the variable \xcd"self".

For example, the \xcd"Vector" class of Figure~\ref{fig:vector}
could be written as shown in Figure~\ref{fig:vprop}.
The class has a type-valued property \xcd"T" and a value property \xcd"len".
\begin{figure}
{
\begin{xtennoindent}
class List(T: *, len: int) {
  val data: Rail(T);

  def this(S: *, n: int, init: (int) => T):
      Vector{self.T==S,self.len==n} {
    super();
    property(S, n);
    ...
  }

  def map(S: *, f: T => S):
      Vector{self.T==S,self.len==this.len} { ... }

  def get(i: int{0 <= self, self < len}): this.T { ... }

  ...
}
\end{xtennoindent}
}
\caption{Vector example, with type properties}
\label{fig:vprop}
\end{figure}

The constructor of \xcd"Vector" takes an additional type-valued
argument \xcd"S", which is used to initialize the type property
\xcd"T".  The \xcd"map" method takes a type argument \xcd"S" among its
formal parameters and returns a \xcd"Vector" whose \xcd"T"
property is constrained to be equal to \xcd"S".

Subtyping constraints on type properties may be used to
provide \emph{use-site variance} constraints.
Use-site variance based on structural virtual types was proposed by
Thorup and Torgerson~\cite{unifying-genericity} and extended for
parametrized type systems by Igarashi and
Viroli~\cite{variant-parametric-types}.  The latter type system lead
to the development of wildcards in
Java~\cite{Java3,adding-wildcards,wildcards-safe}.  Constrained
type properties
have similar expressive power.

Consider the following subtypes of \xcd"Vector":
\begin{itemize}
\item \xcd"Vector".  This type has no constraints on the type
property \xcd"T".
Any type that constrains \xcd"T"
is a subtype of \xcd"Vector".  The type \xcd"Vector" is equivalent to
\xcd"Vector{true}".
%
For a \xcd"Vector" \xcd"v", the return type of the \xcd"get" method
is \xcd"v.T".
Since the property \xcd"T" is unconstrained,
the caller can only assign the return value of \xcd"get"
to a variable of type \xcd"v.T" or of type \xcd"Object".

\item \xcd"Vector{T==float}".
The type property \xcd"T" is bound to \xcd"float".
For a final expression \xcd"v" of this type,
\xcd"v.T" and \xcd"float" are equivalent types and can be used
interchangeably.
The syntax \xcd"Vector[float]" is used as
shorthand for \xcd"Vector{T==float}".

\item \xcdmath"Vector{T$\extends$Collection}".
This type constrains \xcd"T" to be a subtype of \xcd"Collection".
All instances of this type must bind \xcd"T" to a subtype of
\xcd"Collection"; for example \xcd"Vector[Set]" (i.e.,
\xcd"Vector{T==Set}") is a subtype of
\xcdmath"Vector{T$\extends$Collection}" because \xcd"T==Set" entails
\xcdmath"T"
\xcdmath"$\extends$"
\xcdmath"Collection".
%
If \xcd"v" has the type \xcdmath"Vector{T$\extends$Collection}",
then the return type of \xcd"get" has type \xcd"v.T", which is an unknown but
fixed subtype of \xcd"Collection"; the return value can be
assigned into a variable of type \xcd"Collection".

\item \xcdmath"Vector{T$\super$String}".  This type bounds the type property
\xcd"T"
from below.  For a \xcd"Vector" \xcd"v" of this type, any
supertype of \xcd"String" may flow into a variable of type \xcd"v.T".
The return type of the \xcd"get"
method is known to be a
supertype of \xcd"String" (and implicitly a subtype of \xcd"Object").
\end{itemize}

\label{sec:usability}
\label{sec:parameters-vs-fields}

The key difference between type parameters and type properties
is that type properties are
instance \emph{members} bound during object construction.  Type
properties are thus accessible through expressions---\xcd"e.T" is
a legal type (if \xcd"e" is final)---and are inherited by subclasses.
These features gives type properties more expressive power than
type parameters, but because they 
provide similar functionality with often subtle distinctions
type properties
can be difficult to use, especially for novices.
For this reason, \Xten uses type parameter rather than type properties.

Since type properties are inherited, the language design needs
to account for ambiguities introduced when the same name is
used for different type properties declared in or inherited into a class.
These can be disambiguated
by ``casting'' the target up to the desired supertype,
e.g., \xcd"(e as C).X" specifies
the property \xcd"X" inherited from \xcd"C".

As an example, in the following \Xten code extended with type properties, the
\xcd"HashMap"  class inherits the properties \xcd"K" and \xcd"V" from the
\xcd"Map" interface.
\begin{xten}
interface Map(K:*, V:*) {
  def get(K): V;
  def put(K, V): V;
}

class HashMap implements Map {
  def get(k: K): V = ...;
  def put(k: K, v: V): V = ...;
}
\end{xten}
A user more used to type parameters might declare \xcd"HashMap" as follows:
\begin{xten}
class HashMap(K:*,V:*) implements Map(K,V) {
  def get(k: K): V = ...;
  def put(k: K, v: V): V = ...;
}
\end{xten}
This declaration would introduce a new pair of type properties
named \xcd"K" and
\xcd"V" that shadow the inherited properties.
A na{\"\i}ve implementation of type properties would store run-time
type information for all four properties in each instance
of \xcd"HashMap".


\subsection{Virtual types}

\emph{Virtual types}~\cite{beta,mp89-virtual-classes,ernst06-virtual}
are a language-based extensibility
mechanism 
originally introduced in the language
BETA~\cite{beta} that---along with
similar constructs built on path-dependent types found in
languages such as Scala~\cite{scala}, J\&~\cite{nqm06},
and Tribe~\cite{cdnw07-tribe}---share many similarities with type properties.
A virtual type is a type binding nested within an enclosing instance.
Subclasses are permitted to override the binding of inherited virtual types. 

Virtual types
may be used to provide genericity; indeed,
one of the first proposals for genericity in Java was based on
virtual types~\cite{thorup97}, and
Java
wildcards (i.e., parameters with use-site variance)
were developed from a line of work beginning with virtual
types~\cite{unifying-genericity,variant-parametric-types,adding-wildcards}.

%Igarashi and Pierce~\cite{ip99-virtual-types}
%model the semantics of virtual types
%and several variants
%in a typed lambda-calculus with subtyping and dependent types.

Constrained types are more expressive than virtual
types in that they can be constrained at the use-site,
can be refined on a per-object basis without explicit subclassing,
and can be refined contravariantly as well as covariantly.

Thorup~\cite{thorup97}
proposed adding genericity to Java using virtual types.  For example,
a generic \xcd"List" class can be written as follows:
{
\begin{xten}
abstract class List {
  abstract typedef T;
  T get(int i) { ... }
}
\end{xten}}
\noindent
The virtual type \xcd"T" is unbound in \xcd"List", but 
can be refined by binding \xcd"T" in a subclass:
{
\begin{xten}
abstract class NumberList extends List {
  abstract typedef T as Number;
}
class IntList extends NumberList {
  final typedef T as Integer;
}
\end{xten}}
\noindent
Only classes where \xcd"T" is final bound, such as \xcd"IntList",
can be non-abstract.
%
The analogous definition of 
\xcd"List" in \Xten{} using type properties is as follows:
{
\begin{xten}
class List[T] {
  def get(i: int): T { ... }
}
\end{xten}}

\noindent
Unlike the virtual-type version,
the \Xten{} version of \xcd"List" is not abstract;
\xcd"T" need not be instantiated by a subclass because it can be
instantiated on a per-object basis.
Rather than declaring subclasses of \xcd"List",
one uses the constrained subtypes
\xcdmath"List{T$\extends$Number}" and \xcd"List{T==Integer}".

Type properties can also be refined contravariantly.
For instance, one can write the type \xcdmath"List{T$\super$Integer}".

Dependent classes~\cite{dependent-classes} generalize virtual
classes to express similar semantics via parametrization rather
than nesting.  Virtual classes depend only other their enclosing
instance; dependent classes, in contrast, depend on any number
of objects in which they are parametrized.  With type
properties, classes are not parametrized on their values;
rather properties are members and types are constructed by
constraining these properties.  Parametrization can be 
encoded with type properties using equality constraints.

\subsection{Wildcards}
\label{sec:wildcards}

Constraints on type properties can also provide a characterization of
wildcards in Java~\cite{Java3,adding-wildcards,wildcards-safe}.
We leave to future work a formal proof of this
characterization.
Wildcards can be  motivated
by the following example from Torgersen et al.~\cite{adding-wildcards}:
Consider a \xcd"Set" class and a variable \xcd"EMPTY" containing
the empty set.  What should be the type of \xcd"EMPTY"?
In Java, one can use a wildcard and 
assign the type \xcd"Set<?>", i.e., the type of all \xcd"Set"
instantiated on \emph{some} parameter.  Clients of this
type do not know what parameter to which the actual instance of \xcd"Set"
is bound.
With type properties,
a similar effect is achieved simply by leaving the
element type property of \xcd"Set" unconstrained.

Wildcards can
also be bounded above and below with
``\Xcd{?} \Xcd{extends} \Xcd{T}'' and ``\Xcd{?} \Xcd{super} \Xcd{T}'',
respectively.
%
Again, with type properties, 
a similar effect is achieved by constraining
the
element type property \xcd"X" of \xcd"Set" with subtyping constraints,
e.g., \xcd"Set{X<=T}" and \xcd"Set{X>=T}".

\eat{
Finally,
one can also specify lower bounds on types.  These are useful for
comparators:
{\footnotesize
\begin{xtenmath}
class TreeSet[T] {
  def this[T](cmp: Comparator{T $\super$ this.T}) { ... }
}
\end{xtenmath}}
Here, the comparator for any supertype of \xcd"T" can be used as
to compare \xcd"TreeSet" elements.

Another use of lower bounds is for list operations.
The \xcd"map" method below takes a function that maps a supertype
of the class parameter \xcd"T" to the method type parameter \xcd"S":
{\footnotesize
\begin{xtenmath}
class List[T] {
  def map[S](fun: Object{self $\super$ T} => S) : List[S] { ... }
}
\end{xtenmath}}
}

Like wildcards,
constrained types support
\emph{proper abstraction}~\cite{adding-wildcards}.  To illustrate, a
\xcd"reverse"
operation can operate on \xcd"List" of any type:
{
\begin{xten}
def reverse(list: List) {
  for (i: int in 0..list.length-1)
    swap(list, i, list.length-1-i);
}
\end{xten}}

\noindent
The client of \xcd"reverse" need not provide the concrete type
on which the list is instantiated; the \xcd"list" itself
provides the element type---it is stored in the \xcd"list"
to implement run-time type introspection.

In Java, this method would be written with a type parameter on
the method; the type system permits it to be called with any
\xcd"List".
However,
the method parameter cannot be omitted: declaring a parameterless version
of \xcd"reverse" requires delegating to a private parametrized version that
``opens up'' the parameter.

%\subsection{Self types}

Type properties can also be used to support a form of self
types~\cite{bruce-binary,bsg95}.
%
Self types can be implemented by introducing a
type property \Xcd{class} to the root of the class hierarchy, \Xcd{Object}:
{\footnotesize
\begin{xtenmath}
class Object[class] { $\dots$ }
\end{xtenmath}}

\noindent
Scala's path-dependent types~\cite{scala} and J\&'s
dependent classes~\cite{nqm06}
take a similar approach.

Self types are achieved by
implicitly constraining types so that if a path expression \Xcd{p}
has type \Xcd{C}, then
$\Xcd{p}.\Xcd{class} \subtype \Xcd{C}$.  In particular,
$\Xcd{this}.\Xcd{class}$ is guaranteed to be a subtype
of the lexically enclosing class; the type
$\Xcd{this}.\Xcd{class}$ represents all instances of the fixed,
but statically unknown, run-time class referred to by the \Xcd{this}
parameter.

Self types address the binary method problem~\cite{bruce-binary}.
In the following
example, the class \xcd"BitSet" can be written with a
\xcd"union" method that takes a self type as argument.

{\footnotesize
\begin{xtenmath}
interface Set {
  def union(s: this.class): void;
}

class BitSet implements Set {
  int bits;
  def union(s: this.class): void {
    this.bits |= s.bits;
  }
}
\end{xtenmath}}

\noindent
Since \xcd"s" has type this \Xcd{this}.\Xcd{class}, and the class
invariant of \xcd"BitSet" implies
$\Xcd{this}.\Xcd{class} \subtype \Xcd{BitSet}$,
the implementation of the method is free to access the
\xcd"bits" field of \xcd"s".

Callers of \xcd"BitSet".\xcd"union()" must call the method with
an argument that has the same run-time class as the
receiver.  For a receiver \xcd"p", the
type of the actual argument of the call must have a constraint
that entails \xcd"self".\xcd"class==p".\xcd"class".



\subsection{Self types}

Type properties can also be used to support a form of self
types~\cite{bruce-binary,bsg95}.
%
Self types can be implemented by introducing a
type property \Xcd{type} to the root of the class hierarchy, \Xcd{Object}:
\begin{xtenmath}
class Object(type:*){type <= Object} { $\dots$ }
\end{xtenmath}

\noindent
For any final path expression \Xcd{p}, the type
$\Xcd{p}.\Xcd{type}$ represents all instances of the fixed,
but statically unknown, run-time class referred to by \Xcd{p}.
Scala's path-dependent types~\cite{scala} and J\&'s
dependent classes~\cite{nqm06}
take a similar approach.

Self types are achieved by
constraining types so that if a path expression \Xcd{p}
has type \Xcd{C}, then
$\Xcd{p}.\Xcd{type} \subtype \Xcd{C}$.
In particular, one can add the class invariant
$\Xcd{this}.\Xcd{type} \subtype \Xcd{C}$ to every class \Xcd{C}.
This invariant ensures that
$\Xcd{this}.\Xcd{type}$ is a subtype
of the lexically enclosing class.

\eat{
\subsection{Ownership types}

Consider the following example of generic ownership
derived from Potanin et al.~\cite{ogj-oopsla06}.

\begin{xtenmath}
class Object(owner: Object) { }

// Map inherits Object.owner
// No need to add explicit vOwner and kOwner properties for Key, Value
class Map[Key, Value]{Key $\extends$ Comparable, Value $\extends$ Object}
{
    private nodes: Vector[Node[Key, Value](this)](this);

    public def put(key: Key, value: Value): Void {
        nodes.add(new Node[Key, Value](key, value, this)());
    }

    public def get(key: Key): Value {
        for (mn: Node[Key, Value](this) in nodes) {
            if (mn.key.equals(key))
                return mn.value;
        }
        return null;
    }

    // OGJ will prevent this from being called, since caller
    // can only assign the result to a supertype of Vector(this),
    // which would be only Vector(this) or Object(this)
    // BUT: we have Vector $\super$ Vector(this)
    // Need to require that all class types have an equality constraint
    // on the owner property
    public def exposeVector(): Vector(this) { return nodes; }
}

class Node[Key, Value]
    {Key $\extends$ Comparable, Value $\extends$ Object}
{
    val key: Key;
    val value: Value;

    public def this[K, V](k: Key, v: Value, o: Object): Node[K, V](o) {
        super(o);               // set the owner
        property[K, V];         // set the type properties
        this.key = k;
        this.value = v;
    }
}
\end{xten}

Restrictions:
\begin{itemize}
\item owner property must be constrained (define this!)
\item owner is always equal to or inside the owner of all other type properties
\item types with an actual owner == this, can only be accessed via this
\end{itemize}

}


\subsection{Run-time casts}
\label{sec:casts}

While constraints are normally solved at compile time, 
constraints can be evaluated at run time by using casts.
The expression 
\xcd"xs as List{length==n}" checks not only 
that \xcd"xs"
is an instance of
the \xcd"List" class, but also that \xcd"xs.length" equals \xcd"n".
A \xcd"ClassCastException" is thrown if the check fails.
%
In this example, the test of the constraint does not require
run-time constraint
solving; the constraint can be checked by simply
evaluating the \xcd"length" property of \xcd"xs" and comparing against \xcd"n".

However, the situation is more complicated when casting to a
generic type.  Unlike Java, \Xten does not erase type
parameters at run-time.  Instead each instance of a generic type
contains a description of the types that its parameters are
instantiated upon.  This extra run-time type information
enables checked casts to generic types.

The implementation becomes complicated when variant type
parameters are permitted.
While our formalism does not model
variance, \Xten does support them, as described in
Section~\ref{sec:lang}.
Consider a declaration of class \xcd"C" with a covariant type
parameter \xcd"X":
\begin{xtenmath}
class C[+X](x: X) {
   def this(y: X) { property(y); }
}
\end{xtenmath}
\noindent
Because the static type of an expression may involve one or more
type parameters,
checking if an expression is an instance of, say, \xcd"C[A{c}]"
may require a run-time constraint entailment test.  That is, if a value \xcd"v"
has run-time type \xcd"C[B{d}]", then because \xcd"C"'s
parameter is covariant,
\xcd"v" \xcd"as" \xcd"C[A{c}]" must check that \xcd"B" is a subclass of
\xcd"A" and that \xcd"d" entails \xcd"c".\footnote{If the
\xcd"X" parameter were declared invariant, the cast would only
need to check that \xcd"A" and \xcd"B" are the same class and
that \xcd"d" and \xcd"c" are equivalent, which might be
accomplished by representing constraints at run-time in a
canonical form.}

One approach is to restrict the language 
to rule out casts to type parameters 
and to generic types with subtyping constraints, ensuring that
entailment checks are not needed at run time.

Alternatively, 
the constraint solver could be embedded into the runtime system.
This is the solution used in the \Xten{} implementation; however, this
solution can result in inefficient run-time casts
if entailment checking for the given constraint system is expensive.

We are exploring alternative implementations or future work.

\eat{
A different approach to have the compiler pre-compute the results of
entailment checks.
This might be done by analyzing the program to identify which pairs of
constraints might be tested for entailment at run time and then generating a
graph were each node is a constraint and there is a directed edge between
nodes in an entailment relationship.  Run-time entailment
checking can then be implemented as reachability checking. 
This solution is a whole program analysis; all
constraints must be visible to generate the graph.
\todo{
We leave the design of this analysis for future work.
}

\todo{this is vague.  and very likely wrong.}
If \xcd"e as T" occurs in the program text and \xcd"e" has
type \xcd"U", the analysis identifies all \xcd"new" expressions
$a$,
that create a subtype of \xcd"U" and identifies all type
expressions $t$ that could instantiate \xcd"T".
For each pair $(a,t)$, the analysis checks that the
type of $a$ (determined by the constructor invoked by $a$) is a subtype of $t$.
Since $a$ and $t$ may be in different environments, \xcd"e" must
be substituted for \xcd"this" in $a$ and \xcd"self" in $t$.
}

\eat{
Another option is to test objects cast to \xcd"T" not for membership in the
type \xcd"T", but rather to test against the
\emph{interpretation} of \xcd"T".
Observe that
if an instance of a generic class \xcd"C[X]"
is a member of the type \xcd"C{X==U}", then
all fields ${\tt f}_i$ of the instance with
declared type ${\tt S}_i$ contain values
that are instances of ${\tt S}_i[{\tt U}/{\tt X}]$.
For example, given the following declaration of class \xcd"List":
{
\begin{xten}
class List[X] {
  val head: X;
  val tail: List[X];
}
\end{xten}
}
\noindent
if \xcd"xs" is an instance of \xcd"List{X==String}", then
by checking that \xcd"xs.head" is an instance of \xcd"String"
and \xcd"xs.tail" is, recursively, an instance of \xcd"List{X==String}".
This property can be exploited by implementing cast to check the
types of all fields of the object.
For this check to be sound, it is vital that all fields whose
type depends on the type property \xcd"X" be transitively final;
otherwise, the test is not invariant---the
result of the test could change as the data is mutated.
Care must also be taken to implement the
test so that it terminates for cyclic data structures.
This implementation is inefficient for large data structures.

This solution has a more permissive semantics
than those implemented in \Xten or \FXGL{Q}.
The difference is best illustrated by considering an empty generic class:
{
\begin{xten}
class Nil[X] { }
\end{xten}
}
\noindent
In this case, there is no field of type \xcd"X" to test;
therefore, an object instantiated as \xcd"Nil[int]" 
can be considered a member of \xcd"Nil[String]". 
However, the solution remains sound:
Given a class \xcd"C[X]" and an expression \xcd"e" of type \xcd"t.X", if
a run-time check finds that \xcd"t" has type \xcd"C[T]", 
the compiler \emph{cannot} use this information to derive 
that \xcd"e" has type \xcd"T".
We leave to future work a proof of this claim.
}



\section{Related work}
\label{sec:related}
Constraint-based type systems enjoy a long history.
The use of constraints for type inference and subtyping were
developed by
Mitchell~\cite{mitchell84}
and
Reynolds~\cite{reynolds85}.
%
Trifonov and Smith~\cite{trifonov96}
proposed a type system where types are refined by subtyping
constraints.  Dependent types are not supported.
%
Pottier~\cite{pottier96simplifying,pottier01b}
presents a constraint-based type system for an ML-like language with
subtyping.

HM(X)~\cite{sulzmann97type,pottier01a,pottier-remy-attapl}
is a constraint-based framework
for Hindley--Milner style type systems.
The framework is parameterized on the specific constraint system
X; instantiating X yields extensions of the HM type system.
Constraints in HM(X) are over types, not values.
%
% XXX HM(X) introduced {\em term constraint systems}; constraints in
% CFJ are term constraints?

% Cardelli~\cite{cardelli86}, type checking dependent types and
% subtypes.

% Russell
% \cite{fuh88}
% \cite{curtis90}
% \cite{aiken93}

% Aiken, Wimmers, and Lakshman proposed {\em conditional
% types}~\cite{conditional-types}, which have the ability to
% encode control-flow analysis of {\tt case} expressions.
% Conditional types are not dependent.

% \cite{smith94}



% \cite{palsberg95}
% constraint-based inference algorithm for object calculus, 

% Henglein (TAPOS) set constraints for OO language type-inference.

% Bane~\cite{fahndrich99}

% Pottier

% CLP(X) framework in constraint logic programming (JM94)
% HM(X)

Several systems have been proposed that refine types in a base
type system through constraints.
%
{\em Refinement types}~\cite{refinement-types} extend the 
Hindley--Milner type system with intersection, union, and
constructor types, enabling specification and inference of
more precise type information.
%
{\em Conditional
types}~\cite{conditional-types} extend refinement types to
encode control-flow information in the types.
%
Jones introduced {\em qualified types}, which permit
types to be constrained by a finite set of
predicates~\cite{jones94}.
%
{\em Sized types}~\cite{sized-types}
annotate types with the sizes of recursive data structures.
Sizes are linear functions of size variables.
Size inference is performed using a constraint solver for
Presburger arithmetic~\cite{omega}.
% constraints on types, support primitive recursion only

% Indexed types~\cite{indexed-types}

% Index objects must be pure.
% Singleton types int(n).
% ML$^{\Pi}_0$:
% Refinement of the ML type system: does not affect the
% operational semantics.  Can erase to ML$_0$.

% Jay and Sekanina 1996: array bounds checking based on shape
% types.

With hybrid type-checking~\cite{flanagan-popl06,flanagan-fool06},
types can be constrained by arbitrary boolean expressions.
While typing is undecidable, dynamic checks are inserted into
the program when necessary if the type-checker cannot determine
type safety statically.
In \Xten{} dynamic type checks, including tests of dependent
clauses, are inserted only at explicit casts.

% Ada dependent types.
% Ada has constrained array definitions.  A constraint
% \cite{ada-ref-man}.  Not clear if they're dependent.  Are
% there other dependent types?  Generics are dependent?

Singleton types~\cite{aspinall-singletons,stone00} are dependent
types containing only one value.  
In Stone's formulation~\cite{stone00},
$S(e : \tau)$
is the type of all values of type $\tau$ that are equal to $e$.
Term equivalence is
constructed so that type-checking is decidable.
The singleton $S(e: \tau)$ can be encoded in \Xten{} as
$\tau$\xcd{(:self ==}~$e$\xcd{)}.

        % Used for array bounds by Morrisett et al (I think--need
        % to find paper)

% Singleton types~\cite{aspinall-singletons}.

Several languages---gbeta~\cite{ernst99-gbeta},
Scala~\cite{scala-overview,scala-oopsla05}, J\&~\cite{nqm06}, and
others~\cite{oz01,ocrz-ecoop03,dependent-classes}---provide {\em path-dependent
types}.  For a final access path \xcd{p}, {\tt p.type}
in Scala is the singleton type containing the object \xcd{p}.
In J\&, {\tt p.class} is a type containing all objects
whose run-time class is the same as \xcd{p}'s.
Scala's {\tt p.type} can be encoded in \Xten{} using an equality
constraint \xcd{C(:self == p)}, where \xcd{C} is a supertype of
\xcd{p}'s static type.
\eat{
These types can be encoded in CFJ by introducing a
\xcd{type} property.
\rn{T-constr}, as
described in Section~\ref{sec:examples}.
}

% Where clauses for F-bounded polymorphism~\cite{where-clauses}
% Bounded quantification: Cardelli and Wegner.  Bound T with T'
% In F-bounded polymorphism~\cite{f-bounds}, type variables are bounded by a function of 
% the type variable. 
% Not dependent types.

\eat{
conditional types:

type of an expression can be constrained using information about
the results of run-time tests in the context surrounding the
expression.

e.g., can express that e2 is evaluated only if e1 is true

\begin{verbatim}
\y. case y of true => e1 | false => e2 :
        'a -> (true ? ('a ^ typeof(e1)) U (false ? ('a ^ typeof(e2))
\end{verbatim}

Types include type constructors applied to types.

\begin{verbatim}
        so,  true      : true
        but, (\x . x)  : 'a -> 'a
             node(l,r) : node('a tree, 'a tree)
\end{verbatim}


when checking a case branch, type of the expression being
matched refined to the include the type constructor for that
branch

captures some control flow analysis in the types

types
\begin{verbatim}
        t ::= t1 -> t2
                | c(..ti..) <-- type constructor
                | alpha
                | t1 U t2
                | t1 ^ t2
                | t1 ? t2
                | 0
                | 1

        sigma ::= t | \forall ..alpha.. t where ..ti <= tj..
\end{verbatim}
}


Cayenne~\cite{cayenne} is a Haskell-like language with fully dependent types.
There is no distinction between static and dynamic types.
Type-checking is undecidable.
There is no notion of datatype refinement as in DML.

Epigram~\cite{epigram,epigram-matter}
is a dependently typed functional programming language based on
a type theory with inductive families.
Epigram does not have a phase distinction between values and
types.

\eat{
$\lambda^{\sf Con}$ is a lambda calculus with assertions.
Findler, Felleisen, Contracts for higher-order functions (ICFP02)

  example: int[> 9]

contracts are either simple predicates or function contracts.
defined by (define/contract ...)

enforced at run-time.
}

% Jif~\cite{jif,jflow} is an extension of Java in which
% types are labeled with security policies enforced by the
% compiler.

ESC/Java~\cite{esc-java}
allow programmers to write object invariants and pre- and
post-conditions that are enforced statically
by the compiler using an automated theorem prover.
Static checking is undecidable and, in the presence of loops,
is unsound (but still useful) unless the programmer supplies loop invariants.
ESC/Java can enforce invariants on mutable state.

% and Spec$\sharp$~\cite{specsharp}

Pluggable and optional type systems were proposed by
Bracha~\cite{bracha04-pluggable} and provide another means of
specifying refinement types.
Type annotations, implemented in compiler plugins, serve only to
reject programs statically that might otherwise have dynamic
type errors.
CQual~\cite{foster-popl02} extends C with user-defined type
qualifiers.  These
qualifiers may be flow-sensitive and may be inferred. 
CQual supports only a fixed set of typing rules
for all qualifiers.
In contrast, the {\em semantic type qualifiers} of
Chin, Markstrum, and Millstein~\cite{chin05-qualifiers}
allow programmers to define typing rules for qualifiers
in a meta language that allows type-checking rules to be
specified declaratively.
JavaCOP~\cite{javacop-oopsla06} is a pluggable type system
framework for Java.  Annotations are defined in a meta language
that allows type-checking rules to be specified declaratively.
JSR 308~\cite{jsr308} is a proposal for adding user-defined type qualifiers
to Java.

% Holt, Cordy, the Turing programming language

% Ou, Tan, Mandelbaum, Walker, Dynamic typing with dependent types
% Separate dependent and simple parts of the program.
% Statically type the dependent parts.
% Dynamic checks when passing values into dependent part.

Our work is most closely related to \DML{}, \cite{xi99dependent}, an
extension of ML with dependent types. \DML{} is also built
parametrically on a constraint solver. Types are refinement types;
they do not affect the operational semantics and erasing the
constraints yields a legal ML program.

At a conceptual level the intuitions behind the development of \DML{}
and constrained types are similar. Both are intended for practical
programming by mainstream programmers, both introduce a strict
separation between compile-time and run-time processing, are
parametric on a constraint solver, and deal with mutually recursive
data-structures, mutable state, and higher-order functions (encoded as
objects in the case of constrained types). Both support existential
types.

The most obvious distinction between the two lies in the target
domain: \DML{} is designed for functional programming, specifically
ML, whereas constrained types are designed for imperative, concurrent
OO languages. Hence technically our development of constrained types
takes the route of an extension to \FJ. But there are several other
crucial differences as well.

\lstnewenvironment{displayml}
  {\lstset{language=ML,basicstyle=\tt,tabsize=4,columns=fullflexible,captionpos=b,xleftmargin=1em,xrightmargin=1em,keywordstyle=,keepspaces}}
  {}

First, \DML{} achieves its separation by not permitting program
variables to be used in types. Instead, a parallel set of (universally
or existentially quantified) ``index'' variables are
introduced. For instance the typing of the \xcd{append} operation on
lists is provided by:
\begin{displayml}
fun('a)
  append(nil, ys) = ys
| append(cons(x, xs), ys)
    = cons(x, append(xs,ys))
where append <| {m:nat}{n:nat} 
    'a list(m) * 'a list(n) -> 'a list(m+n)  
\end{displayml}
\noindent in contrast with the direct embedded expression with constrained types:
\begin{xten}
class List(int(:self >= 0) n) {
  Object item;
  List(n-1) tail;
  List(n+a.n) append(final List a) { 
    return n==0
      ? a : new List(item, tail.app(a));
  }
  ...
}
\end{xten}

Second, \DML{} permits only variables of basic index sorts known to
the constraint solver (e.g., \xcd{bool}, \xcd{int}, \xcd{nat}) to
occur in types. In contrast, constrained types permit program
variables at any type to occur in constrained types. As with \DML,
only operations specified by the constraint system are permitted in
types. However, these operations always include field selection and
equality on object references.  (As we have seen permitting arbitrary
type/property graphs may lead to undecidability.) Note that \DML{}
style constraints are easily encoded in constrained types.

% A reviewer says:
% The third criticism of DML is technically correct but highly
% misleading.  Instead of casts, DML allows "if tests" or case
% analysis as dynamic tests that then yield static information
% about the type in the appropriate branch of the if or case.
% Either omit this criticism or describe how DML does the same
% thing--or if DML's system is weaker in some way, give a
% particular example to justify that.

% Third, \DML{} does not permit any runtime checking of constraints
% (dynamic casts).


\section{Conclusions}
\label{sec:conclusions}

We have presented a constraint-based framework \FX{} for type-
and value-dependent types in an object-oriented language.
%
The use of constraints on type properties allows the design to
capture many features of generics in object-oriented languages
and then to extend these features with more
expressive power.  We have proved the type system sound.

The type system \FXGD{} formalizes the semantics of the \Xten{}
programming language.  The design admits an efficient
implementation for generics and dependent types in \Xten{}.
To improve the expressiveness of \Xten{}, we plan to implement
a type inference algorithm that infers constraints over types
and values, and to support user-defined constraints.


\eat{
\section*{Acknowledgments} 

The authors thank Bob Blainey,
Doug Lea, Jens Palsberg, and Lex Spoon
for valuable feedback on versions of the language.
We thank
Andrew Myers and
Michael Clarkson for providing us with their implementation of
PolyJ, on which our implementation was based, and for many
discussions over the years about parametrized types in Java.
}

\bibliographystyle{plain}
\bibliography{master}

% \appendix
% \onecolumn

% \section{An extended example}
% {\footnotesize
\begin{verbatim}
/**
   A distributed binary tree.
   @author Satish Chandra 4/6/2006
   @author vj
 */
//                             ____P0
//                            |     |
//                            |     |
//                          _P2  __P0
//                         |  | |   |
//                         |  | |   |
//                        P3 P2 P1 P0
//                         *  *  *  *
// Right child is always on the same place as its parent;
// left child is at a different place at the top few levels of the tree,
// but at the same place as its parent at the lower levels.

class Tree(localLeft: boolean,
           left: nullable Tree(& localLeft => loc=here),
           right: nullable Tree(& loc=here),
           next: nullable Tree) extends Object {
    def postOrder:Tree = {
        val result:Tree = this;
        if (right != null) {
            val result:Tree = right.postOrder();
            right.next = this;
            if (left != null) return left.postOrder(tt);
        } else if (left != null) return left.postOrder(tt);
        this
    }
    def postOrder(rest: Tree):Tree = {
        this.next = rest;
        postOrder
    }
    def sum:int = size + (right==null => 0 : right.sum()) + (left==null => 0 : left.sum)
}
value TreeMaker {
    // Create a binary tree on span places.
    def build(count:int, span:int): nullable Tree(& localLeft==(span/2==0)) = {
        if (count == 0) return null;
        {val ll:boolean = (span/2==0);
         new Tree(ll,  eval(ll => here : place.places(here.id+span/2)){build(count/2, span/2)},
           build(count/2, span/2),count)}
    }
}
\end{verbatim}}

\subsection{Places}
{\footnotesize
\begin{verbatim}
/**

 * This class implements the notion of places in X10. The maximum
 * number of places is determined by a configuration parameter
 * (MAX_PLACES). Each place is indexed by a nat, from 0 to MAX_PLACES;
 * thus there are MAX_PLACES+1 places. This ensures that there is
 * always at least 1 place, the 0'th place.

 * We use a dependent parameter to ensure that the compiler can track
 * indices for places.
 *
 * Note that place(i), for i <= MAX_PLACES, can now be used as a non-empty type.
 * Thus it is possible to run an async at another place, without using arays---
 * just use async(place(i)) {...} for an appropriate i.

 * @author Christoph von Praun
 * @author vj
 */

package x10.lang;

import x10.util.List;
import x10.util.Set;

public value class place (nat i : i <= MAX_PLACES){

    /** The number of places in this run of the system. Set on
     * initialization, through the command line/init parameters file.
     */
    config nat MAX_PLACES;

    // Create this array at the very beginning.
    private constant place value [] myPlaces = new place[MAX_PLACES+1] fun place (int i) {
	return new place( i )(); };

    /** The last place in this program execution.
     */
    public static final place LAST_PLACE = myPlaces[MAX_PLACES];

    /** The first place in this program execution.
     */
    public static final place FIRST_PLACE = myPlaces[0];
    public static final Set<place> places = makeSet( MAX_PLACES );

    /** Returns the set of places from first place to last place.
     */
    public static Set<place> makeSet( nat lastPlace ) {
	Set<place> result = new Set<place>();
	for ( int i : 0 .. lastPlace ) {
	    result.add( myPlaces[i] );
	}
	return result;
    }

    /**  Return the current place for this activity.
     */
    public static place here() {
	return activity.currentActivity().place();
    }

    /** Returns the next place, using modular arithmetic. Thus the
     * next place for the last place is the first place.
     */
    public place(i+1 % MAX_PLACES) next()  { return next( 1 ); }

    /** Returns the previous place, using modular arithmetic. Thus the
     * previous place for the first place is the last place.
     */
    public place(i-1 % MAX_PLACES) prev()  { return next( -1 ); }

    /** Returns the k'th next place, using modular arithmetic. k may
     * be negative.
     */
    public place(i+k % MAX_PLACES) next( int k ) {
	return places[ (i + k) % MAX_PLACES];
    }

    /**  Is this the first place?
     */
    public boolean isFirst() { return i==0; }

    /** Is this the last place?
     */
    public boolean isLast() { return i==MAX_PLACES; }
}
\end{verbatim}}
\subsection{$k$-dimensional regions}
{\footnotesize
\begin{verbatim}
package x10.lang;

/** A region represents a k-dimensional space of points. A region is a
 * dependent class, with the value parameter specifying the dimension
 * of the region.
 * @author vj
 * @date 12/24/2004
 */

public final value class region( int dimension : dimension >= 0 )  {

    /** Construct a 1-dimensional region, if low <= high. Otherwise
     * through a MalformedRegionException.
     */
    extern public region (: dimension==1) (int low, int high)
        throws MalformedRegionException;

    /** Construct a region, using the list of region(1)'s passed as
     * arguments to the constructor.
     */
    extern public region( List(dimension)<region(1)> regions );

    /** Throws IndexOutOfBoundException if i > dimension. Returns the
        region(1) associated with the i'th dimension of this otherwise.
     */
    extern public region(1) dimension( int i )
        throws IndexOutOfBoundException;


    /** Returns true iff the region contains every point between two
     * points in the region.
     */
    extern public boolean isConvex();

    /** Return the low bound for a 1-dimensional region.
     */
    extern public (:dimension=1) int low();

    /** Return the high bound for a 1-dimensional region.
     */
    extern public (:dimension=1) int high();

    /** Return the next element for a 1-dimensional region, if any.
     */
    extern public (:dimension=1) int next( int current )
        throws IndexOutOfBoundException;

    extern public region(dimension) union( region(dimension) r);
    extern public region(dimension) intersection( region(dimension) r);
    extern public region(dimension) difference( region(dimension) r);
    extern public region(dimension) convexHull();

    /**
       Returns true iff this is a superset of r.
     */
    extern public boolean contains( region(dimension) r);
    /**
       Returns true iff this is disjoint from r.
     */
    extern public boolean disjoint( region(dimension) r);

    /** Returns true iff the set of points in r and this are equal.
     */
    public boolean equal( region(dimension) r) {
        return this.contains(r) && r.contains(this);
    }

    // Static methods follow.

    public static region(2) upperTriangular(int size) {
        return upperTriangular(2)( size );
    }
    public static region(2) lowerTriangular(int size) {
        return lowerTriangular(2)( size );
    }
    public static region(2) banded(int size, int width) {
        return banded(2)( size );
    }

    /** Return an \code{upperTriangular} region for a dim-dimensional
     * space of size \code{size} in each dimension.
     */
    extern public static (int dim) region(dim) upperTriangular(int size);

    /** Return a lowerTriangular region for a dim-dimensional space of
     * size \code{size} in each dimension.
     */
    extern public static (int dim) region(dim) lowerTriangular(int size);

    /** Return a banded region of width {\code width} for a
     * dim-dimensional space of size {\code size} in each dimension.
     */
    extern public static (int dim) region(dim) banded(int size, int width);


}

\end{verbatim}}

\subsection{Point}
{\footnotesize
\begin{verbatim}
package x10.lang;

public final class point( region region ) {
    parameter int dimension = region.dimension;
    // an array of the given size.
    int[dimension] val;

    /** Create a point with the given values in each dimension.
     */
    public point( int[dimension] val ) {
        this.val = val;
    }

    /** Return the value of this point on the i'th dimension.
     */
    public int valAt( int i) throws IndexOutOfBoundException {
        if (i < 1 || i > dimension) throw new IndexOutOfBoundException();
        return val[i];
    }

    /** Return the next point in the given region on this given
     * dimension, if any.
     */
    public void inc( int i )
        throws IndexOutOfBoundException, MalformedRegionException {
        int val = valAt(i);
        val[i] = region.dimension(i).next( val );
    }

    /** Return true iff the point is on the upper boundary of the i'th
     * dimension.
     */
    public boolean onUpperBoundary(int i)
        throws IndexOutOfBoundException {
        int val = valAt(i);
        return val == region.dimension(i).high();
    }

    /** Return true iff the point is on the lower boundary of the i'th
     * dimension.
     */
    public boolean onLowerBoundary(int i)
        throws IndexOutOfBoundException {
        int val = valAt(i);
        return val == region.dimension(i).low();
    }
}
\end{verbatim}}

\subsection{Distribution}
{\footnotesize
\begin{verbatim}
package x10.lang;

/** A distribution is a mapping from a given region to a set of
 * places. It takes as parameter the region over which the mapping is
 * defined. The dimensionality of the distribution is the same as the
 * dimensionality of the underlying region.

   @author vj
   @date 12/24/2004
 */

public final value class distribution( region region ) {
    /** The parameter dimension may be used in constructing types derived
     * from the class distribution. For instance,
     * distribution(dimension=k) is the type of all k-dimensional
     * distributions.
     */
    parameter int dimension = region.dimension;

    /** places is the range of the distribution. Guranteed that if a
     * place P is in this set then for some point p in region,
     * this.valueAt(p)==P.
     */
    public final Set<place> places; // consider making this a parameter?

    /** Returns the place to which the point p in region is mapped.
     */
    extern public place valueAt(point(region) p);

    /** Returns the region mapped by this distribution to the place P.
        The value returned is a subset of this.region.
     */
    extern public region(dimension) restriction( place P );

    /** Returns the distribution obtained by range-restricting this to Ps.
        The region of the distribution returned is contained in this.region.
     */
    extern public distribution(:this.region.contains(region))
        restriction( Set<place> Ps );

    /** Returns a new distribution obtained by restricting this to the
     * domain region.intersection(R), where parameter R is a region
     * with the same dimension.
     */
    extern public (region(dimension) R) distribution(region.intersection(R))
        restriction();

    /** Returns the restriction of this to the domain region.difference(R),
        where parameter R is a region with the same dimension.
     */
    extern public (region(dimension) R) distribution(region.difference(R))
        difference();

    /** Takes as parameter a distribution D defined over a region
        disjoint from this. Returns a distribution defined over a
        region which is the union of this.region and D.region.
        This distribution must assume the value of D over D.region
        and this over this.region.

        @seealso distribution.asymmetricUnion.
     */
    extern public (distribution(:region.disjoint(this.region) &&
                                dimension=this.dimension) D)
        distribution(region.union(D.region)) union();

    /** Returns a distribution defined on region.union(R): it takes on
        this.valueAt(p) for all points p in region, and D.valueAt(p) for all
        points in R.difference(region).
     */
    extern public (region(dimension) R) distribution(region.union(R))
        asymmetricUnion( distribution(R) D);

    /** Return a distribution on region.setMinus(R) which takes on the
     * same value at each point in its domain as this. R is passed as
     * a parameter; this allows the type of the return value to be
     * parametric in R.
     */
    extern public (region(dimension) R) distribution(region.setMinus(R))
        setMinus();

    /** Return true iff the given distribution D, which must be over a
     * region of the same dimension as this, is defined over a subset
     * of this.region and agrees with it at each point.
     */
    extern public (region(dimension) r)
        boolean subDistribution( distribution(r) D);

    /** Returns true iff this and d map each point in their common
     * domain to the same place.
     */
    public boolean equal( distribution( region ) d ) {
        return this.subDistribution(region)(d)
            && d.subDistribution(region)(this);
    }

    /** Returns the unique 1-dimensional distribution U over the region 1..k,
     * (where k is the cardinality of Q) which maps the point [i] to the
     * i'th element in Q in canonical place-order.
     */
    extern public static distribution(:dimension=1) unique( Set<place> Q );

    /** Returns the constant distribution which maps every point in its
        region to the given place P.
    */
    extern public static (region R) distribution(R) constant( place P );

    /** Returns the block distribution over the given region, and over
     * place.MAX_PLACES places.
     */
    public static (region R) distribution(R) block() {
        return this.block(R)(place.places);
    }

    /** Returns the block distribution over the given region and the
     * given set of places. Chunks of the region are distributed over
     * s, in canonical order.
     */
    extern public static (region R) distribution(R) block( Set<place> s);


    /** Returns the cyclic distribution over the given region, and over
     * all places.
     */
    public static (region R) distribution(R) cyclic() {
        return this.cyclic(R)(place.places);
    }

    extern public static (region R) distribution(R) cyclic( Set<place> s);

    /** Returns the block-cyclic distribution over the given region, and over
     * place.MAX_PLACES places. Exception thrown if blockSize < 1.
     */
    extern public static (region R)
        distribution(R) blockCyclic( int blockSize)
        throws MalformedRegionException;

    /** Returns a distribution which assigns a random place in the
     * given set of places to each point in the region.
     */
    extern public static (region R) distribution(R) random();

    /** Returns a distribution which assigns some arbitrary place in
     * the given set of places to each point in the region. There are
     * no guarantees on this assignment, e.g. all points may be
     * assigned to the same place.
     */
    extern public static (region R) distribution(R) arbitrary();

}
\end{verbatim}}

\subsection{Arrays}
Finally we can now define arrays. An array is built over a
distribution and a base type.

{\footnotesize
\begin{verbatim}
package x10.lang;

/** The class of all  multidimensional, distributed arrays in X10.

    <p> I dont yet know how to handle B@current base type for the
    array.

 * @author vj 12/24/2004
 */

public final value class array ( distribution dist )<B@P> {
    parameter int dimension = dist.dimension;
    parameter region(dimension) region = dist.region;

    /** Return an array initialized with the given function which
        maps each point in region to a value in B.
     */
    extern public array( Fun<point(region),B@P> init);

    /** Return the value of the array at the given point in the
     * region.
     */
    extern public B@P valueAt(point(region) p);

    /** Return the value obtained by reducing the given array with the
        function fun, which is assumed to be associative and
        commutative. unit should satisfy fun(unit,x)=x=fun(x,unit).
     */
    extern public B reduce(Fun<B@?,Fun<B@?,B@?>> fun, B@? unit);


    /** Return an array of B with the same distribution as this, by
        scanning this with the function fun, and unit unit.
     */
    extern public array(dist)<B> scan(Fun<B@?,Fun<B@?,B@?>> fun, B@? unit);

    /** Return an array of B@P defined on the intersection of the
        region underlying the array and the parameter region R.
     */
    extern public (region(dimension) R)
        array(dist.restriction(R)())<B@P>  restriction();

    /** Return an array of B@P defined on the intersection of
        the region underlying this and the parametric distribution.
     */
    public  (distribution(:dimension=this.dimension) D)
        array(dist.restriction(D.region)())<B@P> restriction();

    /** Take as parameter a distribution D of the same dimension as *
     * this, and defined over a disjoint region. Take as argument an *
     * array other over D. Return an array whose distribution is the
     * union of this and D and which takes on the value
     * this.atValue(p) for p in this.region and other.atValue(p) for p
     * in other.region.
     */
    extern public (distribution(:region.disjoint(this.region) &&
                                dimension=this.dimension) D)
        array(dist.union(D))<B@P> compose( array(D)<B@P> other);

    /** Return the array obtained by overlaying this array on top of
        other. The method takes as parameter a distribution D over the
        same dimension. It returns an array over the distribution
        dist.asymmetricUnion(D).
     */
    extern public (distribution(:dimension=this.dimension) D)
        array(dist.asymmetricUnion(D))<B@P> overlay( array(D)<B@P> other);

    extern public array<B> overlay(array<B> other);

    /** Assume given an array a over distribution dist, but with
     * basetype C@P. Assume given a function f: B@P -> C@P -> D@P.
     * Return an array with distribution dist over the type D@P
     * containing fun(this.atValue(p),a.atValue(p)) for each p in
     * dist.region.
     */
    extern public <C@P, D>
        array(dist)<D@P> lift(Fun<B@P, Fun<C@P, D@P>> fun, array(dist)<C@P> a);

    /**  Return an array of B with distribution d initialized
         with the value b at every point in d.
     */
    extern public static (distribution D) <B@P> array(D)<B@P> constant(B@? b);

}
\end{verbatim}}


\begin{example}
 The code for {\tt List} translates as given in Table~\ref{List-translation}.
\end{example}

\begin{figure*}
{\footnotesize
\begin{verbatim}
  public value class List <Node> {
    public final nat n;   // is a parameter
    nullable Node node = null;
    nullable List<Node> rest = null;  // All assignments must check n = this.n-1.

    /** Returns the empty list. Defined only when the parameter n
        has the value 0. Invocation: new List(0)<Node>().
     */
    public List ( final nat n ) {
      assume n==0;
      this.n = n;
    }

    /** Returns a list of length 1 containing the given node.
        Invocation: new List(1)<Node>( node ).
     */
    public List ( final nat n, Node node ) {
      assume n==1;                         // From the constructor precondition.
      assert 0==0 : "DependentTypeError"; // For the constructor call.
      assert n>=1 : "DependentTypeError"; // For the this call.
      this(n, node, new List<Node>(0));
    }

    public List ( final nat n, Node node, List<Node> rest ) {
      assume n>=1;                               // From the constructor precondition
      assume rest.n==n-1 : "DependentTypeError"; // From the argument type.
      this.n = n;
      this.node = node;
      assert rest.n==n-1 : "DependentTypeError"; // For the field assignment.
      this.rest = rest;
    }

    public  List<Node> append( List<Node> arg ) {
      if (n == 0) {
          final List<Node> result = arg;
          assert n+arg.n == result.n : "DependentTypeError"; // For the return value
          return result;
      } else {
          assume rest.n == n-1;
          final List<Node> argval = rest.append(arg);
          assume argval.n == rest.n+arg.n;
          assert n+arg.n-1== argval.n : "DependentTypeError"; // For the constructor call.
          final List<Node> result = new List<Node>(n+arg.n, node, argval);
          assume result.n == n+arg.n;
          assert n+arg.n == result.n : "DependentTypeError"; // For the return value
          return result;
      }
    }

\end{verbatim}}
\caption{Translation of {\tt List} (contd in Table~\ref{List-translation-2}).}\label{List-translation}
\end{figure*}
\begin{figure*}
{\footnotesize
\begin{verbatim}
    public  List<Node> rev() {
      final List<Node> arg = new List<Node>(0);
      assume arg.n = 0;                           // From the constructor call.
      final List<Node> result = rev( arg );
      assume result.n == n+arg.n;                  // From the method signature
      assert n == result.n : "DependentTypeError"; // For the return value.
      return result;
    }

    public  List(n+arg.n)<Node> rev( final List<Node> arg) {
      if (n==0) {
         assert n+arg.n == arg.n : "DependentTypeError"; // For the return value.
         return arg;
      } else {
        assert 1+arg.n-1=arg.n : "DependentTypeError"; // For the argument to the constructor
        final List<Node> arg2 = new List<Node>(1+arg.n,node, arg));
        assume arg2.n==1+arg.n;                      // From the constructor invocation
        final List<Node> restval = rest;             // Read from a mutable field of parametric type
        assume restval.n == n-1;                     // From the field read.
        final List(restval.n+arg2.n)<Node> result = restval.rev( arg2 );
        assume result.n=restval.n+arg2.n
        assert n+arg.n == result.n                   // For the return value
        return result;
    }

    /** Return a list of compile-time unknown length, obtained by filtering
        this with f. */
    public List<Node> filter(fun<Node, boolean> f) {
         if (n==0) return this;
         if (f(node)) {
           final List<Node> l = rest.filter(f);
           assert l.n+1-1==l.n : "DependentTypeError"; // For the constructor call
           return new List<Node>(l.n+1,node, l);
         } else {
           return rest.filter(f);
         }
    }

    /** Return a list of m numbers from o..m-1. */
    public static  List<nat> gen( final nat m ) {
         assert 0 <= m : "DependentTypeError";        // Precondition for method call.
         final List<nat> result = gen(0,m);
         assume result.n=m-0 : "DependentTypeError";  // From the method signature
         assert m == result.n : "DependentTypeError"; // For the return value
         return result;
    }

    /** Return a list of (m-i) elements, from i to m-1. */
    public static List<nat> gen(final nat i, final nat m) {
      assume i <= m;                                   // Method precondition.
      if (i==m) {
        assert m-i == 0 : "DependentTypeError";        // For the constructor call
        final List result = new List<nat>(m-i);
        assume result.n == 0;                          // From the constructor call.
        assert m-i == result.n : "DependentTypeError"; // For the return value.
        return result;
      } else {
        assert i+1 <= m : "DependentTypeError";        // For the method call.
        final List<nat> arg = gen(i+1,m);
        assume arg.n = m-(i+1);                        // From the method call.
        assert m-i-1 = arg.n;                          // For the constructor invocation.
        final List result = new List<nat>(m-i, i, arg);
        assume result.n = m-i;                         // From the constructor invocation.
        assert m-i == result.n : "DependentTypeError"; // For the return value
        return result;
    }
  }
\end{verbatim}}
\caption{Translation of {\tt List} (continued).}\label{List-translation-2}
\end{figure*}

\section{Type-checking dependent classes}

Each programming language---such as \Xten{}---will specify the base
underlying classes (and the operations on them) which can occur as
types in parameter lists. For instance, in the code for {\tt List}
above, the only type that appears in parameter lists is {\tt int}, and
the only operations on {\tt int} are addition, subtraction, {\tt >=},
{\tt ==}, and the only constants are {\tt 0} and {\tt 1}.  (This
language falls within Presburger arithmetic, a decidable fragment of
arithmetic.)  The compiler must come equipped with a constraint solver
(decision procedure) that can answer questions of the form: does one
constraint entail another?  Constraints are atomic formulas built up
from these operations, using variables. For instance, the compiler
must answer each one of:
{\footnotesize
\begin{verbatim}
  n >= 2 |- n-1 >= 0
  n >= 0, m >= 0 |- m+n >= 0
\end{verbatim}}

Ultimately, the only variables that will occur in constraints are
those that correspond to {\tt config} parameters and those that are
defined by implicit parameter definitions. We need to establish that
the verification of any class will generate only a finite number of
constraints, hence only a finite constraint problem for the constraint
solver.

Second, it should be possible for instances of user-defined classes
(and operations on them) to occur as type parameters. For the compiler
to check conditions involving such values, it is necessary that the
underlying constraint solver be extended.

There are two general ways in which the constraint solver may be
extended.  Both require that the programmer single out some classes
and methods on those classes as {\em pure}. (We shall think of
constants as corresponding to zero-ary methods.) Only instances of
pure classes and expressions involving pure methods on these instances
are allowed in parameter expressions.

How shall constraints be generated for such pure methods? First, the
programmer may explicitly supply with each pure method {\tt T m(T1 x1,
..., Tn xn)} a constraint on {\tt n+2} variables in the constraint
system of the underlying solver that is entailed by {\tt y =
o.m(x1,..., xn)}. Whenever the compiler has to perform reasoning on an
expression involving this method invocation, it uses the constraint
supplied by the programmer. A second more ambitious possibility is
that a symbolic evaluator of the language may be run on the body of
the method to automatically generate the corresponding constraint.

Finally an additional possibility is that the constraint solver itself
be made extensible. In this case, when a user writes a class which is
intended to be used in specifying parameters, he also supplies an
additional program which is used to extend the underlying constraint
solver used by the compiler. This program adds more primitive
constraints and knows how to perform reasoning using these
constraints. This is how I expect we will initially implement the
\Xten{} language. As language designers and implementers we will
provide constraint solvers for finite functions and {\tt Herbrand}
terms on top of arithmetic.





\end{document}
