\section{Regions}\label{XtenRegions}\index{region}

A region is a set of points all having a common rank.  {}\Xten{}
provides a built-in value class, \xcd"x10.lang.Region", to allow the
creation of new regions and to perform operations on regions.
%This
%class is {\tt final} in {}\XtenCurrVer; future versions of the
%language may permit user-definable regions.

Each region \xcd"R" has a constant rank, \xcd"R.rank", which is a
non-negative integer. The literal \xcd"[]" represents the {\em empty
region} and has rank \xcd"0".

Here are several examples of region declarations:
\begin{xten}
Null: Region = new Region();  // Empty 0-dimensional region          
R1: Region = 1..100; // 1-dim region with extent 1..100
R2: Region = [1..100]; // same as R1
R3: Region = (0..99) * (-1..MAX_HEIGHT);   
R4: Region = [0..99, -1..MAX_HEIGHT]; // same as R3  
R5: Region = Region.factory.upperTriangular(N);
R6: Region = Region.factory.banded(N, K);
R7: Region = R4 && R5; // intersection of two regions
R8: Region = R4 || R5; // union of two regions
\end{xten}

The expression \xcdmath"a$_1$..a$_2$"
is shorthand for the rectangular, rank-1 region
consisting of the points
$\{$\xcdmath"[a$_1$]", \dots, \xcdmath"[a$_2$]"$\}$.
Each subexpression of \xcdmath"a$_i$" must be of type \xcd"Int".
If \xcdmath"a$_1$"
is greater than \xcdmath"a$_2$", the region is empty.

A region may be constructed by converting from a rail of
regions or from a rail of points.  Regions are typically 
initialized using the rail constructor syntax
(\Sref{RailConstructors})
(e.g., \xcd"R4" above).
The region constructed from a rail of points represents the
region containing just those points.
The region constructed from a rail of regions
represents
the Cartesian product of each of the arguments.
%\XtenCurrVer{} does not support hierarchical regions.

\index{region!upperTriangular}
\index{region!lowerTriangular}\index{region!banded}

Various built-in regions are provided through  factory
methods on \xcd"Region".  For instance:
\begin{itemize}
\item \xcd"Region.upperTriangular(N)" returns a region corresponding
to the non-zero indices in an upper-triangular \xcd"N x N" matrix.
\item \xcd"Region.lowerTriangular(N)" returns a region corresponding
to the non-zero indices in a lower-triangular \xcd"N x N" matrix.
\item \xcd"Region.banded(N, K)" returns a region corresponding to
the non-zero indices in a banded \xcd"N x N" matrix where the width of
the band is \xcd"K"
\end{itemize}

All the points in a region are ordered canonically by the
lexicographic total order. Thus the points of a region \xcd"R=(1..2)*(1..2)"
are ordered as 
\begin{xten}
(1,1), (1,2), (2,1), (2,2)
\end{xten}
Sequential iteration statements such as \xcd"for" (\Sref{ForAllLoop})
iterate over the points in a region in the canonical order.

A region is said to be {\em convex}\index{region!convex} if it is of
the form \xcdmath"(T$_1$ * $\cdots$ * T$_k$)" for some set of enumerations
\xcdmath"T$_i$". Such a
region satisfies the property that if two points $p_1$ and $p_3$ are
in the region, then so is every point $p_2$ between them. (Note that
\xcd"||" may produce non-convex regions from convex regions, e.g.,
\xcd"[1,1] || [3,3]" is a non-convex region.)

For each region \xcd"R", the {\em convex closure} of \xcd"R" is the
smallest convex region enclosing \xcd"R".  For each integer \xcd"i"
less than \xcd"R.rank", the term \xcd"R(i)" represents the enumeration
in the \xcd"i"th dimension of the convex closure of \xcd"R". It may be
used in a type expression wherever an enumeration may be used.

\subsection{Operations on regions}
Various non side-effecting operators (i.e., pure functions) are
provided on regions. These allow the programmer to express sparse as
well as dense regions.

Let \xcd"R" be a region. A subset of \xcd"R" is also called a
{\em sub-region}.\index{region!sub-region}

Let \xcdmath"R$_1$" and \xcdmath"R$_2$" be two regions.

\xcdmath"R$_1$ && R$_2$" is the intersection of \xcdmath"R$_1$" and
\xcdmath"R$_2$".\index{region!intersection}

\xcdmath"R$_1$ || R$_2$" is the union of the \xcdmath"R$_1$" and
\xcdmath"R$_2$".\index{region!union}

\xcdmath"R$_1$ - R$_2$" is the set difference of \xcdmath"R$_1$" and
\xcdmath"R$_2$".\index{region!set difference}

\xcdmath"R$_1$ * R$_2$" is the Cartesian product of \xcdmath"R$_1$" and
\xcdmath"R$_2$", 
formed by pairing each point in \xcdmath"R$_1$"
with every the point in \xcdmath"R$_2$".
\index{region!product}
Thus, \xcd"(1..2) * (3..4)"
is the region consisting of the points
$\{$\xcd"(1,3)", \xcd"(1,4)", \xcd"(2,3)", \xcd"(2,4)"$\}$.

Two regions are equal (\xcd"==") if they represent the same set of
points.\index{region!==}

